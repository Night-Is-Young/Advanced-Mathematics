\documentclass{ctexart}
\usepackage{template}
\begin{document}\pagestyle{empty}
\begin{center}\large
    不定积分练习
\end{center}
\begin{problem}[Problem 1.]
    求不定积分$$\int{\dfrac{\sqrt{x^2-a^2}}{x}}\dx$$
\end{problem}
\begin{solution}[Solution(Method I).]
    采取换元法.置$\dfrac{a}{x}=\sin t$,则
    $$\dfrac{\dx}{\di t}=-a\cdot\dfrac{1}{\sin^2x}\cdot\cos x=-\dfrac{a\cos x}{\sin^2x}$$
    从而
    $$\begin{aligned}
        \int{\dfrac{\sqrt{x^2-a^2}}{x}\dx}
        &= \int{\dfrac{-a\cos^2t\di t}{\sin^2 t}} \\
        &= -a\int\dfrac{1-\sin^2t}{\sin^2t}\di t \\
        &= -a\left(\int{\dfrac{\di t}{\sin^2t}}-\int\di t\right) \\
        &= -a\left(-\cot t-t\right)+C
    \end{aligned}$$
    当$x>a$时有$t\in\left(0,\dfrac{\pi}{2}\right)$,于是
    $$\cot t=\sqrt{\dfrac{\cos^2t}{\sin^2t}}=\sqrt{\dfrac{1-\left(\dfrac{a}{x}\right)^2}{\left(\dfrac{a}{x}\right)^2}}=\dfrac{\sqrt{x^2-a^2}}{a}$$
    当$x<-a$时亦可知$\cot t=-\dfrac{\sqrt{x^2-a^2}}{a}$.
    于是
    $$
    \int{\dfrac{\sqrt{x^2-a^2}}{x}\dx}=
        \left\{\begin{array}{l}
            \sqrt{x^2-a^2}+a\arcsin\dfrac{a}{x}+C,x>a \\
            -\sqrt{x^2-a^2}+a\arcsin\dfrac{a}{x}+C,x<-a \\
        \end{array}\right.
    $$
\end{solution}
\begin{solution}[Solution(Method II).]
    采取换元法.置$\dfrac{a}{x}=\cos t$,则$\displaystyle\dfrac{\dx}{\di t}=\dfrac{a\sin t}{\cos^2t}$.
    从而\\\ \\
    $$\begin{aligned}
        \int{\dfrac{\sqrt{x^2-a^2}}{x}\dx}
        &= \int{\dfrac{a\sin^2t\di t}{\cos^2t}} \\
        &= a\int{\dfrac{1-\cos^2t}{\cos^2t}\di t} \\
        &= a\left(\int{\dfrac{\di t}{\cos^2t}}-\int{\di t}\right) \\
        &= a\left(\tan t-t\right)+C
    \end{aligned}$$
    于是可以得到与\textbf{Method I}相似的结果.
    $$
    \int{\dfrac{\sqrt{x^2-a^2}}{x}\dx}=
        \left\{\begin{array}{l}
            \sqrt{x^2-a^2}-a\arccos\dfrac{a}{x}+C,x>a \\
            -\sqrt{x^2-a^2}-a\arccos\dfrac{a}{x}+C,x<-a \\
        \end{array}\right.
    $$
\end{solution}
\begin{solution}[Solution(Method III).]
    直接采取分部积分法.
    $$\begin{aligned}
        \int{\dfrac{\sqrt{x^2-a^2}}{x}\dx}
        &= \sqrt{x^2-a^2}-\int x\di\left(\dfrac{\sqrt{x^2-a^2}}{x}\right) \\
        &= \sqrt{x^2-a^2}-\int x\cdot\dfrac{x\cdot\dfrac{x}{\sqrt{x^2-a^2}}-\sqrt{x^2-a^2}}{x^2}\dx \\
        &= \sqrt{x^2-a^2}-\int\dfrac{a^2x\dx}{x^2\sqrt{x^2-a^2}} \\
        &= \sqrt{x^2-a^2}-\dfrac{a^2}{2}\int\dfrac{\dx^2}{x^2\sqrt{x^2-a^2}}
    \end{aligned}$$
    置$u=\sqrt{x^2-a^2}$,则
    $$\begin{aligned}
        \int\dfrac{\dx^2}{x^2\sqrt{x^2-a^2}}
        &= \int\dfrac{\di u^2}{(u^2+a^2)u} \\
        &= 2\int{\dfrac{\di u}{u^2+a^2}} \\
        &= \dfrac{2}{a}\arctan{\dfrac{u}{a}}+C
    \end{aligned}$$
    从而$$\int{\dfrac{\sqrt{x^2-a^2}}{x}\dx}=\sqrt{x^2-a^2}-a\arctan\dfrac{\sqrt{x^2-a^2}}{a}+C$$
\end{solution}
\begin{problem}[Problem 2.]
    求不定积分$$\int{\dfrac{\dx}{x^6\sqrt{1+x^2}}}$$
\end{problem}
\begin{solution}[Solution.]
    置$x=\tan t,t\in\left(0,\dfrac{\pi}{2}\right)$,从而$\dfrac{\dx}{\di t}=\dfrac{1}{\cos^2t}$.\\
    从而
    $$\begin{aligned}
        \int{\dfrac{\dx}{x^6\sqrt{1+x^2}}}
        &= \int{\dfrac{\di t}{\tan^6t\cdot\cos^2t\cdot\dfrac{1}{\cos t}}} \\
        &= \int{\dfrac{\cos^5t}{\sin^6t}\di t}
    \end{aligned}$$
    置$u=\sin t$,则$u^2=\dfrac{\sin^2t}{\sin^2t+\cos^2t}=\dfrac{\tan^2t}{\tan^2t+1}=\dfrac{x^2}{x^2+1}$.\\
    由于$x$与$u$同号,则$u=\dfrac{x}{\sqrt{1+x^2}}$.\\
    于是
    $$\begin{aligned}
        \int{\dfrac{\cos^5t}{\sin^6t}\di t}
        &= \int{\dfrac{\cos^4t\left(\cos t\di t\right)}{\sin^6t}} \\
        &= \int{\dfrac{\left(1-u^2\right)^2\di u}{u^6}} \\
        &= \int{\left(\dfrac{1}{u^6}-\dfrac{2}{u^4}+\dfrac{1}{u^2}\right)\di u} \\
        &= -\dfrac{1}{5u^5}+\dfrac{2}{3u^3}-\dfrac{1}{u}+C \\
        &= -\dfrac{\left(x^2+1\right)^\frac{5}{2}}{5x^5}+\dfrac{2\left(x^2+1\right)^\frac{3}{2}}{3x^3}-\dfrac{\left(x^2+1\right)^\frac{1}{2}}{x}+C
    \end{aligned}$$
\end{solution}
\begin{problem}[Problem 3.]
    求不定积分$$\int{\dfrac{\dx}{\sqrt[3]{(x+1)(x-1)^5}}}$$
\end{problem}
\begin{solution}[Solution.]
    置$t=\sqrt[3]{\dfrac{x+1}{x-1}}$,则
    $$\dfrac{\di t}{\dx}=\dfrac{1}{3}\left(\dfrac{x+1}{x-1}\right)^{-\frac{2}{3}}\dfrac{-2}{(x-1)^2}=-\dfrac{2}{3}\dfrac{1}{\sqrt[3]{(x+1)^2(x-1)^4}}$$
    从而
    $$\begin{aligned}
        \int{\dfrac{\dx}{\sqrt[3]{(x+1)(x-1)^5}}}
        &= -\dfrac{3}{2}\int t\di t \\
        &= -\dfrac{3}{4}t^2+C \\
        &= -\dfrac{3}{4}\left(\dfrac{x+1}{x-1}\right)^{\frac{2}{3}}+C
    \end{aligned}$$
\end{solution}
\begin{problem}[Problem 4.]
    求不定积分$$\int\dfrac{x\arccos x}{\left(1-x^2\right)^2}\dx$$
\end{problem}
\begin{solution}[Solution(Method I).]
    置$t=\arccos x$,则$x=\cos t$,从而$\dfrac{\dx}{\di t}=-\sin t$.置$u=\sin t$.\\
    从而
    $$\begin{aligned}
        \int\dfrac{x\arccos x}{\left(1-x^2\right)^2}\dx
        &= -\int{\dfrac{t\cos t\sin t\di t}{\sin^4t}} \\
        &= -\int{\dfrac{t\di\left(\sin t\right)}{\sin^3t}}
        = -\int{\dfrac{t\di u}{u^3}} \\
        &= \dfrac{1}{2}\int{t\di\left(\dfrac{1}{u^2}\right)}
        = \dfrac{1}{2}\left(\dfrac{t}{u^2}-\int{\dfrac{\di t}{u^2}}\right) \\
        &= \dfrac{1}{2}\left(\dfrac{\arccos x}{1-x^2}-\int{\dfrac{\di t}{\sin^2 t}}\right) \\
        &= \dfrac{1}{2}\left(\dfrac{\arccos x}{1-x^2}+\cot t\right)+C \\
        &= \dfrac{1}{2}\left(\dfrac{\arccos x}{1-x^2}+\dfrac{x}{\sqrt{1-x^2}}\right)+C
    \end{aligned}$$
\end{solution}
\begin{solution}[Solution(Method II).]
    直接分部积分有
    $$\begin{aligned}
        \int\dfrac{x\arccos x}{\left(1-x^2\right)^2}\dx
        &= \dfrac{1}{2}\int{\arccos{x}\di\left(\dfrac{1}{1-x^2}\right)} \\
        &= \dfrac{1}{2}\left(\dfrac{\arccos{x}}{1-x^2}-\int\dfrac{\di\arccos{x}}{1-x^2}\right) \\
        &= \dfrac{1}{2}\left(\dfrac{\arccos{x}}{1-x^2}+\int\dfrac{\dx}{\left(1-x^2\right)^\frac{3}{2}}\right) \\
    \end{aligned}$$
    置$x=\sin t$,则有
    $$\int\dfrac{\dx}{\left(1-x^2\right)^\frac{3}{2}}=\int{\dfrac{\di\sin t}{\cos^3t}}=\int{\dfrac{\di t}{\cos^2t}}=\tan t+C=\dfrac{x}{\sqrt{1-x^2}}+C$$
    于是$$\int\dfrac{x\arccos x}{\left(1-x^2\right)^2}\dx=\dfrac{1}{2}\left(\dfrac{\arccos x}{1-x^2}+\dfrac{x}{\sqrt{1-x^2}}\right)+C$$
\end{solution}
\begin{problem}[Problem 5.]
    求不定积分$$\int{x\ln{\left(x+\sqrt{1+x^2}\right)}}\dx$$ 
\end{problem}
\begin{solution}[Solution.]
    $$\begin{aligned}
        \int{x\ln{\left(x+\sqrt{1+x^2}\right)}}\dx
        &= x^2\ln{\left(x+\sqrt{1+x^2}\right)}-\int{x\di\left(x\ln{\left(x+\sqrt{1+x^2}\right)}\right)} \\
        &= x^2\ln{\left(x+\sqrt{1+x^2}\right)}-\int{x\left(\ln{\left(x+\sqrt{1+x^2}\right)}+\dfrac{x\left(1+\dfrac{x}{\sqrt{1+x^2}}\right)}{x+\sqrt{1+x^2}}\right)\dx} \\
        &= x^2\ln{\left(x+\sqrt{1+x^2}\right)}-\int{x\ln{\left(x+\sqrt{1+x^2}\right)}\dx}-\int{\dfrac{x^2\di x}{\sqrt{1+x^2}}}
    \end{aligned}$$
    而$$\int{\dfrac{x^2\dx}{\sqrt{1+x^2}}}=\int{x\di{\left(\sqrt{1+x^2}\right)}}=x\sqrt{1+x^2}-\int{\sqrt{1+x^2}\dx}$$
    又$$\int{\dfrac{x^2\dx}{\sqrt{1+x^2}}}=\int{\dfrac{\left(x^2+1-1\right)\dx}{\sqrt{1+x^2}}}=\int{\sqrt{1+x^2}\dx}-\int{\dfrac{\dx}{\sqrt{1+x^2}}}$$
    两式相加有$$2\int{\dfrac{x^2\dx}{\sqrt{1+x^2}}}=x\sqrt{1+x^2}-\int{\dfrac{\dx}{\sqrt{1+x^2}}}$$
    从而$$\begin{aligned}
        \int{x\ln{\left(x+\sqrt{1+x^2}\right)}}\dx
        &= \dfrac{1}{2}\left(x^2\ln{\left(x+\sqrt{1+x^2}\right)}-\int{\dfrac{x^2\dx}{\sqrt{1+x^2}}}\right) \\
        &= \dfrac{1}{2}x^2\ln{\left(x+\sqrt{1+x^2}\right)}-\dfrac{1}{4}x\sqrt{1+x^2}+\dfrac{1}{4}\int{\dfrac{\dx}{\sqrt{1+x^2}}} \\
        &= \dfrac{1}{2}x^2\ln{\left(x+\sqrt{1+x^2}\right)}-\dfrac{1}{4}x\sqrt{1+x^2}+\dfrac{1}{4}\ln{\left(x+\sqrt{x^2+1}\right)}+C
    \end{aligned}$$
\end{solution}
\begin{problem}[Problem 6.]
    求不定积分$$\int\dfrac{\e^{\arctan{x}}\dx}{\left(1+x^2\right)^\frac{3}{2}}$$
\end{problem}
\begin{solution}[Solution(Method I).]
    注意到$\dfrac{\di\left(\e^{\arctan{x}}\right)}{\dx}=\dfrac{\e^{\arctan{x}}}{1+x^2}$,从而
    $$\int\dfrac{\e^{\arctan{x}}\dx}{\left(1+x^2\right)^\frac{3}{2}}
    =\int{\dfrac{\di\left(\e^{\arctan{x}}\right)}{\sqrt{1+x^2}}}
    =\dfrac{\e^{\arctan{x}}}{\sqrt{1+x^2}}-\int{\dfrac{x\e^{\arctan{x}}\dx}{\left(1+x^2\right)^\frac{3}{2}}}$$
    而
    $$\int{\dfrac{x\e^{\arctan{x}}\dx}{\left(1+x^2\right)^\frac{3}{2}}}
    =\int{\dfrac{x\di\left(\e^{\arctan{x}}\right)}{\sqrt{1+x^2}}}
    =\dfrac{x\e^{\arctan{x}}}{\sqrt{1+x^2}}-\int{\dfrac{\e^{\arctan{x}}\dx}{\left(1+x^2\right)^\frac{3}{2}}}$$
    两式相加可得
    $$\int\dfrac{\e^{\arctan{x}}\dx}{\left(1+x^2\right)^\frac{3}{2}}
    =\dfrac{\e^{\arctan{x}}(1+x)}{2\sqrt{1+x^2}}+C$$
\end{solution}
\begin{solution}[Solution(Method II).]
    置$x=\tan t$,则
    $$\begin{aligned}
        \int\dfrac{\e^{\arctan{x}}\dx}{\left(1+x^2\right)^\frac{3}{2}}
        &= \int{\cos^3t\e^t\di(\tan t)} \\
        &= \int{\e^t\cos t\di t} \\
        &= \int{\cos t\di\left(\e^t\right)} \\
        &= \e^t\cos t+\int{\e^t\sin t\di t} \\
        &= \e^t\cos t+\e^t\sin t-\int{\e^t\cos t\di t}
    \end{aligned}$$
    从而$$\begin{aligned}
        \int\dfrac{\e^{\arctan{x}}\dx}{\left(1+x^2\right)^\frac{3}{2}}
        &= \int{\e^t\cos t\di t} \\
        &= \dfrac{1}{2}\left(\e^t\cos t+\e^t\sin t\right) \\
        &= \dfrac{\e^{\arctan{x}}(1+x)}{2\sqrt{1+x^2}}+C
    \end{aligned}$$
\end{solution}
\begin{problem}[Problem 7.]
    求不定积分$$\int{\dfrac{x\ln{x}}{\left(1+x^2\right)^2}\dx}$$
\end{problem}
\begin{solution}[Solution.]
    置$u=x^2$,于是
    $$\begin{aligned}
        \int{\dfrac{x\ln{x}}{\left(1+x^2\right)^2}\dx}
        &= \dfrac{1}{4}\int\dfrac{\ln u\di u}{\left(1+u\right)^2} \\
        &= -\dfrac{1}{4}\int\ln u\di\left(\dfrac{1}{u+1}\right) \\
        &= -\dfrac{1}{4}\left(\dfrac{\ln u}{u+1}-\int\dfrac{\di(\ln u)}{u+1}\right) \\
        &= -\dfrac{1}{4}\left(\dfrac{\ln u}{u+1}-\int\left(\dfrac{1}{u}-\dfrac{1}{u+1}\right)\di u\right) \\
        &= -\dfrac{1}{4}\left(\dfrac{\ln u}{u+1}+\ln\left|u+1\right|-\ln\left|u\right|\right)+C \\
        &= \dfrac{x^2\ln x}{2\left(x^2+1\right)}-\dfrac{1}{4}\ln\left|x^2+1\right|+C
    \end{aligned}$$
\end{solution}
\begin{problem}[Problem 8.]
    求不定积分$$\int{\dfrac{\dx}{x\left(x^5+1\right)}}$$
\end{problem}
\begin{solution}[Solution.]
    拆分原式可得
    $$\begin{aligned}
        \int{\dfrac{\dx}{x\left(x^5+1\right)}}
        &= \int{\dfrac{\left(x^5+1\right)-x^5}{x\left(x^5+1\right)}\dx} \\
        &= \int\dfrac{\dx}{x}-\int\dfrac{x^4\dx}{x^5+1} \\
        &= \int\dfrac{\dx}{x}-\dfrac{1}{5}\int\dfrac{\di\left(x^5\right)}{x^5+1} \\
        &= \ln{\left|x\right|}-\dfrac{1}{5}\ln{\left|x^5+1\right|}+C
    \end{aligned}$$
\end{solution}
\begin{problem}[Problem 9.]
    求不定积分$$\int\dfrac{\cos x}{5-3\cos x}\dx$$
\end{problem}
\begin{solution}[Solution.]
    置$t=\tan{\dfrac{x}{2}}$,于是$\cos x=\dfrac{1-t^2}{1+t^2},\dfrac{\dx}{\di t}=\dfrac{2}{1+t^2}$.于是
    $$\begin{aligned}
        \int\dfrac{\cos x}{5-3\cos x}\dx
        &= \int\dfrac{\dfrac{1-t^2}{1+t^2}}{5-3\cdot\dfrac{1-t^2}{1+t^2}}\cdot\dfrac{2}{1+t^2}\di t \\
        &= \int\dfrac{2\left(1-t^2\right)}{\left(1+t^2\right)\left(2+8t^2\right)}\di t \\
        &= \int\dfrac{1-t^2}{\left(1+t^2\right)\left(4t^2+1\right)}\di t \\
        &= \int\left(\dfrac{5}{3}\cdot\dfrac{1}{4t^2+1}-\dfrac{2}{3}\cdot\dfrac{1}{t^2+1}\right)\di t \\
        &= \dfrac{5}{6}\arctan 2t-\dfrac{2}{3}\arctan t+C \\
        &= -\dfrac{x}{3}+\dfrac{5}{6}\arctan\left(2\tan\dfrac{x}{2}\right)+C
    \end{aligned}$$
\end{solution}
\begin{problem}[Problem 10.]
    求不定积分$$\int\dfrac{x\ln x}{\left(x^2+1\right)^{\frac{3}{2}}}\dx$$
\end{problem}
\begin{solution}[Solution.]
    置$u=x^2$,于是
    $$\begin{aligned}
        \int\dfrac{x\ln x}{\left(x^2+1\right)^{\frac{3}{2}}}\dx
        &= \dfrac{1}{4}\int\dfrac{\ln u\di u}{\left(u+1\right)^{\frac{3}{2}}} \\
        &= 
    \end{aligned}$$
\end{solution}
\begin{problem}[Problem 11.]
    求定积分$$\int_0^{2m\pi}\dfrac{\dx}{\sin^4x+\cos^4x}$$
\end{problem}
\end{document}