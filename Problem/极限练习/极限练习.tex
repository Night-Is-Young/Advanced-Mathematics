\documentclass{ctexart}
\usepackage{geometry}
\usepackage[dvipsnames,svgnames]{xcolor}
\usepackage{framed}
\usepackage{enumerate}
\usepackage{amsmath,amsthm,amssymb}
\usepackage{enumitem}
\usepackage{template}

\allowdisplaybreaks
\geometry{left=2cm, right=2cm, top=2.5cm, bottom=2.5cm}

\begin{document}\pagestyle{empty}
\begin{center}\large
    极限练习
\end{center}
\begin{problem}[Problem 1.]
    求序列极限$$\lim_{n\to\infty}{\prod_{i=1}^{n}\left(1+\dfrac{i}{n^2}\right)}$$
\end{problem}
\begin{solution}[Solution.]
    熟知当$x\in(1,+\infty)$时有$$1-\dfrac{1}{x}<\ln x<x-1$$
    我们对原式两端取对数后可得
    $$\sum_{i=1}^{n}\left(1-\dfrac{1}{1+\dfrac{i}{n^2}}\right)<\sum_{i=1}^{n}\ln\left(1+\dfrac{i}{n^2}\right)<\sum_{i=1}^{n}\dfrac{i}{n^2}$$
    即$$\sum_{i=1}^{n}\dfrac{i}{n^2+i}<\sum_{i=1}^{n}\ln\left(1+\dfrac{i}{n^2}\right)<\dfrac{n+1}{2n}$$
    又$i\in[1,n]$时
    $$\dfrac{i}{n^2+n}<\dfrac{i}{n^2+i}<\dfrac{i}{n^2}$$
    于是$$\dfrac{1}{2}=\sum_{i=1}^{n}\dfrac{i}{n^2+n}<\sum_{i=1}^{n}\dfrac{i}{n^2+i}<\sum_{i=1}^{n}\dfrac{i}{n^2}=\dfrac{1}{2}+\dfrac{1}{2n}$$
    夹逼可得$$\lim_{n\to\infty}\sum_{i=1}^{n}\ln\left(1+\dfrac{i}{n^2}\right)=\dfrac{1}{2}$$
    于是$$\lim_{n\to\infty}{\prod_{i=1}^{n}\left(1+\dfrac{i}{n^2}\right)}=\sqrt{e}$$
\end{solution}
\begin{problem}[Problem 2.]
    设$S_n=1^1+2^2+3^3+\cdots+n^n$,试证明:当$n\geqslant2$时有
    $$n^n\left(1+\dfrac{1}{4(n-1)}\right)\leqslant S_n<n^n\left(1+\dfrac{2}{\e(n-1)}\right)$$
\end{problem}
\begin{solution}[Proof.]
    记$A_n=\dfrac{S_n}{n^n}$,则
    $$\begin{aligned}
        A_n
        &= \dfrac{\sum_{i=1}^{n}i^i}{n^n}
        = 1+\dfrac{\sum_{i=1}^{n-1}i^i}{n^n} \\
        &= 1+\dfrac{(n-1)^{n-1}}{n^n}\cdot\dfrac{\sum_{i=1}^{n-1}i^i}{(n-1)^{n-1}} \\
        &= 1+\dfrac{(n-1)^{n-1}}{n^n}A_{n-1} \\
        &= 1+\dfrac{(n-1)^{n-1}}{n^n}\left(1+\dfrac{(n-2)^{n-2}}{(n-1)^{n-1}}A_{n-2}\right) \\
    \end{aligned}$$
    一方面,我们有
    $$\begin{aligned}
        A
        &= 1+\dfrac{(n-1)^{n-1}}{n^n}\left(1+\dfrac{(n-2)^{n-2}}{(n-1)^{n-1}}A_{n-2}\right) \\
        &\leqslant 1+\dfrac{(n-1)^{n-1}}{n^n}\left(1+\dfrac{(n-2)^{n-2}}{(n-1)^{n-1}}\cdot\dfrac{(n-2)(n-2)^{n-2}}{(n-2)^{n-2}}\right) \\
        &= 1+\left(1-\dfrac{1}{n}\right)^{n}\cdot\dfrac{1}{n-1}\left(1+\dfrac{(n-2)^{n-1}}{(n-1)^{n-1}}\right) \\
        &< 1+\dfrac{1}{\e(n-1)}(1+1) \\
        &= 1+\dfrac{2}{\e(n-1)}
    \end{aligned}$$
    另一方面,我们有
    $$\begin{aligned}
        A
        &= 1+\dfrac{(n-1)^{n-1}}{n^n}\left(1+\dfrac{(n-2)^{n-2}}{(n-1)^{n-1}}A_{n-2}\right) \\
        &\geqslant 1+\dfrac{(n-1)^{n-1}}{n^n} \\
        &= 1+\left(1-\dfrac{1}{n}\right)^{n}\cdot\dfrac{1}{n-1} \\
        &\geqslant 1+\left(1-\dfrac{1}{2}\right)^{2}\cdot\dfrac{1}{n-1} \\
        &= 1+\dfrac{1}{4(n-1)}
    \end{aligned}$$
    于是$$1+\dfrac{1}{4(n-1)}<\dfrac{S_n}{n^n}<1+\dfrac{2}{\e(n-1)}$$
    于是原命题得证.
\end{solution}
\begin{problem}[Problem 3.]
    对于实数$p>1$,序列$\left\{a_n\right\}$满足$$a_n=\sum_{i=1}^{n}\dfrac{1}{i^p}$$
    试证明$\left\{a_n\right\}$收敛.
\end{problem}
\begin{solution}[Proof.]
    我们仿照证明调和级数发散的方法放缩之,则$n\geqslant2$时有
    $$\begin{aligned}
        \dfrac{1}{1^p}+\dfrac{1}{2^p}+\dfrac{1}{3^p}+\cdots+\dfrac{1}{n^p}
        &< \dfrac{1}{1^p}+\dfrac{1}{2^p}+\dfrac{1}{3^p}+\cdots+\dfrac{1}{\left(2^n-1\right)^p} \\
        &< \dfrac{1}{1^p}+\dfrac{1}{2^p}+\dfrac{1}{2^p}+\dfrac{1}{4^p}+\dfrac{1}{4^p}+\dfrac{1}{4^p}+\dfrac{1}{4^p}+\cdots+\dfrac{2^{n-1}}{2^{(n-1)p}} \\
        &= \sum_{i=1}^{n}\dfrac{2^{i-1}}{2^{p(i-1)}}=\sum_{i=1}^{n}\left(2^{1-p}\right)^{i-1} \\
        &= \dfrac{1-2^{(1-p)n}}{1-2^{1-p}}<\dfrac{1}{1-2^{1-p}}\\
    \end{aligned}$$
    又$\left\{a_n\right\}$单调递增.于是$\left\{a_n\right\}$存在极限.
\end{solution}
\begin{problem}[Problem 4.]
    设序列$\left\{a_n\right\}$满足$$a_n=\sum_{i=1}^{n}\dfrac{1}{\sqrt{i}}-2\sqrt{n}$$
    试证明$\left\{a_n\right\}$收敛.
\end{problem}
\begin{analyze}[Analysis.]
    根据朴素的认识,我们有
    $$\sum_{i=1}^n\dfrac{1}{\sqrt{i}}\simeq\int_1^n\dfrac{1}{\sqrt{x}}\dx=2\sqrt{n}-2$$
    我们要做的就是通过放缩证明求和与积分的差距越来越小.
\end{analyze}
\begin{solution}[Proof.]
    构造函数$f(x)=\dfrac{1}{\sqrt{x}}$,则$\forall k\in\N^*$有
    $$f(k+1)<f(x)<f(k)$$
    于是
    $$\int_k^{k+1}f(x)\dx<\int_k^{k+1}f(k)\dx=\dfrac{1}{\sqrt{k}}$$
    $$\int_k^{k+1}f(x)\dx>\int_k^{k+1}f(k+1)\dx=\dfrac{1}{\sqrt{k+1}}$$
    于是$$\sum_{i=1}^{n}\dfrac{1}{\sqrt{i}}>\sum_{i=1}^{n}\int_i^{i+1}f(x)\dx=\int_{1}^{n+1}f(x)\dx=2\sqrt{n+1}-2$$
    $$\sum_{i=1}^{n}\dfrac{1}{\sqrt{i}}<\sum_{i=1}^{n}\int_{i-1}^{i}f(x)\dx=\int_{0}^{n}f(x)\dx=2\sqrt{n}$$
    于是$$2\sqrt{n+1}-2\sqrt{n}-2<a_n<0$$
    即$$-2<a_n<0$$
    又$$a_{n+1}-a_n=\dfrac{1}{\sqrt{n+1}}-2\sqrt{n+1}+2\sqrt{n}=\dfrac{1}{\sqrt{n+1}}-\dfrac{2}{\sqrt{n+1}+\sqrt{n}}<0$$
    即$\left\{a_n\right\}$单调递减.于是原命题得证.
\end{solution}
\begin{problem}[Problem 5.]
    设序列$\left\{a_n\right\}$满足$$a_n=\underbrace{\sqrt{1+\sqrt{1+\cdots+\sqrt{1}}}}_{n\text{重根号}}$$
    试证明$\displaystyle\lim_{n\to\infty}a_n$存在,并求其值.
\end{problem}
\begin{solution}[Proof.]
    观察递推式,我们可以得到$$a_{n+1}=\sqrt{1+a_n}$$
    下面证明$\forall n\in\N^*,1\leqslant{a_n}<\dfrac{\sqrt{5}+1}{2}$.\\
    首先,$n=1$时$a_1=1$,成立.\\
    假定$n=k$时上述命题成立,那么当$n=k+1$时有$$\sqrt{1+1}\leqslant a_{k+1}<\sqrt{1+\dfrac{\sqrt{5}+1}{2}}$$
    即$$\sqrt{2}\leqslant a_{k+1}<\dfrac{\sqrt{5}}{2}$$
    于是$$\dfrac{a_{n+1}}{a_n}=\dfrac{\sqrt{1+a_n}}{a_n}=\sqrt{\dfrac{1}{a_n^2}+\dfrac{1}{a_n}}>1$$
    由此可得$\left\{a_n\right\}$递增且有上界.设$\lim_{n\to\infty}a_n=A$,对递推式两边求极限有
    $$A=\sqrt{1+A}$$
    解得$A=\dfrac{\sqrt{5}+1}{2}$,于是$\displaystyle\lim_{n\to\infty}a_n=\dfrac{\sqrt{5}+1}{2}$.
\end{solution}
\begin{problem}[Problem 6.]
    设序列$\left\{a_n\right\}$满足$$a_n=\underbrace{\sqrt{1+\sqrt{2+\cdots+\sqrt{n}}}}_{n\text{重根号}}$$
    试证明$\displaystyle\lim_{n\to\infty}a_n$存在.
\end{problem}
\begin{solution}[Solution(Method I).]
    我们有$$1<\sqrt{1+\sqrt{2+\cdots+\sqrt{n}}}<\sqrt{4+\sqrt{16+\cdots+\sqrt{2^{2^n}}}}=\dfrac{1}{2}\sqrt{1+\sqrt{1+\cdots+\sqrt{1}}}<\sqrt{5}+1$$
    又$\left\{a_n\right\}$显然递增.于是原命题得证.
\end{solution}
\begin{solution}[Solution(Method II).]
    注意到$\forall n>4,\sqrt{2n}\leqslant n-1$.\\
    于是$$\begin{aligned}
        a_n
        &= \sqrt{1+\sqrt{2+\cdots+\sqrt{n-1+\sqrt{n}}}} \\
        &< \sqrt{1+\sqrt{2+\cdots+\sqrt{n-1+\sqrt{2n}}}} \\
        &\leqslant \sqrt{1+\sqrt{2+\cdots+\sqrt{n-2+\sqrt{2(n-1)}}}} \\
        &< \cdots \\
        &<\sqrt{1+\sqrt{2+\sqrt{3+2\sqrt{4}}}}
    \end{aligned}$$
    即$\left\{a_n\right\}$有上界.又$\left\{a_n\right\}$显然递增,于是原命题得证.
\end{solution}
\begin{problem}[Problem 7.]
    设序列$\left\{a_n\right\}$满足$$a_n=\underbrace{\sqrt[n]{n+\sqrt[n]{n+\cdots+\sqrt[n]{n}}}}_{n\text{重根号}}$$
    试证明$\displaystyle\lim_{n\to\infty}a_n$存在,并求其值.
\end{problem}
\begin{solution}[Proof.]
    注意到$n>2$时$\sqrt[n]{n+2}<2$,于是
    $$\begin{aligned}
        a_n
        &= \sqrt[n]{n+\sqrt[n]{n+\cdots+\sqrt[n]{n+\sqrt[n]{n}}}} \\
        &< \sqrt[n]{n+\sqrt[n]{n+\cdots+\sqrt[n]{n+2}}} \\
        &< \cdots \\
        &< \sqrt[n]{n+2} \\
    \end{aligned}$$
    即$$\sqrt[n]{n}<a_n<\sqrt[n]{n+2}\ \ \ \ \forall n>2$$
    对上述不等式两边求极限,夹逼可得$\displaystyle\lim_{n\to\infty}a_n=1$.
\end{solution}
\begin{problem}[Problem 8.]
    设$\displaystyle x_n=\dfrac{1}{n^2}\sum_{k=0}^{n}\ln{\binom{n}{k}}$,求$\displaystyle\lim_{n\to\infty}x_n$.
\end{problem}
\begin{solution}[Solution.]
    置$\displaystyle a_n=\sum_{k=0}^{n}\ln{\binom{n}{k}},b_n=n^2$.\\
    于是据Stolz定理有
    $$\lim_{n\to\infty}x_n=\lim_{n\to\infty}\dfrac{a_n}{b_n}=\lim_{n\to\infty}\dfrac{a_{n+1}-a_n}{b_{n+1}-b_n}$$
    而$$a_{n+1}-a_n=\ln{\left(\dfrac{(n+1)!^{n+2}\prod_{k=0}^{n}k!^2}{n!^{n+1}\prod_{k=0}^{n+1}k!^2}\right)}=\ln\dfrac{(n+1)!^n}{n!^{n+1}}=\ln\dfrac{(n+1)^n}{n!}$$
    于是$$\lim_{n\to\infty}\dfrac{a_{n+1}-a_n}{b_{n+1}-b_n}=\lim_{n\to\infty}\dfrac{1}{2n+1}\ln\dfrac{(n+1)^n}{n!}$$
    置$c_n=\ln\dfrac{(n+1)^n}{n!},d_n=2n+1$.\\
    于是据Stolz定理有
    $$\begin{aligned}
        \lim_{n\to\infty}\dfrac{a_{n+1}-a_n}{b_{n+1}-b_n}
        &= \lim_{n\to\infty}\dfrac{c_n}{d_n} = \lim_{n\to\infty}\dfrac{c_n-c_{n-1}}{d_n-d_{n-1}} \\
        &= \dfrac{1}{2}\lim_{n\to\infty}\ln\left(\dfrac{(n+1)^n}{n!}\cdot\dfrac{(n-1)!}{n^{n-1}}\right) \\
        &= \dfrac{1}{2}\lim_{n\to\infty}\ln\left[\left(\dfrac{n+1}{n}\right)^n\right] \\
        &= \dfrac{1}{2}\ln\e = \dfrac{1}{2}
    \end{aligned}$$
\end{solution}
\begin{problem}[Problem 9.]
    序列$\left\{a_n\right\}$满足$\displaystyle\lim_{n\to\infty}a_n\sum_{i=1}^{n}a_i^2=1$,试证明:$\displaystyle\lim_{n\to\infty}\sqrt[3]{3n}a_n=1$.
\end{problem}
\begin{solution}[Proof.]
    置$\displaystyle S_n=\sum_{i=1}^na_i^2$,于是
    $$\lim_{n\to\infty}\sqrt[3]{3n}a_n=\lim_{n\to\infty}\sqrt[3]{3na_n^3}=\lim_{n\to\infty}\sqrt[3]{\dfrac{3n}{S_n^3}\cdot\left(a_nS_n\right)^3}=\lim_{n\to\infty}\sqrt[3]{\dfrac{3n}{S_n^3}}$$
    依Stolz定理有
    $$\begin{aligned}
        \lim_{n\to\infty}\dfrac{S_n^3}{3n}
        &= \lim_{n\to\infty}\dfrac{S_n^3-S_{n-1}^3}{3n-3(n-1)} \\
        &= \lim_{n\to\infty}\dfrac{(S_n-S_{n-1})(S_n^2+S_nS_{n-1}+S_{n-1}^2)}{3} \\
        &= \lim_{n\to\infty}\dfrac{a_n^2(S_n^2+S_nS_{n-1}+S_{n-1}^2)}{3} \\
        &= \lim_{n\to\infty}\left(a_nS_n\right)^2\dfrac{1+\dfrac{S_{n-1}}{S_n}+\left(\dfrac{S_{n-1}}{S_n}\right)^2}{3}
    \end{aligned}$$
    下面证明$\displaystyle\lim_{n\to\infty}a_n=0$.\\
    若否,则$\exists \ep>0,\forall N\in\N^*,\exists n>N\st \left|a_n\right|>\ep$,也即$\left\{a_n\right\}$中有无穷多项$a_i$满足$\left|a_i\right|>\ep$.\\
    我们取这样的$\left[\dfrac{1}{\ep^3}\right]+2$个$a_i$,记$\displaystyle \max_{\left|a_i\right|>\ep}\left\{i\right\}=N_0$,于是当$n>N_0$且$\left|a_n\right|>\ep$时总有
    $$a_n\sum_{i=1}^{n}a_i^2>\ep\sum_{i=1}^{\left[\frac{1}{\ep^3}\right]+2}\ep^2>1+\ep^3$$
    于是$\exists \ep'=\ep^3>0,\forall N'\in\N^*,\exists n=\max\left\{N',N_0\right\}\text{且}\left|a_n\right|>\ep\st\left|a_nS_n-1\right|>\ep'$.\\
    这与题设中$\displaystyle\lim_{n\to\infty}a_n\sum_{i=1}^{n}a_i^2=1$矛盾,于是$\displaystyle\lim_{n\to\infty}a_n=0$.
    从而$$\lim_{n\to\infty}\dfrac{S_{n-1}}{S_n}=1-\lim_{n\to\infty}{\dfrac{a_n^2}{S_n}}=1-\lim_{n\to\infty}a_n^3=1$$
    从而$$\lim_{n\to\infty}\dfrac{S_n^3}{3n}=1$$
    于是原命题得证.
\end{solution}
\end{document}