\documentclass{ctexart}
\usepackage{geometry}
\usepackage[dvipsnames,svgnames]{xcolor}
\usepackage[strict]{changepage}
\usepackage{framed}
\usepackage{enumerate}
\usepackage{amsmath,amsthm,amssymb}
\usepackage{enumitem}
\usepackage{template}

\allowdisplaybreaks
\geometry{left=2cm, right=2cm, top=2.5cm, bottom=2.5cm}

\begin{document}
\pagestyle{empty}
\begin{problem}[古早计算大赛题目]
    试证明:球坐标系下Laplace算子为
    $$\Delta=\dfrac{1}{r^2}\dfrac{\p}{\p{r}}\left(r^2\dfrac{\p}{\p{r}}\right)+\dfrac{1}{r^2\sin\theta}\dfrac{\p}{\p\theta}\left(\sin\theta\dfrac{\p}{\p\theta}\right)+\dfrac{1}{r^2\sin^2\theta}\dfrac{\p^2}{\p\phi^2}$$
\end{problem}
\begin{solution}[Proof.]
    我们知道笛卡尔坐标系下的Laplace算子为
    $$\Delta=\dfrac{\p^2}{\p{x^2}}+\dfrac{\p^2}{\p{y^2}}+\dfrac{\p^2}{\p{z^2}}$$
    以及笛卡尔坐标系向球坐标系的变换
    $$\left\{\begin{array}{l}
    x=r\sin\theta\cos\phi\\
    y=r\sin\theta\sin\phi\\
    z=r\cos\theta
    \end{array}\right.$$
    考虑函数$f(x,y,z):\R^3\to\R$,则有
    $$\dfrac{1}{r^2}\dfrac{\p}{\p{r}}\left(r^2\dfrac{\p{f}}{\p{r}}\right)
    =\dfrac{1}{r^2}\left(2r\dfrac{\p{f}}{\p{r}}+r^2\dfrac{\p^2 f}{\p{r^2}}\right)
    =\dfrac{2}{r}\dfrac{\p{f}}{\p{r}}+\dfrac{\p^2f}{\p{r^2}}
    $$
    $$\dfrac{1}{r^2\sin\theta}\dfrac{\p}{\p\theta}\left(\sin\theta\dfrac{\p{f}}{\p\theta}\right)
    =\dfrac{1}{r^2\sin\theta}\left(\cos\theta\dfrac{\p{f}}{\p{\theta}}+\sin\theta\dfrac{\p^2f}{\p\theta^2}\right)
    =\dfrac{1}{r^2\tan\theta}\dfrac{\p{f}}{\p{\theta}}+\dfrac{1}{r^2}\dfrac{\p^2f}{\p\theta^2}$$
    而
    $$\dfrac{\p^2{f}}{\p{r^2}}=\dfrac{\p}{\p{r}}\left(\dfrac{\p{f}}{\p{r}}\right)=\dfrac{\p}{\p{r}}\left(\dfrac{\p{f}}{\p{x}}\cdot\dfrac{\p{x}}{\p{r}}+\dfrac{\p{f}}{\p{y}}\cdot\dfrac{\p{y}}{\p{r}}+\dfrac{\p{f}}{\p{z}}\cdot\dfrac{\p{z}}{\p{r}}\right)$$
    我们知道$$\dfrac{\p}{\p{r}}\left(\dfrac{\p{f}}{\p{x}}\cdot\dfrac{\p{x}}{\p{r}}\right)=\dfrac{\p^2x}{\p{r^2}}\cdot\dfrac{\p{f}}{\p{x}}+\dfrac{\p^2f}{\p{x^2}}\cdot\left(\dfrac{\p{x}}{\p{r}}\right)^2$$
    而
    $$\dfrac{\p{x}}{\p{r}}=\sin\theta\cos\phi,\dfrac{\p^2x}{\p{r^2}}=0$$
    $$\dfrac{\p{y}}{\p{r}}=\sin\theta\sin\phi,\dfrac{\p^2y}{\p{r^2}}=0$$
    $$\dfrac{\p{z}}{\p{r}}=\cos\theta,\dfrac{\p^2z}{\p{r^2}}=0$$
    代入可得
    $$\begin{aligned}
        \dfrac{1}{r^2}\dfrac{\p}{\p{r}}\left(r^2\dfrac{\p{f}}{\p{r}}\right)
        &= \sin^2\theta\cos^2\phi\dfrac{\p^2f}{\p{x^2}}+\sin^2\theta\sin^2\phi\dfrac{\p^2f}{\p{y^2}}+\sin^2\theta\dfrac{\p^2f}{\p{z^2}}\\
        &+ \dfrac{2}{r}\left(\sin\theta\cos\phi\dfrac{\p{f}}{\p{x}}+\sin\theta\sin\phi\dfrac{\p{f}}{\p{y}}+\cos\theta\dfrac{\p{f}}{\p{z}}\right)
    \end{aligned}$$
    又
    $$\dfrac{\p{x}}{\p{\theta}}=r\cos\theta\cos\phi,\dfrac{\p^2x}{\p{\theta^2}}=-r\sin\theta\cos\phi$$
    $$\dfrac{\p{y}}{\p{\theta}}=r\cos\theta\sin\phi,\dfrac{\p^2y}{\p{\theta^2}}=-r\sin\theta\sin\phi$$
    $$\dfrac{\p{z}}{\p{\theta}}=-r\sin\theta,\dfrac{\p^2z}{\p{\theta^2}}=-r\cos\theta$$
    于是
    $$\begin{aligned}
        \dfrac{1}{r^2\tan\theta}\dfrac{\p{f}}{\p{\theta}}+\dfrac{1}{r^2}\dfrac{\p^2f}{\p\theta^2}
        &= \cos^2\theta\cos^2\phi\dfrac{\p^2f}{\p{x^2}}+\cos^2\theta\sin^2\phi\dfrac{\p^2f}{\p{y^2}}+\cos^2\theta\dfrac{\p^2f}{\p{z^2}} \\
        &+ \dfrac{1}{r}\left(\dfrac{\cos2\theta}{\sin\theta}\cos\phi\dfrac{\p{f}}{\p{x}}+\dfrac{\cos2\theta}{\sin\theta}\sin\phi\dfrac{\p{f}}{\p{y}}-2\cos\theta\dfrac{\p{f}}{\p{z}}\right)
    \end{aligned}$$
    又
    $$\dfrac{\p{x}}{\p{\phi}}=-r\sin\theta\sin\phi,\dfrac{\p^2{x}}{\p{\phi^2}}=-r\sin\theta\cos\phi$$
    $$\dfrac{\p{y}}{\p{\phi}}=r\sin\theta\cos\phi,\dfrac{\p^2{y}}{\p{\phi^2}}=-r\sin\theta\sin\phi$$
    $$\dfrac{\p{z}}{\p\phi}=\dfrac{\p^2z}{\p\phi^2}=0$$
    于是
    $$\begin{aligned}
        \dfrac{1}{r^2\sin^2\theta}\dfrac{\p^2f}{\p\phi^2}
        &=\sin^2\phi\dfrac{\p^2f}{\p{x^2}}+\cos^2\phi\dfrac{\p^2f}{\p{y^2}}-\dfrac{1}{r}\left(\dfrac{\cos\phi}{\sin\theta}\dfrac{\p{f}}{\p{x}}+\dfrac{\sin\phi}{\sin\theta}\dfrac{\p{f}}{\p{y}}\right)
    \end{aligned}$$
    将三项相加,考虑如下几项偏导数的系数
    $$\begin{aligned}
        \dfrac{\p{f}}{\p{x}}&:\dfrac{1}{r}\left(2\sin\theta\cos\phi+\dfrac{\cos2\theta}{\sin\theta}\cos\phi-\dfrac{\cos\phi}{\sin\theta}\right)=\dfrac{\cos\phi}{r\sin\theta}\left(2\sin^2\theta-1+\cos2\theta\right)=0\\
        \dfrac{\p{f}}{\p{y}}&:\dfrac{1}{r}\left(2\sin\theta\sin\phi+\dfrac{\cos2\theta}{\sin\theta}\sin\phi-\dfrac{\sin\phi}{\sin\theta}\right)=\dfrac{\sin\phi}{r\sin\theta}\left(2\sin^2\theta-1+\cos2\theta\right)=0\\
        \dfrac{\p{f}}{\p{z}}&:\dfrac{2}{r}\cos\theta-\dfrac{1}{r}\cdot2\cos\theta=0\\
        \dfrac{\p^2{f}}{\p{x^2}}&:\sin^2\theta\cos^2\phi+\cos^2\theta\cos^2\phi+\sin^2\phi=1\\
        \dfrac{\p^2{f}}{\p{y^2}}&:\sin^2\theta\sin^2\phi+\cos^2\theta\sin^2\phi+\cos^2\phi=1\\
        \dfrac{\p^2{f}}{\p{y^2}}&:\sin^2\theta+\cos^2\theta=1
    \end{aligned}$$
    于是
    $$\dfrac{1}{r^2}\dfrac{\p}{\p{r}}\left(r^2\dfrac{\p{f}}{\p{r}}\right)+\dfrac{1}{r^2\sin\theta}\dfrac{\p}{\p\theta}\left(\sin\theta\dfrac{\p{f}}{\p\theta}\right)+\dfrac{1}{r^2\sin^2\theta}\dfrac{\p^2f}{\p\phi^2}
    =\dfrac{\p^2f}{\p{x^2}}+\dfrac{\p^2f}{\p{y^2}}+\dfrac{\p^2f}{\p{z^2}}$$
    从而原命题得证.
\end{solution}
\end{document}