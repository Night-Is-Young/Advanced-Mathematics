\documentclass{ctexart}
\usepackage{geometry}
\usepackage[dvipsnames,svgnames]{xcolor}
\usepackage[strict]{changepage}
\usepackage{framed}
\usepackage{enumerate}
\usepackage{amsmath,amsthm,amssymb}
\usepackage{enumitem}
\usepackage{solution}
\usepackage{extarrows}

\allowdisplaybreaks
\geometry{left=2cm, right=2cm, top=2.5cm, bottom=2.5cm}

\begin{document}\pagestyle{empty}
\begin{center}
    \large\tbf{Lecture 3 Triple integral(三重积分)}
\end{center}
\begin{problem}[L.3.1]
    求$I=\displaystyle\iiint_\Omega\left(y^2+z^2\right)\di V$,其中$\Omega=\{(x,y,z)|0\leqslant z\leqslant x^2+y^2\leqslant 1\}$.
\end{problem}
\begin{solution}
    做柱坐标变换,则变换后的积分区域$\Omega'=\{(r,\theta,z)|0\leqslant r\leqslant 1,0\leqslant\theta\leqslant2\pi,0\leqslant z\leqslant r^2\}$.我们有
    \[\begin{aligned}
        I
        &= \iiint_{\Omega}\left(y^2+z^2\right)\di V \\
        &= \iiint_{\Omega'}\left(r^2\sin^2\theta+z^2\right)r\di r\di\theta\di z \\
        &= \int_0^{2\pi}\di\theta\int_0^1\di r\int_0^{r^2}\left(r^3\sin^2\theta+rz^2\right)\di z \\
        &= \int_0^{2\pi}\di\theta\int_0^1\left(r^5\sin^2\theta+\dfrac13r^7\right)\di r \\
        &= \int_0^{2\pi}\left(\dfrac16\sin^2\theta+\dfrac{1}{24}\right)\di\theta \\
        &= \dfrac{\pi}{4}
    \end{aligned}\]
\end{solution}
\begin{problem}[L.3.2]
    求$I=\displaystyle\iiint_{\Omega}z(x^2+y^2+z^2)\di V$,其中$\Omega$为球体$x^2+y^2+z^2\leqslant2z$.
\end{problem}
\begin{solution}
    注意到$\Omega=\{(x,y,z)|x^2+y^2+(z-1)^2\leqslant 1\}$.\\
    做球坐标变换,可得变换后的积分区域为$\Omega'=\{(\rho,\theta,\varphi)|0\leqslant r\leqslant2\cos\varphi,0\leqslant\theta\leqslant2\pi,0\leqslant\varphi\leqslant\dfrac{\pi}{2}\}$.于是
    \[\begin{aligned}
        I
        &= \iiint_{\Omega}z(x^2+y^2+z^2)\di V \\
        &= \iiint_{\Omega'}\rho\cos\varphi\cdot\rho^2\cdot\rho^2\sin\varphi\di\rho\di\theta\di\varphi \\
        &= \int_0^{2\pi}\di\theta\int_0^{\frac\pi2}\di\varphi\int_0^{2\cos\varphi}\rho^5\sin\varphi\cos\varphi\di\rho \\
        &= \int_0^{2\pi}\di\theta\int_0^{\frac\pi2}\dfrac{32\cos^7\varphi\sin\varphi}{3}\di\varphi \\
        &\xlongequal{t=\cos\varphi}2\pi\int_{0}^{1}\dfrac{32t^7\di t}{3} \\
        &= \dfrac{8\pi}{3}
    \end{aligned}\]
\end{solution}
\end{document}