\documentclass[b5paper,oneside]{ctexart}
\usepackage{amsmath, amsthm, amssymb,geometry,enumerate}
\geometry{left=2cm, right=2cm, top=3cm, bottom=3cm}
\newcommand{\e}{\mathrm{e}}
\linespread{1.55}
\date{\today}
\begin{document}
\pagestyle{empty}
\begin{center}\Large Lecture 3 Function limit theory(函数极限)\end{center}
\begin{enumerate}[1.]
    \item 问极限$\lim_{x \to \infty} \frac{\sin{x^2}\sqrt{x}}{\ln{(1+x^\frac{3}{2})}\arctan{5x}}$是否存在,若存在请计算该极限的值,若不存在请证明.
    \newpage
    \item 使用等价无穷小方法计算下列极限.
        \begin{enumerate}[(1)]
            \item $\lim_{x \to 1} \tan{\frac{\sin{\pi x}}{4(x-1)}}$.
            \item $\lim_{n \to \infty} n(\sqrt[n]{a}-1)$,其中参数$a>0$.
            \item $\lim_{x \to 0+0} \frac{1-\sqrt{\cos{x}}}{x-x\cos{\sqrt{x}}}$.
        \end{enumerate}
    \newpage
    \item 计算下列方幂型极限.
        \begin{enumerate}[(1)]
            \item $\lim_{x \to 0} (\frac{x\mathrm{e}^x+1}{x\pi^x+1})^{\frac{1}{x^2}}$.
            \item $\lim_{x \to 0+0} x^x$.
            \item $\lim_{x \to 0+0} (1+\frac{1}{x})^x$
        \end{enumerate}
    \newpage
    \item 设$a_1,\cdots,a_p$为$p$个正实数,满足$a_1\geqslant a_2\geqslant\cdots\geqslant a_p$.计算下面两个极限,并对比其区别.
        \begin{enumerate}[(1)]
            \item $\lim_{x \to 0+0} (\frac{\sum_{k=1}^{p} a_k^x}{p})^{\frac{1}{x}}$.
            \item $\lim_{x \to +\infty} (\frac{\sum_{k=1}^{p} a_k^x}{p})^{\frac{1}{x}}$.
        \end{enumerate}
    \newpage
    \item 判断下列命题正误,若正确请证明,若不正确请给出反例,其中$f(x)$是定义在$\mathbb{R}$上的函数.
        \begin{enumerate}[(1)]
            \item 若对任意$a \in \mathbb{R}$均有$\lim_{n \to \infty} f(n+a)=0$,则必有$\lim_{x \to +\infty}{f(x)}=0$.
            \item 若对任意$a \in \mathbb{R}$均有$\lim_{n \to \infty} f(\frac{a}{n})=0$,则必有$\lim_{x \to 0}{f(x)}=0$.
        \end{enumerate}
    \newpage
    \item 计算下列极限.
        \begin{enumerate}[(1)]
            \item $\lim_{x \to +\infty} (\sqrt{x+\sqrt{x+\sqrt{x}}}-x)$.
            \item $\lim_{x \to 0} \frac{2\sin{x}-\sin{2x}}{x^3}$.
            \item $\lim_{x \to +\infty} (\sin{\frac{1}{x}}+\cos{\frac{1}{x}})^x$.
            \item $\lim_{x \to 0} \frac{\sqrt{1+x}-\sqrt[6]{1+x}}{\sqrt[3]{1+x}-1}$.
        \end{enumerate}
    \newpage
    \item 设$f(x)$和$g(x)$在$x=a$的一个去心邻域有定义,设$\lim_{x \to a} f(x)=c$且$\lim_{x \to a} g(x)=+\infty$,其中$0<c<1$.
        用$\epsilon-\delta$语言证明$\lim_{x \to a} {f(x)}^{g(x)}=0$.
    \newpage
    \item 已知$\lim_{x \to +\infty} (\sqrt{x^2-x+1}-ax-b)=0$,计算参数$a,b$的所有可能取值.
    \newpage
    \item 设$f(x)$在$x=0$的邻域有定义,回答下列问题.
        \begin{enumerate}[(1)]
            \item $\lim_{x \to 0} f(x^3)$收敛是$\lim_{x \to 0} f(x)$收敛的充分必要条件.
            \item 问$\lim_{x \to 0} f(x)$收敛和$\lim_{x \to 0} f(x^2)$收敛的命题充分必要性如何,说明你的结论.
        \end{enumerate}
\end{enumerate}
\end{document}