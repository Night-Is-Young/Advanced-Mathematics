\documentclass[b5paper,oneside]{ctexart}
\usepackage{amsmath, amsthm, amssymb,geometry,enumerate}
\geometry{left=2cm, right=2cm, top=3cm, bottom=3cm}
\newcommand{\e}{\mathrm{e}}
\linespread{1.55}
\date{\today}
\begin{document}
\pagestyle{empty}
\begin{center}\Large Lecture 2 Sequence limit theory(序列极限)\end{center}
\begin{enumerate}[1.]
    \item 计算下列和式型极限.
        \begin{enumerate}[(1)]
            \item $\lim_{n \to \infty}  \sum_{k=1}^{n}  \frac{1}{(k+1)\sqrt{k}+k\sqrt{k+1}}$
            \item $\lim_{n \to \infty}  \prod_{k=1}^{n}  \cos{\frac{1}{2^k}}$
            \item $\lim_{n \to \infty}  \sum_{k=n^2}^{(n+1)^2}  \frac{1}{\sqrt{k}}$
            \item $\lim_{n \to \infty}  \frac{1}{n^2} \sum_{k=1}^{n}  \sqrt{(2n+k)(2n+k+1)}$
        \end{enumerate}
    \newpage
    \item 计算下列极限.
        \begin{enumerate}[(1)]
            \item $\lim_{n \to \infty} \sqrt{n}(\sqrt[n]{n}-1)$
            \item $\lim_{n \to \infty} \sqrt{n}(\sqrt{n+1}+2\sqrt{n+2}-3\sqrt{n+3})$
            \item $\lim_{n \to \infty} \sin{\pi\sqrt{4n^2+1}}$
            \item $\lim_{n \to \infty} \frac{1}{n} \ln{(\sum_{k=3}^{2021} k^n)}$
        \end{enumerate}
    \newpage
    \item 证明序列$x_n=\tan n$发散.
    \item 设$x_n$是方程$\sum_{k=1}^{n} x^k=1$在$(0,1)$上唯一的根,证明$\lim_{n \to \infty} x_n=\frac{1}{2}$.
    \newpage
    \item 证明:$\mathrm{e}=\lim_{n \to \infty} \sum_{k=0}^{n} \frac{1}{k!}$.
    \newpage
    \item 用柯西命题证明或计算:
        \begin{enumerate}[(1)]
            \item 设序列$\left\{ a_n \right\}$满足$\lim_{n \to \infty} (a_{n+1}-a_n)=A$,那么$\lim_{n \to \infty} \frac{a_n}{n}=A$.
            \item 设正序列$\left\{ a_n \right\}$满足$\lim_{n \to \infty} \frac{a_{n+1}}{a_n}=A$,那么$\lim_{n \to \infty} \sqrt[n]{a_n}=A$.
            \item 求极限$\lim_{n \to \infty} \frac{n}{\sqrt[n]{n!}}$.
        \end{enumerate}
    \newpage
        \item 计算下列极限.
        \begin{enumerate}[(1)]
            \item $\lim_{n \to \infty} \sin^2{(\pi\sqrt{n^2+n})}$.
            \item $\lim_{n \to \infty} (\frac{n-2}{n-1})^{2n+1}$.
            \item $\lim_{n \to \infty} \sqrt[3]{n}(\sqrt[3]{n+1}-\sqrt[3]{n})$.
            \item $\lim_{n \to \infty} \frac{2^nn!}{n^n}$.
        \end{enumerate}
    \newpage
    \item 设$x_1>0$,且对一切$n \in \mathbb{N}^*$有$x_{n+1}=\frac{1}{2}(x_n+\frac{1}{x_n})$,求证$\lim_{n \to \infty} x_n$存在并求该极限的值.
    \newpage
    \item 考虑序列$\left\{ a_n \right\}$,定义$S_n=\sum_{k=1}^{n}a_k$,回答下列问题.
        \begin{enumerate}[(1)]
            \item 如果$S_n$收敛,证明$\lim_{n \to \infty} a_n=0$.
            \item 如果$\lim_{n \to \infty} a_n=0$,$\lim_{n \to \infty} S_n$是否一定存在?
        \end{enumerate}
    \newpage
    \item 定义Fibonacci数列$F_{n+1}=F_n+F_{n-1}$对一切$n \in \mathbb{N}^*$成立,且$F_0=F_1=1$,设$x_n=\frac{F_n}{F_{n+1}}$,问序列$\left\{ x_n \right\}$是否收敛,如收敛请计算其极限,如发散请说明理由.
\end{enumerate}
\end{document}