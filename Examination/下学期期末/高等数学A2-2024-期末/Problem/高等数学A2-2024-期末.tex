\documentclass{ctexart}
\usepackage{template}
\usepackage{esint}

\begin{document}\pagestyle{empty}
\begin{center}\Large
    北京大学数学科学学院2023-24高等数学A2期末考试
\end{center}
\begin{enumerate}[leftmargin=*,label=\textbf{\arabic*.},topsep=0pt,parsep=0pt,itemsep=0pt,partopsep=0pt]
    \item \textbf{(15\songti{分})}\ 回答下列问题并简述理由.
        \begin{enumerate}[label=\tbf{(\arabic*)},topsep=0pt,parsep=0pt,itemsep=0pt,partopsep=0pt]
            \item \textbf{(5\songti{分})}\ 设$\left\{a_n\right\}_{n=1}^{+\infty}$是一给定数列,$\displaystyle\lim_{n\to\infty}na_n=0$,问级数$\displaystyle\sum_{n=1}^{\infty}a_n$的敛散性如何:
                \begin{center}
                    $(a)$一定收敛\ \ \ \ \ $(b)$一定发散\ \ \ \ \ $(c)$敛散性不确定
                \end{center}
            \item \textbf{(5\songti{分})}\ 设$\left\{a_n\right\}_{n=1}^{+\infty}$是一给定数列,$\displaystyle\lim_{n\to\infty}na_n$不存在,问级数$\displaystyle\sum_{n=1}^{\infty}a_n$的敛散性如何:
                \begin{center}
                    $(a)$一定收敛\ \ \ \ \ $(b)$一定发散\ \ \ \ \ $(c)$敛散性不确定
                \end{center}
            \item \textbf{(5\songti{分})}\ 设$\left\{a_n\right\}_{n=1}^{+\infty}$是一给定数列,$\displaystyle\lim_{n\to\infty}\left|na_n\right|=+\infty$,问级数$\displaystyle\sum_{n=1}^{\infty}a_n$的敛散性如何:
                \begin{center}
                    $(a)$一定收敛\ \ \ \ \ $(b)$一定发散\ \ \ \ \ $(c)$敛散性不确定
                \end{center}
        \end{enumerate}

    \item \textbf{(10\songti{分})}\ 试求幂级数的收敛半径.
        \begin{enumerate}[label=\tbf{(\arabic*)},topsep=0pt,parsep=0pt,itemsep=0pt,partopsep=0pt]
            \item \textbf{(5\songti{分})}
                \[\sum_{n=1}^{\infty}\dfrac{\left(n!\right)^2}{(2n)!}x^n\]
            \item \textbf{(5\songti{分})}
                \[\sum_{n=1}^{\infty}\dfrac{(-1)^n+(-2)^n+3^n}{n(n+1)(n+2)}x^n\]

        \end{enumerate}

    \item \textbf{(10\songti{分})}\ 求微分方程
        \[y''+y'=x^2+x\]
        的通解.

    \item \textbf{(10\songti{分})}\ 判断下列级数的敛散性.
        \begin{enumerate}[label=\tbf{(\arabic*)},topsep=0pt,parsep=0pt,itemsep=0pt,partopsep=0pt]
            \item \textbf{(5\songti{分})}
                \[\sum_{n=2}^{\infty}\dfrac{1}{(\ln n)^{\ln n}}\]
            \item \textbf{(5\songti{分})}
                \[\sum_{n=2}^{\infty}\dfrac{(-1)^n}{\sqrt{n}+(-1)^n}\]

        \end{enumerate}
        
    \item \textbf{(15\songti{分})}\ 证明函数
        \[f(x)=\sum_{n=2}^{+\infty}\left(\dfrac{x\sin x}{\ln n}\right)^n\]
        是$(-\infty,+\infty)$上的连续函数.

    \item \textbf{(10\songti{分})}\ 计算含参变量$t$的无穷积分
        \[I(t)=\int_0^{+\infty}\e^{-x^2}\cos(tx)\di x(-\infty<t<+\infty)\]
        已知$I(0)=\dfrac{\sqrt{\pi}}{2}$.

    \item \textbf{(12\songti{分})}\ 回答下列问题.
        \begin{enumerate}[label=\tbf{(\arabic*)},topsep=0pt,parsep=0pt,itemsep=0pt,partopsep=0pt]
            \item \textbf{(2\songti{分})}\ 写出$\Gamma$函数的表达式.
            \item \textbf{(5\songti{分})}\ 试证明$\Gamma(s)$在$s\in(0,+\infty)$上连续.
            \item \textbf{(5\songti{分})}\ 用$\Gamma$函数表示积分
                \[\int_0^{+\infty}\e^{-x^n}\di x\]
                并求极限
                \[\lim_{n\to+\infty}\int_0^{+\infty}\e^{-x^n}\di x\]

        \end{enumerate}

    \item \textbf{(13\songti{分})}
        设函数
        \[f(x)=\left\{\begin{array}{l}
            x,0\leqslant x\leqslant 1\\
            0,1<x\leqslant 2
        \end{array}\right.\]
        计算$f(x)$在$[0,2]$上的Fourier展开式,并证明
        \[\sum_{n=0}^{\infty}\dfrac{1}{(2n+1)^2}=\dfrac{\pi^2}{8}\ \ \ \ \ \sum_{n=1}^{\infty}\dfrac{1}{n^2}=\dfrac{\pi^2}{6}\]

    \item \textbf{(5\songti{分})}\ 试证明
        \[\int_0^1\dfrac{1}{x^x}\di x=\sum_{n=1}^{\infty}\dfrac{1}{n^n}\]
    
\end{enumerate}
\end{document}