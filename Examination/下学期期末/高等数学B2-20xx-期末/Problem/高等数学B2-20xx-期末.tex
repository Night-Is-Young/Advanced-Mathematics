\documentclass{ctexart}
\usepackage{template}
\usepackage{esint}

\begin{document}\pagestyle{empty}
\begin{center}\Large
    北京大学数学科学学院20xx高等数学B2期末考试
\end{center}
\begin{enumerate}[leftmargin=*,label=\textbf{\arabic*.},topsep=0pt,parsep=0pt,itemsep=0pt,partopsep=0pt]
    \item \textbf{(10\songti{分})}\ 对于$n\in\N^*$,设
        \[u_{2n-1}=\dfrac1n\ \ \ \ \ u_{2n}=\int_n^{n+1}\dfrac{\di x}{x}\]
        判断级数
        \[\sum_{n=1}^{\infty}(-1)^{n-1}u_n\]
        的敛散性.

    \item \textbf{(10\songti{分})}\ 求积分
        \[\int_0^{+\infty}\dfrac{\e^{-\alpha x}-\e^{-\beta x}}{x}\sin mx\di x\]
        其中$\alpha>0,\beta>0,m\neq0$.

    \item \textbf{(10\songti{分})}\ 设
        \[\sum_{n=1}^{\infty}a_n\]
        为正项级数,并有
        \[\lim_{n\to\infty}\dfrac{\ln\dfrac{1}{a_n}}{\ln n}=b\]
        \begin{enumerate}[label=\tbf{(\arabic*)},topsep=0pt,parsep=0pt,itemsep=0pt,partopsep=0pt]
            \item \textbf{(5\songti{分})}\ 试证明:当$b>1$时,级数
                \[\sum_{n=1}^{\infty}a_n\]
                收敛.
            \item \textbf{(5\songti{分})}\ 试求$b$的取值范围,使得上述级数一定发散.
        \end{enumerate}

    \item \textbf{(10\songti{分})}\ 求下列函数项级数的收敛区间和收敛域.
        \begin{enumerate}[label=\tbf{(\arabic*)},topsep=0pt,parsep=0pt,itemsep=0pt,partopsep=0pt]
            \item \textbf{(5\songti{分})}\ 
                \[\sum_{n=2}^{\infty}\dfrac{x^{n-1}}{n\cdot3^n\cdot\ln n}\]
            \item \textbf{(5\songti{分})}\ 
                \[\sum_{n=1}^{\infty}\left(\sum_{k=1}^{n}\dfrac{1}{k}\right)x^n\]

        \end{enumerate}
        
    \item \textbf{(10\songti{分})}\ 讨论数项级数
        \[\sum_{n=1}^{\infty}\dfrac{\sin n\varphi}{n^p}\]
        的敛散性,其中$\varphi\in(0,\pi)$为取定的参数.

    \item \textbf{(10\songti{分})}\ 判断下列广义积分的敛散性.
        \begin{enumerate}[label=\tbf{(\arabic*)},topsep=0pt,parsep=0pt,itemsep=0pt,partopsep=0pt]
            \item \textbf{(5\songti{分})}
                \[\int_0^{+\infty}\dfrac{\ln(1+x)}{x^p}\di x\]
            \item \textbf{(5\songti{分})}
                \[\int_0^{\frac12}\dfrac{\ln x}{\sqrt{x}\left(1-x\right)^2}\di x\]

        \end{enumerate}
    \item \textbf{(10\songti{分})}\ 求含参变量$x$的无穷积分
        \[I(x)=\int_0^{+\infty}\e^{-t^2}\cos 2xt\di t\]
    \item \textbf{(15\songti{分})}\ 设$f(x)$以$2\pi$为周期,且在$[-\pi,\pi]$上可积的函数.$a_n,b_n$为$f(x)$的Fourier系数.
        \begin{enumerate}[label=\tbf{(\arabic*)},topsep=0pt,parsep=0pt,itemsep=0pt,partopsep=0pt]
            \item \textbf{(3\songti{分})}\ 试求延迟函数$f(x+t)$的Fourier系数.
            \item \textbf{(12\songti{分})}\ 设$f(x)$连续且在$[-\pi,\pi]$上分段光滑,试求卷积函数
                \[F(x)=\dfrac{1}{\pi}\int_{-\pi}^{\pi}f(t)f(x+t)\di t\]
                的Fourier展开式,并由此推出Parseval等式.

        \end{enumerate}
    \item \textbf{(15\songti{分})}\ 回答下列问题.
        \begin{enumerate}[label=\tbf{(\arabic*)},topsep=0pt,parsep=0pt,itemsep=0pt,partopsep=0pt]
            \item \textbf{(5\songti{分})}\ 把$f(x)=x^2$在$(-\pi,\pi]$上展开为Fourier级数.
            \item \textbf{(5\songti{分})}\ 利用\tbf{(1)}的结论证明
                \[\sum_{n=1}^{\infty}\dfrac{1}{n^2}=\dfrac{\pi^2}{6}\ \ \ \ \ \sum_{n=1}^{\infty}\dfrac{1}{(2n-1)^2}=\dfrac{\pi^2}{8}\]
            \item \textbf{(5\songti{分})}\ 利用Parseval等式求级数
                \[\sum_{n=1}^{\infty}\dfrac{1}{n^4}\]
                的值.
        \end{enumerate}
\end{enumerate}
\end{document}