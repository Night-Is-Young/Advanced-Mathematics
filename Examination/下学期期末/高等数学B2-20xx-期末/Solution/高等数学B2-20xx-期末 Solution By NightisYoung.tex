\documentclass{ctexart}
\usepackage{template}
\usepackage{esint,extarrows}

\geometry{left=2cm, right=2cm, top=2.5cm, bottom=2.5cm}

\begin{document}\pagestyle{empty}
\begin{center}\Large
    北京大学数学科学学院20xx高等数学B2期末考试
\end{center}
\begin{problem}[1.(10\songti{分})]
    对于$n\in\N^*$,设
    \[u_{2n-1}=\dfrac1n\ \ \ \ \ u_{2n}=\int_n^{n+1}\dfrac{\di x}{x}\]
    判断级数
    \[\sum_{n=1}^{\infty}(-1)^{n-1}u_n\]
    的敛散性.

\end{problem}

\begin{problem}[2.(10\songti{分})]
    求积分
    \[\int_0^{\infty}\dfrac{\e^{-\alpha x}-\e^{-\beta x}}{x}\sin mx\di x\]
    其中$\alpha>0,\beta>0,m\neq0$.

\end{problem}

\begin{problem}[3.(10\songti{分})]
    设
    \[\sum_{n=1}^{\infty}a_n\]
    为正项级数,并有
    \[\lim_{n\to\infty}\dfrac{\ln\dfrac{1}{a_n}}{\ln n}=b\]
    \begin{enumerate}[label=\tbf{(\arabic*)},topsep=0pt,parsep=0pt,itemsep=0pt,partopsep=0pt]
        \item \textbf{(5\songti{分})}\ 试证明:当$b>1$时,级数
            \[\sum_{n=1}^{\infty}a_n\]
            收敛.
        \item \textbf{(5\songti{分})}\ 试求$b$的取值范围,使得上述级数一定发散.
    \end{enumerate}
\end{problem}

\begin{problem}[4.(10\songti{分})]
    求下列函数项级数的收敛区间和收敛域.
    \begin{enumerate}[label=\tbf{(\arabic*)},topsep=0pt,parsep=0pt,itemsep=0pt,partopsep=0pt]
        \item \textbf{(5\songti{分})}\ 
            \[\sum_{n=2}^{\infty}\dfrac{x^{n-1}}{n\cdot2^n\cdot\ln n}\]
        \item \textbf{(5\songti{分})}\ 
            \[\sum_{n=1}^{\infty}\left(\sum_{k=1}^{n}\dfrac{1}{k}\right)x^n\]

    \end{enumerate}
        
\end{problem}

\begin{problem}[5.(10\songti{分})]
    讨论级数
    \[\sum_{n=1}^{\infty}\dfrac{\sin nx}{n^p}(0<x<\pi)\]
    的敛散性.
\end{problem}

\begin{problem}[6.(10\songti{分})]
    判断下列广义积分的敛散性.
    \begin{enumerate}[label=\tbf{(\arabic*)},topsep=0pt,parsep=0pt,itemsep=0pt,partopsep=0pt]
        \item \textbf{(5\songti{分})}
            \[\int_0^{+\infty}\dfrac{\ln(1+x)}{x^p}\di x\]
        \item \textbf{(5\songti{分})}
            \[\int_0^{\frac12}\dfrac{\ln x}{\sqrt{x}\left(1-x\right)^2}\di x\]

    \end{enumerate}
\end{problem}

\begin{problem}[7.(10\songti{分})]
    求含参变量$x$的无穷积分
    \[I(x)=\int_0^{+\infty}\e^{-t^2}\cos 2xt\di t\]
\end{problem}

\begin{problem}[8.(15\songti{分})]
    设$f(x)$以$2\pi$为周期,且在$[-\pi,\pi]$上可积的函数.$a_n,b_n$为$f(x)$的Fourier系数.
    \begin{enumerate}[label=\tbf{(\arabic*)},topsep=0pt,parsep=0pt,itemsep=0pt,partopsep=0pt]
        \item \textbf{(3\songti{分})}\ 试求延迟函数$f(x+t)$的Fourier系数.
        \item \textbf{(12\songti{分})}\ 设$f(x)$连续且在$[-\pi,\pi]$上分段光滑,试求卷积函数
            \[F(x)=\dfrac{1}{\pi}\int_{-\pi}^{\pi}f(t)f(x+t)\di t\]
            的Fourier展开式,并由此推出Parseval等式.

    \end{enumerate}
\end{problem}
\begin{problem}[9.(15\songti{分})]
    回答下列问题.
    \begin{enumerate}[label=\tbf{(\arabic*)},topsep=0pt,parsep=0pt,itemsep=0pt,partopsep=0pt]
        \item \textbf{(5\songti{分})}\ 把$f(x)=x^2$在$(-\pi,\pi]$上展开为Fourier级数.
        \item \textbf{(5\songti{分})}\ 利用\tbf{(1)}的结论证明
            \[\sum_{n=1}^{\infty}\dfrac{1}{n^2}=\dfrac{\pi^2}{6}\ \ \ \ \ \sum_{n=1}^{\infty}\dfrac{1}{(2n-1)^2}=\dfrac{\pi^2}{8}\]
        \item \textbf{(5\songti{分})}\ 利用Parseval等式求级数
            \[\sum_{n=1}^{\infty}\dfrac{1}{n^4}\]
            的值.
    \end{enumerate}
\end{problem}
\end{document}