\documentclass{ctexart}
\usepackage{template}
\usepackage{esint,extarrows}

\geometry{left=2cm, right=2cm, top=2.5cm, bottom=2.5cm}

\begin{document}\pagestyle{empty}
\begin{center}\Large
    北京大学数学科学学院20xx高等数学B2期末考试
\end{center}
\begin{problem}[1.(10\songti{分})]
    对于$n\in\N^*$,设
    \[u_{2n-1}=\dfrac1n\ \ \ \ \ u_{2n}=\int_n^{n+1}\dfrac{\di x}{x}\]
    判断级数
    \[\sum_{n=1}^{\infty}(-1)^{n-1}u_n\]
    的敛散性.

\end{problem}
\begin{solution}
    对原级数变形可得
    \[\sum_{n=1}^{\infty}(-1)^{n-1}u_n
    =\sum_{n=1}^{\infty}\left(\dfrac1n-\int_n^{n+1}\dfrac{\di x}{x}\right)
    =\sum_{n=1}^{\infty}\left(\dfrac{1}{n}-\ln\dfrac{n+1}{n}\right)\]
    又因为对任意$x>0$有
    \[\ln(1+x)>x-\dfrac{x^2}{2}\]
    成立,于是对任意$n\in\N^*$有
    \[\dfrac{1}{n}-\ln\left(1+\dfrac1n\right)<\dfrac1n-\left(\dfrac1n-\dfrac{1}{2n^2}\right)=\dfrac{1}{2n^2}\]
    又因为
    \[\sum_{n=1}^{\infty}\dfrac{1}{2n^2}<\dfrac12+\dfrac12\sum_{n=2}^{\infty}\left(\dfrac{1}{n-1}-\dfrac1n\right)=1\]
    收敛,于是根据比较判别法可知原级数收敛.
\end{solution}
\begin{problem}[2.(10\songti{分})]
    求积分
    \[\int_0^{+\infty}\dfrac{\e^{-\alpha x}-\e^{-\beta x}}{x}\sin mx\di x\]
    其中$\alpha>0,\beta>0,m\neq0$.

\end{problem}
\begin{solution}
    我们有
    \[\int_0^{+\infty}\dfrac{\e^{-\alpha x}-\e^{-\beta x}}{x}\sin mx\di x
    =\int_0^{+\infty}\left(\int_{\alpha}^{\beta}\e^{-xy}\sin mx\di y\right)\di x\]
    由于二元函数$f(x,y)=\e^{-xy}\sin mx$在$[0,+\infty)\times[\alpha,\beta]$上连续,于是
    \[\begin{aligned}
        \int_0^{+\infty}\left(\int_{\alpha}^{\beta}\e^{-xy}\sin mx\di y\right)\di x
        &= \int_{\alpha}^{\beta}\left(\int_0^{+\infty}\e^{-yx}\sin mx\di x\right)\di y \\
        &= \int_{\alpha}^{\beta}\left(\left.-\dfrac{\e^{-yx}(m\cos mx+y\sin mx)}{m^2+y^2}\right|_0^{+\infty}\right)\di y \\
        &= \int_{\alpha}^{\beta}\dfrac{m}{m^2+y^2}\di y \\
        &= \arctan\dfrac{\beta}{m}-\arctan\dfrac{\alpha}{m}
    \end{aligned}\]

\end{solution}
\begin{problem}[3.(10\songti{分})]
    设
    \[\sum_{n=1}^{\infty}a_n\]
    为正项级数,并有
    \[\lim_{n\to\infty}\dfrac{\ln\dfrac{1}{a_n}}{\ln n}=b\]
    \begin{enumerate}[label=\tbf{(\arabic*)},topsep=0pt,parsep=0pt,itemsep=0pt,partopsep=0pt]
        \item \textbf{(5\songti{分})}\ 试证明:当$b>1$时,级数
            \[\sum_{n=1}^{\infty}a_n\]
            收敛.
        \item \textbf{(5\songti{分})}\ 试求$b$的取值范围,使得上述级数一定发散.
    \end{enumerate}
\end{problem}
\begin{solution}
    \begin{enumerate}[label=\tbf{(\arabic*)},topsep=0pt,parsep=0pt,itemsep=0pt,partopsep=0pt]
        \item 注意到题设条件等价于
            \[\lim_{n\to\infty}\dfrac{\ln a_n}{\ln\dfrac{1}{n^b}}=1\]
            于是
            \[\lim_{n\to\infty}\dfrac{a_n}{n^{-b}}=1\]
            于是级数$\displaystyle\sum_{n=1}^{\infty}a_n$与$\displaystyle\sum_{n=1}^{\infty}\dfrac{1}{n^b}$同敛散.因此当$b>1$时该级数收敛.
        \item 若$b=0$,那么对任意$\ep>0$都存在$N\in\N^*$使得$\forall n>N$都有$\left|\dfrac{\ln\frac{1}{a_n}}{\ln n}\right|<\ep$,即$a_n>\dfrac{\e^{\ep}}{n}$,于是根据比较判别法可知级数发散.\\
            若$b\neq0$且$b<1$,由\tbf{(1)}可知于是级数$\displaystyle\sum_{n=1}^{\infty}a_n$与$\displaystyle\sum_{n=1}^{\infty}\dfrac{1}{n^b}$同敛散,于是此时级数发散.\\
            综上所述,当$b<1$时级数发散.
    \end{enumerate}
\end{solution}
\begin{problem}[4.(10\songti{分})]
    求下列函数项级数的收敛区间和收敛域.
    \begin{enumerate}[label=\tbf{(\arabic*)},topsep=0pt,parsep=0pt,itemsep=0pt,partopsep=0pt]
        \item \textbf{(5\songti{分})}\ 
            \[\sum_{n=2}^{\infty}\dfrac{x^{n-1}}{n\cdot3^n\cdot\ln n}\]
        \item \textbf{(5\songti{分})}\ 
            \[\sum_{n=1}^{\infty}\left(\sum_{k=1}^{n}\dfrac{1}{k}\right)x^n\]

    \end{enumerate}
\end{problem}
\begin{solution}
    题中给出的两个函数项级数均为幂级数.
    \begin{enumerate}[label=\tbf{(\arabic*)},topsep=0pt,parsep=0pt,itemsep=0pt,partopsep=0pt]
        \item 令$u_n=\dfrac{1}{n\cdot 3^n\cdot\ln n}$,于是
            \[\lim_{n\to\infty}\dfrac{u_{n+1}}{u_n}=\lim_{n\to\infty}\dfrac{n\ln n}{3(n+1)\ln(n+1)}=\dfrac13\]
            根据D'Alembert判别法可知收敛半径$R=3$,即收敛区间为$(-3,3)$.\\
            当$x=3$时,有
            \[\sum_{n=2}^{\infty}\dfrac{1}{3n\ln n}>\dfrac{1}{3}\int_2^{+\infty}\ln\ln x\di x\]
            发散.当$x=-3$时,原级数为交错级数,由于
            \[\lim_{n\to\infty}\dfrac{1}{n\ln n}\]
            \[\begin{aligned}
                \sum_{n=2}^{\infty}\dfrac{(-3)^{n-1}}{n\cdot3^n\cdot\ln n}
                &= -\dfrac13\sum_{n=1}^{\infty}\left(\dfrac{1}{2n\ln 2n}-\dfrac{1}{(2n+1)\ln(2n+1)}\right) \\
                &> -\dfrac13\sum_{n=1}^{\infty}\dfrac{1}{4n^2+2n}
            \end{aligned}\]
            收敛.于是收敛域为$[-3,3)$.
        \item 令$u_n=\displaystyle\sum_{k=1}^n\dfrac{1}{k}$,于是
            \[\lim_{n\to\infty}\dfrac{u_{n+1}}{u_n}=\lim_{n\to\infty}1+\dfrac{1}{(n+1)u_n}=1\]
            根据D'Alembert判别法可知收敛半径$R=1$,收敛区间为$(-1,1)$.\\
            当$x=1$时有
            \[\sum_{n=1}^{\infty}u_n>\sum_{n=1}^{\infty}\dfrac1n\]
            发散.当$x=-1$时有
            \[\sum_{n=1}^{\infty}u_n(-1)^n=\sum_{n=1}^{\infty}\dfrac{1}{2n}\]
            发散.于是收敛域为$(-1,1)$.
    \end{enumerate}
\end{solution}
\begin{problem}[5.(10\songti{分})]
    讨论数项级数
    \[\sum_{n=1}^{\infty}\dfrac{\sin n\varphi}{n^p}\]
    的敛散性,其中$\varphi\in(0,\pi)$为取定的参数.
\end{problem}
\begin{solution}
    对$p$的取值分类讨论.
    \begin{enumerate}[label=\tbf{\roman*.},topsep=0pt,parsep=0pt,itemsep=0pt,partopsep=0pt,leftmargin=*]
        \item $p>1$.此时有
            \[\sum_{n=1}^{\infty}\left|\dfrac{\sin n\varphi}{n^p}\right|\leqslant
            \sum_{n=1}^{\infty}\dfrac{1}{n^p}\]
            于是级数绝对收敛.
        \item $0<p\leqslant 1$.令$a_n=\dfrac{1}{n^p},b_n=\sin n\varphi$.我们有
            \[\left|\sum_{n=1}^{\infty}b_n\right|=\left|\dfrac{1}{\sin\frac{\varphi}{2}}\sin\dfrac{n\varphi}{2}\sin\dfrac{(n+1)\varphi}{2}\right|\leqslant\left|\dfrac{1}{\sin\frac{\varphi}{2}}\right|\]
            于是$\displaystyle\sum_{n=1}^{\infty}b_n$一致有界.而当$p>0$时,对任意$x\in(0,\pi)$有$\displaystyle\lim_{n\to\infty}a_n=0$,即$a_n$一致收敛于$0$,且对任意$x$单调递减.\\
            于是根据Dirichlet判别法,原级数条件收敛.现在考虑该级数是否绝对收敛.我们有
            \[\left|\dfrac{\sin n\varphi}{n^p}\right|\geqslant\dfrac{\sin^2n\varphi}{n^p}=\dfrac{1-\cos 2n\varphi}{2n^p}\]
            与前面同理可得级数
            \[\sum_{n=1}^{\infty}\dfrac{\cos 2n\varphi}{2n^p}\]
            收敛,而级数
            \[\sum_{n=1}^{\infty}\dfrac{1}{2n^p}\]
            发散,于是根据比较判别法可知
            \[\sum_{n=1}^{\infty}\left|\dfrac{\sin n\varphi}{n^p}\right|=1\]
            发散.于是原级数条件收敛.
        \item $p\leqslant 0$.如果
            \[\lim_{n\to\infty}\dfrac{\sin n\varphi}{n^p}=0\]
            成立,那么必然有
            \[\lim_{n\to\infty}\sin n\varphi=0\]
            于是要求
            \[\lim_{n\to\infty}\left[\sin (n+1)\varphi-\sin(n-1)\varphi\right]=0\]
            而
            \[\sin (n+1)\varphi-\sin (n-1)\varphi=2\cos n\varphi\sin\varphi\]
            又因为$\sin\varphi\neq0$,于是
            \[\lim_{n\to\infty}\cos n\varphi=0\]
            因而
            \[\lim_{n\to\infty}\cos^2 n\varphi+\sin^2 n\varphi=0\]
            这与$\cos^2 n\varphi+\sin^2 n\varphi=1$矛盾,于是原级数发散.
    \end{enumerate}
\end{solution}
\begin{problem}[6.(10\songti{分})]
    判断下列广义积分的敛散性.
    \begin{enumerate}[label=\tbf{(\arabic*)},topsep=0pt,parsep=0pt,itemsep=0pt,partopsep=0pt]
        \item \textbf{(5\songti{分})}
            \[\int_0^{+\infty}\dfrac{\ln(1+x)}{x^p}\di x\]
        \item \textbf{(5\songti{分})}
            \[\int_0^{\frac12}\dfrac{\ln x}{\sqrt{x}\left(1-x\right)^2}\di x\]

    \end{enumerate}
\end{problem}
\begin{solution}
    \begin{enumerate}[label=\tbf{(\arabic*)},topsep=0pt,parsep=0pt,itemsep=0pt,partopsep=0pt]
        \item 先考虑在$(0,1)$上的积分.当$x\to0^+$时,有
            \[\lim_{n\to0^+}\dfrac{\frac{\ln(1+x)}{x^p}}{\frac{1}{x^{p-1}}}
            =\lim_{n\to0^+}\dfrac{\ln(1+x)}{x}=1\]
            当$p\geqslant 2$时
            \[\int_0^1\dfrac{1}{x^{p-1}}\di x\]
            发散,而$p<2$时上述瑕积分收敛.根据比较判别法可知当$p<2$时瑕积分
            \[\int_0^1\dfrac{\ln(1+x)}{x^p}\di x\]
            收敛.\\
            现在考虑在$(1,+\infty)$上的积分.对$p$的取值分类讨论.
            \begin{enumerate}[label=\tbf{\roman*.},topsep=0pt,parsep=0pt,itemsep=0pt,partopsep=0pt,leftmargin=*]
                \item $p>1$.此时有
                    \[\lim_{n\to+\infty}\dfrac{\frac{\ln(x+1)}{x^p}}{\frac{1}{x^{\frac{p+1}{2}}}}
                    =\lim_{n\to+\infty}\dfrac{\ln(x+1)}{x^{\frac{p-1}{2}}}=0\]
                    于是无穷积分
                    \[\int_1^{+\infty}\dfrac{\ln(1+x)}{x^p}\di x\]
                    收敛.
                \item $p\leqslant1$.此时有
                    \[\int_1^{+\infty}\dfrac{\ln(1+x)}{x^p}\di x
                    \geqslant\int_1^{+\infty}\dfrac{\ln(1+x)}{x}\di x
                    >\int_1^{+\infty}\dfrac{\ln x}{x}\di x
                    \xlongequal{u=\ln x}\int_0^{+\infty}u\di u\to\infty\]
                    发散.
            \end{enumerate}
            综上所述,当$1<p<2$时原积分收敛,当$p\geqslant 2$或$p\leqslant 1$时原积分发散.
        \item 我们有
            \[\int_0^{\frac12}\dfrac{\ln x}{\sqrt{x}(1-x)^2}
            >\int_0^{\frac12}\dfrac{4\ln x}{\sqrt{x}}\di x
            \xlongequal{u=\sqrt{x}}\int_0^{\frac{\sqrt2}{2}}4\ln u\di u
            =4\left.\left(u\ln u-u\right)\right|_0^{\frac{\sqrt{2}}{2}}=-\sqrt{2}\left(2+\ln 2\right)\]
            收敛.
    \end{enumerate}
\end{solution}
\begin{problem}[7.(10\songti{分})]
    求含参变量$x$的无穷积分
    \[I(x)=\int_0^{+\infty}\e^{-t^2}\cos 2xt\di t\]
     
\end{problem}
\begin{solution}
    令$f(x,t)=\e^{-t^2}\cos 2xt$,则
    \[\dfrac{\p f}{\p x}(x,t)=-2t\e^{-t^2}\sin 2tx\]
    于是$f(x,t)$与$\dfrac{\p f}{\p t}(x,t)$均在$(-\infty,+\infty)\times[0,+\infty)$上连续.\\
    注意到对任意$x\in\R$有
    \[\left|\dfrac{\p f}{\p x}(x,t)\right|\leqslant 2t\e^{-t^2}\]
    而
    \[\int_0^{+\infty}2t\e^{-t^2}\di t=\int_0^{+\infty}\e^{-u}\di u=1\]
    收敛,于是$\displaystyle\int_0^{+\infty}\dfrac{\p f}{\p x}(x,t)\di t$在$x\in\R$上一致收敛.于是有
    \[\begin{aligned}
        I'(x)
        &= \int_0^{+\infty}\dfrac{\p f}{\p x}(x,t)\di t
        = \int_0^{+\infty}\left(-2t\e^{-t^2}\sin 2xt\right)\di t
        = \int_0^{+\infty}\sin 2xt\di\left(\e^{-t^2}\right) \\
        &= \left.\left(\e^{-t^2}\sin 2xt\right)\right|_0^{+\infty}-\int_0^{+\infty}2x\e^{-t^2}\cos 2xt\di t \\
        &= -2xI(x)
    \end{aligned}\]
   考虑微分方程$I'(x)=-2xI(x)$,其通解为
   \[I(x)=C\e^{-x^2}\]
   又
   \[I(0)=\int_0^{+\infty}\e^{-t^2}\di t=\dfrac{\sqrt{\pi}}{2}\]
   于是
   \[I(x)=\dfrac{\sqrt\pi}{2}\e^{-x^2}\]

\end{solution}
\begin{problem}[8.(15\songti{分})]
    设$f(x)$以$2\pi$为周期,且在$[-\pi,\pi]$上可积的函数.$a_n,b_n$为$f(x)$的Fourier系数.
    \begin{enumerate}[label=\tbf{(\arabic*)},topsep=0pt,parsep=0pt,itemsep=0pt,partopsep=0pt]
        \item \textbf{(3\songti{分})}\ 试求延迟函数$f(x+t)$的Fourier系数.
        \item \textbf{(12\songti{分})}\ 设$f(x)$连续且在$[-\pi,\pi]$上分段光滑,试求卷积函数
            \[F(x)=\dfrac{1}{\pi}\int_{-\pi}^{\pi}f(t)f(x+t)\di t\]
            的Fourier展开式,并由此推出Parseval等式.

    \end{enumerate}
\end{problem}
\begin{solution}
    \begin{enumerate}[label=\tbf{(\arabic*)},topsep=0pt,parsep=0pt,itemsep=0pt,partopsep=0pt]
        \item 设$f(x+t)$的Fourier系数为$a_n',b_n'$.我们有
            \[\begin{aligned}
                a_n'
                &= \dfrac{1}{\pi}\int_{-\pi}^{\pi}f(x+t)\cos nx\di x \\
                &\xlongequal{u=x+t}\dfrac{1}{\pi}\int_{-\pi+t}^{\pi+t}f(u)\left(\cos(nu)\cos(nt)+\sin(nu)\sin(nt)\right)\di u \\
                &= \dfrac{1}{\pi}\left(\cos(nt)\int_{-\pi}^{\pi}f(u)\cos(nu)\di u+\sin(nt)\int_{-\pi}^{\pi}f(u)\sin(nu)\di u\right) \\
                &= a_n\cos(nt)+b_n\sin(nt)
            \end{aligned}\]
            其中由于$f(x)$和$\sin nx,\cos nx$均以$2\pi$为周期,因此积分区域平移后不改变积分值.同理可知
            \[b_n'=b_n\cos(nt)-a_n\sin(nt)\]
        \item 设$F(x)$的Fourier系数为$A_n,B_n$.我们有
            \[\begin{aligned}
                A_n
                &= \dfrac{1}{\pi}\int_{-\pi}^{\pi}F(x)\di x \\
                &= \dfrac{1}{\pi}\int_{-\pi}^{\pi}\left(\dfrac{1}{\pi}\int_{-\pi}^{\pi}f(t)f(x+t)\di t\right)\cos(nx)\di x \\
                &= \dfrac{1}{\pi^2}\int_{-\pi}^{\pi}f(t)\left(\int_{-\pi}^{\pi}f(t+x)\cos(nx)\di x\right)\di t \\
                &= \dfrac{1}{\pi}\int_{-\pi}^{\pi}f(t)\left(a_n\cos(nt)+b_n\sin(nt)\right)\di t \\
                &= \dfrac{a_n}{\pi}\int_{-\pi}^{\pi}f(t)\cos(nt)\di t+\dfrac{b_n}{\pi}\int_{-\pi}^{\pi}f(t)\sin(nt)\di t \\
                &= a_n^2+b_n^2
            \end{aligned}\]
            同理可知
            \[B_n=\dfrac{b_n}{\pi}\int_{-\pi}^{\pi}f(t)\cos(nt)\di t-\dfrac{a_n}{\pi}\int_{-\pi}^{\pi}f(t)\sin(nt)\di t=a_nb_n-b_na_n=0\]
            于是$F(x)$的Fourier展开式为
            \[F(x)=\dfrac{a_0^2}{2}+\sum_{n=1}^{\infty}\left(a_n^2+b_n^2\right)\cos(nx)\]
            现在,取$x=0$,就有
            \[F(0)=\dfrac{a_0^2}{2}+\sum_{n=1}^{\infty}\left(a_n^2+b_n^2\right)=\dfrac{1}{\pi}\int_{-\pi}^{\pi}\left(f(t)\right)^2\di t\]
            这就是Parseval等式.
    \end{enumerate}
\end{solution}
\begin{problem}[9.(15\songti{分})]
    回答下列问题.
    \begin{enumerate}[label=\tbf{(\arabic*)},topsep=0pt,parsep=0pt,itemsep=0pt,partopsep=0pt]
        \item \textbf{(5\songti{分})}\ 把$f(x)=x^2$在$(-\pi,\pi]$上展开为Fourier级数.
        \item \textbf{(5\songti{分})}\ 利用\tbf{(1)}的结论证明
            \[\sum_{n=1}^{\infty}\dfrac{1}{n^2}=\dfrac{\pi^2}{6}\ \ \ \ \ \sum_{n=1}^{\infty}\dfrac{1}{(2n-1)^2}=\dfrac{\pi^2}{8}\]
        \item \textbf{(5\songti{分})}\ 利用Parseval等式求级数
            \[\sum_{n=1}^{\infty}\dfrac{1}{n^4}\]
            的值.
    \end{enumerate}
\end{problem}
\begin{solution}
    \begin{enumerate}[label=\tbf{(\arabic*)},topsep=0pt,parsep=0pt,itemsep=0pt,partopsep=0pt]
        \item 这是一个偶函数,因此只需考虑$a_n$即可.有
            \[a_0=\dfrac{1}{\pi}\int_{-\pi}^{\pi}x^2\di x=\dfrac{2\pi^2}{3}\]
            \[a_n=\dfrac{1}{\pi}\int_{-\pi}^{\pi}x^2\cos(nx)\di x
            =\dfrac{1}{\pi}\left.\left(\dfrac{x^2\sin(nx)}{n}+\dfrac{2x\cos(nx)}{n^2}-\dfrac{2}{n^3}\sin(nx)\right)\right|_{-\pi}^{\pi}
            =\dfrac{4\cdot(-1)^n}{n^2}\]
            因而
            \[f(x)=\dfrac{\pi^2}{3}+\sum_{n=1}^{\infty}\dfrac{4\cdot(-1)^n\cos(nx)}{n^2}\]
        \item 在\tbf{(1)}的结论中令$x=0$,即可得
            \[\sum_{n=1}^{\infty}\dfrac{(-1)^n}{n^2}=-\dfrac{\pi^2}{12}\]
            由于
            \[\sum_{n=1}^{\infty}\dfrac{(-1)^n}{n^2}=-\sum_{n=1}^{\infty}\dfrac{1}{n^2}+2\sum_{n=1}^{\infty}\dfrac{1}{(2n)^2}=-\dfrac12\sum_{n=1}^{\infty}\dfrac{1}{n^2}\]
            于是
            \[\sum_{n=1}^{\infty}\dfrac{1}{n^2}=\dfrac{\pi^2}{6}\]
            由此亦可得
            \[\sum_{n=1}^{\infty}\dfrac{1}{(2n-1)^2}=\sum_{n=1}^{\infty}\dfrac{1}{n^2}-\sum_{n=1}^{\infty}\dfrac{1}{(2n)^2}=\dfrac34\cdot\dfrac{\pi^2}{6}=\dfrac{\pi^2}{8}\]
        \item 对$f(x)$使用Parseval等式可得
            \[\dfrac{a_0^2}{2}+\sum_{n=1}^{\infty}a_n^2=\dfrac{1}{\pi}\int_{-\pi}^{\pi}x^4\di x\]
            即
            \[\dfrac{2\pi^4}{9}+\sum_{n=1}^{\infty}\dfrac{16}{n^4}=\dfrac{2\pi^4}{5}\]
            于是
            \[\sum_{n=1}^{\infty}\dfrac{1}{n^4}=\dfrac{\pi^4}{16}\left(\dfrac{2}{5}-\dfrac{2}{9}\right)=\dfrac{\pi^4}{90}\]
    \end{enumerate}
\end{solution}
\end{document}