\documentclass{ctexart}
\usepackage{template}
\usepackage{esint,extarrows}

\geometry{left=2cm, right=2cm, top=2.5cm, bottom=2.5cm}

\begin{document}\pagestyle{empty}
\begin{center}\Large
    北京大学数学科学学院2021-22高等数学B2期末考试
\end{center}
\begin{problem}[1.(10\songti{分})]
    求函数
    \[f(x)=\dfrac{\sqrt{|x|}}{2}\ln\dfrac{1+\sqrt{|x|}}{1-\sqrt{|x|}}\]
    在$x=0$处的幂级数展开式,并指出此幂级数的收敛域.

\end{problem}

\begin{problem}[2.(15\songti{分})]
    设$f:\R\to\R$是周期为$2\pi$的而函数,$f(x)$在$(-\pi,\pi]$上等于$\e^x$.%
    求$f(x)$的傅里叶级数,以及此傅里叶级数在$x=\pi$处的收敛值.

\end{problem}

\begin{problem}[3.(10\songti{分})]
    求无穷积分
    \[\int_0^{+\infty}\sqrt{x^3}\e^{-x}\di x\]
    和瑕积分
    \[\int_0^1\sqrt{\dfrac{x^3}{1-x}}\di x\]
    的值.

\end{problem}

\begin{problem}[4.(10\songti{分})]
    求幂级数
    \[\sum_{n=0}^{\infty}(n+1)(n+2)x^n\]
    的收敛区间,以及此幂级数的和函数.    

\end{problem}

\begin{problem}[5.(10\songti{分})]
    任意给定常数$r>0$,试证明函数项级数
    \[\sum_{n=1}^\infty n^2\e^{-nx}\]
    在$[r,+\infty)$上一致收敛.
\end{problem}

\begin{problem}[6.(15\songti{分})]
    函数项级数
    \[\sum_{n=1}^\infty\dfrac{(-1)^n}{n^x+2n}\]
    的收敛域,全体绝对收敛点,全体条件收敛点.
\end{problem}

\begin{problem}[7.(15\songti{分})]
    定义函数$\theta:[0,+\infty)\to[0,+\infty)$为
    \[\theta(x)=\int_0^x\sqrt{(t+1)(t+2)(t+3)}\di t\]
    试证明无穷积分
    \[\int_0^{+\infty}\cos\left(\theta(x)\right)\di x\]
    收敛.本题要求写出详细过程和依据.

\end{problem}

\begin{problem}[8.(15\songti{分})]
    设$n$是正整数.
    \begin{enumerate}[label=\tbf{(\arabic*)},topsep=0pt,parsep=0pt,itemsep=0pt,partopsep=0pt]
        \item \textbf{(5\songti{分})}\ 任意给定$a>0$,试证明含参变量$t$的无穷积分
            \[\int_0^{+\infty}\dfrac{1}{\left(t+x^2\right)^n}\di x\]
            在$[a,+\infty)$上一致收敛.
        \item \textbf{(10\songti{分})}\ 对每个$t\in(0,+\infty)$,求出
            \[\int_0^{+\infty}\dfrac{1}{\left(t+x^2\right)^n}\di x\]
            的值.
    \end{enumerate}
    本题要求写出详细过程和依据.
\end{problem}

\end{document}