\documentclass{ctexart}
\usepackage{template}
\usepackage{esint,extarrows}

\geometry{left=2cm, right=2cm, top=2.5cm, bottom=2.5cm}

\begin{document}\pagestyle{empty}
\begin{center}\Large
    北京大学数学科学学院2022-23高等数学B2期末考试
\end{center}
\begin{problem}[1.(15\songti{分})]
    判断下列级数的敛散性.
    \begin{enumerate}[label=\tbf{(\arabic*)},topsep=0pt,parsep=0pt,itemsep=0pt,partopsep=0pt]
        \item \textbf{(5\songti{分})}
            \[\sum_{n=3}^{\infty}\dfrac{2}{n\ln\sqrt{n}}\]
        \item \textbf{(5\songti{分})}
            \[\sum_{n=1}^{\infty}\dfrac{3\sqrt[5]{n}+1}{\left(\sqrt[4]{n}+n\right)\left(\sqrt[3]{n}+n\right)}\]
        \item \textbf{(5\songti{分})}
            \[\sum_{n=1}^{\infty}\left(\dfrac{4\cdot3^n}{5^n}-\dfrac{1}{n}+\dfrac{1}{n+2}\right)\]
    \end{enumerate}

\end{problem}
\begin{solution}
    \begin{enumerate}[label=\tbf{(\arabic*)},topsep=0pt,parsep=0pt,itemsep=0pt,partopsep=0pt]
        \item 注意到
            \[\sum_{n=3}^{\infty}\dfrac{2}{n\ln\sqrt{n}}
            =\sum_{n=3}^{\infty}\dfrac{4}{n\ln n}
            >\int_3^{+\infty}\dfrac{4}{x\ln x}\di x
            =\left.\left(4\ln\ln x\right)\right|_2^{+\infty}\]
            而$\displaystyle\lim_{x\to+\infty}\ln\ln x=+\infty$,于是原级数发散.
        \item 注意到
            \[\dfrac{3\sqrt[5]{n}+1}{\left(\sqrt[4]{n}+n\right)\left(\sqrt[3]{n}+n\right)}<\dfrac{3\sqrt[5]{n}+1}{n^2}<\dfrac{1}{n^{\frac{9}{5}}}\]
            而
            \[\sum_{n=1}^{\infty}\dfrac{1}{n^{\frac{9}{5}}}\]
            收敛,于是根据比较判别法可知原级数收敛.
        \item 首先有
            \[\sum_{n=1}^{\infty}\dfrac{4\cdot3^n}{5^n}=\dfrac{12}{5}\lim_{n\to\infty}\cdot\dfrac{1-\left(\frac35\right)^n}{1-\frac35}=6\]
            又有
            \[\sum_{n=1}^{\infty}\left(\dfrac{1}{n}-\dfrac{1}{n+2}\right)
            =\lim_{n\to\infty}\left(1+\dfrac{1}{2}-\dfrac{1}{n+1}-\dfrac{1}{n+2}\right)
            =\dfrac32\]
            于是
            \[\sum_{n=1}^{\infty}\left(\dfrac{4\cdot3^n}{5^n}-\dfrac{1}{n}+\dfrac{1}{n+2}\right)=\dfrac{9}{2}\]
            因而原级数收敛.
    \end{enumerate}
\end{solution}
\begin{problem}[2.(10\songti{分})]
    讨论函数序列
    \[f_n(x)=\left(1-\dfrac{1}{\sqrt{n}}\right)^{x^2}\ \ \ \ \ n=1,2,\cdots\]
    在$(0,+\infty)$上的一致收敛性.

\end{problem}
\begin{solution}
    对任意$x\in(0,+\infty)$,都有
    \[\lim_{n\to\infty}f_n(x)=\lim_{n\to\infty}\left(1-\dfrac{1}{\sqrt{n}}\right)^{x^2}=1\]
    于是函数序列$\left\{f_n(x)\right\}$逐点收敛于极限函数$f(x)=1$.然而,对于点列$x_n=\sqrt[4]{n}$,有
    \[\lim_{n\to\infty}\left|f_n\left(x_n\right)-f\left(x_n\right)\right|
    =\lim_{n\to\infty}1-\left(1-\dfrac{1}{\sqrt{n}}\right)^{\sqrt{n}}=1-\e\neq0\]
    于是原函数序列在$(0,+\infty)$上不一致收敛.
\end{solution}
\begin{problem}[3.(15\songti{分})]
    求幂级数
    \[\sum_{n=1}^{\infty}\dfrac{(-1)^n}{n}x^n\]
    的收敛半径,收敛域,和函数.

\end{problem}
\begin{solution}
    令$a_n=\dfrac{(-1)^n}{n}$,则有
    \[\lim_{n\to\infty}\left|\dfrac{a_{n+1}}{a_n}\right|
    =\lim_{n\to\infty}\dfrac{n}{n+1}=1\]
    于是收敛半径$R=1$.当$x=1$时,
    \[\sum_{n=1}^{\infty}\dfrac{(-1)^n}{n}x^n=\sum_{n=1}^{\infty}\left(\dfrac{1}{2n-1}-\dfrac{1}{2n}\right)=\sum_{n=1}^{\infty}\dfrac{1}{4n^2-2n}\]
    收敛,而当$x=-1$时
    \[\sum_{n=1}^{\infty}\dfrac{(-1)^n}{n}x^n=\sum_{n=1}^{\infty}\dfrac{1}{n}\]
    发散.于是原级数的收敛域为$(-1,1]$.设该级数的和函数为$S(x)$,在收敛域上对级数逐项求导可得
    \[S'(x)=\sum_{n=1}^{\infty}(-1)^nx^{n-1}=-2+\sum_{n=1}^{\infty}(-1)^{n+1}x^{n}=-1+\dfrac{1}{1+x}\]
    于是
    \[S(x)=\int S'(x)\di x+C=-2x+\ln(1+x)+C\]
    又$S(0)=0$,于是
    \[S(x)=\sum_{n=1}^{\infty}\dfrac{(-1)^n}{n}x^n=-2x+\ln(1+x)\]

\end{solution}
\begin{problem}[4.(10\songti{分})]
    求函数
    \[f(x)=\dfrac{1}{x^2-2x-3}\]
    于$x=1$处的泰勒展开式,并计算$f^{(2022)}(1),f^{(2023)}(1)$的值.
        
\end{problem}
\begin{solution}
    注意到
    \[f(x)=\dfrac{1}{x^2-2x-3}=\dfrac{1}{(x-1)^2-4}=-\dfrac14\cdot\dfrac{1}{1-\left(\frac{x-1}{2}\right)^2}\]
    而$g(x)=\dfrac{1}{1+x}$在$x=0$处的泰勒展开式为
    \[g(x)=\dfrac{1}{1+x}=\sum_{n=0}^{\infty}(-1)^nx^n\]
    于是
    \[f(x)=-\dfrac14\sum_{n=0}^{\infty}(-1)^n\left(-\left(\dfrac{x-1}{2}\right)^2\right)^n
    =-\dfrac14\sum_{n=0}^{\infty}\dfrac{(x-1)^{2n}}{4^n}=\sum_{n=0}^{\infty}\dfrac{-(x-1)^{2n}}{4^{n+1}}\]
    又因为$f(x)$的泰勒展开式的通项为
    \[f(x)=\sum_{n=0}^{\infty}\dfrac{f^{(n)}(1)(x-1)^n}{n!}\]
    于是$f^{(2022)}(1)=-\dfrac{2022!}{4^{1012}},f^{(2023)}=0$.
\end{solution}
\begin{problem}[5.(10\songti{分})]
    讨论无穷积分
    \[\int_1^{+\infty}\dfrac{\sin x}{\sqrt{x}}\arctan x\di x\]
    的敛散性.
\end{problem}
\begin{solution}
    首先考虑积分
    \[\int_1^{+\infty}\left|\dfrac{\sin x}{\sqrt{x}}\arctan x\right|\di x\]
    为了说明其无界性,考虑一个充分大的$n\in\N$.我们有
    \[\int_1^{+\infty}\left|\dfrac{\sin x}{\sqrt{x}}\arctan x\right|\di x
    >\int_{\pi}^{n\pi}\left|\dfrac{\sin x}{\sqrt{x}}\arctan x\right|\di x
    >\arctan\pi\int_{\pi}^{n\pi}\left|\dfrac{\sin x}{\sqrt{x}}\right|\di x\]
    而
    \[\int_{\pi}^{n\pi}\left|\dfrac{\sin x}{\sqrt{x}}\right|\di x
    =\sum_{k=1}^{n-1}\int_{k\pi}^{(k+1)\pi}\left|\dfrac{\sin x}{\sqrt{x}}\right|\di x
    >\sum_{k=1}^{n-1}\int_{k\pi}^{(k+1)\pi}\dfrac{\left|\sin x\right|\di x}{\sqrt{k+1}}
    =\sum_{k=1}^{n-1}\dfrac{1}{\sqrt{k+1}}\]
    注意到
    \[\lim_{n\to\infty}\sum_{k=1}^{n-1}\dfrac{1}{\sqrt{k+1}}=\sum_{n=2}^{\infty}\dfrac{1}{\sqrt{n}}\]
    发散,于是这积分不绝对收敛.\\
    现在令$f(x)=\sin x,g(x)=\dfrac{\arctan x}{\sqrt{x}}$.对任意$A>a>1$总有
    \[\left|\int_a^{A}f(x)\di x\right|=\left|\cos A-\cos a\right|\leqslant 2\]
    一致有界,而
    \[g'(x)=\dfrac{1}{(x^2+1)\sqrt{x}}-\dfrac{\arctan x}{2x\sqrt{x}}\]
    当$x>\tan 1>1$时总有$\arctan x>1$且$x^2+1>2x$,因此$g(x)$在$(\tan 1,+\infty)$单调递减.又因为
    \[\lim_{x\to+\infty}g(x)=\lim_{x\to+\infty}\dfrac{\pi}{2\sqrt{x}}=0\]
    于是$g(x)$单调递减且收敛于$0$.根据Dirichlet判别法,无穷积分
    \[\int_1^{+\infty}\dfrac{\sin x}{\sqrt{x}}\arctan x\di x\]
    收敛.综上所述,该积分条件收敛.
\end{solution}
\begin{problem}[6.(10\songti{分})]
    讨论级数
    \[\sum_{n=1}^{\infty}\dfrac{\sin^n x}{1+\sin^{2n}x}\]
    在$x\in(-\infty,+\infty)$上的绝对收敛性和条件收敛性.
\end{problem}
\begin{solution}
    考虑到$\sin x$的周期性,只需考虑$x\in(-\pi,\pi]$的情形即可.令$u=\sin x,a_n=\dfrac{u^n}{1+u^{2n}}$.\\
    当$x=0,\pi$时$u=0$,于是$a_n=0$,原级数显然收敛.
    当$x\in\left(0,\dfrac\pi2\right)\cup\left(\dfrac\pi2,\pi\right)$时,有$0<u<1$.于是
    \[\sum_{n=1}^{\infty}\dfrac{u^n}{1+u^{2n}}<\sum_{n=1}^{\infty}u^n=\dfrac{u}{1-u}\]
    收敛.\\
    当$x=\dfrac\pi2$时$u=1$,$a_n=\dfrac12$,不满足$\displaystyle\lim_{n\to\infty}a_n=0$的条件,原级数发散.\\
    当$x\in\left(-\pi,-\dfrac\pi2\right)\cup\left(-\dfrac\pi2,0\right)$时,有$-1<u<0$.由前面的推导同理可得
    \[\sum_{n=1}^{\infty}\left|\dfrac{u^n}{1+u^{2n}}\right|<\sum_{n=1}^{\infty}\left|u^n\right|=\dfrac{-u}{1+u}\]此时级数绝对收敛.\\
    当$x=-\dfrac\pi2$时$u=-1$,$a_n=\dfrac{(-1)^n}{2}$,仍然发散.\\
    因此,原级数在$\left(k\pi-\dfrac\pi2,k\pi+\dfrac{\pi}{2}\right)(k\in\mathbb{Z})$上绝对收敛,在$x=k\pi+\dfrac\pi2$时发散.
\end{solution}
\begin{problem}[7.(20\songti{分})]
    设以$2\pi$为周期的函数$f(x)$在$[-\pi,\pi]$上的表达式为
    \[f(x)=x^2\]
    求$f(x)$的Fourier级数及其和函数,并给出级数
    \[\sum_{n=1}^{\infty}\dfrac{1}{n^2}
    \ \ \ \ \ \sum_{n=1}^{\infty}\dfrac{(-1)^n}{n^2}
    \ \ \ \ \ \sum_{n=1}^{\infty}\dfrac{1}{n^4}\]
    的值.

\end{problem}
\begin{solution}
    这是一个偶函数,因此只需考虑$a_n$即可.有
    \[a_0=\dfrac{1}{\pi}\int_{-\pi}^{\pi}x^2\di x=\dfrac{2\pi^2}{3}\]
    \[a_n=\dfrac{1}{\pi}\int_{-\pi}^{\pi}x^2\cos(nx)\di x
    =\dfrac{1}{\pi}\left.\left(\dfrac{x^2\sin(nx)}{n}+\dfrac{2x\cos(nx)}{n^2}-\dfrac{2}{n^3}\sin(nx)\right)\right|_{-\pi}^{\pi}
    =\dfrac{4\cdot(-1)^n}{n^2}\]
    因而
    \[f(x)=\dfrac{\pi^2}{3}+\sum_{n=1}^{\infty}\dfrac{4\cdot(-1)^n\cos(nx)}{n^2}\]
    令$x=0$,即可得
    \[\sum_{n=1}^{\infty}\dfrac{(-1)^n}{n^2}=-\dfrac{\pi^2}{12}\]
    由于
    \[\sum_{n=1}^{\infty}\dfrac{(-1)^n}{n^2}=-\sum_{n=1}^{\infty}\dfrac{1}{n^2}+2\sum_{n=1}^{\infty}\dfrac{1}{(2n)^2}=-\dfrac12\sum_{n=1}^{\infty}\dfrac{1}{n^2}\]
    于是
    \[\sum_{n=1}^{\infty}\dfrac{1}{n^2}=\dfrac{\pi^2}{6}\]
    对$f(x)$使用Parseval等式可得
    \[\dfrac{a_0^2}{2}+\sum_{n=1}^{\infty}a_n^2=\dfrac{1}{\pi}\int_{-\pi}^{\pi}x^4\di x\]
    即
    \[\dfrac{2\pi^4}{9}+\sum_{n=1}^{\infty}\dfrac{16}{n^4}=\dfrac{2\pi^4}{5}\]
    于是
    \[\sum_{n=1}^{\infty}\dfrac{1}{n^4}=\dfrac{\pi^4}{16}\left(\dfrac{2}{5}-\dfrac{2}{9}\right)=\dfrac{\pi^4}{90}\]

\end{solution}
\begin{problem}[8.(10\songti{分})]
    证明和计算下列各题.
    \begin{enumerate}[label=\tbf{(\arabic*)},topsep=0pt,parsep=0pt,itemsep=0pt,partopsep=0pt]
        \item 证明含参变量$t$的无穷积分
            \[I(t)=\int_0^{+\infty}\e^{-tx}\dfrac{\sin x}{x}\di x\]
            在$t\in[0,+\infty)$上一致收敛.
        \item 证明上述$I(t)$在$t\in(0,+\infty)$上可导.
        \item 求出上述$I(t)$.
        \item 计算无穷积分
            \[\int_0^{+\infty}\dfrac{\sin x}{x}\di x\]
            的值.
    \end{enumerate}

\end{problem}
\begin{solution}
    \begin{enumerate}[label=\tbf{(\arabic*)},topsep=0pt,parsep=0pt,itemsep=0pt,partopsep=0pt]
        \item 令$f(x,t)=\e^{-tx},g(x,t)=\dfrac{\sin x}{x}$.对任意$t\in[0,+\infty)$和$x\in[0,+\infty)$,总有
            \[\left|f(x,t)\right|=\left|\e^{-tx}\right|\leqslant e^0=1\]
            即$f(x,t)$对任意$t\in[0,+\infty)$一致收敛.\\
            令$a(x)=\sin x,b(x)=\dfrac1x$.我们有
            \[\int_0^1\dfrac{\sin x}{x}\di x<\int_0^1\di x=1\]
            又对任意$1<a<A$有
            \[\left(\int_a^Aa(x)\di x\right)\left|\cos A-\cos a\right|\leqslant 2\]
            对$x$一致有界,$b(x)=\dfrac{1}{x}$单调递减且收敛于$0$,于是根据Dirichlet判别法可知
            \[\int_0^{+\infty}g(x,t)\di x=\int_0^{+\infty}a(x)b(x)\di x\]
            收敛,即对任意$t\in[0,+\infty)$都一致收敛.\\
            于是根据Abel判别法可知含参变量$t$的无穷积分
            \[I(t)=\int_0^{+\infty}\e^{-tx}\dfrac{\sin x}{x}\di x\]
            在$t\in[0,+\infty)$上一致收敛.
        \item 令$F(x,t)=\dfrac{\e^{-tx}\sin x}{x}$.注意到$F(x)$在矩形域$(0,+\infty)\times(0,+\infty)$上二元连续.由\tbf{(1)}可得
            \[\int_0^{+\infty}F(x,t)\di x\]
            在$t\in(0,+\infty)$一致收敛,又
            \[\int_0^{+\infty}\dfrac{\p F}{\p t}(x,t)\di x=\int_0^{+\infty}-\e^{-tx}\sin x\di x\]
            令$\alpha(x,t)=x\e^{-tx},\beta(x,t)=\sin x$.同理有
            \[\left|\int_a^A\beta(x)\di x\right|=\left|\cos A-\cos a\right|\leqslant 2\]
            一致有界,并且$\alpha(x,t)$对$x$单调递减,且$\displaystyle\lim_{x\to\infty}\alpha(x,t)=0$.\\
            根据Dirichlet判别法可知上述级数对任意$t_0\in(0,+\infty)$和包含$t_0$的闭区间$[c,d]$上都一致收敛.\\
            因此,$I(t)$在$(0,+\infty)$上可导,并且有
            \[I'(t)=-\int_0^{+\infty}\e^{-tx}\sin x\di x\]
        \item 我们有
            \[\int-\e^{-tx}\sin x\di x=\dfrac{\e^{-tx}(t\sin x+\cos x)}{t^2+1}+C\]
            于是
            \[I'(t)=\int_0^{+\infty}-\e^{-tx}\sin x\di x
            =\left.\left(\dfrac{\e^{-tx}(t\sin x+\cos x)}{t^2+1}\right)\right|_0^{+\infty}
            =-\dfrac{1}{t^2+1}\]
            即
            \[I(t)=-\arctan t+C\]
            又因为
            \[0<\left|\int_0^{+\infty}\e^{-tx}\dfrac{\sin x}{x}\di x\right|
            <\left|\int_0^{+\infty}\e^{-tx}\di x\right|=\dfrac1t\]
            由夹逼准则可得
            \[\lim_{t\to\infty}I(t)=0\]
            从而$C=\dfrac\pi2$.因此有
            \[\int_0^{+\infty}\dfrac{\sin x}{x}\di x
            =\lim_{t\to0}\int_0^{+\infty}\e^{-tx}\dfrac{\sin x}{x}\di x
            =\lim_{t\to0}I(t)=I(0)=\dfrac\pi2\]

    \end{enumerate}
\end{solution}
\end{document}