\documentclass{ctexart}
\usepackage{template}
\usepackage{esint}

\begin{document}\pagestyle{empty}
\begin{center}\Large
    北京大学数学科学学院2024-25高等数学B2期末考试
\end{center}
\begin{enumerate}[leftmargin=*,label=\textbf{\arabic*.},topsep=0pt,parsep=0pt,itemsep=0pt,partopsep=0pt]
    \item \textbf{(14\songti{分})}\ 讨论下列级数的敛散性.
        \begin{enumerate}[label=\tbf{(\arabic*)},topsep=0pt,parsep=0pt,itemsep=0pt,partopsep=0pt]
            \item \[\sum_{n=1}^{\infty}2^n\left(\dfrac{n-1}{n}\right)^{n^2}\]
            \item \[\sum_{n=2}^{\infty}\dfrac{1}{n\sqrt{\ln n}}\]
        \end{enumerate}

    \item \textbf{(14\songti{分})}\ 判断下列级数的敛散性.如果收敛,请判断其为绝对收敛还是条件收敛.
        \begin{enumerate}[label=\tbf{(\arabic*)},topsep=0pt,parsep=0pt,itemsep=0pt,partopsep=0pt]
            \item \[\sum_{n=2}^{\infty}\dfrac{(-1)^n}{\sqrt{n}+(-1)^n}\]
            \item \[\sum_{n=1}^{\infty}\dfrac{(-1)^{n-1}}{n^{1+\frac1n}}\]
        \end{enumerate}

    \item \textbf{(16\songti{分})}\ 求幂级数
        \[\sum_{n=1}^{\infty}\dfrac{(-1)^{n-1}}{(2n-1)(2n+1)}x^{2n+1}\]
        的收敛半径,收敛区间,收敛域以及和函数.

    \item \textbf{(12\songti{分})}\ 求含参变量$a$的积分
        \[I(a)=\int_0^{\frac\pi2}\ln\left(a^2\sin^2x+\cos^2x\right)\di x(a>0)\]
        
    \item \textbf{(12\songti{分})}\ 判断广义积分
        \[\int_0^{+\infty}\dfrac{1}{\left(1+x^2\right)\left(\sin^2 x\right)^\alpha}\di x\]
        的敛散性,其中$0<\alpha<\dfrac12$.

    \item \textbf{(12\songti{分})}\ 讨论积分
        \[\int_1^{+\infty}t\e^{-tx}\dfrac{\cos x}{x}\di x\]
        在$0\leqslant t<+\infty$上的一致收敛性.

    \item \textbf{(20\songti{分})}\ 设$f(x)$是以$2\pi$为周期的函数,满足$f(x)=|x|,x\in[-\pi,\pi]$.求出$f(x)$的傅里叶级数及其和函数,并求出级数
        \[\sum_{n=1}^{\infty}\dfrac{1}{n^4}\]
    
\end{enumerate}
\end{document}