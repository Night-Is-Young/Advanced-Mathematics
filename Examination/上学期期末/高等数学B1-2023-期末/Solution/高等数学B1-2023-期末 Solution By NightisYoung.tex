\documentclass{ctexart}
\usepackage{geometry}
\usepackage[dvipsnames,svgnames]{xcolor}
\usepackage[strict]{changepage}
\usepackage{framed}
\usepackage{enumerate}
\usepackage{amsmath,amsthm,amssymb}
\usepackage{enumitem}
\usepackage{template}

\geometry{left=2cm, right=2cm, top=2.5cm, bottom=2.5cm}

\begin{document}\pagestyle{empty}
\begin{center}\Large
    北京大学数学科学学院2023-24高等数学B1期末考试
\end{center}
\begin{problem}[1.(10\songti{分})]
    求极限\[\lim_{x\to0}\dfrac{\sin\left(x-\displaystyle\int_0^x\sqrt{1+t^2}\di t\right)}{x^3}\]
\end{problem}
\begin{solution}
    我们有
    \[\begin{aligned}
        \lim_{x\to0}\dfrac{\sin\left(x-\displaystyle\int_0^x\sqrt{1+t^2}\di t\right)}{x^3}
        &= \lim_{x\to0}\dfrac{\cos\left(x-\displaystyle\int_0^x\sqrt{1+t^2}\di t\right)\left(1-\sqrt{1+x^2}\right)}{3x^2} \\
        &= 1\cdot\lim_{x\to0}\dfrac{-x^2}{3x^2\left(1+\sqrt{1-x^2}\right)} \\
        &= 1\cdot\left(-\dfrac13\right)\cdot\dfrac12 \\
        &= -\dfrac16
    \end{aligned}\]
\end{solution}
\begin{problem}[2.(10\songti{分})]
    设函数$f:[0,7]\to\R$为\[f(x)=x^3-6x^2+9x-1\]称区间$[a,b]$是$f$的单调区间,当$0\leqslant a<b\leqslant 7$且限制在$[a,b]$上的$f$严格单调.求$f$的长度最大的单调区间.
\end{problem}
\begin{solution}
    对$f(x)$求导有
    \[f'(x)=3x^2-12x+9=3(x-1)(x-3)\]
    于是当$x\in[0,1)$或$x\in(3,7]$时$f(x)>0$.$x\in(1,3)$时$f(x)<0$.$x=1$或$3$时$f(x)=0$.\\
    因而$f(x)$在$[0,1]$单调递增,在$[1,3]$单调递减,在$[3,7]$单调递增.\\
    于是$f$的长度最大的单调区间为$[3,7]$,其长度为$4$.
\end{solution}
\begin{problem}[3.(10\songti{分})]
    设欧氏空间$\R^3$中的平面$T:2x-y+3z=6$.设$T$与$x,y,z$三轴的交点分别为$A,B,C$.以原点$O(0,0,0)$为球心,与$T$相切的球面记作$S$.
    \begin{enumerate}[label=\tbf{(\arabic*)}]
        \item \textbf{(5\songti{分})}\ 求$\triangle ABC$的面积.
        \item \textbf{(5\songti{分})}\ 求球面$S$与$T$相切的点的坐标.
    \end{enumerate}
\end{problem}
\begin{solution}
    \begin{enumerate}[label=\tbf{(\arabic*)}]
        \item 分别令$x,y,z$三者中的两者为$0$,可解得$A(3,0,0),B(0,-6,0),C(0,0,2)$.于是
            \[S_{\triangle ABC}=\dfrac{1}{2}\left|\overrightarrow{AB}\times\overrightarrow{AC}\right|=\dfrac12\begin{vmatrix}
                i&j&k\\-3&-6&0\\-3&0&2
            \end{vmatrix}=\dfrac12\left|(12,6,-18)\right|=3\sqrt{14}\]
        \item 设切点为$P(x,y,z)$.设$\vec{u}$为$T$的法向量,根据$T$的一般式可知$\vec{u}=(2,-1,3)$.\\
            由于$S$于$T$相切,因此$\overrightarrow{OP}\bot T$,于是$\overrightarrow{OP}//\vec{u}$.于是可以列出方程组
            \[\left\{\begin{array}{l}
                2x-y+3z=6\\
                \dfrac x2=-\dfrac y1=\dfrac z3
            \end{array}\right.\]
            解得$x=\dfrac67,y=-\dfrac3{7},z=\dfrac9{7}$.于是$P\left(\dfrac67,-\dfrac37,\dfrac97\right)$.
    \end{enumerate}
\end{solution}
\begin{problem}[4.(10\songti{分})]
    设二元函数$z=f(x,y)$是由方程$F(x,y,z)=z^3+z\e^x+y=0$确定的隐函数.求$z=f(x,y)$在$(0,2)$处函数值下降最快的方向上的单位向量.
\end{problem}
\begin{solution}
    首先求$F$的偏导,有\[\dfrac{\p F}{\p x}=z\e^x\ \ \ \ \ \dfrac{\p F}{\p y}=1\ \ \ \ \ \dfrac{\p F}{\p z}=3z^2+\e^x\]
    当$(x,y)=(0,2)$时$F(x,y,z)=0$有$z^3+z+2=0$,即$(z+1)(z^2-z+2)=0$.故这方程仅有$z=-1$一实根.\\
    当$(x,y,z)=(0,2,-1)$时$F_z(0,2,-1)=4\neq0$.于是根据隐函数存在定理,$F(x,y,z)=0$在$(0,2)$处附近确定唯一的隐函数$z=f(x,y)$,并且有
    \[\dfrac{\p f}{\p x}=-\dfrac{F_x(0,2,-1)}{F_z(0,2,-1)}=\dfrac14\ \ \ \ \ \dfrac{\p f}{\p y}=-\dfrac{F_y(0,2,-1)}{F_z(0,2,-1)}=-\dfrac14\]
    于是$f$在$(0,2)$处的梯度向量为$\left(\dfrac14,-\dfrac14\right)$.\\
    取负梯度后单位化有$\left(-\dfrac{\sqrt2}{2},\dfrac{\sqrt2}{2}\right)$,此即所求向量.
\end{solution}
\begin{problem}[5.(10\songti{分})]
    求函数$f(x,y)=x^y$在$(1,1)$处的二阶泰勒多项式.
\end{problem}
\begin{solution}
    我们有
    \[\dfrac{\p f}{\p x}=yx^{y-1}\ \ \ \ \ \dfrac{\p f}{\p y}=x^y\ln x\]
    \[\dfrac{\p^2 f}{\p x^2}=y(y-1)x^{y-2}\ \ \ \ \ \dfrac{\p^2 f}{\p x\p y}=x^{y-1}\left(1+y\ln x\right)\ \ \ \ \ \dfrac{\p^2 f}{\p y^2}=x^y\left(\ln x\right)^2\]
    代入$x=1,y=1$可知
    \[f(x,y)=1+(x-1)+(x-1)(y-1)\]
\end{solution}
\begin{problem}[6.(10\songti{分})]
    设$D$是由直线$x+y=2\pi$,$x$轴和$y$轴围成的有界闭区域.求$D$上的二元函数$f(x,y)=\sin x+\sin y-\sin(x+y)$达到最大值的$D$中所有点.
\end{problem}
\begin{solution}
    我们有\[\dfrac{\p f}{\p x}=\cos x-\cos(x+y)\ \ \ \ \ \dfrac{\p f}{\p y}=\cos y-\cos(x+y)\]
    令$\dfrac{\p f}{\p x}=\dfrac{\p f}{\p y}=0$,有\[\cos x=\cos y=\cos(x+y)\]
    若$x=y$,则有$\cos x=\cos 2x=2\cos^2x-1$,解得$\cos x=-\dfrac12$或$1$.\\
    由于$x,y\geqslant0$且$x+y\leqslant2\pi$,于是$x=y=\dfrac{2\pi}{3}$或$x=y=0$.此时有
    \[f\left(\dfrac{2\pi}{3},\dfrac{2\pi}{3}\right)=\dfrac{3\sqrt3}{2}\ \ \ \ \ f(0,0)=0\]
    若$x+y=2\pi$,又要求$\cos x=\cos(x+y)$,这是无解的.\\
    于是$D$中所有使$f(x,y)$取得最大值的点为$\left(\dfrac{2\pi}{3},\dfrac{2\pi}{3}\right)$.
\end{solution}
\begin{problem}[7.(10\songti{分})]
    回答下列问题.
    \begin{enumerate}[label=\tbf{(\arabic*)}]
        \item \textbf{(2\songti{分})}\ 举例说明:当$z$是$(x,y)$的函数,也是$(t,u)$的函数时,$x\equiv t\nRightarrow \dfrac{\p z}{\p x}\equiv\dfrac{\p z}{\p t}$.
        \item \textbf{(8\songti{分})}\ 给定方程\[x^2\dfrac{\p z}{\p x}+y^2\dfrac{\p z}{\p y}=z^2\]作变量代换\[x=t,y=\dfrac{t}{1+tu},z=\dfrac{t}{1+tW}\]试证明:\[\dfrac{\p W}{\p t}=0\]
    \end{enumerate}
\end{problem}
\begin{solution}
    \begin{enumerate}[label=\tbf{(\arabic*)}]
        \item 令$z=x+y=tu+t$,于是$\dfrac{\p z}{\p x}=1,\dfrac{\p z}{\p t}=1+u$.显然当$x=t$时不一定有$\dfrac{\p z}{\p x}=\dfrac{\p z}{\p t}$.
        \item 由$z=\dfrac{t}{1+tW}$有$W(t,u)=\dfrac{1}{z(x(t,u),y(t,u))}-\dfrac{1}{t}$.于是
            \[\begin{aligned}
                \dfrac{\p W}{\p t}
                &= -\dfrac{1}{z^2}\cdot\dfrac{\p z}{\p t}+\dfrac{1}{t^2} \\
                &= -\dfrac{1}{z^2}\left(\dfrac{\p z}{\p x}+\dfrac{1}{(1+tu)^2}\dfrac{\p z}{\p y}\right)+\dfrac{1}{t^2} \\
                &= -\dfrac{1}{z^2}\left(\dfrac{\p z}{\p x}+\left(\dfrac{y}{x}\right)^2\dfrac{\p z}{\p y}\right)+\dfrac{1}{x^2} \\
                &= \dfrac{1}{x^2}-\dfrac{1}{x^2} \\
                &= 0
            \end{aligned}\]
            于是命题得证.
    \end{enumerate}
\end{solution}
\begin{problem}[8.(15\songti{分})]
    证明下列恒等式.
    \begin{enumerate}[label=\tbf{(\arabic*)}]
        \item \textbf{(3\songti{分})}\ 对于任意$x\in\left(-\dfrac{1}{\sqrt2},\dfrac{1}{\sqrt2}\right)$,有\[2\int_0^x\dfrac{\di t}{\sqrt{1-t^2}}=\int_0^{2x\sqrt{1-x^2}}\dfrac{\di t}{\sqrt{1-t^2}}\]
        \item \textbf{(12\songti{分})}\ 对于任意$x\in\left(-\dfrac{1}{\sqrt[4]{6}},\dfrac{1}{\sqrt[4]{6}}\right)$,有\[2\int_0^x\dfrac{\di t}{\sqrt{1-t^4}}=\int_{0}^{\frac{2x\sqrt{1-x^4}}{1+x^4}}\dfrac{\di t}{\sqrt{1-t^4}}\]
    \end{enumerate}
\end{problem}
\begin{proof}
    \begin{enumerate}[label=\tbf{(\arabic*)}]
        \item 令$\displaystyle F(x)=2\int_0^x\dfrac{\di t}{\sqrt{1-t^2}},G(x)=\int_0^{2x\sqrt{1-x^2}}\dfrac{\di t}{\sqrt{1-t^2}}$.于是$\dfrac{\di F}{\dx}=\dfrac{2}{\sqrt{1-x^2}}$.
            \[\begin{aligned}
                \dfrac{\di G}{\di x}
                &= \dfrac{2}{\sqrt{1-\left(2x\sqrt{1-x^2}\right)^2}}\cdot\left(\sqrt{1-x^2}-\dfrac{x^2}{\sqrt{1-x^2}}\right) \\
                &= \dfrac{2}{\sqrt{4x^4-4x^2+1}}\cdot\dfrac{1-2x^2}{\sqrt{1-x^2}}=\dfrac{2}{\sqrt{1-x^2}}
            \end{aligned}\]
            即$F'(x)=G'(x)$.又$F(0)=G(0)=0$.令$H(x)=F(x)-G(x)$.\\
            根据Lagrange中值定理,对任意$x\in\left(-\dfrac{1}{\sqrt2},\dfrac{1}{\sqrt2}\right)$且$x\neq0$,存在$\xi\in(x,0)$或$(0,x)$使得
            \[H'(\xi)=\dfrac{H(x)-H(0)}{x}=0\]
            即$H(x)=H(0)=0$,于是$F(x)=G(x)$对任意$x\in\left(-\dfrac{1}{\sqrt2},\dfrac{1}{\sqrt2}\right)$成立,从而命题得证.
        \item 令$\displaystyle F(x)=2\int_0^x\dfrac{\di t}{\sqrt{1-t^4}},G(x)=\int_{0}^{\frac{2x\sqrt{1-x^4}}{1+x^4}}\dfrac{\di t}{\sqrt{1-t^4}}$.于是$\dfrac{\di F}{\dx}=\dfrac{2}{\sqrt{1-x^4}}$.
            \[\begin{aligned}
                \dfrac{\di G}{\dx}
                &= \dfrac{2}{\sqrt{1-\left(\frac{2x\sqrt{1-x^4}}{1+x^4}\right)^4}}\cdot\left(\dfrac{\left(\sqrt{1-x^4}-\frac{2x^4}{\sqrt{1-x^4}}\right)(1+x^4)-4x^4\sqrt{1-x^4}}{\left(1+x^4\right)^2}\right) \\
                &= \dfrac{2}{\sqrt{\frac{x^{16}-12x^{12}+38x^8-12x^4+1}{\left(1+x^4\right)^4}}}\cdot\dfrac{x^8-6x^4+1}{\sqrt{1-x^4}\left(1+x^4\right)^2} \\
                &= \dfrac{2}{\sqrt{\left(\frac{x^8-6x^4+1}{(1+x^4)^2}\right)^2}}\cdot\dfrac{x^8-6x^4+1}{\sqrt{1-x^4}\left(1+x^4\right)^2} \\
                &= \dfrac{2}{\sqrt{1-x^4}}
            \end{aligned}\]
            于是$F'(x)=G'(x)$,又$F(0)=G(0)=0$.与\tbf{(1)}同理可知$F(x)=G(x)$,命题得证.
    \end{enumerate}
\end{proof}
\begin{problem}[9.(15\songti{分})]
    设函数$P(x)$在$[0,1]$连续,有$P(0)=0,P(1)=1$.$P(x)$在$(0,1)$可导,且对任意$x\in(0,1)$有$P'(x)>0$.任意取定$A,B\in\R,n\in\N^*$.试证明:在$(0,1)$上存在$\theta_0,\cdots,\theta_n\in\R$使得
    \[(A+B)^n=\sum_{k=0}^n\dfrac{1}{P'(\theta_k)}\dfrac{n!}{k!(n-k)!}A^{n-k}B^k\]
    并且\[0<\theta_0<\cdots<\theta_n<1\]
\end{problem}
\begin{proof}
    设$a=\dfrac{A}{A+B},b=\dfrac{B}{A+B}$.于是要证的等式即
    \[1=\sum_{k=0}^n\dfrac{1}{P'(\theta_k)}\dfrac{n!}{k!(n-k)!}a^{n-k}b^k\]
    由于对于任意$x\in(0,1),P'(x)>0$,因而$P(x)$严格单调递增,因而$P(x)$是单射.\\
    据Lagrange中值定理,对于任意$0\leqslant\alpha<\beta\leqslant1$,存在$\xi\in(\alpha,\beta)$使得
    \[P'(\xi)=\dfrac{P(\beta)-P(\alpha)}{\beta-\alpha}\]
    即\[\dfrac{1}{P'(\xi)}=\dfrac{\beta-\alpha}{P(\beta)-P(\alpha)}\]
    现在,考虑序列$\left\{\phi_i\right\}_{i=0}^{n+1}$为
    \[\phi_{0}=0,\phi_{i}=\sum_{k=0}^{i-1}\dfrac{n!a^{n-k}b^k}{k!(n-k!)}\]
    于是$\left\{\phi_i\right\}$严格递增.记序列$\left\{\psi_i\right\}_{i=0}^{n+1}$使得$P(\psi_i)=\phi_i$.由于$P(x)$严格单调递增,于是$\psi_i$也是严格单调递增序列.\\
    又因为$P(0)=0=\phi_0,P(1)=1=\phi_{n+1}$,于是$\psi_0=0,\psi_{n+1}=1$.\\
    现在,对于任意$k\in[0,n]$,根据Lagrange中值定理都存在$\xi_k\in(\psi_k,\psi_{k+1})$使得
    \[P'(\xi_k)=\dfrac{\psi_{k+1}-\psi_k}{P(\psi_{k+1})-P(\psi_k)}=\dfrac{\psi_{k+1}-\psi_k}{\phi_{k+1}-\phi_k}\]
    于是我们有
    \[\begin{aligned}
        \sum_{k=0}^n\dfrac{1}{P'(\theta_k)}\dfrac{n!}{k!(n-k)!}a^{n-k}b^k
        &= \sum_{k=0}^n\dfrac{\psi_{k+1}-\psi_k}{\phi_{k+1}-\phi_k}\dfrac{n!}{k!(n-k)!}a^{n-k}b^k \\
        &= \sum_{k=0}^n\left(\psi_{k+1}-\psi_k\right) \\
        &= \psi_{n+1}-\psi_0 \\
        &= 1
    \end{aligned}\]
    于是命题得证.
\end{proof}
\end{document}