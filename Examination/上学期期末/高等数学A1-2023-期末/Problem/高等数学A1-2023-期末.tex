\documentclass{ctexart}
\usepackage{geometry}
\usepackage[dvipsnames,svgnames]{xcolor}
\usepackage[strict]{changepage}
\usepackage{framed}
\usepackage{enumerate}
\usepackage{amsmath,amsthm,amssymb}
\usepackage{enumitem}
\usepackage{template}

\allowdisplaybreaks
\geometry{left=2cm, right=2cm, top=2.5cm, bottom=2.5cm}

\begin{document}\pagestyle{empty}

\begin{center}\Large
    北京大学数学科学学院2023-24高等数学A1期末考试
\end{center}
\begin{enumerate}[leftmargin=*,label=\textbf{\arabic*.}]
    \item \textbf{(20\songti{分})}\ 求下列函数的极限.
        \begin{enumerate}[label=\tbf{(\arabic*)}]
            \item \textbf{(10\songti{分})}\ 求极限$\displaystyle\lim_{x\to0}\dfrac1x\left(\left(1+x\right)^{\frac1x}-\e\right)$.
            \item \textbf{(10\songti{分})}\ 设函数$f(x)$在$x=0$处$n+1$阶可导,且满足\[f(0)=f'(0)=\cdots=f^{(n-1)}=0\ \ \ \ \ f^{(n)}(0)=a\]求极限$\displaystyle\lim_{x\to0}\dfrac{f\left(\e^x-1\right)-f(x)}{x^{n+1}}$.
        \end{enumerate}
    \item \textbf{(20\songti{分})}\ 回答下列问题.
        \begin{enumerate}[label=\tbf{(\arabic*)}]
            \item \textbf{(10\songti{分})}\ 设函数$F(u,v)$有连续的二阶偏导数,$z=z(x,y)$是由方程$F(x-z,y-z)=0$确定的隐函数.计算并化简
                \[\dfrac{\p^2z}{\p x^2}+\dfrac{\p^2z}{\p x\p y}+\dfrac{\p^2z}{\p y\p x}+\dfrac{\p^2z}{\p y^2}\]
            \item \textbf{(10\songti{分})}\ 给定方程组
                \[\left\{\begin{array}{l}
                    xy+yz^2+4=0\\
                    x^2y+yz-z^2+5=0
                \end{array}\right.\]
                试讨论上述方程在$P_0\left(1,-2,1\right)$处能确定的隐函数,并计算其在$P_0$处的导数.
        \end{enumerate}
    \item \textbf{(20\songti{分})}\ 求函数$f(x,y)=\left(y-x^2\right)\left(y-x^3\right)$的极值.
    \item \textbf{(20\songti{分})}\ 回答下列问题.
        \begin{enumerate}[label=\tbf{(\arabic*)}]
            \item \textbf{(10\songti{分})}\ 设函数$f(x,y)$在点$(0,0)$的某邻域内有定义且在$(0,0)$处连续.若极限$\displaystyle\lim_{(x,y)\to(0,0)}\dfrac{f(x,y)}{x^2+y^2}$存在,试证明$f(x,y)$在$(0,0)$处可微.
            \item \textbf{(10\songti{分})}\ 欧氏空间$\R^3$中平面$T:x+y+z=1$截圆柱面$S:x^2+y^2=1$得一椭圆周$R$.求$R$上到原点最近和最远的点.
        \end{enumerate}
    \item \textbf{(20\songti{分})}\ 回答下列问题.
        \begin{enumerate}[label=\tbf{(\arabic*)}]
            \item \textbf{(10\songti{分})}\ 设$f(x)$是一个定义在$\R$上的周期为$T\neq0$的无穷阶光滑函数.试证明:对于任意$k\in\N^*$,总存在$\xi\in\R$使得$f^{(k)}(\xi)=0$.
            \item \textbf{(10\songti{分})}\ 设函数$f(u,v)$有连续的偏导数$f_u(u,v)$和$f_v(u,v)$且满足$f(x,1-x)=1$.试证明:在单位圆周$S:u^2+v^2=1$上至少存在两个不同的点$(u_1,v_1)$和$(u_2,v_2)$使得$v_if_u(u_i,v_i)=u_if_v(u_i,v_i)$,其中$i=1,2$.
        \end{enumerate}
\end{enumerate}
\end{document}