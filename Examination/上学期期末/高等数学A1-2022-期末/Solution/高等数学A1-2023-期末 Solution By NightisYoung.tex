\documentclass{ctexart}
\usepackage{geometry}
\usepackage[dvipsnames,svgnames]{xcolor}
\usepackage[strict]{changepage}
\usepackage{framed}
\usepackage{enumerate}
\usepackage{amsmath,amsthm,amssymb}
\usepackage{enumitem}
\usepackage{template}

\geometry{left=2cm, right=2cm, top=2.5cm, bottom=2.5cm}

\begin{document}\pagestyle{empty}
\begin{center}\Large
    北京大学数学科学学院2023-24高等数学A1期末考试
\end{center}
\begin{problem}[1.(20\songti{分})]
    求下列函数的极限.
    \begin{enumerate}[label=\tbf{(\arabic*)}]
        \item \textbf{(10\songti{分})}\ 求极限$\displaystyle\lim_{x\to0}\dfrac1x\left(\left(1+x\right)^{\frac1x}-\e\right)$.
        \item \textbf{(10\songti{分})}\ 设函数$f(x)$在$x=0$处$n+1$阶可导,且满足\[f(0)=f'(0)=\cdots=f^{(n-1)}=0\ \ \ \ \ f^{(n)}(0)=a\]求极限$\displaystyle\lim_{x\to0}\dfrac{f\left(\e^x-1\right)-f(x)}{x^{n+1}}$.
    \end{enumerate}
\end{problem}
\begin{solution}
    \begin{enumerate}[label=\tbf{(\arabic*)}]
        \item 令$u=\left(1+x\right)^{\frac1x}$,于是$\ln u=\dfrac{\ln(1+x)}{x}$,于是
            \[\dfrac{\di u}{\dx}=\dfrac{\di u}{\di\ln u}\cdot\dfrac{\di\ln u}{\dx}=\left(1+x\right)^{\frac1x}\dfrac{x-(x+1)\ln(x+1)}{x^2(x+1)}\]
            因此
            \[\begin{aligned}
                \lim_{x\to0}\dfrac1x\left(\left(1+x\right)^{\frac1x}-\e\right)
                &= \lim_{x\to0}\left(1+x\right)^{\frac1x}\cdot\lim_{x\to0}\dfrac{x-(x+1)\ln(x+1)}{x^2(x+1)} \\
                &= \e\cdot\lim_{x\to0}\dfrac{\frac{1}{(x+1)^2}-\frac{1}{1+x}}{2x} \\
                &= \e\cdot\lim_{x\to0}\dfrac{1-(1+x)}{2x(x+1)^2} \\
                &= -\dfrac{\e}{2}
            \end{aligned}\]
        \item 设$f^{(n+1)}(0)=b$.考虑$f(x)$在$x=0$处的泰勒展开
            \[f(x)=f(0)+\sum_{k=1}^{n+1}\dfrac{f^{(k)}(0)x^k}{k!}+o(x^{n+1})=\dfrac{ax^n}{n!}+\dfrac{bx^{n+1}}{(n+1)!}+o(x^{n+1})\]
            因为$\e^x-1\sim x$,于是
            \[f(\e^x-1)-f(x)=\dfrac{a}{n!}\left((\e^x-1)^n-x^n\right)+\dfrac{b}{(n+1)!}\left((\e^x-1)^{n+1}-x^{n+1}\right)+o(x^{n+1})\]
            于是
            \[\begin{aligned}
                \lim_{x\to0}\dfrac{f(\e^x-1)-f(x)}{x^{n+1}}
                &= \dfrac{a}{n!}\lim_{x\to0}\dfrac{(\e^x-1)^n-x^n}{x^{n+1}}+\dfrac{b}{(n-1)!}\lim_{x\to0}\left[\left(\dfrac{\e^x-1}{x}\right)^{n+1}-1+\dfrac{o(x^{n+1})}{x^{n+1}}\right] \\
                &= \dfrac{a}{n!}\lim_{x\to0}\dfrac{\left(x+\frac{x^2}{2}+o(x^2)\right)^n-x^n}{x^{n+1}}+0 \\
                &= \dfrac{a}{n!}\lim_{x\to0}\dfrac{x^n+\frac{n}{2}x^{n+1}+o(x^{n+1})-x^n}{x^{n+1}} \\
                &= \dfrac{a}{2(n-1)!}
            \end{aligned}\]
    \end{enumerate}
\end{solution}
\begin{problem}[2.(20\songti{分})]
    回答下列问题.
    \begin{enumerate}[label=\tbf{(\arabic*)}]
        \item \textbf{(10\songti{分})}\ 设函数$F(u,v)$有连续的二阶偏导数,$z=z(x,y)$是由方程$F(x-z,y-z)=0$确定的隐函数.计算并化简
            \[\dfrac{\p^2z}{\p x^2}+\dfrac{\p^2z}{\p x\p y}+\dfrac{\p^2z}{\p y\p x}+\dfrac{\p^2z}{\p y^2}\]
        \item \textbf{(10\songti{分})}\ 给定方程组
            \[\left\{\begin{array}{l}
                xy+yz^2+4=0\\
                x^2y+yz-z^2+5=0
            \end{array}\right.\]
            试讨论上述方程在$P_0\left(1,-2,1\right)$处能确定的隐函数,并计算其在$P_0$处的导数.
    \end{enumerate}
\end{problem}
\begin{solution}
    \begin{enumerate}[label=\tbf{(\arabic*)}]
        \item 设$G(x,y,z)=F(x-z,y-z)$.于是有
            \[G_x(x,y,z)=F_u(x-z,y-z)\ \ \ \ \ G_y(x,y,z)=F_v(x-z,y-z)\]
            \[G_z(x-z,y-z)=-F_u(x-z,y-z)-F_v(x-z,y-z)\]
            于是根据隐函数存在定理,$G(x,y,z)=F(x-z,y-z)=0$确定的隐函数$z=z(x,y)$满足
            \[\dfrac{\p z}{\p x}=-\dfrac{G_x}{G_z}=\dfrac{F_u}{F_u+F_v}\ \ \ \ \ \dfrac{\p z}{\p y}=-\dfrac{G_y}{G_z}=\dfrac{F_v}{F_u+F_v}\]
            其中$F_u,F_v$均指代$F_u(x-z,y-z),F_v(x-z,y-z)$.于是有
            \[\dfrac{\p z}{\p x}+\dfrac{\p z}{\p y}=\dfrac{F_u+F_v}{F_u+F_v}=1\]
            将上式对$x$求偏导有
            \[\dfrac{\p^2 z}{\p x^2}+\dfrac{\p^2 z}{\p x\p y}=0\]
            对$y$求偏导有
            \[\dfrac{\p^2 z}{\p y^2}+\dfrac{\p^2 z}{\p y\p x}=0\]
            于是\[\dfrac{\p^2z}{\p x^2}+\dfrac{\p^2z}{\p x\p y}+\dfrac{\p^2z}{\p y\p x}+\dfrac{\p^2z}{\p y^2}=0\]
        \item 设$F(x,y,z)=xy+yz^2+4,G(x,y,z)=x^2y+yz-z^2+5$.计算$F,G$在$P_0$处的各偏导有
            \[F_x=-2\ \ \ \ \ F_y=2\ \ \ \ \ F_z=-4\ \ \ \ \ G_x=-4\ \ \ \ \ G_y=2\ \ \ \ \ G_z=-4\]
            计算雅可比行列式有

    \end{enumerate}
\end{solution}
\begin{problem}[3.(20\songti{分})]
    求函数$f(x,y)=\left(y-x^2\right)\left(y-x^3\right)$的极值.
\end{problem}
\begin{solution}
    对$f(x,y)$求偏导有
    \[\dfrac{\p f}{\p x}=5x^4-3yx^2-2yx\ \ \ \ \ \dfrac{\p f}{\p y}=2y-x^2-x^3\]
    令$\dfrac{\p f}{\p x}=\dfrac{\p f}{\p y}=0$,有
    \[\left\{\begin{array}{l}
        x(5x^3-3xy-2y)=0\\
        2y-x^2-x^3=0
    \end{array}\right.\]
    将$y=\dfrac{x^2+x^3}{2}$代入第一个方程中有$x^3\left(-\dfrac{3}{2}x^2+\dfrac52x-1\right)=0$.于是$x=0$或$1$或$\dfrac23$.\\
    而\[f_{xx}(x,y)=20x^3-6xy-2y\ \ \ \ \ f_{xy}=-3x^2-2x\ \ \ \ \ f_{yy}=2\]
    当$(x,y)=(0,0)$时$B^2=AC$,于是$(0,0)$目前无法判断是否是极值点.为此,令$y=0$,则$f(x,y)=x^5$.\\
    于是在$(0,0)$的任意邻域$D$内总存在使$f(x,0)$与$f(-x,0)$异号的$x\neq0$.于是$(0,0)$不是$f(x,y)$的极值点.\\
    当$(x,y)=(1,1)$时$B^2<AC$,于是$(1,1)$不是极值点.\\
    当$(x,y)=\left(\dfrac23,\dfrac{10}{27}\right)$时$B^2>AC$,且$A<0$,于是$\left(\dfrac23,\dfrac{10}{27}\right)$是$f(x,y)$的极大值点,此时$f(x,y)=-\dfrac{4}{729}$.\\
    由于$f(x,y)$在$\R^2$上连续且可微,于是$f(x,y)$没有其它的极值.\\
    于是$f(x,y)$在$\left(\dfrac23,\dfrac{10}{27}\right)$取到极小值$-\dfrac4{729}$,没有极大值.
\end{solution}
\begin{problem}[4.(20\songti{分})]
    回答下列问题.
    \begin{enumerate}[label=\tbf{(\arabic*)}]
        \item \textbf{(10\songti{分})}\ 设函数$f(x,y)$在点$(0,0)$的某邻域内有定义且在$(0,0)$处连续.若极限$\displaystyle\lim_{(x,y)\to(0,0)}\dfrac{f(x,y)}{x^2+y^2}$存在,试证明$f(x,y)$在$(0,0)$处可微.
        \item \textbf{(10\songti{分})}\ 欧氏空间$\R^3$中平面$T:x+y+z=1$截圆柱面$S:x^2+y^2=1$得一椭圆周$R$.求$R$上到原点最近和最远的点.
    \end{enumerate}
\end{problem}
\begin{solution}
    \begin{enumerate}[label=\tbf{(\arabic*)}]
        \item 由题意可知$\displaystyle\lim_{(x,y)\to(0,0)}f(x,y)=0$.又$f(x,y)$在$(0,0)$处连续,于是$f(0,0)=0$.\\
            不妨令$\displaystyle\lim_{(x,y)\to(0,0)}\dfrac{f(x,y)}{x^2+y^2}=t$.\\
            取$(x,y)\to(0,0)$的特殊路径$x\to0,y=0$可知
            \[\lim_{x\to0}\dfrac{f(x,0)-f(0,0)}{x}=\lim_{x\to0}x\cdot\dfrac{f(x,0)}{x^2+0^2}=0\]
            同理$\displaystyle\lim_{y\to0}\dfrac{f(0,y)-f(0,0)}{y}=0$.因而$f_x(0,0)=f_y(0,0)=0$.\\
            于是
            \[\lim_{(x,y)\to(0,0)}\dfrac{f(x,y)-xf_x(0,0)-yf_y(0,0)-f(0,0)}{\sqrt{x^2+y^2}}=\lim_{(x,y)\to(0,0)}\dfrac{f(x,y)}{x^2+y^2}\cdot\sqrt{x^2+y^2}=0\]
            因而$f(x,y)=f_(0,0)+xf_x(0,0)+yf_y(0,0)+o(\rho)$,其中$\rho=\sqrt{x^2+y^2}\to0$.\\
            于是$f(x,y)$在$(0,0)$处可微.
        \item 假定$x^2+y^2=1$.设距离为$d(x,y,z)=\sqrt{x^2+y^2+z^2}=\sqrt{1+z^2}$.不妨令$x=\sin\theta,y=\cos\theta$,则有
            \[z=1-x-y=1-\sin\theta-\cos\theta\]
            令$z(\theta)=(1-\sin\theta-\cos\theta)^2$,于是$z'(\theta)=2(1-\sin\theta-\cos\theta)(\sin\theta-\cos\theta)$.令$z'(\theta)=0$可知$\theta=\dfrac\pi4,\dfrac{5\pi}{4},0,\dfrac\pi2$.\\
            于是我们有
            \[z\left(\dfrac\pi4\right)=\left(\sqrt2-1\right)^2\ \ \ \ \ z\left(\dfrac{5\pi}4\right)=\left(\sqrt2+1\right)^2\ \ \ \ \ z\left(\dfrac\pi2\right)=0\ \ \ \ \ z(0)=0\]
            距离原点最近的点为$(1,0,0)$和$(0,1,0)$,距离原点最远的点为$\left(-\dfrac{\sqrt2}{2},-\dfrac{\sqrt2}{2},1+\sqrt2\right)$.
    \end{enumerate}
\end{solution}
\begin{problem}[5.(20\songti{分})]
    回答下列问题.
    \begin{enumerate}[label=\tbf{(\arabic*)}]
        \item \textbf{(10\songti{分})}\ 设$f(x)$是一个定义在$\R$上的周期为$T\neq0$的无穷阶光滑函数.试证明:对于任意$k\in\N^*$,总存在$\xi\in\R$使得$f^{(k)}(\xi)=0$.
        \item \textbf{(10\songti{分})}\ 设函数$f(u,v)$有连续的偏导数$f_u(u,v)$和$f_v(u,v)$且满足$f(x,1-x)=1$.试证明:在单位圆周$S:u^2+v^2=1$上至少存在两个不同的点$(u_1,v_1)$和$(u_2,v_2)$使得$v_if_u(u_i,v_i)=u_if_v(u_i,v_i)$,其中$i=1,2$.
    \end{enumerate}
\end{problem}
\begin{proof}
    \begin{enumerate}[label=\tbf{(\arabic*)}]
        \item 首先,若$f(x)$是周期为$T$的周期函数,则$f'(x)$也是周期为$T$的周期函数.这可以由
            \[f'(x+T)=\lim_{\Delta x\to0}\dfrac{f(x+T+\Delta x)-f(x+T)}{\Delta x}=\lim_{\Delta x\to0}\dfrac{f(x+\Delta x)-f(x)}{\Delta x}=f'(x)\]
            得到.于是归纳可得对于任意$k\in\N^*$,$f^{(k)}(x)$都是周期为$T$的周期函数.\\
            于是对于任意$k\in\N^*$,考虑某一$x\in\R$,则有$f^{(k-1)}(x)=f^{(k-1)}(x+T)$.\\
            根据Rolle中值定理,存在$\xi\in(x,x+T)$使得$f^{(k)}(\xi)=0$,命题得证.
        \item 考虑单位圆周上的点$T(\cos\theta,\sin\theta)$.令$g(\theta)=f(\cos\theta,\sin\theta)$.于是
            \[g'(\theta)=-\sin\theta f_u(\cos\theta,\sin\theta)+\cos\theta f_v(\cos\theta,\sin\theta)\]
            注意到$g(0)=f(1,0)=1$,$g\left(\dfrac\pi2\right)=f(0,1)=1,g(2\pi)=f(1,0)=1$.\\
            由于$f_u(u,v),f_v(u,v)$均是连续函数,于是$g(\theta)$也是连续函数.\\
            在$\left[0,\dfrac\pi2\right]$应用Rolle中值定理可知存在$\phi_1\in\left(0,\dfrac\pi2\right)$使得$g'(\phi_1)=0$,即
            \[\sin\phi_1 f_u(\cos\phi_1,\sin\phi_1)=\cos\phi_1 f_v(\cos\phi_1,\sin\phi_1)\]
            同理在$\left[\dfrac\pi2,2\pi\right]$应用Rolle中值定理可知存在$\phi_2\in\left(\dfrac\pi2,2\pi\right)$使得
            \[\sin\phi_2 f_u(\cos\phi_2,\sin\phi_2)=\cos\phi_2 f_v(\cos\phi_2,\sin\phi_2)\]
            于是单位圆周上至少存在两个点$(\cos\phi_1,\sin\phi_1)$和$(\cos\phi_2,\sin\phi_2)$满足题意.
    \end{enumerate}
\end{proof}
\end{document}