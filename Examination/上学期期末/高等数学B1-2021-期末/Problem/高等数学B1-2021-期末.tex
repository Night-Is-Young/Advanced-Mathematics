\documentclass{ctexart}
\usepackage{geometry}
\usepackage[dvipsnames,svgnames]{xcolor}
\usepackage[strict]{changepage}
\usepackage{framed}
\usepackage{enumerate}
\usepackage{amsmath,amsthm,amssymb}
\usepackage{enumitem}
\usepackage{template}

\allowdisplaybreaks
\geometry{left=2cm, right=2cm, top=2.5cm, bottom=2.5cm}

\begin{document}\pagestyle{empty}

\begin{center}\Large
    北京大学数学科学学院2021-22高等数学B1期末考试
\end{center}
\begin{enumerate}[leftmargin=*,label=\textbf{\arabic*.}]
    \item \textbf{(10\songti{分})}\ 证明:对于任意$x\in\R$,存在$\theta\in(0,1)$使得
        \[\arctan x=\dfrac{x}{1+\theta^2x^2}\]
        成立.
    \item \textbf{(20\songti{分})}\ 求出下面函数的极限.
        \begin{enumerate}[label=\textbf{(\arabic*)},leftmargin=*]
            \item \textbf{(10\songti{分})}\ $\displaystyle\lim_{x\to0}\dfrac{\tan^4x}{\sqrt{1-\dfrac{x\sin x}{2}}-\sqrt{\cos x}}$.
            \item \textbf{(10\songti{分})}\ 设$n\in\N^*$.对于实序列$\left\{a_k\right\}_{k=1}^n$,求\[\lim_{x\to0}\left(\dfrac{\sum_{k=1}^na_k^x}{n}\right)^{\frac{1}{x}}\]
        \end{enumerate}
    \item \textbf{(15\songti{分})}\ 设函数\[f(x)=\dfrac{1-2x+5x^2}{(1-2x)(1+x^2)}\]在$x=0$处的$2n+1$阶泰勒公式.
    \item \textbf{(10\songti{分})}\ 定义三元函数$f:\R^3\to\R$为\[f(x,y,z)=\left\{\begin{array}{l}
            \dfrac{xyz}{x^2+y^2+z^2},(x,y,z)\neq(0,0,0)\\
            0,(x,y,z)=(0,0,0)
        \end{array}\right.\]回答下列问题.
        \begin{enumerate}[label=\textbf{(\arabic*)},leftmargin=*]
            \item \textbf{(5\songti{分})}\ 求函数$f(x,y,z)$在$(0,0,0)$处的三个偏导数.
            \item \textbf{(5\songti{分})}\ $f(x,y,z)$在$(0,0,0)$处是否可微?试证明之.
        \end{enumerate}
    \item \textbf{(15\songti{分})}\ 设$f,g:\R\to\R$都有连续的二阶导数.对于任意$x,y\in\R$,
        定义$h(x,y)=xg\left(\dfrac{y}{x}\right)+f\left(\dfrac{y}{x}\right)$,
        试计算$x^2h_{xx}(x,y)+2xyh_{yx}(x,y)+y^2h_{yy}(x,y)$.
    \item \textbf{(20\songti{分})}\ 设函数$F:\R^3\to\R$为\[F(x,y,z)=x^3+(y^2-1)z^3-xyz\]回答下列问题.
        \begin{enumerate}[label=\textbf{(\arabic*)},leftmargin=*]
            \item \textbf{(5\songti{分})}\ 证明:存在$\R^2$上$(1,1)$的邻域$D$使得$D$上由$F(x,y,z)\equiv0$确定唯一隐函数$z=f(x,y)$且$f(1,1)=1$.
            \item \textbf{(5\songti{分})}\ 求出在$(1,1)$处函数$z=f(x,y)$减少最快的方向上的单位向量$\vec{v}$.
            \item \textbf{(10\songti{分})}\ 设$\R^3$中平面$x+2y-2z=1$的$z$分量为正的法向量记为$\vec{u}$.向量$(\vec{v},0)\in\R^3$.求$\vec{u}$与$(\vec{v},0)$的夹角余弦.
        \end{enumerate}
    \item \textbf{(10\songti{分})}\ 给定正整数$n\geqslant 3$,求单位圆的内接$n$边形面积的最大值.
\end{enumerate}
\end{document}