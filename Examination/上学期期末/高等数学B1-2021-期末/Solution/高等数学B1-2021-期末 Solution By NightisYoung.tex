\documentclass{ctexart}
\usepackage{geometry}
\usepackage[dvipsnames,svgnames]{xcolor}
\usepackage[strict]{changepage}
\usepackage{framed}
\usepackage{enumerate}
\usepackage{amsmath,amsthm,amssymb}
\usepackage{enumitem}
\usepackage{template}

\geometry{left=2cm, right=2cm, top=2.5cm, bottom=2.5cm}

\begin{document}\pagestyle{empty}
\begin{center}\Large
    北京大学数学科学学院2021-22高等数学B1期末考试
\end{center}
\begin{problem}[1.(10\songti{分})]
    证明:对于任意$x\in\R$,存在$\theta\in(0,1)$使得
    \[\arctan x=\dfrac{x}{1+\theta^2x^2}\]
    成立.
\end{problem}
\begin{proof}
    当$x=0$时,任取$\theta>0$即可使等式成立.\\
    当$x\neq 0$时,不妨设$x>0$.令$f(u)=\arctan u$,于是$f(u)$在$\R$上连续且可导.\\
    根据Lagrange中值定理,存在$\xi\in(0,x)$使得$f'(\xi)=\dfrac{f(x)-f(0)}{x-0}$.\\
    即$\dfrac{1}{1+\xi^2}=\dfrac{\arctan x}{x}$.令$\theta=\dfrac{\xi}{x}$,就有$\arctan x=\dfrac{x}{1+(\theta x)^2}$.\\
    当$x<0$时亦同理.于是命题得证.
\end{proof}
\begin{problem}[2.(20\songti{分})]
    求出下面函数的极限.
    \begin{enumerate}[label=\textbf{(\arabic*)},leftmargin=*]
        \item \textbf{(10\songti{分})}\ $\displaystyle\lim_{x\to0}\dfrac{\tan^4x}{\sqrt{1-\dfrac{x\sin x}{2}}-\sqrt{\cos x}}$.
        \item \textbf{(10\songti{分})}\ 设$n\in\N^*$.对于实序列$\left\{a_k\right\}_{k=1}^n$,求\[\lim_{x\to0}\left(\dfrac{\sum_{k=1}^na_k^x}{n}\right)^{\frac{1}{x}}\]
    \end{enumerate}
\end{problem}
\begin{solution}
    \begin{enumerate}[label=\textbf{(\arabic*)},leftmargin=*]
        \item \[\begin{aligned}
            \lim_{x\to0}\dfrac{\tan^4x}{\sqrt{1-\dfrac{x\sin x}{2}}-\sqrt{\cos x}}
            &= \lim_{x\to0}\dfrac{\tan^4x}{1-\dfrac{x\sin x}{2}-\cos x}\cdot\left(\sqrt{1-\dfrac{x\sin x}{2}}+\sqrt{\cos x}\right) \\
            &= \lim_{x\to0}\dfrac{(x+o(x))^4}{1-\dfrac{x\left(x-\dfrac{x^3}{3}+o(x^3)\right)}{2}-\left(1-\dfrac{x^2}{2}+\dfrac{x^4}{24}+o(x^4)\right)} \\
            &= \lim_{x\to0}\dfrac{x^4+o(x^4)}{\dfrac{x^4}{8}+o(x^4)} = 8
            \end{aligned}\]
        \item 我们有
            \[\lim_{x\to0}\dfrac{\ln\left(\displaystyle\sum_{k=1}^na_k^x\right)-\ln n}{x}=\lim_{x\to0}\dfrac{\displaystyle\sum_{k=1}^{n}a_k^x\ln a_k}{\displaystyle\sum_{k=1}^{n}a_k^x}=\dfrac{\displaystyle\sum_{k=1}^n\ln a_k}{n}\]
            于是
            \[\lim_{x\to0}\left(\dfrac{\sum_{k=1}^na_k^x}{n}\right)^{\frac{1}{x}}=\lim_{x\to0}\text{exp}\left(\dfrac{\ln\left(\sum_{k=1}^na_k^x\right)-\ln n}{x}\right)=\e^{\dfrac{\sum_{k=1}^n\ln a_k}{n}}=\sqrt[n]{a_1\cdots a_k}\]
    \end{enumerate}
\end{solution}
\begin{problem}[3.(15\songti{分})]
    设函数\[f(x)=\dfrac{1-2x+5x^2}{(1-2x)(1+x^2)}\]在$x=0$处的$2n+1$阶泰勒公式.
\end{problem}
\begin{solution}
    我们有\[f(x)=-\dfrac{2x}{1+x^2}+\dfrac{1}{1-2x}\]
    令$g(x)=\dfrac{1}{1+x}$,对$g(x)$泰勒展开有
    \[g(x)=1-x+x^2-x^3+\cdots=\sum_{i=0}^{n}(-x)^i\]
    于是
    \[\begin{aligned}
        f(x)
        &= -2xg(x^2)+g(-2x) \\
        &= -2x\sum_{i=0}^{n}(-x^2)^i+\sum_{i=0}^{2n+1}(2x)^i+o(x^{2n+1}) \\
        &= 1+\sum_{i=1}^{n}\left((2x)^{2i}+(2^{2i+1}-2(-1)^i)x^{2i+1}\right)+o(x^{2n+1})
    \end{aligned}\]
\end{solution}
\begin{problem}[4.(10\songti{分})]
    定义三元函数$f:\R^3\to\R$为\[f(x,y,z)=\left\{\begin{array}{l}
        \dfrac{xyz}{x^2+y^2+z^2},(x,y,z)\neq(0,0,0)\\
        0,(x,y,z)=(0,0,0)
    \end{array}\right.\]回答下列问题.
    \begin{enumerate}[label=\textbf{(\arabic*)},leftmargin=*]
        \item \textbf{(5\songti{分})}\ 求函数$f(x,y,z)$在$(0,0,0)$处的三个偏导数.
        \item \textbf{(5\songti{分})}\ $f(x,y,z)$在$(0,0,0)$处是否可微?试证明之.
    \end{enumerate}
\end{problem}
\begin{solution}
    \begin{enumerate}[label=\textbf{(\arabic*)},leftmargin=*]
        \item 我们有
            \[\left.\dfrac{\p f}{\p x}\right|_{(0,0,0)}=\lim_{x\to0}\dfrac{f(x,0,0)-f(0,0,0)}{x}=\lim_{x\to0}\dfrac{0-0}{x}=0\]
            同理$\left.\dfrac{\p f}{\p y}\right|_{(0,0,0)}=\left.\dfrac{\p f}{\p z}\right|_{(0,0,0)}=0$.
        \item 记$g(x,y,z)=\dfrac{\p f}{\p z}$.\\
            当$(x,y,z)\neq(0,0,0)$时,$g(x,y,z)=\dfrac{xy(x^2+y^2+z^2)-2z(xyz)}{(x^2+y^2+z^2)^2}=\dfrac{xy(x^2+y^2-z^2)}{(x^2+y^2+z^2)^2}$.\\
            当$(x,y,z)=(0,0,0)$时,$g(x,y,z)=0$.\\
            令$x=y=kz$,于是$\displaystyle\lim_{(x,y,z)\to(0,0,0)}g(x,y,z)=\dfrac{k^2(2k^2-1)}{(2k^2+1)^2}$.\\
            于是$\displaystyle\lim_{(x,y,z)\to(0,0,0)g(x,y,z)}$不存在,因而$f(x)$在$(0,0,0)$处的偏导数不连续,因而$f$不可微.
    \end{enumerate}
\end{solution}
\begin{problem}[5.(15\songti{分})]
    设$f,g:\R\to\R$都有连续的二阶导数.对于任意$x,y\in\R$,
    定义$h(x,y)=xg\left(\dfrac{y}{x}\right)+f\left(\dfrac{y}{x}\right)$,
    试计算$x^2h_{xx}(x,y)+2xyh_{yx}(x,y)+y^2h_{yy}(x,y)$.
\end{problem}
\begin{solution}
    我们有
    \[h_x(x,y)=g\left(\dfrac{y}{x}\right)-\dfrac{y}{x}g'\left(\dfrac{y}{x}\right)-\dfrac{y}{x^2}f'\left(\dfrac{y}{x}\right)\]
    \[h_y(x,y)=g'\left(\dfrac{y}{x}\right)+\dfrac{1}{x}f'\left(\dfrac{y}{x}\right)\]
    \[h_{xx}(x,y)=-\dfrac{y}{x^2}g'\left(\dfrac{y}{x}\right)+\dfrac{y}{x^2}g'\left(\dfrac{y}{x}\right)+\dfrac{y^2}{x^3}g''\left(\dfrac{y}{x}\right)+\dfrac{2y}{x^3}f'\left(\dfrac{y}{x}\right)+\dfrac{y^2}{x^4}f''\left(\dfrac{y}{x}\right)\]
    \[h_{yy}(x,y)=\dfrac{1}{x}g''\left(\dfrac{y}{x}\right)+\dfrac{1}{x^2}f''\left(\dfrac{y}{x}\right)\]
    \[h_{yx}(x,y)=-\dfrac{y}{x^2}g''\left(\dfrac{y}{x}\right)-\dfrac{1}{x^2}f'\left(\dfrac{y}{x}\right)-\dfrac{y}{x^3}f''\left(\dfrac{y}{x}\right)\]
    于是
    \[\begin{aligned}
        &x^2h_{xx}(x,y)+2xyh_{yx}(x,y)+y^2h_{yy}(x,y)\\
        &=\dfrac{y^2}{x}g''\left(\dfrac{y}{x}\right)+\dfrac{2y}{x}f'\left(\dfrac{y}{x}\right)+\dfrac{y^2}{x^2}f''\left(\dfrac{y}{x}\right)+\dfrac{y^2}{x}g''\left(\dfrac{y}{x}\right)+\dfrac{y^2}{x^2}f''\left(\dfrac{y}{x}\right)-\dfrac{2y^2}{x}g''\left(\dfrac{y}{x}\right)-\dfrac{2y}{x}f'\left(\dfrac{y}{x}\right)-\dfrac{2y^2}{x^2}f''\left(\dfrac{y}{x}\right) \\
        &=0
    \end{aligned}\]
\end{solution}
\begin{problem}[6.(20\songti{分})]
    设函数$F:\R^3\to\R$为\[F(x,y,z)=x^3+(y^2-1)z^3-xyz\]回答下列问题.
    \begin{enumerate}[label=\textbf{(\arabic*)},leftmargin=*]
        \item \textbf{(5\songti{分})}\ 证明:存在$\R^2$上$(1,1)$的邻域$D$使得$D$上由$F(x,y,z)\equiv0$确定唯一隐函数$z=f(x,y)$且$f(1,1)=1$.
        \item \textbf{(5\songti{分})}\ 求出在$(1,1)$处函数$z=f(x,y)$减少最快的方向上的单位向量$\vec{v}$.
        \item \textbf{(10\songti{分})}\ 设$\R^3$中平面$x+2y-2z=1$的$z$分量为正的法向量记为$\vec{u}$.向量$(\vec{v},0)\in\R^3$.求$\vec{u}$与$(\vec{v},0)$的夹角余弦.
    \end{enumerate}
\end{problem}
\begin{solution}
    \begin{enumerate}[label=\textbf{(\arabic*)},leftmargin=*]
        \item 首先求一阶偏导数.
            \[\dfrac{\p F}{\p x}=3x^2-yz\ \ \ \ \ \dfrac{\p F}{\p y}=2z^3y-xz\ \ \ \ \ \dfrac{\p F}{\p z}=3(y^2-1)z^2-xy\]
            于是$F$在$(1,1,1)$附近有连续的偏导数且$\left.\dfrac{\p F}{\p z}\right|_{(1,1,1)}=-1\neq0$.\\
            根据隐函数存在定理,存在$(1,1,1)$的邻域使得$F(x,y,z)\equiv0$确定唯一隐函数$z=f(x,y)$,且$f(1,1)=1$.
        \item 我们有
            \[F_x(1,1,1)=2\ \ \ \ \ F_y(1,1,1)=1\]
            于是
            \[\dfrac{\p f}{\p x}=-\dfrac{F_x(1,1,1)}{F_z(1,1,1)}=2\ \ \ \ \ \dfrac{\p f}{\p y}=-\dfrac{F_y(1,1,1)}{F_z(1,1,1)}=1\]
            于是$\vec{v}=-\left.\tbf{grad}f\right|_{(1,1)}=-\dfrac{(2,1)}{\left|(2,1)\right|}=\left(-\dfrac{2\sqrt5}{5},-\dfrac{\sqrt5}{5}\right)$.
        \item 法向量$\vec{u}=(-1,-2,2)$.于是
            \[\cos\langle\vec{u},(\vec{v},0)\rangle=\dfrac{\vec{u}\cdot(\vec{v},0)}{|\vec{u}||(\vec{v},0)|}=\dfrac{4}{3\cdot\sqrt{5}}=-\dfrac{4\sqrt5}{15}\]
    \end{enumerate}
\end{solution}
\begin{problem}[7.(10\songti{分})]
    给定正整数$n\geqslant 3$,求单位圆的内接$n$边形面积的最大值.
\end{problem}
\begin{proof}
    我们将这$n$边形记作$A_1A_2\cdots A_n$,单位圆的圆心记作$O$.令$\theta_k=\angle A_kOA_{k+1}(1\leqslant k<n),\theta_n=\angle A_1OA_n$.\\
    考虑将多边形分成$n$个三角形.于是
    \[S_{\triangle A_kOA_k+1}=\dfrac12r^2\sin\theta_k=\dfrac12\sin\theta_k\]
    其中,当$\theta>\pi$时这三角形的面积为负.由于这在图形中表示需要减去这部分面积,于是这是不矛盾的.\\
    令$\displaystyle S(\theta_1,\cdots,\theta_n)=\dfrac12\sum_{k=1}^n\sin\theta_k$.我们要求$\displaystyle\sum_{k=1}^n\theta_k=2\pi$的约束条件下$S$的最大值.\\
    令$\displaystyle\phi(\theta_1,\cdots,\theta_n)=\sum_{k=1}^n\theta_k-2\pi$.\\
    构造辅助函数$\displaystyle F(\theta_1,\cdots,\theta_n,\phi)=S(\theta_1,\cdots,\theta_n)-\lambda\phi(\theta_1,\cdots,\theta_n)$.求$F$对$\theta_1,\cdots,\theta_n,\lambda$的偏导数可得
    \[F_{\theta_k}(\theta_1,\cdots,\theta_n,\lambda)=\dfrac{1}{2}\cos\theta_k-\lambda,\forall 1\leqslant k\leqslant n\]
    \[F_{\phi}(\theta_1,\cdots,\theta_n,\lambda)=\phi(\theta_1,\cdots,\theta_n)=0\]
    于是当各$F_{\theta_k}=0$时$F$取到最大值.\\
    此时$\cos\theta_k=2\lambda$.为使得各$\cos\theta_k$相等,又$n\geqslant3$,于是$\theta_1=\cdots=\theta_n=\dfrac{2\pi}{n}$.\\
    于是面积的最大值为$S=\dfrac{n}{2}\sin\dfrac{2\pi}{n}$.
\end{proof}
\end{document}