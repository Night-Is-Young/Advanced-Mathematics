\documentclass{ctexart}
\usepackage{geometry}
\usepackage[dvipsnames,svgnames]{xcolor}
\usepackage[strict]{changepage}
\usepackage{framed}
\usepackage{enumerate}
\usepackage{amsmath,amsthm,amssymb}
\usepackage{enumitem}
\usepackage{template}

\geometry{left=2cm, right=2cm, top=2.5cm, bottom=2.5cm}

\begin{document}\pagestyle{empty}
\begin{center}\Large
    北京大学数学科学学院2023-24高等数学A1期末考试
\end{center}
\begin{problem}[1.(11\songti{分})]
    求极限\[\lim_{(x,y)\to(0,0)}\left(x^2+y^2\right)^{x^2}\]
\end{problem}
\begin{solution}
    置$x=r\sin\theta,y=r\cos\theta,u=r^2$,于是
    \[\lim_{(x,y)\to(0,0)}\left(x^2+y^2\right)^{x^2}=\lim_{r\to0^+}\left(r^2\right)^{r^2\sin^2\theta}=\left(\lim_{u\to0+}u^u\right)^{\sin^2\theta}\]
    而
    \[\lim_{u\to0^+}u^u=\text{exp}\left(\lim_{u\to0^+}u\ln u\right)=\text{exp}\left(\lim_{u\to0^+}\dfrac{\ln u}{\frac{1}{u}}\right)=\text{exp}\left(\lim_{u\to0^+}\dfrac{\frac{1}{u}}{-\frac{1}{u^2}}\right)=\e^0=1\]
    于是
    \[\lim_{(x,y)\to(0,0)}\left(x^2+y^2\right)^{x^2}=0^{\sin^2\theta}=0\]
\end{solution}
\begin{problem}[2.(11\songti{分})]
    求极限\[\lim_{x\to+\infty}\left(\dfrac{\pi}{2}-\arctan x\right)^{\frac{1}{\ln x}}\]
\end{problem}
\begin{problem}[3.(11\songti{分})]
    求函数\[f(x)=\cos(2x)\cdot\ln(1+x)\]在$x=0$处的四阶泰勒多项式.
\end{problem}
\begin{problem}[4.(12\songti{分})]
    设$f(x)$在$[a,b]$二阶可导,满足$f(a)=f(b)=0$且存在$c\in(a,b)$使得$f(c)>0$.试证明:存在$\xi\in(a,b)$使得$f''(\xi)<0$.
\end{problem}
\begin{problem}[5.(10\songti{分})]
    回答下列问题.本题只需给出结果,无需证明.
    \begin{enumerate}[label=\tbf{(\arabic*)}]
        \item \textbf{(5\songti{分})}\ 设平面$\Sigma$过点$P_0$,其法向量为$\vec{n}$.点$P_1$是平面$\Sigma$之外的一点.试用$\overrightarrow{P_0P_1}$和$\vec{n}$表示$P_1$到$\Sigma$的距离.
        \item \textbf{(3\songti{分})}\ 设直线$L$过点$P_0$,其方向向量为$\vec{\tau}$.点$P_1$是直线$L$之外的一点.试用$\overrightarrow{P_0P_1}$和$\vec{\tau}$表示$P_1$到$L$的距离.
        \item \textbf{(2\songti{分})}\ 设异面直线$L_1,L_2$的方向向量分别为$\vec{\tau_1},\vec{\tau_2}$.点$P_1,P_2$分别是$L_1,L_2$上的点.试用$\overrightarrow{P_1P_2}$和$\vec{\tau_1},\vec{\tau_2}$表示$L_1$\\和$L_2$间的距离.
    \end{enumerate}
\end{problem}
\begin{problem}[6.(10\songti{分})]
    设二元函数\[f(x,y)=\left\{\begin{array}{l}
        y\arctan\dfrac{1}{\sqrt{x^2+y^2}},(x,y)\neq(0,0)\\
        0,(x,y)=(0,0)
    \end{array}\right.\]讨论$f(x,y)$在$(0,0)$处是否可微.
\end{problem}
\begin{problem}[7.(10\songti{分})]
    \begin{enumerate}[label=\tbf{(\arabic*)}]
        \item \textbf{(5\songti{分})}\ 设二元函数\[f(x,y)=\left\{\begin{array}{l}
                \dfrac{2xy^3}{x^2+y^4},(x,y)\neq(0,0)\\
                0,(x,y)=(0,0)
            \end{array}\right.\]计算方向导数$\left.\dfrac{\p f}{\p\mbf{l}}\right|_{(0,0)}$.其中单位向量$\mbf{l}=(\cos\alpha,\sin\alpha)$,$\alpha\in[0,2\pi)$.
        \item \textbf{(3\songti{分})}\ 若二元函数$g(x,y)$在$(x_0,y_0)$处取到极小值,那么对于某一$\alpha\in[0,2\pi)$,$t=0$是否一定是$h(t)=g(x_0+t\cos\alpha,y_0+t\sin\alpha)$的极小值点?说明理由.
        \item \textbf{(2\songti{分})}\ 若对于任意$\alpha\in[0,2\pi)$,$t=0$是是$h(t)=g(x_0+t\cos\alpha,y_0+t\sin\alpha)$的极小值点,那么$(x_0,y_0)$是否一定是$g(x,y)$的极小值点?说明理由.
    \end{enumerate}
\end{problem}
\begin{problem}[8.(15\songti{分})]
    设二元函数$z=z(x,y)$是由方程
    \[(x^2+y^2)z+\ln z+2(x+y+1)=0\]确定的隐函数,试求$z=z(x,y)$的极值.
\end{problem}
\begin{problem}[9.(10\songti{分})]
    设函数$f:[a,b]\to\R$在闭区间$[a,b]$上二阶可导,满足$f(a)=f(b)=f'(a)=f'(b)=0$,且对于任意$x\in[a,b]$都有$|f''(x)|\leqslant M$.试证明:对于任意$x\in[a,b]$,都有$|f(x)|\leqslant\dfrac{M}{16}(b-a)^2$.
\end{problem}
\begin{proof}
    考虑$|f(x)|$在$x=x_0$处取到极大值,于是$f'(x_0)=0$.\\
    考虑$x\in(a,x_0)$.将$f(x)$在$x=x_0$和$x=a$处分别做泰勒展开有
    \[f(x)=f(x_0)+(x-x_0)f'(x_0)+\dfrac{1}{2}(x-x_0)^2f''(\xi_1)\]
    \[f(x)=f(a)+(x-a)f'(a)+\dfrac{1}{2}(x-a)^2f''(\xi_2)\]
    其中$a<\xi_2<x<\xi_1<x_0$.于是我们有
    \[f(x_0)+\dfrac{1}{2}(x-x_0)^2f''(\xi_1)=\dfrac{1}{2}(x-a)^2f''(\xi_2)\]
    令$x=\dfrac{x_0+a}{2}$,则有
    \[f(x_0)=\dfrac{(x_0-a)^2}{8}\left(f''(\xi_2)-f''(\xi_1)\right)\leqslant\dfrac{M(x_0-a)^2}{4}\]
    同理,考虑$x\in(x_0,b)$可得
    \[f(x_0)\leqslant\dfrac{M(x_0-b)^2}{4}\]
    两式相加可得
    \[f(x_0)\leqslant\dfrac{1}{2}\left(\dfrac{M(x_0-a)^2}{4}+\dfrac{M(x_0-b)^2}{4}\right)=\dfrac{M}{8}\left((x_0-a)^2+(b-x_0)^2\right)\leqslant\dfrac{M}{16}(b-a)^2\]
\end{proof}
\end{document}