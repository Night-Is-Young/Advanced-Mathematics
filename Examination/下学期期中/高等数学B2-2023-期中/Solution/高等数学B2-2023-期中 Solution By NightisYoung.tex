\documentclass{ctexart}
\usepackage{template}
\usepackage{esint,extarrows}

\geometry{left=2cm, right=2cm, top=2.5cm, bottom=2.5cm}

\begin{document}\pagestyle{empty}
\begin{center}\Large
    北京大学数学科学学院2022-23高等数学B2期中考试
\end{center}
\begin{problem}[1.(10\songti{分})]
    求常微分方程
    \[\left(xy-x^3y^3\right)\dx+\left(1+x^2\right)\di y=0\]
    的满足$y(0)=1$的解.
\end{problem}
\begin{solution}
    对原方程变形可得
    \[\dfrac{\di y}{\di x}=\dfrac{x^3y^3-xy}{1+x^2}\]
    令$u=\dfrac{1}{y^2}$,则有
    \[\dfrac{\di u}{\di x}=\dfrac{x^3y^3-xy}{1+x^2}\cdot\left(-\dfrac{2}{y^3}\right)=\dfrac{2ux-2x^3}{1+x^2}\]
    即
    \[\dfrac{\di u}{\di x}-\dfrac{2ux}{1+x^2}=-\dfrac{2x^3}{1+x^2}\]
    对应齐次方程为
    \[\dfrac{\di u}{u}=\dfrac{2x\di x}{1+x^2}\]
    通解为
    \[u=C(1+x^2)\]
    现在将$u=C(x)\left(1+x^2\right)$代入原方程可得
    \[C'(x)\left(1+x^2\right)=-\dfrac{2x^3}{1+x^2}\]
    于是
    \[C(x)=-\dfrac{1}{1+x^2}-\ln(1+x^2)+C\]
    于是
    \[u(x)=-1-\left(1+x^2\right)\ln(1+x^2)+C\left(1+x^2\right)\]
    $y(0)=1$要求$u(0)=1$,于是$C=2$,即原方程的解为
    \[y=\dfrac{1}{\sqrt{1+2x^2+\left(1+x^2\right)\ln\left(1+x^2\right)}}\]
\end{solution}
\begin{problem}[2.(10\songti{分})]
    求常微分方程
    \[x^2y''+3xy'+4y=0(x>0)\]
    的满足$y(1)=y'(1)=1$的解.
\end{problem}
\begin{solution}
    做代换$x=\e^t$,于是
    \[\dfrac{\di y}{\di x}=\dfrac{\di y}{\di t}\cdot\dfrac{\di t}{\dx}=\e^{-t}\dfrac{\di y}{\di t}\]
    \[\dfrac{\di^2y}{\dx^2}=\dfrac{\di}{\di t}\left(\dfrac{\di y}{\di x}\right)\dfrac{\di t}{\di x}=\e^{-2t}\left(\dfrac{\di^2y}{\di t^2}-\dfrac{\di y}{\di t}\right)\]
    于是
    \[y''_t-y'_t+3y'_t+4y=0\]
    这是一个齐次方程,对应的特征根为$\lambda_{1,2}=-1\pm\sqrt3\i$,其通解为
    \[y=\e^{-t}\left(C_1\cos\sqrt3t+C_2\sin\sqrt3t\right)\]
    于是原方程的通解为
    \[y=\dfrac{C_1\cos\sqrt3\ln x+C_2\sin\sqrt3\ln x}{x}\]
    要求$y(1)=1$则有$C_1=1$.而
    \[y'_x=\e^{-t}y'_t=\e^{-2t}\left(\left(\sqrt3C_2-C_1\right)\cos\sqrt3t-\left(\sqrt3C_1+C_2\right)\sin\sqrt3t\right)\]
    要求$y'(1)=1$就有$\sqrt3C_2-C_1=1$.于是$C_1=1,C_2=\dfrac{2\sqrt3}{3}$,于是原方程的解为
    \[y=\dfrac{3\cos\sqrt3\ln x+2\sqrt{3}\sin\sqrt3\ln x}{3x}\]
\end{solution}
\begin{problem}[3.(10\songti{分})]
    求常微分方程
    \[y''+y'-2y=x+\e^x+\sin x\]
    的满足$y(0)=-\dfrac{7}{20},y'(0)=\dfrac{38}{15}$的解.
\end{problem}
\begin{solution}
    原方程对应的齐次方程的特征根为$\lambda_1=1,\lambda_2=-2$,于是原方程的通解为
    \[y=C_1\e^{x}+C_2\e^{-2x}\]
    考虑方程$y''+y'-2y=x$的特解,设$y=Ax^2+Bx+C$,则
    \[2A+(2Ax+B)-2\left(Ax^2+Bx+C\right)=x\]
    于是$A=0,B=-\dfrac12,C=-\dfrac14$.\\
    考虑方程$y''+y'-2y=\e^x$的特解,设$y=Ax\e^x$,则有
    \[A\e^x\left(x+2+x+1-2x\right)=\e^x\]
    于是$A=\dfrac13$.\\
    考虑方程$y''+y'-2y=\sin x$的特解,设$y=A\cos x+B\sin x$,则有
    \[(-B-A-2B)\sin x+(-A+B-2A)\cos x=\sin x\]
    于是$A=-\dfrac{1}{10},B=-\dfrac{3}{10}$.\\
    于是原方程的通解为
    \[y=C_1\e^{x}+C_2\e^{-2x}-\dfrac12x-\dfrac14+\dfrac13x\e^x-\dfrac{1}{10}\cos x-\dfrac{3}{10}\sin x\]
    于是
    \[\left\{\begin{array}{l}
        y(0)=C_1+C_2-\dfrac14-\dfrac{1}{10}=-\dfrac{7}{20}\\
        y'(0)=C_1-2C_2-\dfrac12+\dfrac{1}{3}-\dfrac{3}{10}=\dfrac{38}{15}
    \end{array}\right.\]
    解得$C_1=1,C_2=-1$,于是原方程的解为
    \[y=\e^{x}-\e^{-2x}-\dfrac12x-\dfrac14+\dfrac13x\e^x-\dfrac{1}{10}\cos x-\dfrac{3}{10}\sin x\]

\end{solution}
\begin{problem}[4.(10\songti{分})]
    设关于$R$的函数
    \[I(R)=\oint_{x^2+y^2=R^2}\dfrac{x\di y-y\di x}{\left(x^2+xy+y^2\right)^2}\]
    试证明
    \[\lim_{R\to\infty}I(R)=0\]
    其中积分方向为圆周的逆时针方向.
\end{problem}
\begin{proof}
    考虑二元函数
    \[P(x,y)=\dfrac{-y}{\left(x^2+xy+y^2\right)^2}\ \ \ \ \ Q(x,y)=\dfrac{x}{\left(x^2+xy+y^2\right)^2}\]
    积分曲线$S_R:x^2+y^2=R^2$的单位切向量为$\mbf n=\frac{1}{R}(-y,x)$,于是
    \[\begin{aligned}
        I(R)
        &= \oint_{S_R^+}\left(P,Q\right)\cdot\mbf n\di s \\
        &= \oint_S\dfrac{x^2+y^2}{R\left(x^2+xy+y^2\right)^2}\di s \\
        &= \int_0^{2\pi}\dfrac{R}{\left(R^2\left(1+\cos t\sin t\right)\right)^2}\di t \\
        &\leqslant \int_0^{2\pi}\dfrac{4}{R^3}\di t \\
        &= \dfrac{8\pi}{R^3}
    \end{aligned}\]
    于是
    \[0\leqslant I(R)\leqslant \dfrac{8\pi}{R^3}\]
    由夹逼准证可知
    \[\lim_{R\to+\infty}I(R)=0\]

\end{proof}
\begin{problem}[5.(10\songti{分})]
    设$L$为空间曲线
    \[\left\{\begin{array}{l}
        x^2+y^2=1\\
        x+z=1
    \end{array}\right.\]
    其方向为自$z$轴正方向向负方向看的逆时针方向.计算曲线积分
    \[\int_{L}\left(y-z+\sin^2x\right)\dx+(z-x+\sin^2y)\di y+\left(x-y+\sin^2z\right)\di z\]

\end{problem}
\begin{solution}
    令$P(x,y,z)=y-z+\sin^2x,Q(x,y,z)=z-x+\sin^2y,R(x,y,z)=x-y+\sin^2z$.%
    于是$P,Q,R$在$\R^3$上可微.$L$围成的空间曲面$S:z=1-x,x^2+y^2\leqslant 1$,其单位外法向量为$\mbf n=\dfrac{(1,0,1)}{\sqrt2}$.\\
    于是据Stokes公式有
    \[\begin{aligned}
        \int_LP\di x+Q\di y+R\di z
        &= \iint_{S^+}\begin{vmatrix}
            \di x\di y&\di y\di z&\di z\di x\\
            \dfrac{\p}{\p x}&\dfrac{\p}{\p y}&\dfrac{\p}{\p z}\\
            P&Q&R
        \end{vmatrix}\\
        &= -2\iint_{S^+}\dx\di y+\di y\di z+\di z\di x \\
        &= -4\iint_{x^2+y^2\leqslant 1}\di\sigma \\
        &= -4\pi
    \end{aligned}\]
    
\end{solution}
\begin{problem}[6.(10\songti{分})]
    设$D$是单位圆$x^2+y^2\leqslant1$,求积分
    \[\iint_D\left(x+y+xy\right)^2\di\sigma\]

\end{problem}
\begin{solution}
    做极坐标变换$x=r\cos\theta,y=r\sin\theta$,于是积分区域变为$D':0\leqslant r\leqslant 1,0\leqslant\theta\leqslant2\pi$.于是
    \[\begin{aligned}
        \iint_{D}\left(x+y+xy\right)^2\di\sigma
        &= \iint_{D}\left(x^2+y^2+x^2y^2+2xy+2x^2y+2xy^2\right)\di\sigma \\
        &= \iint_{D}\left(x^2+y^2+x^2y^2\right)\di\sigma \\
        &= \int_0^{2\pi}\di\theta\int_0^1\left(r^2+r^4\sin^2\theta\cos^2\theta\right)r\di r \\
        &= \int_0^{2\pi}\left(\dfrac14+\dfrac16\sin^2\theta\cos^2\theta\right)\di\theta \\
        &\xlongequal{t=2\theta} \dfrac\pi2+\dfrac{1}{48}\int_0^{4\pi}\sin^2 t\di t \\
        &= \dfrac{13}{24}\pi
    \end{aligned}\]
\end{solution}
\begin{problem}[7.(10\songti{分})]
    设平面闭区域$D$由直线$y=x$和曲线$y=x^3$围成,求积分
    \[\iint_D\left(\dfrac{3x^2\sin y}{y}+2\e^{x^2}\right)\di\sigma\]

\end{problem}
\begin{solution}
    首先考虑第一项,有
    \[\begin{aligned}
        \iint_D\dfrac{3x^2\sin y}{y}\di x\di y
        &= \int_{-1}^0\di y\int_{\sqrt[3]y}^{y}\dfrac{3x^2\sin y}{y}\di x+\int_{0}^1\di y\int_{y}^{\sqrt[3]y}\dfrac{3x^2\sin y}{y}\di x \\
        &\xlongequal{t=x^3} \int_{-1}^0\di y\int_y^{y^3}\dfrac{\sin y}{y}\di t+\int_{0}^1\di y\int_{y^3}^{y}\dfrac{\sin y}{y}\di t\\
        &= \int_{-1}^0\sin y\left(y^2-1\right)\di y+\int_{0}^1\sin y\left(1-y^2\right)\di y\\
        &= 2\int_{0}^1\sin y\left(1-y^2\right)\di y \\
        &= 2\left.\left(y^2\cos y-2y\sin y-3\cos y\right)\right|_0^1 \\
        &= 6-4\cos 1-4\sin 1
    \end{aligned}\]
    再考虑第二项,有
    \[\begin{aligned}
        \iint_D2\e^{x^2}\di\sigma
        &= \int_{-1}^0\dx\int_{x}^{x^3}2\e^{x^2}\di y+\int_{0}^1\dx\int_{x^3}^{x}2\e^{x^2}\di y \\
        &= 4\int_{0}^1\left(x-x^3\right)\e^{x^2}\dx \\
        &\xlongequal{t=x^2}2\int_0^1\left(1-t\right)\e^t\di t \\
        &= 2\left.\left(\e^t(2-t)\right)\right|_0^1 \\
        &= 2\e-4
    \end{aligned}\]
    于是
    \[\iint_D\left(\dfrac{3x^2\sin y}{y}+2\e^{x^2}\right)\di\sigma=2\e+2-4\cos 1-4\sin 1\]

\end{solution}
\begin{problem}[8.(10\songti{分})]
    设空间闭区域$\Omega$由曲面$z=\sqrt{1+x^2+y^2},z=\sqrt{3\left(1+x^2+y^2\right)}$和$x^2+y^2=1$围成,求积分
    \[\iiint_{\Omega}\dfrac{(x+y+z)^2\sqrt{1+x^2+y^2}}{\left(x^2+y^2+z^2\right)\left(1+x^2+y^2+z^2\right)}\di V\]

\end{problem}
\begin{solution}
    做柱坐标变换$z=z,x=r\cos\theta,y=r\sin\theta$,于是积分区域变为
    \[\Omega':0\leqslant r\leqslant 1,0\leqslant\theta\leqslant2\pi,\sqrt{1+r^2}\leqslant z\leqslant \sqrt3\sqrt{1+r^2}\]
    于是
    \[\begin{aligned}
        &\iiint_{\Omega}\dfrac{(x+y+z)^2\sqrt{1+x^2+y^2}}{\left(x^2+y^2+z^2\right)\left(1+x^2+y^2+z^2\right)}\di V \\
        =&\iiint_{\Omega'}\dfrac{\left(r^2+z^2+r^2\sin\theta\cos\theta+2rz\left(\sin\theta+\cos\theta\right)\right)\sqrt{1+r^2}}{\left(r^2+z^2\right)\left(1+r^2+z^2\right)}r\di V \\
        =&\int_0^{2\pi}\di\theta\int_0^1\di r\int_{\sqrt{1+r^2}}^{\sqrt3\sqrt{1+r^2}}\dfrac{r\sqrt{1+r^2}}{\left(z^2+r^2+1\right)}\di z \\
        =&\int_0^{2\pi}\di\theta\int_0^1\di r\left.\left(r\arctan\dfrac{z}{\sqrt{1+r^2}}\right)\right|_{\sqrt{1+r^2}}^{\sqrt3\sqrt{1+r^2}} \\
        =&\int_0^{2\pi}\di\theta\int_0^1\dfrac{\pi}{12}r\di r \\
        =&\dfrac{\pi^2}{12}
    \end{aligned}\]

\end{solution}
\begin{problem}[9.(10\songti{分})]
    设$\Gamma$由闭曲线$x^2+y^2=9(y\geqslant0)$和$\dfrac{x^2}{9}+\dfrac{y^2}{16}=1(y\leqslant0)$组成,方向沿逆时针方向,求曲线积分
    \[\oint_{\Gamma}\left(\dfrac{y^2+y+4x^2}{4x^2+y^2}+\sin x^2\right)\dx+\left(\dfrac{4x^2-x+y^2}{4x^2+y^2}+\sin y^2\right)\di y\]

\end{problem}
\begin{solution}
    设$\Omega$为$\Gamma$围成的闭区域.\\
    首先令
    \[P(x,y)=\dfrac{y}{4x^2+y^2}\ \ \ \ \ Q(x,y)=\dfrac{-x}{4x^2+y^2}\]
    再令
    \[A(x,y)=1+\sin x^2\ \ \ \ \ B(x,y)=1+\sin y^2\]
    于是原曲线积分即为
    \[\oint_{\Gamma}(A+P)\dx+(B+Q)\di y\]
    首先有
    \[\dfrac{\p A}{\p y}=\dfrac{\p B}{\p x}=0\]
    在$\R^2$上成立,于是据Green公式有
    \[\oint_{L}A\di x+B\di y=\iint_{\Omega}0\di\sigma=0\]
    考虑小椭圆$\Omega_\ep:4x^2+y^2\leqslant\ep^2$和其边界$S_\ep$,其中$0<\ep<3$.在$\Omega\backslash\Omega_\ep$上根据Green公式有
    \[\begin{aligned}
        \oint_{\Gamma+S_\ep^-}P\di x+Q\di y
        &= \iint_{\Omega\backslash\Omega_\ep}\left(\dfrac{\p Q}{\p x}-\dfrac{\p P}{\p y}\right)\di\sigma \\
        &= \iint_{\Omega\backslash\Omega_\ep}-\dfrac{\left(4x^2+y^2\right)-8x^2+\left(4x^2+y^2\right)-2y^2}{\left(4x^2+y^2\right)^2}\di\sigma \\
        &= 0
    \end{aligned}\]
    又有
    \[\begin{aligned}
        \oint_{\Gamma}P\di x+Q\di y
        &= \oint_{S_\ep^+}P\dx+Q\di y \\
        &= \dfrac{1}{\ep^2}\oint_Sy\dx-x\di y \\
        &= -\dfrac{2}{\ep^2}\iint_{\Omega_\ep}\di\sigma \\
        &= -\pi
    \end{aligned}\]
    于是
    \[\oint_{\Gamma}\left(\dfrac{y^2+y+4x^2}{4x^2+y^2}+\sin x^2\right)\dx+\left(\dfrac{4x^2-x+y^2}{4x^2+y^2}+\sin y^2\right)\di y=-\pi\]

\end{solution}
\begin{problem}[10.(10\songti{分})]
    设曲面$S$是柱体$\Omega=\left\{(x,y,z)\in\R^3:x^2+y^2\leqslant1,0\leqslant z\leqslant1\right\}$的表面的外侧.
    \begin{enumerate}[label=\tbf{(\arabic*)}]
        \item 求曲面积分
            \[\iint_S(y-z)|x|\di y\di z+(z-x)|y|\di z\di x+(x-y)z\di x\di y\]
        \item 求曲面积分
            \[\iint_S(y-z)x^2\di y\di z+(z-x)y^2\di z\di x+(x-y)z^2\di x\di y\]
        \item 求曲面积分
            \[\iint_S(y-z)x^3\di y\di z+(z-x)y^3\di z\di x+(x-y)z^3\di x\di y\]
    \end{enumerate}
\end{problem}
\begin{solution}
    $S$可分为三个部分,即
    \[D_1=\left\{(x,y,z)\in\R^3:z=0,x^2+y^2\leqslant1\right\}\]
    \[D_2=\left\{(x,y,z)\in\R^3:z=1,x^2+y^2\leqslant1\right\}\]
    \[D_3=\left\{(x,y,z)\in\R^3:0\leqslant z\leqslant 1,x^2+y^2=1\right\}\]
    三个面的单位外法向量分别为$\mbf n_1=(0,0,-1),\mbf n_2=(0,0,1),\mbf n_3=(x,y,0)$.\\
    各区域都关于$Oxz$和$Oyz$平面对称,也关于$z$轴对称.
    \begin{enumerate}[label=\tbf{(\arabic*)}]
        \item 在$D_1$面上有
            \[I_1=\iint_{D_1}-1(x-y)z\di S=0\]
            在$D_2$面上有
            \[I_2=\iint_{D_1}-1(x-y)z\di S=-\iint_{x^2+y^2\leqslant 1}(x-y)\di\sigma=0\]
            在$D_3$面上有
            \[I_3=\iint_{D_3}x|x|(y-z)+y|y|(z-x)\di S=0\]
            于是
            \[\iint_S(y-z)|x|\di y\di z+(z-x)|y|\di z\di x+(x-y)z\di x\di y=0\]
        \item 同理不难有$I_1=I_2=0$,而
            \[I_3=\iint_{D_3}(y-z)x^3+(z-x)y^3\di S=0\]
            于是
            \[\iint_S(y-z)x^2\di y\di z+(z-x)y^2\di z\di x+(x-y)z^2\di x\di y=0\]
        \item 仍然不难有$I_1=I_2=0$,而
            \[I_3=\iint_{D_3}(y-z)x^4+(z-x)y^4\di S=\iint_{D_3}x^4y-xy^4+z\left(y^4-x^4\right)\di S=0\]
            于是
            \[\iint_S(y-z)x^3\di y\di z+(z-x)y^3\di z\di x+(x-y)z^3\di x\di y=0\]
    \end{enumerate}
\end{solution}
\end{document}