\documentclass{ctexart}
\usepackage{template}
\usepackage{esint}

\begin{document}\pagestyle{empty}
\begin{center}\Large
    北京大学数学科学学院2022-23高等数学B2期中模拟考试
\end{center}
\begin{enumerate}[leftmargin=*,label=\textbf{\arabic*.}]
    \item \textbf{(10\songti{分})}\ 设$D$是由直线$y=0,y=1,y=x,y=x+1$所围成的有界闭区域,求二重积分
        \[\iint_D(4y-2x)\di x\di y\]
    \item \textbf{(10\songti{分})}\ 设$V$是由平面$x=0,y=0,z=0,x+y+z=1$围成的四面体,求三重积分
        \[\iiint_V\dfrac{1}{\left(1+x+y+z\right)^2}\di V\]
    \item \textbf{(10\songti{分})}\ 设$E$是椭圆$x^2+\dfrac{y^2}{4}=1$,求曲线积分
        \[\int_E\left|xy\right|\di s\]
    \item \textbf{(15\songti{分})}\ 设$n\in\N^*$,有向曲线$L_n=\left\{\left(t,\left|\sin t\right|\right):0\leqslant t\leqslant n\pi\right\}$.求极限
        \[\lim_{n\to\infty}\int_{L_n}\e^{y^2-x^2}\cos(2xy)\di x+\e^{y^2-x^2}\sin(2xy)\di y\]
    \item \textbf{(10\songti{分})}\ 设$S$是曲面$\left\{(x,y,z)\in\R^3:x^2+z^2=1,x\geqslant 0,z\geqslant 0,0\leqslant y\leqslant 1\right\}$.求曲面积分
        \[\iint_Sx\di S\]
    \item \textbf{(10\songti{分})}\ 设$S$是单位球面$x^2+y^2+z^2=1$,求曲面积分
        \[\iint_{S^+}x\di y\di z+y\di z\di x+z\di x\di y\]
    \item \textbf{(15\songti{分})}\ 设平面直角坐标系$Oxy$中有曲线$L:\left\{(x,y(x)):x\geqslant0\right\}$,其中$y(0)=1$,%
        $y(x)$是严格递减的正的可导函数.任取$L$上一点$M$,$L$在$M$点的切线交$x$轴于$A$点,假定$\left|MA\right|\equiv1$.%
        写出$y=y(x)$满足的一阶常微分方程,并求解该方程对应的初值问题$y(0)=1$.
    \item \textbf{(10\songti{分})}\ 求常微分方程
        \[y''+4y=\sin3x\]
        的通解.
    \item \textbf{(10\songti{分})}\ 回答下列问题.
        \begin{enumerate}[label=\tbf{(\arabic*)}]
            \item 设$D=\R^2\backslash\left\{(x,0)\in\R^2:x\geqslant 0\right\}$,写出一个在$D$上可微的函数$T:D\to\R$且满足
                \[\dfrac{\p T}{\p x}=\dfrac{-y}{x^2+y^2}\ \ \ \ \ \dfrac{\p T}{\p y}=\dfrac{x}{x^2+y^2}\]
            \item 设$\Omega=\R^2\backslash\left\{(0,0)\right\}$,试证明:不存在函数$U\to\R$使得$U$在$\Omega$上可微,且满足
                \[\dfrac{\p U}{\p x}=\dfrac{-y}{x^2+y^2}\ \ \ \ \ \dfrac{\p U}{\p y}=\dfrac{x}{x^2+y^2}\]
        \end{enumerate}
\end{enumerate}
\end{document}