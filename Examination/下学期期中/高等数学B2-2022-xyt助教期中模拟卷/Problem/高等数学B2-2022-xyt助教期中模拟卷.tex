\documentclass{ctexart}
\usepackage{template}
\usepackage{esint}

\begin{document}\pagestyle{empty}
\begin{center}\Large
    北京大学数学科学学院2022-23高等数学B2期中模拟考试
\end{center}
\begin{enumerate}[leftmargin=*,label=\textbf{\arabic*.}]
    \item \textbf{(10\songti{分})}\ 设$D$是由直线$x=1,y=x,y=2x$所围成的有界闭区域,求二重积分
        \[\iint_D(\sqrt x+y)\di x\di y\]
    \item \textbf{(10\songti{分})}\ 求三重积分
        \[\iiint_\Omega\dfrac{1}{\sqrt{x^2+y^2+z^2}}\di V\]
        其中$\Omega$是球$x^2+y^2+z^2\leqslant 2x$.
    \item \textbf{(10\songti{分})}\ 求曲线积分
        \[\int_L\left|\dfrac{\di y-\di x}{x-y+1}\right|\di s\]
        其中$L$是圆周$x^2+y^2-2x=0$再$y\leqslant 0$的部分,沿逆时针方向.
    \item \textbf{(15\songti{分})}\ 求曲面积分
        \[\iint_Sy(x-z)\di y\di z+x^2\di z\di x+\left(y^2+xz\right)\di x\di y\]
        其中$S$是球面$x^2+y^2+z^2=5$在$z\geqslant 1$的部分,取外侧.
    \item \textbf{(10\songti{分})}\ 求常微分方程
        \[xy'+2y=\sin x\]
        满足$y(\pi)=\dfrac1\pi$的特解.
    \item \textbf{(10\songti{分})}\ 求常微分方程
        \[y''+2y'=3+4\sin 2x\]
        的通解.
    \item \textbf{(10\songti{分})}\ 设参数$a>0$,$\R^3$中的圆柱$x^2+z^2=a^2$和$y^2+z^2=a^2$相交的区域为$\Omega$,求$\Omega$的体积.
    \item \textbf{(10\songti{分})}\ 设$u(x,y)$是闭矩形$[a,b]\times[c,d]$上的连续函数,在$D$上存在连续的二阶偏导数,并且$u(x,y)=0$对$D$的边界上任意一点都成立,试证明
        \[\iint_D\left|u(x,y)\right|^2\di\sigma\leqslant\left(\iint_D\left|\dfrac{\p u}{\p x}\right|\di\sigma\right)\left(\iint_D\left|\dfrac{\p u}{\p y}\right|\di\sigma\right)\]
        
    \item \textbf{(15\songti{分})}\ 在如下的常微分方程
        \[\left\{\begin{array}{l}
            u''(x)+\lambda u(x)=0\\
            -a_1u'(0)+a_2u(0)=b_1u'(L)+b_2u(L)
        \end{array}\right.\]
        中,我们想求解定义在$[0,L]$上的函数$u(x)$,其中参数$a_1,a_2,b_1,b_2,L$都是给定的函数.\\
        对于大多数参数$\lambda$,上述常微分方程并没有解.如果对于某个$\lambda$,上述方程存在不恒等于$0$的解,%
        我们就成$\lambda$为一个\tbf{本征值},对应的解$u_\lambda(x)$称为\tbf{本征函数}.据此,回答下列问题.
        \begin{enumerate}[label=\tbf{(\arabic*)}]
            \item 当$a_1=-1,b_1=1,a_2=b_2=0$时,试证明$0$是本征值.
            \item 对于$a_1=-1,b_1=1,a_2=b_2=0$以及$a_1=b_1=0,a_2=b_2=1$的情形,分别求解所有本征值和每个本征值对应的所有本征函数.
            \item 如果$a_1,a_2,b_1,b_2>0$,试不通过求解方程证明所有本征值都是正数.
            \item 如果$a_1,a_2,b_1,b_2>0$,设$\lambda$和$\mu$是两个不同的本征值,对应本征函数分别为$u_\lambda(x)$和$u_\mu(x)$.试证明
                \[\int_0^Lu_\lambda(x)u_\mu(x)\di x=0\]

        \end{enumerate}
\end{enumerate}
\end{document}