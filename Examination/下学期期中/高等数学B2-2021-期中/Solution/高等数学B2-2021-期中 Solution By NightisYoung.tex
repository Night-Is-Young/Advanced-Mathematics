\documentclass{ctexart}
\usepackage{template}
\usepackage{esint,extarrows}

\geometry{left=2cm, right=2cm, top=2.5cm, bottom=2.5cm}

\begin{document}\pagestyle{empty}
\begin{center}\Large
    北京大学数学科学学院2021-22高等数学B2期中考试
\end{center}
\begin{problem}[1.(10\songti{分})]
    计算二重积分
    \[\iint_{D}\ln\left(1+x^2+y^2\right)\dx\di y\ \ \ \ \ D:x^2+y^2\leqslant1,x\geqslant0,y\geqslant0\]
\end{problem}
\begin{solution}
    做极坐标变换$x=r\cos\theta,y=r\sin\theta$.于是变换后的积分区域$D':0\leqslant r\leqslant 1,0\leqslant\theta\leqslant\dfrac\pi2$.于是
    \[\begin{aligned}
        \iint_{D}\ln\left(1+x^2+y^2\right)\dx\di y
        &= \iint_{D'}\ln\left(1+r^2\right)r\di r\di\theta \\
        &= \int_0^{\frac\pi2}\di\theta\int_0^1\ln\left(1+r^2\right)r\di r \\
        &\xlongequal{t=r^2} \dfrac12\int_0^{\frac\pi2}\di\theta\int_1^2\ln t\di t \\
        &= \dfrac\pi4\left(\ln 2-1\right)
    \end{aligned}\]
\end{solution}
\begin{problem}[2.(10\songti{分})]
    计算三重积分
    \[\iiint_{\Omega}\left(y^2+z^2\right)\di V\ \ \ \ \ \Omega:0\leqslant z\leqslant x^2+y^2\leqslant1\]
\end{problem}
\begin{solution}
    做柱坐标变换$x=r\cos\theta,y=r\sin\theta,z=z$,于是变换后的积分区域
    \[\Omega':0\leqslant z\leqslant r^2\leqslant 1,0\leqslant \theta\leqslant 2\pi\]
    于是
    \[\begin{aligned}
        \iiint_{\Omega}\left(y^2+z^2\right)\di V
        &= \iiint_{\Omega'}\left(r^2\sin^2\theta+z^2\right)r\di r\di z\di\theta \\
        &= \int_0^{2\pi}\di\theta\int_0^1\di r\int_0^{r^2}\left(r^2\sin^2\theta+z^2\right)r\di z \\
        &= \int_0^{2\pi}\di\theta\int_0^1r\left(r^4\sin^2\theta+\dfrac{r^6}{3}\right)\di r\\
        &= \int_0^{2\pi}\left(\dfrac16\sin^2\theta+\dfrac1{24}\right)\di\theta\\
        &= \dfrac\pi6+\dfrac1{24}\cdot2\pi \\
        &= \dfrac\pi4
    \end{aligned}\]
\end{solution}
\begin{problem}[3.(10\songti{分})]
    设曲线$C$为椭圆$\dfrac{x^2}{16}+\dfrac{y^2}{9}=1$沿逆时针方向.计算曲线积分
    \[\oint_{C}\dfrac{x\di y-y\di x}{x^2+y^2}\]
\end{problem}
\begin{problem}[4.(10\songti{分})]
    计算曲面积分
    \[\iint_{S}\left(x^2y^2+y^2z^2+z^2x^2\right)\di S\]
    其中$S$为圆锥面$z=\sqrt{x^2+y^2}$被柱面$x^2+y^2=1$所截下部分.
\end{problem}
\begin{problem}[5.(15\songti{分})]
    计算曲面积分
    \[\oiint_Sx\di y\di z+y\di z\di x+z\di x\di y\]
    其中$S$为抛物面$z=x^2+y^2$被平面$z=4$所截部分的外侧.
\end{problem}
\begin{problem}[6.(10\songti{分})]
    求常微分方程
    \[y'=xy+3x+2y+6\]
    的所有解.
\end{problem}\begin{problem}[7.(15\songti{分})]
    求常微分方程
    \[y''-4y'+3y-4\e^x=0\]
    的通解.
\end{problem}\begin{problem}[8.(10\songti{分})]
    设平面有界闭区域为
    \[D=\left\{(x,y)\in\R^2:\dfrac{x^2}{a^2}+\dfrac{y^2}{b^2}\leqslant\right\}\ \ \ \ \ a,b>0\]
    设曲线$L$为$D$的边界,函数$P(x,y),Q(x,y)$在$D$上有连续的一阶偏导数.记$\mbf{F}=(P,Q)$,$\mbf{n}$为曲线$L$的单位外法向量.试证明
    \[\oint_{L^+}\mbf{F}\cdot\mbf{n}\di s=\iint_{D}\left(\dfrac{\p P}{\p x}+\dfrac{\p Q}{\p y}\right)\di x\di y\]
\end{problem}\begin{problem}[9.(10\songti{分})]
    设$f(x)$为$\R$上的连续函数.试证明
    \[\iint_{S}f(x+y+z)\di S=2\pi\int_{-1}^{1}f\left(\sqrt{3}\xi\right)\di\xi\]
    其中$S$为单位球面$x^2+y^2+z^2=1$.
\end{problem}
\end{document}