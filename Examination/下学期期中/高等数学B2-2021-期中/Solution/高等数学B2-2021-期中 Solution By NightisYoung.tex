\documentclass{ctexart}
\usepackage{template}
\usepackage{esint,extarrows}

\geometry{left=2cm, right=2cm, top=2.5cm, bottom=2.5cm}

\begin{document}\pagestyle{empty}
\begin{center}\Large
    北京大学数学科学学院2021-22高等数学B2期中考试
\end{center}
\begin{problem}[1.(10\songti{分})]
    计算二重积分
    \[\iint_{D}\ln\left(1+x^2+y^2\right)\dx\di y\ \ \ \ \ D:x^2+y^2\leqslant1,x\geqslant0,y\geqslant0\]
\end{problem}
\begin{solution}
    做极坐标变换$x=r\cos\theta,y=r\sin\theta$.于是变换后的积分区域$D':0\leqslant r\leqslant 1,0\leqslant\theta\leqslant\dfrac\pi2$.于是
    \[\begin{aligned}
        \iint_{D}\ln\left(1+x^2+y^2\right)\dx\di y
        &= \iint_{D'}\ln\left(1+r^2\right)r\di r\di\theta \\
        &= \int_0^{\frac\pi2}\di\theta\int_0^1\ln\left(1+r^2\right)r\di r \\
        &\xlongequal{t=r^2} \dfrac12\int_0^{\frac\pi2}\di\theta\int_1^2\ln t\di t \\
        &= \dfrac\pi4\left(\ln 2-1\right)
    \end{aligned}\]
\end{solution}
\begin{problem}[2.(10\songti{分})]
    计算三重积分
    \[\iiint_{\Omega}\left(y^2+z^2\right)\di V\ \ \ \ \ \Omega:0\leqslant z\leqslant x^2+y^2\leqslant1\]
\end{problem}
\begin{solution}
    做柱坐标变换$x=r\cos\theta,y=r\sin\theta,z=z$,于是变换后的积分区域
    \[\Omega':0\leqslant z\leqslant r^2\leqslant 1,0\leqslant \theta\leqslant 2\pi\]
    于是
    \[\begin{aligned}
        \iiint_{\Omega}\left(y^2+z^2\right)\di V
        &= \iiint_{\Omega'}\left(r^2\sin^2\theta+z^2\right)r\di r\di z\di\theta \\
        &= \int_0^{2\pi}\di\theta\int_0^1\di r\int_0^{r^2}\left(r^2\sin^2\theta+z^2\right)r\di z \\
        &= \int_0^{2\pi}\di\theta\int_0^1r\left(r^4\sin^2\theta+\dfrac{r^6}{3}\right)\di r\\
        &= \int_0^{2\pi}\left(\dfrac16\sin^2\theta+\dfrac1{24}\right)\di\theta\\
        &= \dfrac\pi6+\dfrac1{24}\cdot2\pi \\
        &= \dfrac\pi4
    \end{aligned}\]
\end{solution}
\begin{problem}[3.(10\songti{分})]
    设曲线$C$为椭圆$\dfrac{x^2}{16}+\dfrac{y^2}{9}=1$沿逆时针方向.计算曲线积分
    \[\oint_{C}\dfrac{x\di y-y\di x}{x^2+y^2}\]
\end{problem}
\begin{solution}
    考虑椭圆$D:\dfrac{x^2}{16}+\dfrac{y^2}{9}\leqslant1$和圆$D_\ep:x^2+y^2\leqslant\ep^2(\ep>0)$.当$\ep<3$时,$D_\ep\subseteq D$.设$C_\ep$为$D\backslash D_\ep$的正向边界.\\
    在$D\backslash D_\ep$上应用Green公式,则有
    \[\oint_{C+C_\ep}\dfrac{x\di y-y\di x}{x^2+y^2}
    =\iint_{D\backslash D_\ep}\dfrac{\left(x^2+y^2-2x^2\right)+\left(x^2+y^2-2y^2\right)}{\left(x^2+y^2\right)^2}\di\sigma=0\]
    于是
    \[\begin{aligned}
        \oint_C\dfrac{x\di y-y\di x}{x^2+y^2}
        &= -\oint_{C_\ep}\dfrac{x\di y-y\di x}{x^2+y^2} \\
        &\xlongequal{x=\ep\cos t,y=\ep\sin t} \int_0^{2\pi}\dfrac{\ep^2\cos^2t+\ep^2\sin^2t}{\ep^2}\di t \\
        &= \int_0^{2\pi}\di t \\
        &= 2\pi
    \end{aligned}\]
\end{solution}
\begin{problem}[4.(10\songti{分})]
    计算曲面积分
    \[\iint_{S}\left(x^2y^2+y^2z^2+z^2x^2\right)\di S\]
    其中$S$为圆锥面$z=\sqrt{x^2+y^2}$被柱面$x^2+y^2=1$所截下部分.
\end{problem}
\begin{solution}
    曲面方程为$z=g(x,y)=\sqrt{x^2+y^2},0\leqslant x^2+y^2\leqslant 1$.于是
    \[\sqrt{1+g_x^2+g_y^2}=\sqrt{1+\left(\dfrac{x}{\sqrt{x^2+y^2}}\right)^2+\left(\dfrac{y}{\sqrt{x^2+y^2}}\right)^2}=\sqrt2\]
    于是
    \[\begin{aligned}
        \iint_{S}\left(x^2y^2+y^2z^2+z^2x^2\right)\di S
        &= \iint_{S}\left(x^2+y^2+\left(x^2+y^2\right)^2\right)\sqrt2\di x\di y \\
        &= \sqrt2\int_0^{2\pi}\di\theta\int_0^1\left(r^2+r^4\right)r\di r \\
        &= \dfrac{5\sqrt2\pi}{6}
    \end{aligned}\]
\end{solution}
\begin{problem}[5.(15\songti{分})]
    计算曲面积分
    \[\oiint_Sx\di y\di z+y\di z\di x+z\di x\di y\]
    其中$S$为抛物面$z=x^2+y^2$被平面$z=4$所截部分的外侧.
\end{problem}
\begin{solution}
    考虑空间区域$\Omega:0\leqslant x^2+y^2\leqslant z\leqslant 4$,则$\Omega$的外侧为$S$和圆$D:0\leqslant x^2+y^2\leqslant 4,z=4$.\\
    在$\Omega$上应用Gauss公式,则有
    \[\begin{aligned}
        \oiint_{S+D}x\di y\di z+y\di z\di x+z\di x\di y
        &= \iiint_{\Omega}3\di V \\
        &= 3\int_0^{2\pi}\di\theta\int_0^2\di r\int_{r^2}^4r\di z \\
        &= 3\int_0^{2\pi}\di\theta\int_0^2\left(4r-r^3\right)\di r \\
        &= 24\pi
    \end{aligned}\]
    而$D$的单位外法向量为$(0,0,1)$,于是
    \[\begin{aligned}
        \oiint_Dx\di y\di z+y\di z\di x+z\di x\di y
        &= \oiint_D\left(x,y,z\right)\cdot\left(0,0,1\right)\di S \\
        &= \iint_Dz\di S \\
        &= 16\pi
    \end{aligned}\]
    于是
    \[\oiint_Sx\di y\di z+y\di z\di x+z\di x\di y=24\pi-16\pi=8\pi\]
\end{solution}
\begin{problem}[6.(10\songti{分})]
    求常微分方程
    \[y'=xy+3x+2y+6\]
    的所有解.
\end{problem}
\begin{solution}
    做换元$v=x+2,u=y+3$,于是$\dfrac{\di u}{\di v}=\dfrac{\di y}{\di x}=y'$.于是原微分方程即为
    \[\dfrac{\di u}{\di v}=uv\]
    如果$u\equiv0$,那么显然这是该微分方程的解.\\
    如果$u\neq0$,则有
    \[\dfrac{\di u}{u}=v\di v\]
    两边积分可得
    \[\ln\left|u\right|=\dfrac12v^2+C\]
    即
    \[u=C\e^{\frac12v^2}\ \ \ \left(C\neq0\right)\]
    合并两种情况并回代可得原方程的解为
    \[y=C\e^{\frac{(x+2)^2}{2}}-3\ \ \ (C\in\R)\]

\end{solution}
\begin{problem}[7.(15\songti{分})]
    求常微分方程
    \[y''-4y'+3y-4\e^x=0\]
    的通解.
\end{problem}
\begin{solution}
    该常微分方程对应的齐次方程为
    \[\lambda^2-4\lambda+3=0\]
    于是特征根为
    \[\lambda_1=1\ \ \ \lambda_2=3\]
    于是设
    \[y=C_1(x)\e^x+C_2(x)\e^{3x}\]
    代入原方程则有
    \[\left\{\begin{array}{l}
        C_1'(x)\e^x+C_2'(x)\e^{3x}=0\\
        C_1'(x)\e^x+3C_2'(x)\e^{3x}=4\e^x
    \end{array}\right.\]
    解得
    \[C_1'(x)=-2\ \ \ C_2'(x)=2\e^{-2x}\]
    于是
    \[C_1(x)=-2x+C_1\ \ \ C_2(x)=-\e^{-2x}+C_2\]
    于是原微分方程的通解为
    \[y=C_1\e^x+C_2\e^{3x}-\e^x(1+2x)\]
\end{solution}
\begin{problem}[8.(10\songti{分})]
    设平面有界闭区域为
    \[D=\left\{(x,y)\in\R^2:\dfrac{x^2}{a^2}+\dfrac{y^2}{b^2}\leqslant1\right\}\ \ \ \ \ a,b>0\]
    设曲线$L$为$D$的边界,函数$P(x,y),Q(x,y)$在$D$上有连续的一阶偏导数.记$\mbf{F}=(P,Q)$,$\mbf{n}$为曲线$L$的单位外法向量.试证明
    \[\oint_{L^+}\mbf{F}\cdot\mbf{n}\di s=\iint_{D}\left(\dfrac{\p P}{\p x}+\dfrac{\p Q}{\p y}\right)\di x\di y\]
\end{problem}
\begin{proof}
    由于$\mbf n=\left(n_x,n_y\right)$是$L$的单位外法向量,于是$L^+$的单位切向量为$\left(-n_y,n_x\right)$.于是
    \[\oint_{L^+}\mbf F\cdot\mbf n\di s
    =\oint_{L^+}(P,Q)\cdot\left(n_x,n_y\right)\di s
    =\oint_{L^+}(-Q,P)\cdot\left(-n_y,n_x\right)\di s
    =\oint_{L^+}-Q\di x+P\di y\]
    在$D$上运用Green公式可得
    \[\oint_{L^+}-Q\di x+P\di y=\iint_D\left(\dfrac{\p P}{\p x}+\dfrac{\p Q}{\p y}\right)\di x\di y\]
    于是
    \[\oint_{L^+}\mbf{F}\cdot\mbf{n}\di s=\iint_{D}\left(\dfrac{\p P}{\p x}+\dfrac{\p Q}{\p y}\right)\di x\di y\]
    命题得证.
\end{proof}
\begin{problem}[9.(10\songti{分})]
    设$f(x)$为$\R$上的连续函数.试证明
    \[\iint_{S}f(x+y+z)\di S=2\pi\int_{-1}^{1}f\left(\sqrt{3}\xi\right)\di\xi\]
    其中$S$为单位球面$x^2+y^2+z^2=1$.
\end{problem}
\begin{proof}
    对球面上的点$(x,y,z)$做正交变换$T:(x,y,z)\mapsto(u,v,w)$.设$T$关于$\R^3$的规范正交基的矩阵为$A$.\\
    现在,令$u=\dfrac{x+y+z}{\sqrt3}$,则
    \[\begin{pmatrix}x\\y\\z\end{pmatrix}
    =\begin{pmatrix}
        \frac{1}{\sqrt3}&\frac{1}{\sqrt3}&\frac{1}{\sqrt3}\\
        A_{2,1}&A_{2,2}&A_{2,3}\\
        A_{3,1}&A_{3,2}&A_{3,3}
    \end{pmatrix}
    \begin{pmatrix}u\\v\\w\end{pmatrix}\]
    根据Schmidt正交化过程可以求解出矩阵$A$.\\
    由于正交变换后依然保持$u^2+v^2+w^2=1$,我们对变换后的坐标进行换元可得
    \[\left\{\begin{array}{l}
        u=\xi \\
        v=\sqrt{1-\xi^2}\cos\theta \\
        w=\sqrt{1-\xi^2}\sin\theta
    \end{array}\right.\]
    于是
    \[\begin{pmatrix}x_\theta\\y_\theta\\z_\theta\end{pmatrix}
    =A^{-1}\begin{pmatrix}
        u_\theta\\v_\theta\\w_\theta
    \end{pmatrix}
    =A^{-1}\begin{pmatrix}
        0\\-\sqrt{1-\xi^2}\sin\theta\\\sqrt{1-\xi^2}\cos\theta
    \end{pmatrix}\]
    以及
    \[\begin{pmatrix}x_\xi\\y_\xi\\z_\xi\end{pmatrix}
    =A^{-1}\begin{pmatrix}
        u_\xi\\v_\xi\\w_\xi
    \end{pmatrix}
    =A^{-1}\begin{pmatrix}
        1\\-\dfrac{\xi}{\sqrt{1-\xi^2}}\cos\theta\\-\dfrac{\xi}{\sqrt{1-\xi^2}}\sin\theta
    \end{pmatrix}\]
    于是
    \[E=\begin{pmatrix}x_\theta&y_\theta&z_\theta\end{pmatrix}
    \begin{pmatrix}x_\theta\\y_\theta\\z_\theta\end{pmatrix}=1-\xi^2\]
    \[F=\begin{pmatrix}x_\theta&y_\theta&z_\theta\end{pmatrix}
    \begin{pmatrix}x_\xi\\y_\xi\\z_\xi\end{pmatrix}=0\]
    \[G=\begin{pmatrix}x_\xi&y_\xi&z_\xi\end{pmatrix}
    \begin{pmatrix}x_\xi\\y_\xi\\z_\xi\end{pmatrix}=\dfrac{1}{1-\xi^2}\]
    变换后的积分区域为$D:-1\leqslant\xi\leqslant 1,0\leqslant\theta\leqslant2\pi$.于是
    \[\begin{aligned}
        \iint_Sf(x+y+z)\di S
        &= \iint_Df\left(\sqrt3\xi\right)\sqrt{EG-F^2}\di\sigma \\
        &= \int_0^{2\pi}\di\theta\int_{-1}^{1}f\left(\sqrt3\xi\right)\di\xi \\
        &= 2\pi\int_{-1}^{1}f\left(\sqrt3\xi\right)\di\xi
    \end{aligned}\]
\end{proof}
\end{document}