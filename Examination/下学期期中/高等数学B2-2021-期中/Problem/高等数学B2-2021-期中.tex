\documentclass{ctexart}
\usepackage{template}
\usepackage{esint}

\begin{document}\pagestyle{empty}
\begin{center}\Large
    北京大学数学科学学院2021-22高等数学B2期中考试
\end{center}
\begin{enumerate}[leftmargin=*,label=\textbf{\arabic*.}]
    \item \textbf{(10\songti{分})}\ 计算二重积分
        \[\iint_{D}\ln\left(1+x^2+y^2\right)\dx\di y\ \ \ \ \ D:x^2+y^2\leqslant1,x\geqslant0,y\geqslant0\]
    \item \textbf{(10\songti{分})}\ 计算三重积分
        \[\iiint_{\Omega}\left(y^2+z^2\right)\ \ \ \ \ \Omega:0\leqslant z\leqslant x^2+y^2\leqslant1\]
    \item \textbf{(10\songti{分})}\ 设曲线$C$为椭圆$\frac{x^2}{16}+\dfrac{y^2}{9}=1$沿逆时针方向.计算曲线积分
        \[\oint_{C}\dfrac{x\di y-y\di x}{x^2+y^2}\]
    \item \textbf{(10\songti{分})}\ 计算曲面积分
        \[\iint_{S}\left(x^2y^2+y^2z^2+z^2x^2\right)\di S\]
        其中$S$为圆锥面$z=\sqrt{x^2+y^2}$被柱面$x^2+y^2=1$所截下部分.
    \item \textbf{(15\songti{分})}\ 计算曲面积分
        \[\oiint_Sx\di y\di z+y\di z\di x+z\di x\di y\]
        其中$S$为抛物面$z=x^2+y^2$被平面$z=4$所截部分的外侧.
    \item \textbf{(10\songti{分})}\ 求常微分方程
        \[y'=xy+3x+2y+6\]
        的所有解.
    \item \textbf{(15\songti{分})}\ 求常微分方程
        \[y''-4y'+3y-4\e^x=0\]
        的通解.
    \item \textbf{(10\songti{分})}\ 设平面有界闭区域为
        \[D=\left\{(x,y)\in\R^2:\dfrac{x^2}{a^2}+\dfrac{y^2}{b^2}\leqslant\right\}\ \ \ \ \ a,b>0\]
        设曲线$L$为$D$的边界,函数$P(x,y),Q(x,y)$在$D$上有连续的一阶偏导数.记$\mbf{F}=(P,Q)$,$\mbf{n}$为曲线$L$的单位外法向量.试证明
        \[\oint_{L^+}\mbf{F}\cdot\mbf{n}\di s=\iint_{D}\left(\dfrac{\p P}{\p x}+\dfrac{\p Q}{\p y}\right)\di x\di y\]
    \item \textbf{(10\songti{分})}\ 设$f(x)$为$\R$上的连续函数.试证明
        \[\iint_{S}f(x+y+z)\di S=2\pi\int_{-1}^{1}f\left(\sqrt{3}\xi\right)\di\xi\]
        其中$S$为单位球面$x^2+y^2+z^2=1$.
\end{enumerate}
\end{document}