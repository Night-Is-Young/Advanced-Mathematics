\documentclass{ctexart}
\usepackage{geometry}
\usepackage[dvipsnames,svgnames]{xcolor}
\usepackage[strict]{changepage}
\usepackage{framed}
\usepackage{enumerate}
\usepackage{amsmath,amsthm,amssymb}
\usepackage{enumitem}
\usepackage{template}

\allowdisplaybreaks
\geometry{left=2cm, right=2cm, top=2.5cm, bottom=2.5cm}

\begin{document}\pagestyle{empty}

\begin{center}\Large
    北京大学数学科学学院2020-21高等数学B1期末考试
\end{center}
\begin{enumerate}[leftmargin=*,label=\textbf{\arabic*.}]
    \item \textbf{(15\songti{分})}\ 下面函数的极限存在吗?若存在,请求出其值;若不存在,请说明理由.
        \begin{enumerate}[label=\textbf{(\arabic*)},leftmargin=*]
            \item \textbf{(5\songti{分})}\ $\displaystyle\lim_{x\to0}\dfrac{2\cos x-2+x^2}{x^4}$.
            \item \textbf{(5\songti{分})}\ $\displaystyle\lim_{(x,y)\to(0,0)}\dfrac{x^5y^3}{x^8+y^8}$.
            \item \textbf{(5\songti{分})}\ $\displaystyle\lim_{(x,y)\to(0,0)}(x+\sin y)\cos\dfrac{1}{|x|+|y|}$.
        \end{enumerate}
    \item \textbf{(15\songti{分})}\ 求闭区间$[-1,1]$上的函数$f(x)=x^{\frac23}-\left(x^2-1\right)^\frac13$的所有最小值点.
    \item \textbf{(20\songti{分})}\ 回答下列问题.
        \begin{enumerate}[label=\textbf{(\arabic*)},leftmargin=*]
            \item \textbf{(15\songti{分})}\ 设$a,b\in\R$且$b\neq0$.求$f(x,y)=\arctan\dfrac xy$在$(a,b)$处的二阶泰勒多项式.
            \item \textbf{(5\songti{分})}\ 设$a<b$且$n\in\N^*$.函数$f:(a,b)\to\R$在开区间$(a,b)$中有$n+1$阶导数.定义二元函数$T:(a,b)\times(a,b)\to\R$为
                \[T(x,y)=f(x)-f(y)-\sum_{k=1}^n\dfrac{f^{(k)}(y)}{k!}(x-y)^k\]
                求出$T(x,y)$对$y$的一阶偏导函数$\dfrac{\p T}{\p y}$.
        \end{enumerate}
    \item \textbf{(10\songti{分})}\ 证明:对任意给定的实数$p$,存在$1$的开邻域$U$和$W$使得存在唯一的函数$y=f(x):U\to W$满足$x^p+y^p-2xy=0$.
    \item \textbf{(15\songti{分})}\ 设在$\R^3$空间中$Oxy$平面之外的点$(x,y,z)$处的电势$V=\left(\dfrac{2y}z\right)^x$.求出在点$\left(1,\dfrac12,1\right)$处电势$V$下降最快的方向上的单位向量.
    \item \textbf{(25\songti{分})}\ 设$\R^3$空间中的平面$K:x+2y+3z=6$与$x,y,z$三轴分别交于$A,B,C$三点.动点$H\in\R^3$与$K$的距离恒为$1$,其在$K$上的垂直投影记为$M$.设$M$在$\triangle ABC$中,其到三条边$BC,CA,AB$的距离分别为$p,q,r$.
            \begin{enumerate}[label=\textbf{(\arabic*)},leftmargin=*]
                \item \textbf{(5\songti{分})}\ 求出$\triangle ABC$的面积.
                \item \textbf{(5\songti{分})}\ 用$p,q,r$表示以$A,B,C,H$为顶点的四面体的表面积$S(p,q,r)$.
                \item \textbf{(5\songti{分})}\ 写出$p,q,r$必须满足的约束条件.
                \item \textbf{(10\songti{分})}\ 求出$S(p,q,r)$的条件极值的稳定点.
            \end{enumerate}
\end{enumerate}
\end{document}