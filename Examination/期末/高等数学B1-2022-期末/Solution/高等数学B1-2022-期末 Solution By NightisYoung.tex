\documentclass{ctexart}
\usepackage{geometry}
\usepackage[dvipsnames,svgnames]{xcolor}
\usepackage[strict]{changepage}
\usepackage{framed}
\usepackage{enumerate}
\usepackage{amsmath,amsthm,amssymb}
\usepackage{enumitem}
\usepackage{template}

\geometry{left=2cm, right=2cm, top=2.5cm, bottom=2.5cm}

\begin{document}\pagestyle{empty}
\begin{center}\Large
    北京大学数学科学学院2022-23高等数学B1期末考试
\end{center}
\begin{problem}[1.(10\songti{分})]
    设$\R^3$中平面$x+3y+2z=6$与$x$轴交点为$A$,与$y$轴交点为$B$,与$z$轴交点为$C$.
    \begin{enumerate}[label=\textbf{(\arabic*)},leftmargin=*]
        \item \textbf{(5\songti{分})}\ 求$\triangle ABC$的面积.
        \item \textbf{(5\songti{分})}\ 求过四点$A,B,C,O(0,0)$的球面的方程.
    \end{enumerate}
\end{problem}
\begin{solution}[Solution.]
    \begin{enumerate}[label=\textbf{(\arabic*)},leftmargin=*]
        \item 分别令$x,y,z$三者中两者为$0$可解得$A(6,0,0),B(0,2,0),C(0,0,3)$.于是
            $$a=|BC|=\sqrt{13},b=|AC|=3\sqrt{5},c=|AB|=2\sqrt{10}$$
            于是$\cos C=\dfrac{a^2+b^2-c^2}{2ab}=\dfrac{18}{6\sqrt{65}}=\dfrac{3}{\sqrt{65}}$,则$\sin C=\sqrt{1-\cos^2C}=\dfrac{2\sqrt{14}}{\sqrt{65}}$.\\
            于是$S_{\triangle ABC}=\dfrac{1}{2}ab\sin C=3\sqrt{14}$.
        \item 设球心$Q(x,y,z)$.根据$|QO|=|QA|=|QB|=|QC|$有
            $$\left\{\begin{array}{l}
                (x-6)^2+y^2+z^2=x^2+y^2+z^2\\
                x^2+(y-2)^2+z^2=x^2+y^2+z^2\\
                x^2+y^2+(z-3)^2=x^2+y^2+z^2
            \end{array}\right.$$
            解得$x=3,y=1,z=\dfrac{3}{2}$.于是球面方程为$\Gamma:(x-3)^2+(y-1)^2+\left(z-\dfrac{3}{2}\right)^2=\dfrac{49}{4}$.
    \end{enumerate}
\end{solution}
\begin{problem}[2.(15\songti{分})]
    下面的二元函数的极限存在吗?如果存在,请求出其值;如果不存在,请说明理由.
        \begin{enumerate}[label=\textbf{(\arabic*)},leftmargin=*]
            \item \textbf{(5\songti{分})}\ $\displaystyle\lim_{(x,y)\to(0,0)}\dfrac{24\cos\sqrt{x^2+y^2}-24+12\left(x^2+y^2\right)}{\left(\tan\sqrt{x^2+y^2}\right)^4}$.
            \item \textbf{(5\songti{分})}\ $\displaystyle\lim_{(x,y)\to(0,0)}\left(x+\ln(1+y)\right)\cos\dfrac{1}{x^2+y^2}$.
            \item \textbf{(5\songti{分})}\ $\displaystyle\lim_{(x,y)\to(0,0)}\dfrac{x\sin y}{\sin^2x+\sin^2y}$.
        \end{enumerate}
\end{problem}
\begin{solution}[Solution.]
    \begin{enumerate}[label=\textbf{(\arabic*)},leftmargin=*]
        \item 置$z=\sqrt{x^2+y^2}$,于是$(x,y)\to(0,0)$有$z\to 0^+$.于是
            $$\begin{aligned}
                \lim_{(x,y)\to(0,0)}\dfrac{24\cos\sqrt{x^2+y^2}-24+12\left(x^2+y^2\right)}{\left(\tan\sqrt{x^2+y^2}\right)^4}
                &= \lim_{z\to 0^+}\dfrac{24\cos z-24+12z^2}{\tan^4 z} \\
                &= \lim_{z\to 0^+}\dfrac{24\cos z-24+12z^2}{z^4}\cdot\lim_{z\to 0^+}\dfrac{z^4}{\tan^4 z} \\
                &= \lim_{z\to 0^+}\dfrac{-24\sin z+24z}{4z^3}\cdot(1)^4\\
                &= \lim_{z\to 0^+}\dfrac{24-24\cos z}{12z^2} \\
                &= \lim_{z\to 0^+}\dfrac{24\sin z}{24z}\\
                &= 1
            \end{aligned}$$
        \item 我们有$$0<\left|\left(x+\ln(1+y)\right)\cos\dfrac{1}{x^2+y^2}\right|<\left|x+\ln(1+y)\right|<\left|x\right|+\left|\ln(1+y)\right|<\left|x\right|+\left|y\right|$$
            而$\displaystyle\lim_{(x,y)\to(0,0)}|x|+|y|=0+0=0$.\\
            据夹逼定理可知$\displaystyle\lim_{(x,y)\to(0,0)}\left(x+\ln(1+y)\right)\cos\dfrac{1}{x^2+y^2}=0$.
        \item 令$y=kx$,其中$k\neq 0$.于是
            $$\begin{aligned}
                \lim_{(x,y)\to(0,0)}\dfrac{x\sin y}{\sin^2x+\sin^2y}
                &= \lim_{x\to0}\dfrac{x\sin kx}{\sin^2x+\sin^2kx} \\
                &= \lim_{x\to0}\dfrac{k\left(\dfrac{\sin kx}{kx}\right)}{\left(\dfrac{\sin x}{x}\right)^2+k^2\left(\dfrac{\sin kx}{kx}\right)^2} \\
                &= \dfrac{k}{1+k^2}
            \end{aligned}$$
            于是从不同路径接近$(0,0)$时取得的极限不同,因而原极限不存在.
    \end{enumerate}
\end{solution}
\begin{problem}[3.(10\songti{分})]
    设$f,g:\R\to\R$都有连续的二阶导数.对于任意$x,y\in\R$,
    定义$h(x,y)=xg\left(\dfrac{y}{x}\right)+f\left(\dfrac{y}{x}\right)$,
    试计算$x^2h_{xx}(x,y)+2xyh_{yx}(x,y)+y^2h_{yy}(x,y)$.
\end{problem}
\begin{solution}[Solution.]
    令$u=\dfrac{y}{x}$,则$h(x,u)=xg(u)+f(u)$.于是
    $$\dfrac{\p h}{\p x}=\dfrac{\p h}{\p u}\cdot\dfrac{\p u}{\p x}=\left(\right)$$
\end{solution}
\begin{problem}[4.(10\songti{分})]
    求$\R^2$中曲线$\e^{xy}+xy+y^2=2$在$(0,1)$处的切线方程.
\end{problem}
\begin{solution}[Solution.]
    对原式求微分可得
    $$x\e^{xy}\di y+y\e^{xy}\dx+x\di y+y\dx+2y\di y=0$$
    移项整理有$$\left(x\e^{xy}+x+2y\right)\di y=-\left(y+y\e^{xy}\right)\dx$$
    于是$$\dfrac{\di y}{\dx}=-\dfrac{y(1+\e^{xy})}{x(1+\e^{xy})+2y}$$
    代入$x=0,y=1$有$\dfrac{\di y}{\dx}=-\dfrac{1+\e}{2}$.\\
    于是切线方程为$y=-\dfrac{1+\e}{2}x+1$.
\end{solution}
\begin{problem}[5.(10\songti{分})]
    设三元函数$f(x,y,z)=\left(\dfrac{2x}{z}\right)^{y},z\neq0$.求$f$在点$\left(\dfrac{1}{2},1,1\right)$处下降最快的方向上的单位向量.
\end{problem}
\begin{solution}[Solution.]
    由题意$$\dfrac{\p f}{\p x}=\left(\dfrac{2}{z}\right)^yyx^{y-1}\ \ \ \ \ 
    \dfrac{\p f}{\p y}=\left(\dfrac{2x}{z}\right)^y\ln\left(\dfrac{2x}{z}\right)\ \ \ \ \
    \dfrac{\p f}{\p z}=-(2x)^yyz^{-y-1}$$
    将$x=\dfrac12,y=1,z=1$代入可知$\dfrac{\p f}{\p x}=2,\dfrac{\p f}{\p y}=0,\dfrac{\p f}{\p z}=-1$.于是$\tbf{grad}f=\left(2,0,-1\right)$为$f$在该点处的梯度.\\
    于是下降最快的方向与负梯度的方向相同,此方向的单位向量为$\left(-\dfrac{2\sqrt5}{5},0,\dfrac{\sqrt5}{5}\right)$.
\end{solution}
\begin{problem}[6.(10\songti{分})]
    求二元函数$f(x,y)=\arctan\dfrac{y}{x}$在点$(2,2)$处的二阶泰勒多项式.
\end{problem}
\begin{solution}[Solution.]
    我们有\\
    $$\dfrac{\p f}{\p x}=\dfrac{1}{1+\left(\dfrac{y}{x}\right)^2}\cdot\left(-\dfrac{y}{x^2}\right)=\dfrac{-y}{x^2+y^2}$$
    $$\dfrac{\p f}{\p y}=\dfrac{1}{1+\left(\dfrac{y}{x}\right)^2}\cdot\dfrac{1}{x}=\dfrac{x}{x^2+y^2}$$
    $$\dfrac{\p^2f}{\p x^2}=\dfrac{2xy}{(x^2+y^2)^2}\ \ \ \ \ \dfrac{\p^2f}{\p y^2}=\dfrac{-2xy}{(x^2+y^2)^2}$$
    $$\dfrac{\p^2f}{\p x\p y}=\dfrac{-(x^2+y^2)-2y(-y)}{(x^2+y^2)^2}=\dfrac{y^2-x^2}{(x^2+y^2)^2}$$
    于是
    \[\begin{aligned}
        f(x,y)
        &= \dfrac{\pi}{4}-\dfrac14(x-2)+\dfrac14(y-2)+\dfrac{1}{16}(x-2)^2-\dfrac{1}{16}(y-2)^2
    \end{aligned}\]
\end{solution}
\begin{problem}[7.(10\songti{分})]
    求函数$f(x)=\left(\sin x\right)^{\frac{2}{3}}+\left(\cos x\right)^{\frac{2}{3}}$在闭区间$\left[-\dfrac{\pi}{2},\dfrac{\pi}{2}\right]$上的最小值,并指明所有最小值点.
\end{problem}
\begin{solution}[Solution.]
    注意到$f(x)$是偶函数.因此,考虑$f(x)$在$\left[0,-\dfrac{\pi}{2}\right]$上的部分.
    \[\begin{aligned}
        \dfrac{\di f}{\dx}
        &= \dfrac{2}{3}(\sin x)^{-\frac{1}{3}}\cos x-\dfrac{2}{3}(\cos x)^{-\frac{1}{3}}\sin x \\
        &= \dfrac{2}{3}\left(\sin x\cos x\right)^{-\frac13}\left((\cos x)^{\frac{4}{3}}-(\sin x)^{\frac{4}{3}}\right)
    \end{aligned}\]
    令$\dfrac{\di f}{\dx}=0$,解得$x=0,\dfrac\pi4$或$\dfrac\pi2$.又$f(0)=1,f\left(\dfrac\pi4\right)=2\sqrt[3]{\dfrac12}=\sqrt[3]4>1,f(\dfrac{\pi}{2})=1$.\\
    于是$f(x)$的最小值为$1$,最小值点为$\pm\dfrac\pi2,0$.
\end{solution}
\begin{problem}[8.(10\songti{分})]
    证明:对于任意给定的$k\in\R$,存在$0$的开邻域$U$和$W$,存在唯一的函数$y=f(x),x\in U,y\in W$满足方程$\e^{kx}+\e^{ky}-2\e^{x+y}=0$.
\end{problem}
\begin{proof}
    设$F(x,y)=\e^{kx}+\e^{ky}-2\e^{x+y}$.注意到$F(0)=0$.\\
    又$\dfrac{\p F}{\p x}=k\e^{kx}-2\e^{x+y},\dfrac{\p F}{\p y}=k\e^{ky}-2\e^{x+y}$.于是$F(x,y)$的一阶偏导是连续的.\\
    注意到$F_y(0,0)=k-2$.\\
    若$k\leqslant0$,则对任意$x,y\in\R$都有$F_y(x,y)<0$.根据隐函数存在定理可知存在唯一$y=f(x)$使得$F(x,y)\equiv0$.\\
    若$k=2$,则$F(x,y)=0$等价于$\left(\e^x-\e^y\right)^2=0$,当且仅当$y=x$时成立.于是这函数为$f(x)=x$.\\
    若$k>0$且$k\neq2$,则$F_y(x,y)=0$可解得$y=\dfrac{\ln 2-\ln k}{k-1}$.\\
    于是取$y$的开邻域$W=\left\{y\in\R:|y|<\dfrac{\ln 2-\ln k}{k-1}\right\}$.根据隐函数存在定理可知存在唯一$y=f(x)$使得$F(x,y)\equiv0$.
\end{proof}
\begin{problem}[9.(15\songti{分})]
    设$r$是正实数,$D=\left\{(x,y)\vert\sqrt{x^2+y^2}<r\right\}$,函数$f:D\to\R$满足$f\in C^3(D),f(0,0)=0$,
    $f$在点$(0,0)$处的一阶全微分$\di f(0,0)=0$.$f$在点$(0,0)$处的二阶全微分满足
    $$\di^2f(0,0)=E\left(\Delta x\right)^2+2F\Delta x\Delta y+G\left(\Delta y\right)^2$$
    其中$E,F,G$均为常数.
    \begin{enumerate}[label=\textbf{(\arabic*)},leftmargin=*]
        \item \textbf{(10\songti{分})}\ 证明:存在$D$上的两个函数$a,b:D\to\R$使得$\forall (x,y)\in D$有$$f(x,y)=xa(x,y)+yb(x,y),a(0,0)=b(0,0)=0$$
        \item \textbf{(5\songti{分})}\ 若$E>0,EG-F^2<0$,则在$\R^3$中点$(0,0,0)$的充分小邻域中,曲面$z=f(x,y)$充分近似于哪一类二次曲面?画出此类二次曲面的草图.
            从此类二次曲面的几何形状判断是否存在$\R^2$中点$(0,0)$的充分小邻域$D_1$,存在$D_1$上的一一对应的$C^1$变量变换$x=x(u,v),y=y(u,v)$使得
            $$f(x(u,v),y(u,v))=u^2-v^2$$
    \end{enumerate}
\end{problem}
\begin{solution}
    \begin{enumerate}[label=\textbf{(\arabic*)},leftmargin=*]
        \item 给定$(x,y)\in D$,记$\phi(t)=f(tx,ty)$,其中$t\in[0,1]$.由于$f$在$D$上可微,因而$\phi(t)$可微.\\
            现在,令
            \[a(x,y)=\int_{0}^{1}f_x(tx,ty)\di t\ \ \ \ \ b(x,y)=\int_{0}^{1}f_y(tx,ty)\di t\]
            注意到
            \[\int_0^1\phi'(t)\di t=\phi(1)-\phi(0)=f(x,y)\]
            而
            \[\phi'(t)=xf_x(tx,ty)+yf_y(tx,ty)\]
            于是
            \[xa(x,y)+yb(x,y)=\int_0^1 xf_x(tx,ty)\di t+\int_0^1 yf_y(tx,ty)\di t=\int_0^1\phi'(t)\di t=f(x,y)\]
            又$\di f(0,0)=0$,于是$a(0,0)=b(0,0)=0$.于是这样的$a,b$满足题意.
        \item 考虑$f(x,y)$在$(0,0)$处的二阶泰勒多项式
            \[f(x,y)=f(0,0)+xf_x(0,0)+yf_y(0,0)+\dfrac{x^2f_{xx}(0,0)+2xyf_{xy}(0,0)+y^2f_{yy}(0,0)}{2}+o(\rho^2)\]
            其中$\rho=\sqrt{x^2+y^2}$.代入题中的$f$的各阶微分有
            \[f(x,y)=\dfrac{1}{2}\left(Ex^2+2Fxy+Gy^2\right)\]
            考虑旋转变换$S:(x,y)\to(x\sin\theta+y\cos\theta,x\cos\theta-y\sin\theta)$将$(x,y)$绕原点逆时针旋转$\theta$.\\
            不妨令$x'=x\sin\theta+y\cos\theta,y'=x\cos\theta-y\sin\theta$.于是
            \[x'^2=x^2\sin^2\theta+y^2\cos^2\theta+2xy\sin\theta\cos\theta\ \ \ \ \ y'^2=x^2\cos^2\theta+y^2\sin^2\theta-2xy\sin\theta\cos\theta\]
            设$Ex^2+2Fxy+Gy^2=Ax'^2+By'^2$,于是有
            \[\left\{\begin{array}{l}
                A\sin^2\theta+B\cos^2\theta=E\\
                A\cos^2\theta+B\sin^2\theta=G\\
                (A-B)\sin\theta\cos\theta=F
            \end{array}\right.\]
            于是
            \[\begin{aligned}
                EG-F^2
                &= (A^2+B^2)\sin^2\theta\cos^2\theta+AB(\sin^4\theta+\cos^4\theta)-(A^2-2AB+B^2)\sin^2\theta\cos^2\theta \\
                &= AB(\sin^4\theta+\cos^4\theta)+2AB\sin^2\theta\cos^2\theta \\
                &= AB
            \end{aligned}\]
            由于$EG-F^2<0$,于是$AB<0$,因而这是双曲抛物面.题中所指的变换即旋转后伸缩变换,是成立的.
    \end{enumerate}
\end{solution}
\end{document}