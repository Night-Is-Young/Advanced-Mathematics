\documentclass{ctexart}
\usepackage{geometry}
\usepackage[dvipsnames,svgnames]{xcolor}
\usepackage[strict]{changepage}
\usepackage{framed}
\usepackage{enumerate}
\usepackage{amsmath,amsthm,amssymb}
\usepackage{enumitem}
\usepackage{template}

\geometry{left=2cm, right=2cm, top=2.5cm, bottom=2.5cm}

\begin{document}\pagestyle{empty}
\begin{center}\Large
    北京大学数学科学学院2023-24高等数学A1期末考试
\end{center}
\begin{problem}[1.(20\songti{分})]
    求下列函数的极限.
    \begin{enumerate}[label=\tbf{(\arabic*)}]
        \item \textbf{(10\songti{分})}\ 求极限$\displaystyle\lim_{x\to0}\dfrac1x\left(\left(1+x\right)^{\frac1x}-\e\right)$.
        \item \textbf{(10\songti{分})}\ 设函数$f(x)$在$x=0$处$n+1$阶可导,且满足\[f(0)=f'(0)=\cdots=f^{(n-1)}=0\ \ \ \ \ f^{(n)}(0)=a\]求极限$\displaystyle\lim_{x\to0}\dfrac{f\left(\e^x-1\right)-f(x)}{x^{n+1}}$.
    \end{enumerate}
\end{problem}
\begin{solution}
    \begin{enumerate}[label=\tbf{(\arabic*)}]
        \item 令$u=\left(1+x\right)^{\frac1x}$,于是$\ln u=\dfrac{\ln(1+x)}{x}$,于是
            \[\dfrac{\di u}{\dx}=\dfrac{\di u}{\di\ln u}\cdot\dfrac{\di\ln u}{\dx}=\left(1+x\right)^{\frac1x}\dfrac{x-(x+1)\ln(x+1)}{x^2(x+1)}\]
            \[\begin{aligned}
                \lim_{x\to0}\dfrac1x\left(\left(1+x\right)^{\frac1x}-\e\right)
                &= \lim_{x\to0}\left(1+x\right)^{\frac1x}\cdot\lim_{x\to0}\dfrac{x-(x+1)\ln(x+1)}{x^2(x+1)} \\
                &= \e\cdot\lim_{x\to0}\dfrac{\frac{1}{(x+1)^2}-\frac{1}{1+x}}{2x} \\
                &= \e\cdot\lim_{x\to0}\dfrac{1-(1+x)}{2x(x+1)^2} \\
                &= -\dfrac{\e}{2}
            \end{aligned}\]
        \item 考虑$f(x)$在$x=0$处的泰勒展开.
            \[f(x)=f(0)+\sum_{k=1}^{n}\dfrac{f^{(k)}(0)}{k!}x^k+o(x^n)=\dfrac{ax^n}{n!}+o(x^n)\]
            又$\displaystyle\lim_{x\to0}\dfrac{\e^x-1}{x}=1$,于是$\e^x-1$与$x$是同阶无穷小量.\\
            于是有
            \[\begin{aligned}
                \lim_{x\to0}\dfrac{f\left(\e^x-1\right)-f(x)}{x^{n+1}}
                &= \lim_{x\to0}\dfrac{}{}
            \end{aligned}\]
    \end{enumerate}
\end{solution}
\begin{problem}[2.(20\songti{分})]
    回答下列问题.
    \begin{enumerate}[label=\tbf{(\arabic*)}]
        \item \textbf{(10\songti{分})}\ 设函数$F(u,v)$有连续的二阶偏导数,$z=z(x,y)$是由方程$F(x-z,y-z)=0$确定的隐函数.计算并化简
            \[\dfrac{\p^2z}{\p x^2}+\dfrac{\p^2z}{\p x\p y}+\dfrac{\p^2z}{\p y\p x}+\dfrac{\p^2z}{\p y^2}\]
        \item \textbf{(10\songti{分})}\ 给定方程组
            \[\left\{\begin{array}{l}
                xy+yz^2+4=0\\
                x^2y+yz-z^2+5=0
            \end{array}\right.\]
            试讨论上述方程在$P_0\left(1,-2,1\right)$处能确定的隐函数,并计算其在$P_0$处的导数.
    \end{enumerate}
\end{problem}
\begin{solution}
    \begin{enumerate}[label=\tbf{(\arabic*)}]
        \item 设$G(x,y,z)=F(x-z,y-z)$.于是有
            \[G_x(x,y,z)=F_u(x-z,y-z)\ \ \ \ \ G_y(x,y,z)=F_v(x-z,y-z)\]
            \[G_z(x-z,y-z)=-F_u(x-z,y-z)-F_v(x-z,y-z)\]
            于是根据隐函数存在定理,$G(x,y,z)=F(x-z,y-z)=0$确定的隐函数$z=z(x,y)$满足
            \[\dfrac{\p z}{\p x}=-\dfrac{G_x}{G_z}=\dfrac{F_u}{F_u+F_v}\ \ \ \ \ \dfrac{\p z}{\p y}=-\dfrac{G_y}{G_z}=\dfrac{F_v}{F_u+F_v}\]
            其中$F_u,F_v$均指代$F_u(x-z,y-z),F_v(x-z,y-z)$.于是有
            \[\dfrac{\p z}{\p x}+\dfrac{\p z}{\p y}=\dfrac{F_u+F_v}{F_u+F_v}=1\]
            将上式对$x$求偏导有
            \[\dfrac{\p^2 z}{\p x^2}+\dfrac{\p^2 z}{\p x\p y}=0\]
            对$y$求偏导有
            \[\dfrac{\p^2 z}{\p y^2}+\dfrac{\p^2 z}{\p y\p x}=0\]
            于是\[\dfrac{\p^2z}{\p x^2}+\dfrac{\p^2z}{\p x\p y}+\dfrac{\p^2z}{\p y\p x}+\dfrac{\p^2z}{\p y^2}=0\]
    \end{enumerate}
\end{solution}
\begin{problem}[3.(20\songti{分})]
    求函数$f(x,y)=\left(y-x^2\right)\left(y-x^3\right)$的极值.
\end{problem}
\begin{problem}[4.(20\songti{分})]
    回答下列问题.
    \begin{enumerate}[label=\tbf{(\arabic*)}]
        \item \textbf{(10\songti{分})}\ 设函数$f(x,y)$在点$(0,0)$的某邻域内有定义且在$(0,0)$处连续.若极限$\displaystyle\lim_{(x,y)\to(0,0)}\dfrac{f(x,y)}{x^2+y^2}$存在,试证明$f(x,y)$在$(0,0)$处可微.
        \item \textbf{(10\songti{分})}\ 欧氏空间$\R^3$中平面$T:x+y+z=1$截圆柱面$S:x^2+y^2=1$得一椭圆周$R$.求$R$上到原点最近和最远的点.
    \end{enumerate}
\end{problem}
\begin{problem}[5.(20\songti{分})]
    回答下列问题.
    \begin{enumerate}[label=\tbf{(\arabic*)}]
        \item \textbf{(10\songti{分})}\ 设$f(x)$是一个定义在$\R$上的周期为$T\neq0$的无穷阶光滑函数.试证明:对于任意$k\in\N^*$,总存在$\xi\in\R$使得$f^{(k)}(\xi)=0$.
        \item \textbf{(10\songti{分})}\ 设函数$f(u,v)$有连续的偏导数$f_u(u,v)$和$f_v(u,v)$且满足$f(x,1-x)=1$.试证明:在单位圆周$S:u^2+v^2=1$上至少存在两个不同的点$(u_1,v_1)$和$(u_2,v_2)$使得$v_if_u(u_i,v_i)=u_if_v(u_i,v_i)$,其中$i=1,2$.
    \end{enumerate}
\end{problem}
\end{document}