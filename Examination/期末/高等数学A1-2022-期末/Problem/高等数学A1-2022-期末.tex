\documentclass{ctexart}
\usepackage{geometry}
\usepackage[dvipsnames,svgnames]{xcolor}
\usepackage[strict]{changepage}
\usepackage{framed}
\usepackage{enumerate}
\usepackage{amsmath,amsthm,amssymb}
\usepackage{enumitem}
\usepackage{template}

\allowdisplaybreaks
\geometry{left=2cm, right=2cm, top=2.5cm, bottom=2.5cm}

\begin{document}\pagestyle{empty}

\begin{center}\Large
    北京大学数学科学学院2022-23高等数学A1期末考试
\end{center}
\begin{enumerate}[leftmargin=*,label=\textbf{\arabic*.}]
    \item \textbf{(16\songti{分})}\ 回答下列问题。
        \begin{enumerate}[label=\tbf{(\arabic*)}]
            \item \textbf{(8\songti{分})}\ 证明:直线$l:\left\{\begin{array}{l}
                    x-2y+z=0\\
                    5x+2y-5z=-6
                \end{array}\right.$过点$(1,2,3)$,并把此一般方程化为标准方程.
            \item \textbf{(8\songti{分})}\ 求曲线$\left\{\begin{array}{l}
                    x=7t-14\\
                    y=4t^2\\
                    z=3t^3    
                \end{array}\right.$在参数$t=1$对应的点$P$处的法平面方程.
        \end{enumerate}
    \item \textbf{(20\songti{分})}\ 回答下列问题.
        \begin{enumerate}[label=\tbf{(\arabic*)}]
            \item \textbf{(10\songti{分})}\ 设函数$z=\arctan\dfrac{(x-3)y+(x^2+x-1)y^2}{(x-2)y+(x-3)^2y^4}$,求$\left.\dfrac{\p z}{\p y}\right|_{(3,0)}$.
            \item \textbf{(10\songti{分})}\ 设函数$z=z(x,y)$由方程\[m\left(x+\dfrac zy\right)^n\]
        \end{enumerate}
    \item \textbf{(20\songti{分})}\ 求函数$f(x,y)=\left(y-x^2\right)\left(y-x^3\right)$的极值.
    \item \textbf{(20\songti{分})}\ 回答下列问题.
        \begin{enumerate}[label=\tbf{(\arabic*)}]
            \item \textbf{(10\songti{分})}\ 设函数$f(x,y)$在点$(0,0)$的某邻域内有定义且在$(0,0)$处连续.若极限$\displaystyle\lim_{(x,y)\to(0,0)}\dfrac{f(x,y)}{x^2+y^2}$存在,试证明$f(x,y)$在$(0,0)$处可微.
            \item \textbf{(10\songti{分})}\ 欧氏空间$\R^3$中平面$T:x+y+z=1$截圆柱面$S:x^2+y^2=1$得一椭圆周$R$.求$R$上到原点最近和最远的点.
        \end{enumerate}
    \item \textbf{(20\songti{分})}\ 回答下列问题.
        \begin{enumerate}[label=\tbf{(\arabic*)}]
            \item \textbf{(10\songti{分})}\ 设$f(x)$是一个定义在$\R$上的周期为$T\neq0$的无穷阶光滑函数.试证明:对于任意$k\in\N^*$,总存在$\xi\in\R$使得$f^{(k)}(\xi)=0$.
            \item \textbf{(10\songti{分})}\ 设函数$f(u,v)$有连续的偏导数$f_u(u,v)$和$f_v(u,v)$且满足$f(x,1-x)=1$.试证明:在单位圆周$S:u^2+v^2=1$上至少存在两个不同的点$(u_1,v_1)$和$(u_2,v_2)$使得$v_if_u(u_i,v_i)=u_if_v(u_i,v_i)$,其中$i=1,2$.
        \end{enumerate}
\end{enumerate}
\end{document}