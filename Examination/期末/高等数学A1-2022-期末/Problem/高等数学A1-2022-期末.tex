\documentclass{ctexart}
\usepackage{geometry}
\usepackage[dvipsnames,svgnames]{xcolor}
\usepackage[strict]{changepage}
\usepackage{framed}
\usepackage{enumerate}
\usepackage{amsmath,amsthm,amssymb}
\usepackage{enumitem}
\usepackage{template}

\allowdisplaybreaks
\geometry{left=2cm, right=2cm, top=2.5cm, bottom=2.5cm}

\begin{document}\pagestyle{empty}

\begin{center}\Large
    北京大学数学科学学院2022-23高等数学A1期末考试
\end{center}
\begin{enumerate}[leftmargin=*,label=\textbf{\arabic*.}]
    \item \textbf{(16\songti{分})}\ 回答下列问题。
        \begin{enumerate}[label=\tbf{(\arabic*)}]
            \item \textbf{(8\songti{分})}\ 证明:直线$l:\left\{\begin{array}{l}
                    x-2y+z=0\\
                    5x+2y-5z=-6
                \end{array}\right.$过点$(1,2,3)$,并把此一般方程化为标准方程.
            \item \textbf{(8\songti{分})}\ 求曲线$S:\left\{\begin{array}{l}
                    x=7t-14\\
                    y=4t^2\\
                    z=3t^3    
                \end{array}\right.$在参数$t=1$对应的点$P$处的法平面方程.
        \end{enumerate}
    \item \textbf{(20\songti{分})}\ 回答下列问题.
        \begin{enumerate}[label=\tbf{(\arabic*)}]
            \item \textbf{(10\songti{分})}\ 设函数$z=\arctan\dfrac{(x-3)y+(x^2+x-1)y^2}{(x-2)y+(x-3)^2y^4}$,求$\left.\dfrac{\p z}{\p y}\right|_{(3,0)}$.
            \item \textbf{(10\songti{分})}\ 设函数$z=z(x,y)$由方程\[m\left(x+\dfrac zy\right)^n+n\left(y+\dfrac zx\right)^m=1\]确定,其中$m,n\in\N$.计算并化简\[x\dfrac{\p z}{\p x}+y\dfrac{\p z}{\p y}+xy\]
        \end{enumerate}
    \item \textbf{(24\songti{分})}\ 下列极限是否存在?若存在,请求出其值;若不存在,请说明理由.
        \begin{enumerate}[label=\tbf{(\arabic*)}]
            \item \textbf{(8\songti{分})}\ \(\displaystyle\lim_{x\to0}\dfrac{\displaystyle\int_0^{x^3}\sin^32t\di t}{\displaystyle\int_0^{x^2}\tan t^5\di t}\).
            \item \textbf{(8\songti{分})}\ \(\displaystyle\lim_{(x,y)\to(0,0)}\dfrac{x^2y}{x^4+y^2}\).
            \item \textbf{(8\songti{分})}\ \(\displaystyle\lim_{n\to\infty}\sum_{k=1}^{n}\dfrac{n}{n^2+k^2}\).
        \end{enumerate}
    \item \textbf{(24\songti{分})}\ 回答下列问题.
        \begin{enumerate}[label=\tbf{(\arabic*)}]
            \item \textbf{(8\songti{分})}\ 设$P_1(a_1,b_1,c_1),P_2(a_2,b_2,c_2)$是单位球面$S:x^2+y^2+z^2=1$上的两个不同的点,$O(0,0,0)$是坐标原点.求\[\left|\overrightarrow{OP_1}\times\overrightarrow{OP_2}\right|^2+\left(\overrightarrow{OP_1}\cdot\overrightarrow{OP_2}\right)\]
            \item \textbf{(8\songti{分})}\ 计算$\displaystyle\int_{-1}^1\left(\dfrac{\sin^2x}{1+\e^x}+\dfrac{\cos^2x}{1+\e^{-x}}\right)\dx$.
            \item \textbf{(8\songti{分})}\ 计算\(\displaystyle\int_{0}^{\frac{\pi}{4}}\ln\dfrac{\sin\left(x+\frac\pi4\right)}{\cos x}\dx\).
        \end{enumerate}
    \item \textbf{(8\songti{分})}\ 设$f(x)$在$(a,b)$上二阶可导,$f(a)=f(b)=0$,$f\left(\dfrac{a+b}{2}\right)>0$.试证明:存在$\xi\in(a,b)$使得$f''(\xi)<0$.
    \item \textbf{(8\songti{分})}\ 设$f(x)$在$[0,2]$上有连续的导数,$f(0)=f(2)=0$,记$M=\displaystyle\max_{x\in[0,2]}\left\{|f(x)|\right\}$.试证明:
        \begin{enumerate}[label=\tbf{(\arabic*)}]
            \item \textbf{(4\songti{分})}\ 存在$\xi\in(0,2)$使得$\left|f'(\xi)\right|\geqslant M$.
            \item \textbf{(4\songti{分})}\ 若对于任意$x\in(0,2)$都有$\left|f'(x)\right|\leqslant M$,则$f(x)\equiv0$.
        \end{enumerate}
\end{enumerate}
\end{document}