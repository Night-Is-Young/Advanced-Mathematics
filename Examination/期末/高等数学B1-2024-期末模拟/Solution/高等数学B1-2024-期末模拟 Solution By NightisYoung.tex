\documentclass{ctexart}
\usepackage{geometry}
\usepackage[dvipsnames,svgnames]{xcolor}
\usepackage[strict]{changepage}
\usepackage{framed}
\usepackage{enumerate}
\usepackage{amsmath,amsthm,amssymb}
\usepackage{enumitem}
\usepackage{template}

\geometry{left=2cm, right=2cm, top=2.5cm, bottom=2.5cm}

\begin{document}\pagestyle{empty}
\begin{center}\Large
    北京大学数学科学学院2024-25高等数学B1期末考试
\end{center}
\begin{problem}[1.(15\songti{分})]
    下面的函数极限存在吗?如果存在,请求出其值;如果不存在,请说明理由.
    \begin{enumerate}[label=\tbf{(\arabic*)}]
        \item \textbf{(5\songti{分})}\ $\displaystyle\lim_{x\to0}\left(\dfrac{1}{x}-\dfrac{1}{\tan x}\right)\dfrac{1}{\tan2x}$.
        \item \textbf{(5\songti{分})}\ $\displaystyle\lim_{x\to0}\dfrac{\sqrt[3]{1+x^2+x^4}-1}{\left(\ln(1+x)\right)^2}$.
        \item \textbf{(5\songti{分})}\ $\displaystyle\lim_{(x,y)\to(0,0)}\dfrac{xy}{x^2+(\e^y-1)^2}$.
    \end{enumerate}
\end{problem}
\begin{solution}
    \begin{enumerate}[label=\tbf{(\arabic*)}]
        \item 注意到$\tan 2x=\dfrac{2\tan x}{1-\tan^2x}$.于是
            \[\begin{aligned}
                \lim_{x\to0}\left(\dfrac{1}{x}-\dfrac{1}{\tan x}\right)\dfrac{1}{\tan2x}
                &= \lim_{x\to0}\dfrac{(\tan x-x)(1-\tan^2x)}{2x\tan^2 x} \\
                &= \lim_{x\to0}\dfrac{\tan x-x}{2x^3}\cdot\lim_{x\to0}\dfrac{x^2}{\tan^2x} \\
                &= \lim_{x\to0}\dfrac{\tan^2x}{6x^2}\cdot1^2\\
                &= \dfrac16
            \end{aligned}\]
        \item 注意到$\sqrt[3]{1+x}=1+\dfrac13x+o(x)$,$\ln(1+x)=x+o(x)$.于是
            \[\begin{aligned}
                \lim_{x\to0}\dfrac{\sqrt[3]{1+x^2+x^4}-1}{\left(\ln(1+x)\right)^2}
                &= \lim_{x\to0}\dfrac{1+\dfrac13x^2+\dfrac13x^4+o(x^2)-1}{x^2+o(x^2)} \\
                &= \dfrac13
            \end{aligned}\]
        \item 取$y=kx$.注意到$\e^x=1+x+o(x)$,于是
            \[\lim_{(x,y)\to(0,0)}\dfrac{xy}{x^2+(\e^y-1)^2}=\lim_{(x,y)\to(0,0)}\dfrac{kx^2}{x^2+k^2x^2+o(x^2)}=\dfrac{k}{1+k^2}\]
            于是取不同的$k$对应的路径所得的极限不同,因而原函数极限不存在.
    \end{enumerate}
\end{solution}
\begin{problem}[2.(10\songti{分})]
    设欧氏空间$\R^3$中的平面$\Omega$过直线$x+1=y-3=\dfrac{z}2$,且与平面$3x-y-10z=4$垂直.
    \begin{enumerate}[label=\tbf{(\arabic*)}]
        \item \textbf{(5\songti{分})}\ 求$\Omega$的标准方程.
        \item \textbf{(5\songti{分})}\ 已知以原点为球心的球面$S$与$\Omega$相切,求$S$的方程.
    \end{enumerate}
\end{problem}
\begin{solution}
    \begin{enumerate}[label=\tbf{(\arabic*)}]
        \item 由题意可知$(3,-1,-10)$与$\Omega$平行.设$\Omega:ax+by+cz+1=0$,于是有
            \[\left\{\begin{array}{l}
                3a-b-10c=0\\
                ax+b(x+4)+c(2x+2)+1\equiv0
            \end{array}\right.\]
            解得$a=\dfrac17,b=-\dfrac27,c=\dfrac1{14}$.于是$\Omega$的方程为$2x-4y+z+14=0$.
        \item 设切点$P(x,y,z)$,$\Omega$的法向量$\vec{u}=(2,-4,1)$与其上一点$M(-7,0,0)$.我们有
            \[\left|\overrightarrow{OP}\right|=\left|\dfrac{\overrightarrow{OM}\cdot\vec{u}}{|\vec{u}|}\right|=\dfrac{14}{\sqrt{21}}=\dfrac{2\sqrt{21}}{3}\]
            于是$S$的半径为$\dfrac{2\sqrt{21}}{3}$,因而其方程为$x^2+y^2+z^2=\dfrac{28}{3}$.
    \end{enumerate}
\end{solution}
\begin{problem}[3.(10\songti{分})]
    回答下列问题.
    \begin{enumerate}[label=\tbf{(\arabic*)}]
        \item \textbf{(5\songti{分})}\ 设正整数$n\geqslant3$,$n$元函数$u:\R^n\to\R$满足
            \[u(\li x,n)=\left(\sum_{k=1}^nx_k^2\right)^{\frac{2-n}{2}}\]
            其中$\displaystyle\sum_{k=1}^nx_k^2\neq0$.试求$\displaystyle\sum_{k=1}^{n}u_{x_kx_k}$.
        \item \textbf{(5\songti{分})}\ 设常数$a\in\R$,设$h(x,t)=f(x+at)+g(x-at)$,其中$f:\R\to\R$和$g:\R\to\R$均有连续的二阶导函数.试求$h_{tt}-a^2h_{xx}$.
    \end{enumerate}
\end{problem}
\begin{solution}
    \begin{enumerate}[label=\tbf{(\arabic*)}]
        \item 对于任意给定的$k$,不妨令$S=\displaystyle\sum_{k=1}^nx_k^2$,令$S-x_k^2=S_k$.于是
            \[u_{x_k}=\dfrac{\p}{\p x_k}\left(S_k+x_k^2\right)^{\frac{2-n}2}=\left(1-\dfrac{n}{2}\right)\left(S_k+x_k^2\right)^{-\frac{n}{2}}\cdot(2x_k)\]
            \[\begin{aligned}
                u_{x_kx_k}
                &= (2-n)\left((S_k+x_k^2)^{-\frac{n}{2}}+x_k\cdot\left(-\dfrac{n}{2}\right)(S_k+x_k^2)^{-\frac{n+2}{2}}\cdot(2x_k)\right) \\
                &= (2-n)\left(S_k+x_k^2\right)^{-\frac{n}{2}-1}\left(S_k+x_k^2-nx_k^2\right) \\
                &= (2-n)S(S-nx_k^2)
            \end{aligned}\]
            于是
            \[\sum_{k=1}^nu_{x_kx_k}=(2-n)S\left(nS-\sum_{k=1}^nnx_k^2\right)=0\]
        \item 我们有
            \[h_t=af'(x+at)-ag'(x-at)\]
            \[h_{tt}=a^2f''(x+at)+a^2g''(x-at)\]
            同理有\[h_x=f'(x+at)+g'(x-at)\]
            \[h_{xx}=f''(x+at)+g''(x-at)\]
            于是\[h_{tt}-a^2h_{xx}=a^2f''(x+at)+a^2g''(x-at)-a^2\left(f''(x+at)+g''(x-at)\right)=0\]
    \end{enumerate}
\end{solution}
\begin{problem}[4.(10\songti{分})]
    求函数$f(x,y)=x^{\sqrt{y}}$在$(1,1)$处的二阶泰勒多项式.
\end{problem}
\begin{solution}
    在$(1,1)$处,我们有
    \[f_x=\sqrt{y}x^{\sqrt{y}-1}=1\]
    \[f_y=\dfrac{x^{\sqrt{y}}\ln x}{2\sqrt{y}}=0\]
    \[f_{xx}=\sqrt{y}\left(\sqrt{y}-1\right)x^{\sqrt{y}-2}=0\]
    \[f_{yy}=\dfrac{x^{\sqrt{y}}\ln^2x+x^{\sqrt{y}}\ln x\cdot\frac{1}{\sqrt{y}}}{4y}=0\]
    \[f_{yx}=\dfrac{1}{2\sqrt{y}}\left(\sqrt{y}x^{\sqrt{y}-1}\ln x+\dfrac{x^{\sqrt{y}}}{x}\right)=\dfrac12\]
    于是
    \[f(x,y)=1+(x-1)+\dfrac{(x-1)(y-1)}{2}\]
\end{solution}
\begin{problem}[5.(10\songti{分})]
    设$t\in[0,2\pi]$,$R>0$,$a>0$.求螺旋线$S:\left\{\begin{array}{l}
        x=R\cos t\\
        y=R\sin t\\
        z=at
    \end{array}\right.$在$t=\dfrac\pi4$的切线和法平面的方程.
\end{problem}
\begin{solution}
    $t=\dfrac{\pi}{4}$时对应点$\left(\dfrac{\sqrt2R}{2},\dfrac{\sqrt2R}{2},\dfrac{a\pi}{4}\right)$.又有
    \[\dfrac{\di x}{\di t}=-R\sin t\ \ \ \ \ \dfrac{\di y}{\di t}=R\cos t\ \ \ \ \ \dfrac{\di z}{\di t}=a\]
    于是切线为$-\dfrac{R}{\sqrt2}\left(x-\dfrac{R}{\sqrt2}\right)=\dfrac{R}{\sqrt2}\left(y-\dfrac{R}{\sqrt2}\right)=a\left(z-\dfrac{a\pi}{4}\right)$.\\
    法平面为$-\dfrac{R}{\sqrt2}x+\dfrac{R}{\sqrt2}y+az=\dfrac{a^2\pi}{4}$.
\end{solution}
\begin{problem}[6.(10\songti{分})]
    设$a,b,c\in\R$.试证明:方程$\e^x=ax^2+bx+c$的互异实根不超过三个.
\end{problem}
\begin{proof}
    设$f(x)=ax^2+bx+c-\e^x$.于是$f'(x)=2ax+b-\e^x$,$f''(x)=2a-\e^x$,$f'''(x)=-\e^x$.\\
    不妨设$f(x)=0$有至少四个互异实根,设其为$x_1,\cdots,x_4$,满足$x_1<\cdots<x_4$.\\
    对于任意$x_k,x_{k+1}$(其中$1\leqslant k<4$),有$f(x_k)=f(x_{k+1})$.\\
    根据Rolle中值定理,存在$\xi_k\in(x_k,x_{k+1})$使得$f'(\xi_k)=0$.\\
    于是$f'(x)=0$至少有三个实根$\xi_1,\xi_2,\xi_3$.同理可知$f''(x)=0$至少有两个实根,$f'''(x)=0$至少有一个实根.\\
    而$f'''(x)=-\e^x<0$,即$f'''(x)=0$没有实根,这与假设不符,从而$f(x)=0$至多有三个互异实根.
\end{proof}
\begin{problem}[7.(10\songti{分})]
    证明:对于任意给定的$k\in\R$,存在$1$的开邻域$U$和$W$,存在唯一的函数$y=f(x),x\in U,y\in W$满足方程$x^k-3x^2y+3xy^2-y^k=0$.
\end{problem}
\begin{proof}
    设$F(x,y)=x^k-3x^2y+3xy^2-y^k$,则有
    \[F_x(x,y)=kx^{k-1}-6xy+3y^2\ \ \ \ \ F_y(x,y)=-ky^{k-1}+6xy-3x^2\]
    若$k=3$,则有$x^3-3x^2y+3xy^2-y^3=0$,即$(x-y)^3=0$.这确定唯一的函数$y=f(x)=x$.\\
    若$k\neq3$,则有$F_y(1,1)=k-3\neq0$.据隐函数存在定理,在$(1,1)$的邻域内$F(x,y)\equiv0$确定唯一的$y=f(x)$且
    \[f'(x)=-\dfrac{F_x}{F_y}=\dfrac{kx^{k-1}-6xy+3y^2}{ky^{k-1}-6xy+3x^2}\]
\end{proof}
\begin{problem}[8.(15\songti{分})]
    求函数$f(x,y,z)=x^3+y^3+z^3-4xyz$在$D=\left\{(x,y,z)\in\R^3:x^2+y^2+z^2\leqslant1\right\}$上的最大值和最小值.
\end{problem}
\begin{solution}
    我们有
    \[f_x=3x^2-4yz\ \ \ \ \ f_y=3y^2-4xz\ \ \ \ \ f_z=3z^2-4xy\]
    令$f_x=f_y=f_z=0$,解得$x=y=z=0$,而$f(0,0,0)=0$.\\
    现在考虑$D$的边界.令$F(x,y,z,\lambda)=f(x,y,z)-\lambda(x^2+y^2+z^2-1)$.令$F$的各偏导为$0$,有
    \[\left\{\begin{array}{l}
        F_x=3x^2-4yz-2\lambda x=0\\
        F_y=3y^2-4xz-2\lambda y=0\\
        F_z=3z^2-4xy-2\lambda z=0\\
        F_\lambda=x^2+y^2+z^2-1=0
    \end{array}\right.\]
    令$F_x$,$F_y$与$F_z$两两相减有
    \[\left\{\begin{array}{l}
        (x-y)(3x+3y+4z-2\lambda)=0\\
        (x-z)(3x+3z+4y-2\lambda)=0\\
        (y-z)(3y+3z+4x-2\lambda)=0
    \end{array}\right.\]
    若$x=y=z$,不难得出$x=y=z=\dfrac{1}{\sqrt3}$或$x=y=z=-\dfrac{1}{\sqrt3}$.此时有
    \[f\left(\dfrac{1}{\sqrt3},\dfrac{1}{\sqrt3},\dfrac{1}{\sqrt3}\right)=-\dfrac{1}{3\sqrt3}\ \ \ \ \ f\left(-\dfrac{1}{\sqrt3},-\dfrac{1}{\sqrt3},-\dfrac{1}{\sqrt3}\right)=\dfrac{1}{3\sqrt3}\]
    若$x,y,z$不全相等,不妨设$x=y$.于是有$(x-z)(7x+3z-2\lambda)=0$,即$\lambda=\dfrac{7x+3z}{2}$.代入$F_x=0$有
    \[x(7z+4x)=0\]
    若$x=0$,则有$y=0$,$z=\pm1$.此时有$f(0,0,1)=f(0,0,-1)=0$.\\
    若$7z+4x=0$,则有$x^2+x^2+\dfrac{16}{49}x^2=1$,于是$x=y=-\dfrac74z=\dfrac{7}{\sqrt{114}}$或$x=y=-\dfrac74z=-\dfrac{7}{\sqrt{114}}$.此时有
    \[f(x,y,z)=x^3+x^3-\dfrac{64}{343}x^3+\dfrac{16}{7}x^3=\dfrac{1406}{343}x^3\]
    于是$f\left(\dfrac{7}{\sqrt{114}},\dfrac{7}{\sqrt{114}},-\dfrac{4}{\sqrt{114}}\right)=\dfrac{37}{3\sqrt{114}}$,$f\left(-\dfrac{7}{\sqrt{114}},-\dfrac{7}{\sqrt{114}},\dfrac{4}{\sqrt{114}}\right)=-\dfrac{37}{3\sqrt{114}}$.\\
    于是$f(x,y,z)$的最大值为$\dfrac{37}{3\sqrt{114}}$,最小值为$-\dfrac{37}{3\sqrt{114}}$.
\end{solution}
\begin{problem}[9.(10\songti{分})]
    设函数$f$在$[0,1]$二阶可微,且$f(0)=f(1)=0$,$\displaystyle\min_{x\in[0,1]}f(x)=-1$.试证明:存在$\epsilon\in(0,1)$使得$f''(\epsilon)\geqslant8$.
\end{problem}
\begin{proof}
    设$f(x)$在$x=x_0$处取到最小值.将$f(x)$在$x=x_0$处做泰勒展开有
    \[f(x)=f(x_0)+f'(x_0)(x-x_0)+\dfrac{f''(\xi)(x-x_0)^2}{2},x\gtrless\xi\gtrless x_0\]
    由于$f(x)$在$[0,1]$二阶可微,因而最小值点必然满足$f'(x_0)=0$.又$f(x_0)=-1$,于是有
    \[f(x)=\dfrac{f''(\xi)(x-x_0)^2}{2}-1,x\gtrless\xi\gtrless x_0\]
    分别取$x=0,1$有
    \[\left\{\begin{array}{l}
        1=\dfrac{f''(\xi_1)x_0^2}{2}\\
        1=\dfrac{f''(\xi_2)(1-x_0)^2}{2}
    \end{array}\right.\]
    由于$\min\left\{x_0,1-x_0\right\}\leqslant\dfrac12$,于是
    \[\max\left\{f''(\xi_1),f''(\xi_2)\right\}=\max\left\{\dfrac{2}{x_0^2},\dfrac{2}{(1-x_0)^2}\right\}\geqslant\dfrac{2}{\left(\frac12\right)^2}=8\]
\end{proof}
\end{document}