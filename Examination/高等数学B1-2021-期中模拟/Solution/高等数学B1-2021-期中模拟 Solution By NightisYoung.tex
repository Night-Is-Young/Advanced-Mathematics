\documentclass{ctexart}
\usepackage{geometry}
\usepackage[dvipsnames,svgnames]{xcolor}
\usepackage[strict]{changepage}
\usepackage{framed}
\usepackage{enumerate}
\usepackage{amsmath,amsthm,amssymb}
\usepackage{enumitem}
\usepackage{template}

\geometry{left=2cm, right=2cm, top=2.5cm, bottom=2.5cm}

\begin{document}\pagestyle{empty}
\begin{center}\Large
    北京大学数学科学学院2021-22高等数学B1期中模拟
\end{center}
\begin{problem}[1.(10\songti{分})]
    多选题,错选或少选均不得分,无需写出解答过程.
    \begin{enumerate}[label=\textbf{(\arabic*)}]
        \item \textbf{(5\songti{分})}\ 选出下列选项中总是正确的式子.
            \begin{tabbing}
                \hspace{0pt} \= \hspace{225pt} \= \hspace{225pt} \kill
                \> \textbf{A.}$\displaystyle\int_0^{\frac{\pi}{2}}\dfrac{\sin x}{x}\dx<\dfrac{\pi}{2}$
                \> \textbf{B.}$\displaystyle\int_0^{\frac{\pi}{2}}\dfrac{\sin x}{x}\dx>1$ \\\ \\
                \> \textbf{C.}$\displaystyle\int_0^{\frac{\pi}{2}}\dfrac{\sin x}{x}\dx>\dfrac{1}{2}+\dfrac{\pi}{4}$
                \> \textbf{D.}$\displaystyle\int_0^{\frac{\pi}{2}}\dfrac{\sin x}{x}\dx>\dfrac{1}{2}\displaystyle\int_0^{\pi}\dfrac{\sin x}{x}\dx$
            \end{tabbing}
        \item \textbf{(5\songti{分})}\ 设$f(x)$是定义在$[1,+\infty)$上的非负单调递减的连续函数.定义$\displaystyle s_n=\sum_{k=1}^nf(k)$,选出下列选项中总是正确的式子.
            \begin{tabbing}
                \hspace{0pt} \= \hspace{225pt} \= \hspace{225pt} \kill
                \> \textbf{A.}$\displaystyle s_n\leqslant\int_1^nf(x)\dx$
                \> \textbf{B.}$\displaystyle s_n\leqslant f(1)+\int_1^nf(x)\dx$ \\\ \\
                \> \textbf{C.}$\displaystyle s_n\geqslant\int_1^{n+1}f(x)\dx$
                \> \textbf{D.}$\displaystyle s_n\geqslant f(1)+\int_1^{n+1}f(x)\dx$
            \end{tabbing}
    \end{enumerate}
\end{problem}
\begin{solution}
    \begin{enumerate}[label=\textbf{(\arabic*)}]
        \item \textbf{Solution.}\\
            注意到$\displaystyle\dfrac{\sin x}{x}<1\text{,于是}\int_{0}^{\frac{\pi}{2}}\dfrac{\sin x}{x}\dx<\int_{0}^{\frac{\pi}{2}}\dx=\dfrac{\pi}{2}$,于是\textbf{A}项正确.\\
            注意到$\displaystyle\dfrac{\sin x}{x}>\dfrac{2}{\pi}\text{,于是}\int_{0}^{\frac{\pi}{2}}\dfrac{\sin x}{x}\dx>\int_{0}^{\frac{\pi}{2}}\dfrac{2\dx}{\pi}=1$,于是\textbf{B}项正确.\\
            割线放缩可得$\displaystyle\dfrac{\sin x}{x}>\dfrac{1-\frac{\pi}{2}}{\frac{\pi}{2}}x+1$,于是$\displaystyle\int_{0}^{\frac{\pi}{2}}\dfrac{\sin x}{x}\dx>\dfrac{\frac{\pi}{2}\left(1+\frac{\pi}{2}\right)}{2}$,于是\textbf{C}项正确.\\
            注意到$\displaystyle\int_0^{\frac{\pi}{2}}\dfrac{\sin x}{x}\dx-\dfrac{1}{2}\displaystyle\int_0^{\pi}\dfrac{\sin x}{x}\dx=\dfrac{1}{2}\left(\int_0^{\frac{\pi}{2}}\dfrac{\sin x}{x}\dx-\int_{\frac{\pi}{2}}^{\pi}\dfrac{\sin x}{x}\dx\right)>0$,于是\textbf{D}项正确.
        \item \textbf{Solution.}\\
            注意到$$\forall k\in\N^*,\forall x\in(k,k+1),f(k)\geqslant f(x)\geqslant f(k+1)$$
            于是$$\forall k\in\N^*,f(k)\geqslant\int_{k}^{k+1}f(x)\dx \geqslant f(k+1)$$
            于是$$s_n\geqslant\sum_{i=1}^{n}\int_{i}^{i+1}f(x)\dx=\int_{1}^{n+1}f(x)\dx$$
            且$$s_n-f(1)\leqslant\sum_{i=1}^{n-1}\int_{i}^{i+1}=\int_{1}^{n}f(x)\dx$$
            于是选择\textbf{BC}项.
    \end{enumerate}
\end{solution}
\begin{problem}[2.(18\songti{分})]
    计算下列极限.
    \begin{enumerate}[label=\textbf{(\arabic*)}]
        \item \textbf{(6\songti{分})}\ 计算序列极限$$\lim_{n\to\infty}\dfrac{\sum_{i=1}^{n}i^{2021}}{n^{2022}}$$
        \item \textbf{(6\songti{分})}\ 计算函数极限$$\lim_{x\to+\infty}\left(\sin\dfrac{1}{x^{2022}}+\cos\dfrac{1}{x^{1011}}\right)^{x^{2022}}$$
        \item \textbf{(6\songti{分})}\ 计算函数极限$$\lim_{x\to0}\dfrac{\sin3x-3\sin x}{x^3}$$
    \end{enumerate}
\end{problem}
\begin{solution}
    \begin{enumerate}[label=\textbf{(\arabic*)}]
        \item \textbf{Solution.}\\
            注意到$$\dfrac{\sum_{i=1}^{n}i^{2021}}{n^{2022}}=\dfrac{1}{n}\sum_{i=1}^{n}\left(\dfrac{i}{n}\right)^{2021}$$
            于是根据Riemann积分的定义有
            $$\lim_{n\to\infty}\dfrac{\sum_{i=1}^{n}i^{2021}}{n^{2022}}=\int_{0}^{1}x^{2021}\dx=\left.\dfrac{x^{2022}}{2022}\right|_0^1=\dfrac{1}{2022}$$
        \item \textbf{Solution.}\\
            注意到$\cos x=1-\sin^2\dfrac{x}{2}$,于是
            $$\begin{aligned}
                \lim_{x\to+\infty}\left(\sin\dfrac{1}{x^{2022}}+\cos\dfrac{1}{x^{1011}}\right)^{x^{2022}}
                &= \lim_{x\to+\infty}\left(1+\sin\dfrac{1}{x^{2022}}-\sin^2\dfrac{1}{x^{2022}}\right)^{x^{2022}}
            \end{aligned}$$
            作变量代换$u=x^{2022}$,于是
            $$\begin{aligned}
                \lim_{x\to+\infty}\left(1+\sin\dfrac{1}{x^{2022}}-\sin^2\dfrac{1}{x^{2022}}\right)^{x^{2022}}
                &= \lim_{u\to+\infty}\left(1+\sin\dfrac{1}{u}-\sin^2\dfrac{1}{u}\right)^{\frac{1}{\sin\frac{1}{u}-\sin^2\frac{1}{u}}\cdot u\sin\frac{1}{u}\left(1-\frac{1}{u}\right)} \\
                &= \e^{1\cdot(1-0)} \\
                &= \e
            \end{aligned}$$
        \item \textbf{Solution.}\\
            我们有$$\sin(3x)=\sin x\cos2x+\sin2x\cos x=\sin x\left(1-2\sin^2x\right)+2\sin x\left(1-\sin^2x\right)=3\sin x-4\sin^3x$$
            于是$$\lim_{x\to0}\dfrac{\sin3x-3\sin x}{x^3}=-4\left(\lim_{x\to0}\dfrac{\sin x}{x}\right)^3=-4$$
    \end{enumerate}
\end{solution}
\begin{problem}[3.(12\songti{分})]
    计算下列积分.
    \begin{enumerate}[label=\textbf{(\arabic*)}]
        \item \textbf{(6\songti{分})}\ 计算定积分$$\int_{-\frac{\pi}{2}}^{\frac{\pi}{2}}\dfrac{\cos x\arctan\left(\e^x\right)}{1+\sin^2x}\dx$$
        \item \textbf{(6\songti{分})}\ 计算不定积分$$\int\dfrac{2x^2+2x+13}{(x-2)\left(1+x^2\right)^2}\dx$$
    \end{enumerate}
\end{problem}
\begin{solution}
    \begin{enumerate}[label=\textbf{(\arabic*)}]
        \item \textbf{Solution.}\\
            置$f(x)=\dfrac{\cos x\arctan\left(\e^x\right)}{1+\sin^2x}\dx$,则
            $$f(-x)=\dfrac{\cos(-x)\arctan\left(\dfrac{1}{\e^x}\right)}{1+\left(\sin x\right)^2}=\dfrac{\cos x\arctan\left(\dfrac{\pi}{2}-\e^x\right)}{1+\sin^2x}$$
            于是$$\int_{-\frac{\pi}{2}}^{\frac{\pi}{2}}$$
        \item \textbf{Solution.}\\
            设$$\dfrac{2x^2+2x+13}{(x-2)\left(1+x^2\right)^2}=\dfrac{A}{x-2}+\dfrac{Bx+C}{x^2+1}+\dfrac{Dx+E}{\left(x^2+1\right)^2}$$
            两端同乘$x-2$后令$x=2$,于是$A=\dfrac{8+4+13}{5^2}=1$
            则$$2x^2+2x+13=x^4+2x^2+1+\left(x^2+1\right)\left(Bx^2+Cx-2Bx-2C\right)+(Dx+E)(x-2)$$
            解得$B=-1,C=-2,D=-3,E=-4$.\\
            而$$\int\dfrac{x+2}{x^2+1}\dx=\dfrac{1}{2}\int\dfrac{\di x^2}{x^2+1}+2\int\dfrac{\dx}{x^2+1}$$
            $$\int\dfrac{3x+4}{\left(x^2+1\right)^2}\dx=\dfrac{3}{2}\int\dfrac{\di x^2}{\left(x^2+1\right)^2}+2\int\dfrac{\left(1+x^2\right)+\left(1-x^2\right)}{\left(x^2+1\right)^2}\dx$$
            于是$$\begin{aligned}
                \int\dfrac{2x^2+2x+13}{(x-2)\left(1+x^2\right)^2}\dx
                &= \int\dfrac{\dx}{x-2}-4\int\dfrac{\dx}{x^2+1}-\dfrac{1}{2}\int\dfrac{\di x^2}{x^2+1}-\dfrac{3}{2}\int\dfrac{\di x^2}{\left(x^2+1\right)^2}-2\int\dfrac{1-x^2}{\left(1+x^2\right)^2}\dx \\
                &= \ln\left|x-2\right|-4\arctan x-\dfrac{1}{2}\ln(x^2+1)+\dfrac{3}{2\left(x^2+1\right)}-\dfrac{2x}{x^2+1}+C\\
                &= -\dfrac{1}{2}\ln\left(x^2+1\right)+\ln\left|x-2\right|-4\arctan x+\dfrac{3-4x}{2\left(x^2+1\right)}+C
            \end{aligned}$$
    \end{enumerate}
\end{solution}
\begin{problem}[4.(8\songti{分})]
    设函数$f(x)$满足$$f(x)=\left\{\begin{array}{l}ax\e^x+bx^x,x>1\\\left|x\right|,x\leqslant1\end{array}\right.$$
    求所有可能的参数$a,b$使得$f(x)$在$x=1$处可导.
\end{problem}
\begin{solution}[Solution.]
    首先需要$f(x)$在$x=1$处连续,于是$\displaystyle\lim_{x\to1^+}=\lim_{x\to1^-}=f(1)=1$,即$a\e+b=1$.
    又要求$f(x)$在$x=1$处的左右导数相同.易知
    $$\lim_{\Delta x\to0^-}\dfrac{f(x+\Delta x)-f(x)}{\Delta x}=\lim_{\Delta x\to0^-}\dfrac{x+\Delta x-x}{\Delta x}=1$$
    又$x>1$时$$f'(x)=a\e^x(x+1)+bx^x(\ln x+1)$$
    于是$f'_+(1)=2a\e+b=1$.\\
    综上可以解得$a=0,b=1$.
\end{solution}
\begin{problem}[5.(12\songti{分})]
    设函数$f(x)$是定义在$\R$上的以$1$为周期的连续函数,试证明:$\exists c\in\R\st f(c)=f(c+\pi)$.
\end{problem}
\begin{solution}[Proof.]
    置$F(x)=f(x+\pi)-f(x)$.根据连续函数的有界性,可知$\exists c_1,c_2\in[0,1)\st f(c_1)\leqslant f(x)\leqslant f(c_2)$,于是
    $$F(c_1)=f(c_1+\pi)-f(c_1)\geqslant0$$
    $$F(c_2)=f(c_2+\pi)-f(c_2)\leqslant0$$
    根据连续函数的介值定理,$\exists c$介于$c_1,c_2$之间,满足$F(c)=0$,即$f(c)=f(c+\pi)$.
\end{solution}
\begin{problem}[6.(8\songti{分})]
    计算曲线$\displaystyle y=\int_0^x\sqrt{\sin x}\dx$在$x\in[0,\pi]$部分的弧长.
\end{problem}
\begin{solution}[Solution]
    根据弧长公式有
    $$\begin{aligned}
        s
        &= \int_0^\pi\sqrt{1+y'^2}\dx \\
        &= \int_0^\pi\sqrt{1+\sin x}\dx \\
        &= \int_0^\pi\sqrt{\sin^2\dfrac{x}{2}+2\sin\dfrac{x}{2}\cos\dfrac{x}{2}+\cos^2\dfrac{x}{2}}\dx \\
        &= \int_0^\pi\left|\sin\dfrac{x}{2}+\cos\dfrac{x}{2}\right|\dx \\
        &= \left.\left(\sin\dfrac{x}{2}-\cos\dfrac{x}{2}\right)\right|_0^\pi \\
        &= 4
    \end{aligned}$$
\end{solution}
\begin{problem}[7.(12\songti{分})]
    考虑方程$x=\tan x$的正实根.
    \begin{enumerate}[label=\textbf{(\arabic*)}]
        \item \textbf{(4\songti{分})}\ 试证明:$x=\tan x$有无穷多个正实根.
        \item \textbf{(8\songti{分})}\ 将$x=\tan x$的正实根从小到大排列成序列$\left\{x_n\right\}$,试证明$\displaystyle\lim_{n\to\infty}\left(x_{n+1}-x_n\right)=\pi$.
    \end{enumerate}
\end{problem}
\begin{solution}
    \begin{enumerate}[label=\textbf{(\arabic*)}]
        \item \textbf{Proof.}\\
            设$f(x)=\tan x-x$,对于任意$k\in\N^*$,在区间$\left[k\pi,k\pi+\dfrac{\pi}{2}\right)$上总有
            $$f(k\pi)=0-k\pi<0$$
            又$\displaystyle\lim_{x\to\left(k\pi+\frac{\pi}{2}\right)^-}f(x)=\lim_{x\to\left(k\pi+\frac{\pi}{2}\right)^-}\tan x-k\pi-\dfrac{\pi}{2}=+\infty$.\\
            即$\exists x\in\left[k\pi,k\pi+\dfrac{\pi}{2}\right)\st f(x)>0$.\\
            于是根据连续函数的介值定理可知$\exists x_0\in\left[k\pi,k\pi+\dfrac{\pi}{2}\right)\st f(x_0)=0$.\\
            又知这样的$k$有无穷多个,于是$x=\tan x$有无穷多个正实根.
        \item \textbf{Proof.}\\
    \end{enumerate}
\end{solution}
\begin{problem}[8.(12\songti{分})]
    设$n\in\N^*$,定义序列$\left\{x_n\right\}$满足$x_n=\sqrt[n]{n}$.
    \begin{enumerate}[label=\textbf{(\arabic*)}]
        \item \textbf{(6\songti{分})}\ 用$\ep-N$语言证明$\displaystyle\lim_{n\to\infty}x_n=1$.
        \item \textbf{(6\songti{分})}\ 求所有的正实数$a$满足$\displaystyle\lim_{n\to\infty}n(x_n-1)^a$收敛.
    \end{enumerate}
\end{problem}
\begin{solution}
    \begin{enumerate}[label=\textbf{(\arabic*)}]
        \item \textbf{Proof.}\\
            记$t_n=\sqrt[n]{n}-1$,于是$n=\left(t+1\right)^n$,作部分二项展开有
            $$n=1+nt_n+\dfrac{n^2-n}{2}t_n^2+\cdots>\dfrac{n^2-n}{2}t_n$$
            于是$t_n<\dfrac{2n}{n^2-n}=\dfrac{1}{n-2}$.\\
            对于任意$\ep>0$,取$N=\left[\dfrac{1}{\ep}\right]+3$,于是$\forall n>N$有
            $$\left|a_n-1\right|=\left|t_n\right|<\dfrac{1}{n-2}<\ep$$
            于是$\displaystyle\lim_{n\to\infty}a_n=1$,得证.
        \item 
    \end{enumerate}
\end{solution}
\begin{problem}[9.(8\songti{分})]
    给定正整数$a$,定义$f_a(x)=\left(x+\sqrt{x^2+1}\right)^a$,求所有自然数$n$满足$f_a^{(n)}(0)=0$.
\end{problem}
\begin{solution}[Solution.]
    注意到$$f'_a(x)=a\left(x+\sqrt{x^2+1}\right)^{a-1}\cdot\left(1+\dfrac{x}{\sqrt{x^2+1}}\right)=\dfrac{a\left(x+\sqrt{x^2+1}\right)^a}{\sqrt{x^2+1}}=\dfrac{a}{\sqrt{x^2+1}}f_a(x)$$
    设$y=f_a(x)$,于是$$\sqrt{1+x^2}y'=ay$$
    即$$\left(1+x^2\right)y'^2=a^2y^2$$
    对该式两端求导有$$2xy'^2+2\left(1+x^2\right)y'y''=2a^2yy'$$
    即$$xy'+\left(1+x^2\right)y''=a^2y$$
    等式两端求$n$阶导有$$xy^{(n+1)}+ny^{(n)}+\left(1+x^2\right)y^{(n+2)}+2nxy^{(n+1)}+\left(n^2-n\right)y^{(n)}=a^2y^{(n)}$$
    代入$x=0$有$$y^{(n+2)}=\left(a^2-n^2\right)y^{(n)}$$
    由$f_a(0)=1,f'_a(0)=a$可知
    $$\left\{\begin{array}{l}
        f_a^{(2k+1)}(0)=\prod_{i=1}^{k}\left(a^2-\left(2i-1\right)^2\right)a \\
        f_a^{(2k+2)}(0)=\prod_{i=0}^{k}\left(a^2-\left(2i\right)^2\right) \\
    \end{array}\right.$$
    于是当$n\geqslant a+2$且$n$与$a$的奇偶性相同时$f_a^{(n)}(0)=0$.
\end{solution}
\end{document}