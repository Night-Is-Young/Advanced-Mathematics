\documentclass{ctexart}
\usepackage{geometry}
\usepackage[dvipsnames,svgnames]{xcolor}
\usepackage[strict]{changepage}
\usepackage{framed}
\usepackage{enumerate}
\usepackage{amsmath,amsthm,amssymb}
\usepackage{enumitem}

\allowdisplaybreaks
\geometry{left=2cm, right=2cm, top=2.5cm, bottom=2.5cm}

\newcommand{\e}{\mathrm{e}}
\newcommand{\di}{\mathrm{d}}
\newcommand{\R}{\mathbb{R}}
\newcommand{\N}{\mathbb{N}}
\newcommand{\ep}{\varepsilon}
\newcommand{\st}{,\text{s.t.}}
\newcommand{\dx}{\di x}

\begin{document}\pagestyle{empty}
\begin{center}\Large
    北京大学数学科学学院2021-22高等数学B1期中考试
\end{center}
\begin{enumerate}[leftmargin=*,label=\textbf{\arabic*.}]
    \item \textbf{(15\songti{分})}\ 导数类基本计算题.
        \begin{enumerate}[label=\textbf{(\arabic*)}]
            \item \textbf{(5\songti{分})}\ 求函数$$f(x)=x^{\arcsin x},0<x<1$$的导函数$f'(x)$.
            \item \textbf{(5\songti{分})}\ 求函数$$f(x)=\int_{\e}^{\e^x}\dfrac{\di t}{1+\ln t}$$的导函数$f'(x)$.
            \item \textbf{(5\songti{分})}\ 求函数$$f(x)=\arctan x$$在$x=0$处的三阶导数$f^{(3)}(0)$.
        \end{enumerate}
    \item \textbf{(15\songti{分})}\ 积分类基本计算题.
        \begin{enumerate}[label=\textbf{(\arabic*)}]
            \item \textbf{(5\songti{分})}\ 求定积分$$\int_{0}^{\frac{\pi}{2}}{\dfrac{\sin x\cos x}{1+\sin^2x}\dx}$$
            \item \textbf{(5\songti{分})}\ 求欧氏平面直角坐标系中曲线$y=\dfrac{1}{2}x^2$从$(0,0)$到$\left(1,\dfrac{1}{2}\right)$的弧长.
            \item \textbf{(5\songti{分})}\ 设奇数$n\geqslant3$,求极坐标系$(r,\theta)$中曲线$r=\sin(n\theta),0\leqslant\theta\leqslant2\pi$围成的封闭图形的面积.
        \end{enumerate}
    \item \textbf{(15\songti{分})}\ 序列$\left\{x_n\right\}$满足
        $$x_1>0,x_{n+1}=\dfrac{1}{2}\left(x_n+\dfrac{1}{x_n}\right)$$
        试证明$\displaystyle\lim_{n\to\infty}x_n$存在,并求出其值.
    \item \textbf{(20\songti{分})}\ 设$x>0$,定义$$p(x)=\int_0^x\dfrac{\di t}{\sqrt{t^3+2021}}$$
        试证明:方程$p(x+1)=p(x)+\sin x$有无穷多个正实数解.
    \item \textbf{(15\songti{分})}\ 证明:对于任意定义在$[0,1]$上的连续函数$f(x)$有$$\lim_{n\to\infty}\int_0^1f(x)\sin(nx)\dx=0$$
    \item \textbf{(20\songti{分})}\ 设$y=f(x)=x^3,x=g(t)=t^2,y=f(g(t))=t^6,\Delta t=0.1,\Delta x=g(1+0.1)-g(1)=0.21$.
        \begin{enumerate}[label=\textbf{(\arabic*)}]
            \item \textbf{(6\songti{分})}\ 当把$t$作为自变量时,函数$y=f(g(t))=t^6$的二阶微分记为$\di_t^2y$,函数$x=g(t)=t^2$的一阶微分记为$\di_tx$.
                试计算:当$t=1,\Delta t=0.1$时,函数$y=f(g(t))$的二阶微分$\left.\di_t^2y\right|_{t=1,\Delta t=0.1}$和函数$x=g(t)$的一阶微分$\left.\di_tx\right|_{t=1,\Delta t=0.1}$.
            \item \textbf{(7\songti{分})}\ 当把$x$作为自变量时,函数$y=f(x)=x^3$的二阶微分记为$\di_x^2y$,$x$(视作$x$的函数)的一阶微分记为$\di_xx$.
                试计算:当$x=1,\Delta x=0.21$时,函数$y=f(x)$的二阶微分$\left.\di_x^2y\right|_{x=1,\Delta x=0.21}$和函数$x$的一阶微分$\left.\di_xx\right|_{x=1,\Delta x=0.21}$.
            \item \textbf{(6\songti{分})}\ $\left.\dfrac{\di_t^2y}{\left(\di_tx\right)^2}\right|_{t=1,\Delta t=0.1}$和$\left.\dfrac{\di_x^2y}{\left(\di_xx\right)^2}\right|_{x=1,\Delta x=0.21}$相等吗?
        \end{enumerate}
\end{enumerate}
\end{document}