\documentclass{ctexart}
\usepackage{geometry}
\usepackage[dvipsnames,svgnames]{xcolor}
\usepackage[strict]{changepage}
\usepackage{framed}
\usepackage{enumerate}
\usepackage{amsmath,amsthm,amssymb}
\usepackage{enumitem}

\allowdisplaybreaks
\geometry{left=2cm, right=2cm, top=2.5cm, bottom=2.5cm}

\newcommand{\e}{\mathrm{e}}
\newcommand{\di}{\mathrm{d}}
\newcommand{\R}{\mathbb{R}}
\newcommand{\N}{\mathbb{N}}
\newcommand{\ep}{\varepsilon}
\newcommand{\st}{,\text{s.t.}}
\newcommand{\dx}{\di x}

\begin{document}\pagestyle{empty}
\begin{center}\Large
    北京大学数学科学学院2024-25高等数学B1期中考试
\end{center}
\textbf{\songti 这卷子真是我奶奶来了都会做!}
\begin{enumerate}[leftmargin=*,label=\textbf{\arabic*.}]
    \item \textbf{(10\songti{分})}\ 求序列极限$$\lim_{n\to\infty}{\sqrt[n]{2024+\sin\left(\e^n\right)}}$$
    \item \textbf{(10\songti{分})}\ 求函数极限$$\lim_{x\to0}\left(\dfrac{1+2\sin^2x}{\cos2x}\right)^{\csc^2x}$$
    \item \textbf{(10\songti{分})}\ 求定义在$(-1,1)$上的函数$$f(x)=\int_0^{\arcsin x}\dfrac{\di t}{\sqrt{1+\sin^2t}}$$的二阶导函数$f''(x)$.
    \item \textbf{(10\songti{分})}\ 求序列极限$$\lim_{n\to\infty}\dfrac{1}{n}\sum_{k=1}^{n}\cos\left(\dfrac{k}{n}-\dfrac{1}{kn^k}\right)$$
    \item \textbf{(10\songti{分})}\ 求不定积分$$\int{\dfrac{4x^2+4x-11}{(2x-1)(2x+3)(2x-5)}\dx}$$
    \item \textbf{(15\songti{分})}\ 设欧氏空间中$V$是曲线弧$\displaystyle y=\dfrac{\ln{x}}{\sqrt{\pi}}(1\leqslant x\leqslant 2)$与直线$x=1,x=2$围成的曲边三角形绕$x$轴旋转一周形成的旋转体,求$V$的体积.
    \item \textbf{(15\songti{分})}\ 试证明:方程$$x^{18}+x^{12}-\cos x=0$$在$\R$上根的个数为$2$.
    \item \textbf{(15\songti{分})}\ 设$D=[0,1]$,函数$A,B:D\to\R$在$D$上连续,且$$\forall x\in D,0\leqslant A(x)\leqslant 1$$
        对于$D$上的连续函数$f:D\to\R$,定义$$T_f(x)=B(x)+\int_0^xA(x)f(x)$$试证明:$T_f=f$有唯一连续函数解.即对于$f,g:D\to\R$,若$T_f=f,T_g=g$,则$f=g$.
\end{enumerate}
\end{document}