\documentclass{ctexart}
\usepackage{geometry}
\usepackage[dvipsnames,svgnames]{xcolor}
\usepackage[strict]{changepage}
\usepackage{framed}
\usepackage{enumerate}
\usepackage{amsmath,amsthm,amssymb}
\usepackage{enumitem}

\allowdisplaybreaks
\geometry{left=2cm, right=2cm, top=2.5cm, bottom=2.5cm}

\newcommand{\e}{\mathrm{e}}
\newcommand{\di}{\mathrm{d}}
\newcommand{\R}{\mathbb{R}}
\newcommand{\N}{\mathbb{N}}
\newcommand{\ep}{\varepsilon}
\newcommand{\st}{,\text{s.t.}}
\newcommand{\dx}{\di x}

\begin{document}\pagestyle{empty}
\begin{center}\Large
    北京大学数学科学学院2023-24高等数学B1期中考试
\end{center}
\begin{enumerate}[leftmargin=*,label=\textbf{\arabic*.}]
    \item \textbf{(10\songti{分})}\ 求序列极限$$\lim_{n\to\infty}{\left(1+\dfrac{1}{n\e}\right)^n}$$
    \item \textbf{(10\songti{分})}\ 设$[x]$为不超过$x$的最大整数,求函数极限$$\lim_{x\to+\infty}{x\sin{\dfrac{1}{[x]}}}$$
    \item \textbf{(10\songti{分})}\ 设$x>0$,求函数$$f(x)=\int_{0}^{\ln{x}}{\sqrt{1+\e^t}\di t}$$的导函数.
    \item \textbf{(10\songti{分})}\ 求不定积\songti{分}$$\int{\dfrac{4x^2+4x-11}{(2x-1)(2x+3)(2x-5)}\dx}$$
    \item \textbf{(10\songti{分})}\ 求欧氏平面直角坐标系中曲线$$y=\dfrac{1}{2}x\sqrt{x^2-1}-\dfrac{1}{2}\ln{\left(x+\sqrt{x^2-1}\right)}$$在$x=1$到$x=2$的弧长.
    \item \textbf{(10\songti{分})}\ 设欧氏空间中$V$是曲线弧$\displaystyle y=\dfrac{\ln{x}}{\sqrt{2\pi}}(1\leqslant x\leqslant 2)$与直线$x=1,x=2$围成的曲边三角形绕$x$轴旋转一周形成的旋转体,求$V$的体积.
    \item \textbf{(10\songti{分})}\ 无穷序列$\left\{a_n\right\},\left\{a_n\right\}$满足$0<b_1<a_1$,且有以下递推关系$$a_{n+1}=\dfrac{a_n+b_n}{2},b_{n+1}=\sqrt{a_nb_n}$$试证明$\displaystyle\lim_{n\to\infty}{a_n}$存在.
    \item \textbf{(20\songti{分})}\ 本题中每个小问都要求给出证明和计算过程.
        \begin{enumerate}[label=\textbf{(\arabic*)}]
            \item \textbf{(2\songti{分})}\ 试证明:当$x\in\left(-\dfrac{\pi}{2},\dfrac{\pi}{2}\right)$时有$$-1<\dfrac{4\sin{x}}{3+\sin^2{x}}<1$$
            \item \textbf{(8\songti{分})}\ 当$x\in\left(-\dfrac{\pi}{2},\dfrac{\pi}{2}\right)$时,求函数$$f(x)=\arcsin{\left(\dfrac{4\sin{x}}{3+\sin^2{x}}\right)}$$的导函数.
            \item \textbf{(10\songti{分})}\ 试证明$$\int_{0}^{\frac{\pi}{2}}{\dfrac{\dx}{\sqrt{4\cos^2{x}+\sin^2{x}}}}=\int_{0}^{\frac{\pi}{2}}{\dfrac{\dx}{\sqrt{\frac{9}{4}\cos^2{x}+2\sin^2{x}}}}$$
        \end{enumerate}
    \item \textbf{(10\songti{分})}\ 设函数$f:[0,1]\to\R,g:[0,1]\to\R$在$[0,1]$上连续,满足$f(0)=g(0),\sin(f(1))=\sin(g(1)),\cos(f(1))=\cos(g(1))$,且
                  $$\forall x\in[0,1],\left(\cos(f(x))+\cos(g(x))\right)^2+\left(\sin(f(x))+\sin(g(x))\right)^2\neq 0$$
                  证明:$f(1)=g(1)$.
\end{enumerate}
\end{document}