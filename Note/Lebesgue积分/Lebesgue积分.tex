\documentclass[a4paper]{ctexart}
\usepackage{template}
\usepackage{array}

\begin{document}\pagestyle{empty}
\begin{center}
    \large Lebesgue积分
\end{center}
在高等数学中,我们所学习的积分是Riemann积分.
\begin{formal}[Riemann Integral]
    记$a=x_0<x_1<x_2<\cdots<x_n=b$,称点列$\{x_i|0\leqslant i\leqslant n\}$为闭区间$[a,b]$的一个\textbf{\songti 分割}.\\
    置$\xi_i\in[x_{i-1},x_{i+1}]$,其中$i\in[1,n]$,称点列$\{x_i|0\leqslant i\leqslant n\}$和$\{\xi_i|1\leqslant i\leqslant n\}$构成$[a,b]$上的\textbf{\songti 取样分割}.\\
    记$\displaystyle\Delta x_i=x_i-x_{i-1},\lambda=\max_{1\leqslant i\leqslant n}\Delta x_i$,若函数$f:[a,b]\to\R$满足$\displaystyle\lim_{\lambda\to 0+}{\sum_{i=0}^{n-1}f(\xi_i)\Delta x_i}$存在,
    则称$f$在$[a,b]$上\textbf{Riemann\songti 可积},记为
    $$\displaystyle\int_a^bf\dx=\lim_{\lambda\to 0+}{\sum_{i=0}^{n-1}f(\xi_i)\Delta x_i}$$
\end{formal}\noindent
Riemann积分通过无限细分后矩形求和的方式求极限.然而,有些函数并不满足Riemann积分的条件,例如著名的Dirichlet函数$D(x)$.
有没有一种办法来计算更广义的积分呢?
这就引入了Lebesgue积分.在介绍Lebesgue积分之前,我们先来简单介绍Lebesgue测度.\\\\
\textbf{0.Preparation}\\首先,我们需要一些准备知识.
\begin{theorem}[0.1 Characteristic Function(示性函数)]
    记集合$S$的示性函数$\mathbf{1}_S:S\to\{0,1\}$,其中
    $\displaystyle\mathbf{1}_S(x)=\left\{\begin{array}{l}0,x\notin S\\1,x\in S\end{array}\right.$
\end{theorem}
\begin{theorem}[0.2 Countable Set(可数集合)]
    记集合$S$是可数的,当且仅当$S$和自然数集$\N$存在双射.\\
    注:此处的可数指无限可数.熟知$\mathbb{Z},\mathbb{Q}$都是可数的.
\end{theorem}
\begin{theorem}[0.3 扩展实数集与级数]
    记扩展实数集$\R^*=\R\cup\left\{-\infty,+\infty\right\}=[-\infty,+\infty]$,定义$0\cdot(\pm\infty)=0$.\\
    记无穷序列$\left\{x_n\right\}$的级数$\displaystyle S_n=\sum_{i=1}^{\infty}{x_i}$.显然,任意改变加和的顺序不会改变级数的敛散性.
\end{theorem}\noindent
\textbf{1.Length}\\
在\textbf{\songti 直觉}上,我们可以说区间$I=[a,b]$的长度为$l(I)=b-a$.这个直觉对于开区间亦成立.
点集$\left\{a\right\}$等价于闭区间$[a,a]$,其长度为$a-a=0$.
同时,定义空集也是一个区间,满足$l(\varnothing)=0$.\\
现在,我们尝试将这个定义扩展到集合上.
\begin{formal}[Definition 1.1]
    对于两两不交的\textbf{\songti 可数}集合列$(I_i)_{i=1}^{\infty}$,记集合$S$满足
    $$S=\bigcup_{i=1}^{\infty}I_j$$
    则\textbf{\songti 尝试}定义$S$的长度$\displaystyle l(S)=\sum_{i=1}^{\infty}\left(b_i-a_i\right)=\sum_{i=1}^{\infty}l(I_i)$.
\end{formal}\noindent
这个定义是否正确和清晰有待后面的考证.\\
根据这个定义,我们可以知道$l(\N)=\sum_{i=1}^{\infty}0=0$.\\
然而,你也许会想$[0,1]$也是有无穷多个点集构成的,为什么它的长度不为$0$呢?
\begin{theorem}[Lemma 1.2]
    [0,1]不可数.更一般的,$\R$和$\R$上所有的区间(不包括点集和空集)不可数.
\end{theorem}\noindent
读者可以自行查询以上引理的证明.\\
\textbf{Lemma 1.2}说明我们不能把$[0,1]$表示成可数个点集的并集,自然也就不能用上述方法求出其长度.\\
事实上,Lebesgue测度中定义的长度并不满足长度无限可加,即只有可数个不交集合的并集的长度才等于各个集合的长度之和.现在,我们用更正式的语言来描述\textbf{\songti 长度}这一概念.\\
对于集合$A\in\R$,如果开区间$(a_1,b_1),(a_2,b_2),\cdots$覆盖了$A$,即$$A\subseteq \bigcup_{i=1}^{\infty}(a_i,b_i)$$
则应有$\displaystyle l(A)\leqslant\sum_{i=1}^\infty(b_i-a_i)$.我们可以通过调整该开区间列来使得右端的级数尽可能地接近$l(A)$.
有时,并不能使得右端的式子严格等于$l(A)$,因此我们采用更广泛适用的下确界来描述该定义.
\begin{formal}[Definition 1.3 Lebesgue外测度]
    定义集合$A\in\R$的Lebesgue外测度$m^*(A)$,满足
    $$m^*(A)=\inf\left\{\left.\sum_{i=1}^{\infty}(b_i-a_i)\right|A\subseteq \bigcup_{i=1}^{\infty}(a_i,b_i)\right\}$$
\end{formal}\noindent
我们允许用无穷可数个开区间进行覆盖.显然,对于给定的集合$A$,$m^*(A)\geqslant 0$,且唯一存在.\\
我们需要对上述定义进行一些说明.
\begin{theorem}[Lemma 1.3.1]
    可数个开区间的并集总是可以写作可数个不交的开区间的并集.
\end{theorem}
\begin{solution}[Proof.]
    对于$a\in A$,定义集合$\displaystyle S_a=\left\{\left.x\right|x\in\R,[x,a]\in A\right\}$.\\
    容易验证这样的$S_a$是开区间,从而$S_a=(\inf{S_a},\sup{S_a})$.\\
    由于每个这样的开区间都至少包含一个有理数,从而最多有可数个这样的集合.
\end{solution}
\begin{theorem}[Lemma 1.3.2]
    允许使用闭区间或半开半闭区间对Lebesgue外测度进行定义,且它们是等价的.
\end{theorem}
\begin{solution}[Proof.]
    定义$$m^*(A)=\inf\left\{\left.\sum_{i=1}^{\infty}l(I_i)\right|A\subseteq \bigcup_{i=1}^{\infty}I_i\right\}$$
    其中$I_i$为区间且两两不交.记$I_i$的端点为$a_i,b_i$.\\
    构造无穷可数的开区间列$\left\{J_n\right\}$满足$J_i=(a_i-2^{-i}\ep,b_i+2^{-i}\ep)$,则有$\displaystyle A\subseteq\bigcup_{i=1}^{\infty}I_i\subseteq\bigcup_{i=1}^{\infty}J_i$.
    又$\displaystyle l\left(\bigcup_{i=1}^{\infty}J_i\right)=l\left(\bigcup_{i=1}^{\infty}I_i\right)+2\ep$,且$\ep$可以任取大于$0$的数,从而
    $$\inf\left\{\left.\sum_{i=1}^{\infty}l(J_i)\right|\ep>0\right\}=\inf\left\{\sum_{i=1}^{\infty}l(I_i)\right\}=m^*(A)$$
    因此,不论采取何种区间定义$m^*(A)$,其结果都是相同的.
\end{solution}
\begin{theorem}[Lemma 1.3.3]
    闭区间$[a,b]\subset\R$的Lebesgue外测度满足$m^*([a,b])=b-a$.
\end{theorem}
\begin{solution}[Proof.]
    取无穷可数的开区间列$\left\{I_n\right\}$满足$I_1=(a-\ep,b+\ep),I_2=I_3=\cdots=\varnothing$.\\
    则有$\displaystyle [a,b]\subset(a-\ep,b+\ep)=\bigcup_{i=1}^{\infty}I_i$,从而
    $$m^*(a,b)\leqslant\inf\left\{\sum_{i=1}^{\infty}l(I_i)\right\}=\inf\left\{\left.b-a+2\ep\right|\ep>0\right\}=b-a$$
\end{solution}
\end{document}