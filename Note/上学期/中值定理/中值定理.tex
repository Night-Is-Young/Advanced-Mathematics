\documentclass{ctexart}
\usepackage{geometry}
\usepackage[dvipsnames,svgnames]{xcolor}
\usepackage{framed}
\usepackage{enumerate}
\usepackage{amsmath,amsthm,amssymb}
\usepackage{enumitem}
\usepackage{template}

\allowdisplaybreaks
\geometry{left=2cm, right=2cm, top=2.5cm, bottom=2.5cm}

\begin{document}
\pagestyle{empty}
\begin{center}\large 中值定理\end{center}
也许您在高等数学的学习过程中见到了各种各样的中值定理(Mean Value Theorem),它们也构成了微积分的重要组成部分.
下面我们就来列举并证明这些中值定理.\\
一.微分中值定理\\
(1).\textbf{Rolle's Mean Value Theorem}(罗尔中值定理)
\begin{formal}[Rolle's Mean Value Theorem]
    如果函数$f(x)$满足
    \begin{enumerate}[leftmargin=*,label=(\alph*)]
        \item $f(x)$在$[a,b]$连续.
        \item $f(x)$在$(a,b)$可导.
        \item $f(x)$在端点的函数值满足$f(a)=f(b)$.
    \end{enumerate}
    则$\exists\xi\in(a,b)\st f'(\xi)=0.$
\end{formal}
\begin{solution}[Proof.]
    首先,由于$f(x)$在$[a,b]$连续,可知$f(x)$在$[a,b]$上有最大值和最小值.\\
    若最大值和最小值均在端点处取得,又$f(a)=f(b)$,则$\forall \xi\in[a,b],f(x)=f(a)$.\\
    从而$f(x)$为常函数,$\forall\xi\in[a,b],f'(\xi)=0$.\\
    若最大值在$(a,b)$上取到,设最大值点为$\xi$,下面证明$f'(\xi)=0$.\\
    由题意$$\forall x\in(a,\xi),\dfrac{f(x)-f(\xi)}{x-\xi}\geqslant 0$$
    $$\forall x\in(\xi,b),\dfrac{f(x)-f(\xi)}{x-\xi}\leqslant 0$$
    则$$f'(\xi-0)=\lim_{\Delta x\to 0-0}{\dfrac{f(\xi+\Delta x)-f(\xi)}{\Delta x}}\geqslant 0$$
    $$f'(\xi+0)=\lim_{\Delta x\to 0+0}{\dfrac{f(\xi+\Delta x)-f(\xi)}{\Delta x}}\leqslant 0$$
    又$f(x)$在$x=\xi$处可导,则$f'(\xi-0)=f'(\xi+0)=0$,从而$f'(\xi)=0$.\\
    若最小值在$(a,b)$上取到,可以通过类似的方法证明之.\\
    综上所述,原定理得证.
\end{solution}\noindent
(2).\textbf{Lagrange's Mean Value Theorem}(拉格朗日中值定理)
\begin{formal}[Lagrange's Mean Value Theorem]
    如果函数$f(x)$满足
    \begin{enumerate}[leftmargin=*,label=(\alph*)]
        \item $f(x)$在$[a,b]$连续.
        \item $f(x)$在$(a,b)$可导.
    \end{enumerate}
    则$\exists\xi\in(a,b)\st f'(\xi)=\dfrac{f(b)-f(a)}{b-a}.$
\end{formal}
\begin{solution}[Proof.]
    令$g(x)=\dfrac{f(b)-f(a)}{b-a}\cdot(x-a)+f(a)-f(x)$\\
    显然,$g(x)$满足Rolle's Theorem的条件,则$\exists\xi\in(a,b)\st g'(\xi)=0$\\
    即$g'(\xi)=\dfrac{f(b)-f(a)}{b-a}-f'(\xi)=0$\\
    即$f'(\xi)=\dfrac{f(b)-f(a)}{b-a}$,从而原定理得证.
\end{solution}\noindent
(3).\textbf{Cauchy's Mean Value Theorem}(柯西中值定理)
\begin{formal}[Cauchy's Mean Value Theorem]
    如果函数$f(x)$和$g(x)$满足
    \begin{enumerate}[leftmargin=*,label=(\alph*)]
        \item $f(x)$和$g(x)$在$[a,b]$连续.
        \item $f(x)$和$g(x)$在$(a,b)$可导.
        \item $\forall x\in(a,b),g'(x)\neq 0$.
    \end{enumerate}
    则$\exists\xi\in(a,b)\st\dfrac{f(b)-f(a)}{g(b)-g(a)}=\dfrac{f'(\xi)}{g'(\xi)}.$
\end{formal}
\begin{solution}[Proof.]
    令$h(x)=f(x)-\dfrac{f(b)-f(a)}{g(b)-g(a)}\cdot g(x)$\\
    显然,$h(x)$满足Rolle's Theorem的条件,则$\exists\xi\in(a,b)\st h'(\xi)=0$\\
    即$h'(\xi)=f'(\xi)-\dfrac{f(b)-f(a)}{g(b)-g(a)}\cdot g'(\xi)=0$\\
    即$\dfrac{f(b)-f(a)}{g(b)-g(a)}=\dfrac{f'(\xi)}{g'(\xi)}$,从而原定理得证.
\end{solution}\noindent
我们可以发现,上述定理是逐渐推广的,但不管形式如何,都可以通过构造辅助函数,进而利用Rolle's Theorem进行证明.\\
\\
二.积分中值定理 \\
(1).积分第一中值定理
\begin{formal}[积分第一中值定理]
    设$f:[a,b]\to\R$在$[a,b]$上连续,$g:[a,b]\to\R$在$[a,b]$上可积且不变号,则
    $$\exists\xi\in(a,b)\st\int_{a}^{b}{f(x)g(x)\dx}=f(\xi)\int_{a}^{b}{g(x)\dx}$$
\end{formal}
\begin{solution}[Proof.]
    不失一般性的,假定$g(x)\geqslant 0$.\\
    由于$f(x)$在$[a,b]$上连续,因此$f(x)$存在最大值和最小值,分别记为$M,m$.\\
    于是有$mg(x)\leqslant f(x)g(x)\leqslant Mg(x)$.对不等式求积分有
    $$\int_{a}^{b}{mg(x)\dx}\leqslant\int_{a}^{b}{f(x)g(x)\dx}\leqslant\int_{a}^{b}{Mg(x)\dx}$$
    即$$m\int_{a}^{b}{g(x)\dx}\leqslant\int_{a}^{b}{f(x)g(x)\dx}\leqslant M\int_{a}^{b}{g(x)\dx}$$
    若$\displaystyle\int_{a}^{b}{g(x)\dx}=0$,则$\displaystyle\forall\xi\in(a,b),\int_{a}^{b}{f(x)g(x)\dx}=f(\xi)\int_{a}^{b}{g(x)\dx}=0$\\
    若$\displaystyle\int_{a}^{b}{g(x)\dx}\neq 0$,则由$g(x)\geqslant 0$可得$\displaystyle\int_{a}^{b}{g(x)\dx}>0$.于是
    $$m\leqslant\dfrac{\int_{a}^{b}{f(x)g(x)\dx}}{\int_{a}^{b}{g(x)\dx}}\leqslant M$$.
    又$m\leqslant f(x)\leqslant M$,根据介值定理,$\exists\xi\in(a,b)\st f(\xi)=\dfrac{\int_{a}^{b}{f(x)g(x)\dx}}{\int_{a}^{b}{g(x)\dx}}$.\\
    即$\displaystyle\int_{a}^{b}{f(x)g(x)\dx}=f(\xi)\int_{a}^{b}{g(x)\dx}$.\\
    对于$g(x)\leqslant 0$,可以采取相似的方法证明.\\
    综上所述,原定理得证.
\end{solution}\noindent
(2).积分第二中值定理
\begin{formal}[积分第二中值定理]
    若函数$f:[a,b]\to\R$在$[a,b]$上黎曼可积,$g:[a,b]\to\R$在$[a,b]$上单调有界,则
    $$\exists\xi\in[a,b]\st\int_{a}^{b}{f(x)g(x)\dx}=g(a)\int_{a}^{\xi}{f(x)\dx}+g(b)\int_{\xi}^{b}{f(x)\dx}$$
\end{formal}
\begin{solution}[Proof(Method I).]
    该方法只适用于研究的函数较为理想的情况,需要$g(x)$在$[a,b]$上可微.\\
    记$\displaystyle F(x)=\int_{a}^{x}{f(x)\dx}$,则
    $$\begin{aligned}
        \int_{a}^{b}{f(x)g(x)\dx}
        &= \int_{a}^{b}{g(x)\di F(x)} \\
        &= \left.F(x)g(x)\right|_{a}^{b}-\int_{a}^{b}{F(x)\di g(x)} \\
        &= F(b)g(b)-F(a)g(a)-\int_{a}^{b}{F(x)g'(x)\dx} \\
        &= g(b)\int_{a}^{x}{f(x)\dx}-\int_{a}^{b}{F(x)g'(x)\dx}
    \end{aligned}$$
    由于$g(x)$在$[a,b]$上单调,则$g'(x)$在$[a,b]$上不变号.\\
    根据积分第一中值定理,$$\exists\xi\in(a,b)\st\int_{a}^{b}{F(x)g'(x)\dx}=F(\xi)\int_{a}^{b}{g'(x)\dx}=F(\xi)(g(b)-g(a))$$
    则有$$\begin{aligned}
        \int_{a}^{b}{f(x)g(x)\dx}
        &= g(b)\int_{a}^{x}{f(x)\dx}-\int_{a}^{b}{F(x)g'(x)\dx} \\
        &= g(b)\int_{a}^{x}{f(x)\dx}-F(\xi)(g(b)-g(a)) \\
        &= g(b)\int_{a}^{x}{f(x)\dx}-(g(b)-g(a))\int_{a}^{\xi}{f(x)\dx} \\
        &= g(a)\int_{a}^{\xi}{f(x)\dx}+g(b)\int_{\xi}^{b}{f(x)\dx}
    \end{aligned}$$
    从而原定理得证.
\end{solution}\noindent
然而这种方法并不适用于所有情况(例如更一般的不可导的$g(x)$).为此,我们需要采取另外的证法.\\
首先,我们来证明Bonnet Theorem.
\begin{formal}[Bonnet Theorem]
    若函数$f:[a,b]\to\R$在$[a,b]$上黎曼可积,$g:[a,b]\to\R$在$[a,b]$上单调递减且非负,则
    $$\exists\xi\in[a,b]\st\int_{a}^{b}{f(x)g(x)\dx}=g(a)\int_{a}^{\xi}{f(x)\dx}$$
\end{formal}
\begin{solution}[Proof.]
    \textbf{证明:}记$\displaystyle F(x)=\int_{a}^{x}{f(x)\dx}$,则$F(x)$在$[a,b]$上连续.
    记$$a=x_0<x_1<x_2<\cdots<x_n=b,\xi=\max_{1\leqslant i\leqslant n}{\left|x_i-x_{i-1}\right|}$$
    记$g(x)$在$[x_{i-1},x_i]$上的振幅(即最值之差)为$w_i(g)$.\\
    则有
    $$\begin{aligned}
        \int_{a}^{b}{f(x)g(x)\dx}
        &= \lim_{\xi\to 0^+}{\sum_{i=1}^{n}{\int_{x_{i-1}}^{x_i}{f(x)g(x)\dx}}} \\
        &= \lim_{\xi\to 0^+}{\sum_{i=1}^{n}{\int_{x_{i-1}}^{x_i}{f(x)\left[g(x)-g(x_i)\right]\dx}}}+\lim_{\xi\to 0^+}{\sum_{i=1}^{n}{\int_{x_{i-1}}^{x_i}{f(x)g(x_i)\dx}}} \\
        &= \lim_{\xi\to 0^+}{\sum_{i=1}^{n}{\int_{x_{i-1}}^{x_i}{\left|f(x)\left[g(x)-g(x_i)\right]\dx\right|}}}+\lim_{\xi\to 0^+}{\sum_{i=1}^{n}{\int_{x_{i-1}}^{x_i}{f(x)g(x_i)\dx}}} \\
        &= \lim_{\xi\to 0^+}{\sum_{i=1}^{n}{\int_{x_{i-1}}^{x_i}{f(x)g(x_i)\dx}}} \\
        &= \lim_{\xi\to 0^+}{\sum_{i=1}^{n}{g(x_i)\left(F(x_i)-F(x_{i-1})\right)}} \\
        &= \lim_{\xi\to 0^+}{\sum_{i=1}^{n}\left[g(x_i)F(x_i)-g(x_{i-1})F(x_{i-1})\right]}+\lim_{\xi\to 0^+}{\sum_{i=1}^{n}\left[g(x_{i-1})-g(x_i)\right]F(x_{i-1})} \\
        &= \left.g(x)F(x)\right|_a^b+\lim_{\xi\to 0^+}{\sum_{i=1}^{n}\left[g(x_{i-1})-g(x_i)\right]F(x_{i-1})} \\
        &= g(b)F(b)-\lim_{\xi\to 0^+}{\sum_{i=1}^{n}\left[g(x_{i})-g(x_{i-1})\right]F(x_{i-1})}
    \end{aligned}$$
    这其中用到了
    $$0\leqslant \lim_{\xi\to 0^+}{\sum_{i=1}^{n}{\int_{x_{i-1}}^{x_i}{\left|f(x)\left[g(x)-g(x_i)\right]\dx\right|}}}\leqslant \lim_{\xi\to 0^+}\max_{[a,b]}{\left|f(x)\right|}{\sum_{i=1}^{n}{w_i(g)(x_i-x_{i-1})}}=0$$
    由于$F(x)$在$[a,b]$连续,不妨设$\displaystyle\max_{[a,b]}F(x)=M_F,\min_{[a,b]}F(x)=m_F$,则
    $$g(b)F(b)-\lim_{\xi\to 0^+}{\sum_{i=1}^{n}\left[g(x_{i})-g(x_{i-1})\right]F(x_{i-1})}\leqslant M_Fg(b)-M_F\sum_{i=1}^{n}{g(x_i)-g(x_{i-1})}=M_Fg(a)$$
    $$g(b)F(b)-\lim_{\xi\to 0^+}{\sum_{i=1}^{n}\left[g(x_{i})-g(x_{i-1})\right]F(x_{i-1})}\geqslant m_Fg(b)-m_F\sum_{i=1}^{n}{g(x_i)-g(x_{i-1})}=m_Fg(a)$$
    从而根据介值定理有
    $$\displaystyle\exists\xi\in[a,b]\st g(b)F(b)-\lim_{\xi\to 0^+}{\sum_{i=1}^{n}\left[g(x_{i})-g(x_{i-1})\right]F(x_{i-1})}=F(\xi)g(a)$$
    整理可得$$\int_{a}^{b}{f(x)g(x)\dx}=g(a)\int_{a}^{\xi}{f(x)\dx}$$
    原定理得证.
\end{solution}\noindent
该定理还有一个等价的形式.
\begin{theorem}[Bonnet Theorem]
    若函数$f:[a,b]\to\R$在$[a,b]$上黎曼可积,$g:[a,b]\to\R$在$[a,b]$上单调递增且非负,则
    $$\exists\xi\in[a,b]\st\int_{a}^{b}{f(x)g(x)\dx}=g(b)\int_{\xi}^{b}{f(x)\dx}$$
\end{theorem}\noindent
下面我们据此证明积分中值第二定理.
\begin{solution}[Proof(Method II).]
    \textbf{证明:}不失一般性的,假定$g(x)$单调递增.则根据Bonnet Theorem有
    $$\exists\xi\in[a,b]\st\int_{a}^{b}{f(x)\left[g(b)-g(x)\right]\dx}=\left[g(b)-g(a)\right]\int_{a}^{\xi}{f(x)\dx}$$
    整理可得$$\int_{a}^{b}{f(x)g(x)\dx}=g(a)\int_{a}^{\xi}{f(x)\dx}+g(b)\int_{\xi}^{b}{f(x)\dx}$$
    原定理得证.
\end{solution}
\end{document}