\documentclass{ctexart}
\usepackage{geometry}
\usepackage[dvipsnames,svgnames]{xcolor}
\usepackage{framed}
\usepackage{enumerate}
\usepackage{amsmath,amsthm,amssymb}
\usepackage{enumitem}
\usepackage{template}

\allowdisplaybreaks
\geometry{left=2cm, right=2cm, top=2.5cm, bottom=2.5cm}

\begin{document}
\pagestyle{empty}
\begin{center}\large 导函数的连续性\end{center}
19世纪时,大部分数学家认为介值定理已经可以刻画出连续函数.
但在1875年,Darboux证明这个想法是错误的,因为连续函数的导函数仍然具有介值性质,但不一定是连续函数.
一个常见的反例如下:
$$f(x)=\left\{\begin{array}{l}
    x^2\sin\dfrac{1}{x},x\neq 0\\
    0,x=0
\end{array}\right.$$
不难得出
$$f'(x)=\left\{\begin{array}{l}
    2x\sin\dfrac{1}{x}-\cos\dfrac{1}{x},x\neq 0\\
    0,x=0
\end{array}\right.$$
于是我们知道$f'(x)$具有介值性,然而在$x=0$处并不连续(具体来说,$x=0$是$f'(x)$的第二类间断点).\\
那么,更加广泛地说,是否所有的导函数都满足介值性呢?Darboux定理告诉我们,这是成立的.
\begin{formal}[Darboux's Theorem]
    设$f:(A,B)\to\R$在开区间$(A,B)$上可导,闭区间$[a,b]\subset(A,B)$.
    那么对介于$f'(a)$和$f'(b)$的任意实数$\eta$,总存在$\xi\in(a,b)$使得$f(\xi)=\eta$.
\end{formal}
\begin{solution}[Proof.]
    设$g(x)=f(x)-\eta x$,于是$g'(x)=f'(x)-\eta$.\\
    不失一般性地,假定$f'(a)<\eta<f'(b)$,于是$g'(a)<0<g'(b)$.\\
    由连续函数的有界性,可知$g(x)$在$[a,b]$上有最小值.由$g'(a)<0<g'(b)$可知$a,b$均不是$g(x)$的最小值点.\\
    于是$\exists\xi\in(a,b)$使得$\displaystyle g(\xi)=\min_{x\in[a,b]}g(x)$.根据费马引理,$g'(\xi)=0$.\\
    从而$f'(\xi)=\eta$,命题得证.
\end{solution}\noindent
于是我们知道导函数是满足介值性的.那么更特殊一点,它是否能更接近一个连续函数呢?
接下来我们证明:导函数不存在第一类间断点.
为了证明这一点,我们首先引入导数极限定理.
\begin{formal}[导数极限定理]
    设函数$f(x):(a,b)\to\R$在$(a,b)$上可导,其导函数记为$f'(x)$.
    对于任意$x_0\in(a,b)$,如果$f'(x)$的左(右)极限存在,那么$f(x)$在$x=x_0$处的左(右)导数一定等于该极限.
\end{formal}
\begin{solution}[Proof.]
    以右侧导数为例.取$x\in(x_0,b)$,根据Lagrange中值定理
    $$\exists\xi\in(x_0,x)\st f'(\xi)=\dfrac{f(x)-f(x_0)}{x-x_0}$$
    置$\Delta x=x-x_0,k=\dfrac{\xi-x_0}{x-x_0}=\dfrac{\xi-x_0}{\Delta x}\in(0,1)$,于是$\xi=x_0+k\Delta x$.\\
    假定$\displaystyle\lim_{x\to x_0^+}f'(x)$存在,那么我们有
    $$f'_+(x_0)=\lim_{\Delta x\to 0^+}\dfrac{f(x+\Delta x)-f(x)}{\Delta x}=\lim_{\Delta x\to 0^+}f'(\xi)=\lim_{\Delta x\to 0^+}f'(x_0+k\Delta x)=\lim_{x\to x_0^+}f'(x)$$
    左侧导数的证明过程类似.于是命题得证.
\end{solution}\noindent
基于上述定理,我们可以马上得出下一结论.
\begin{problem}[Example 3.]
    试证明:对于$(a,b)$上的可微函数$f(x)$有
    $$\forall x_0\in(a,b),\text{若}\lim_{x\to x_0}f'(x)\text{存在,则}\lim_{x\to x_0}f'(x)=f'(x_0)$$
    更一般的,只需$\displaystyle\lim_{x\to x_0^+}f'(x)$和$\displaystyle\lim_{x\to x_0^-}f'(x)$分别存在,上述命题就成立.
\end{problem}
\begin{solution}[Proof(Method I).]
    根据导数极限定理,我们有
    $$\lim_{x\to x_0^+}f'(x)=f'_+(x_0)$$
    $$\lim_{x\to x_0^-}f'(x)=f'_-(x_0)$$
    又$f(x)$在$x=x_0$处可导,于是
    $$f'_-(x_0)=f'_+(x_0)=f'(x_0)$$
    进而$$\lim_{x\to x_0}f'(x)=f'(x_0)$$
\end{solution}
\begin{solution}[Proof(Method II).]
    我们也可以用Darboux定理证明之.
\end{solution}
\begin{analyze}
    关于\tbf{Example 3}有一个错误的推广证明.\\
    假定$f(x)$在$(a,b)$上可导,于是对于任意$x_0\in(a,b)$,取$x\in(a,b)\backslash\left\{x_0\right\}$,据Lagrange中值定理有
    $$\exists\xi,\text{满足}x_0\lessgtr\xi\lessgtr x,\text{使得}f'(\xi)=\dfrac{f(x)-f(x_0)}{x-x_0}$$
    于是$$f'(x_0)=\lim_{x\to x_0}\dfrac{f(x)-f(x_0)}{x-x_0}=\lim_{x\to x_0}f'(\xi)$$
    注意到$x\to x_0$将迫使$\xi\to x_0$,于是
    $$f'(x_0)=\lim_{x\to x_0}f'(\xi)=\lim_{\xi\to x_0}f'(\xi)$$
    从而$f(x)$在$x=x_0$处连续,进而$f(x)$在$(a,b)$上连续.\\
    然而,这种证明有着根本上的问题.
\end{analyze}
\end{document}