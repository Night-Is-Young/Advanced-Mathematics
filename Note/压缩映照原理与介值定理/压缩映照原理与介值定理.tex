\documentclass{ctexart}
\usepackage{geometry}
\usepackage[dvipsnames,svgnames]{xcolor}
\usepackage[strict]{changepage}
\usepackage{framed}
\usepackage{enumerate}
\usepackage{amsmath,amsthm,amssymb}
\usepackage{enumitem}
\usepackage{template}

\geometry{left=2cm, right=2cm, top=2.5cm, bottom=2.5cm}

\begin{document}
\pagestyle{empty}
\begin{center}\large 介值定理和压缩映照原理\end{center}
在高等数学书中,虽然提到了介值定理,却并不要求它的证明.下面我们来证明介值定理.\\
首先,需要引入确界原理.确界原理实际上是实数完备性的一种表现形式.
\begin{formal}[确界原理]
    任何上(下)方有界的非空集合一定存在上(下)界.
\end{formal}
\begin{theorem}
    下面以上确界为例给出确界的定义.\\
    设集合$S$满足$S\subset \R$.若实数$\alpha$满足:
    \begin{enumerate}[label=(\alph*)]
        \item $\forall x\in S,x\leqslant\alpha$\ (满足这一条即可称$\alpha$为$S$的一个上界)
        \item $\forall\ep>0,\exists x\in S\st x>\alpha-\ep$
    \end{enumerate}
    则称$\alpha$为$S$的上确界.
    下确界的定义类似,在此不再赘述.\\
    我们记$S$的上确界为$\sup{S}$,下确界为$\inf{S}$.
\end{theorem}\noindent
现在我们根据确界原理来证明介值定理.
\begin{formal}[介值定理]
    设$f:[a,b]\to\mathbb{R}$为定义在$[a,b]$上的连续函数,则$\forall\eta\in\R$满足$f(a)\lessgtr\eta\lessgtr f(b),\exists\xi\in(a,b)\st f(\xi)=\eta$.
\end{formal}
\begin{solution}
    \textbf{证明:}不妨设$f(a)<\eta<f(b)$.\\
    记集合$S=\left\{x|f(x)\leqslant \eta\right\}\subset[a,b].$由$f(a)<\eta$可知$S$非空.易知$S$具有一个上界$b$.\\
    根据确界原理,$S$存在上确界,记$c=\sup{S}$.下面证明$f(c)=\eta$.\\
    首先,由于$f(x)$在$x=c$处连续,则$\forall\ep>0,\exists\delta>0\st\forall x\in(c-\delta,c+\delta),\left\lvert f(x)-f(c)\right\rvert<\ep$.\\
    采取反证法.若$f(c)>\eta$,取$\ep=f(c)-\eta$,则$\exists\delta>0\st\forall x\in(c-\delta,c+\delta),\left\lvert f(x)-f(c)\right\rvert<f(c)-\eta$.\\
    由于$c$是$S$的上确界,可知$\exists y\in(c-\delta,c]\st y\in S$.\\
    此时有$f(y)-f(c)>\eta-f(c)$,即$f(y)>\eta$,这与$y\in S$不符.故$f(c)\leqslant\eta$.\\
    若$f(c)<\eta$,同理可知$\exists\delta>0\st\forall x\in(c-\delta,c+\delta),\left\lvert f(x)-f(c)\right\rvert<\eta-f(c)$.\\
    设$y\in(c,c+\delta)$,则$f(y)-f(c)<\eta-f(c)$,即$f(y)<\eta$,从而$y\in S$.又$y>c$,这与$c=\sup{S}$矛盾,故$f(c)\geqslant\eta$.\\
    综上所述,$f(c)=\eta$,原定理得证.
\end{solution}\noindent
现在我们尝试运用介值定理来证明压缩映照原理.
\begin{formal}[压缩映照原理]
    设$f:[a,b]\to[a,b]$连续,且$\exists q\in(0,1)\st\forall x,y\in[a,b],\left\lvert f(x)-f(y)\right\rvert\leqslant q\left\lvert x-y\right\rvert$,
    则$\exists!c\in[a,b]\st f(c)=c$.
\end{formal}
\begin{solution}[Proof.]
    首先说明$f(x)$的连续性.\\
    $\forall x_0\in [a,b],\forall \varepsilon>0,\exists \delta=\dfrac{\varepsilon}{q}>0\st
    \forall x\in[x_0-\delta,x_0+\delta]\backslash\left\{x_0\right\},
    \left\lvert f(x)-f(x_0)\right\rvert\leqslant q\left\lvert x-x_0\right\rvert\leqslant q\delta=\varepsilon$.\\
    根据极限的定义有$\lim_{x\to x_0}{f(x)}=f(x_0)$,即$f(x)$在$x=x_0$处连续,进而$f(x)$在$[a,b]$连续.\\
    下面说明$f(x)$存在唯一不动点.\\
    假定存在$c_1,c_2$满足$f(c_i)=c_i(i=1,2)$,则
    $\left\lvert f(c_1)-f(c_2)\right\rvert=\left\lvert c_1-c_2\right\rvert >q\left\lvert c_1-c_2\right\rvert$,与题意不符.\\
    故$f(x)$至多有一个不动点.\\
    记$g(x)=f(x)-x$,显然$g(x)$在$[a,b]$连续.下面说明$g(x)$在$[a,b]$上必然存在零点.
    \begin{enumerate}[label=(\arabic*)]
        \item 若$g(a)=0$或$g(b)=0$,则$g(x)$显然存在零点.
        \item 若$g(a)>0$且$g(b)<0$,则根据介值定理可知$\exists\xi\in(a,b)$使得$g(\xi)=0$.
    \end{enumerate}
    综上,原命题得证.
\end{solution}\noindent
压缩映照原理还有两个引理.
\begin{formal}[Lemma 1]
    设$f:[a,b]\to[a,b]$连续,且$\exists q\in(0,1)\st\forall x,y\in[a,b],\left\lvert f(x)-f(y)\right\rvert\leqslant q\left\lvert x-y\right\rvert$.\\
    设序列$\left\{ x_n\right\}$满足$x_1\in[a,b],x_{n+1}=f(x_n)$,则$\displaystyle\lim_{n\to\infty}{x_n}=c$且$f(c)=c$.
\end{formal}
\begin{solution}[Proof.]
    我们已经证明$f(x)$在$[a,b]$上连续且存在唯一不动点.\\
    下面说明$\left\{ x_n\right\}$收敛.我们有
    $$\begin{aligned}
        \left|x_{n+1}-c\right| 
        &= \left|f(x_n)-f(c)\right| \\
        &\leqslant q\left|x_n-c\right| \\
        &< \left|x_n-c\right|
    \end{aligned}$$
    考虑序列$\left\{ y_n\right\}$,其中$y_n=\left|x_n-c\right|$.\\
    根据上式可知$\left| y_n\right|$单调递减,又$y_n>0$,故$\left\{ y_n\right\}$有极限.\\
    从而$\left\{x_n-c\right\}$的极限存在.设$\displaystyle\lim_{n\to\infty}{x_n}=A$,对递推式$x_{n+1}=f(x_n)$两边取极限有
    $$\lim_{n\to\infty}{x_{n+1}}=\lim_{n\to\infty}{f(x_n)}$$
    即$$A=f(A)$$
    从而$A=c$,即$\displaystyle\lim_{n\to\infty}{x_n}=c.$
\end{solution}
\begin{formal}[Lemma 2]
    设$f:\mathbb{R}\to\mathbb{R}$,$\exists q\in(0,1)\st\forall x,y\in\mathbb{R},\left\lvert f(x)-f(y)\right\rvert\leqslant q\left\lvert x-y\right\rvert$,则$f(x)$连续且$\exists!c\in\mathbb{R}\st f(c)=c$.
\end{formal}
\begin{solution}[Proof.]
    易知$f(x)$的连续性和不动点少于或等于$1$个.下面我们来证明$f(x)$存在不动点.\\
    记$g(x)=f(x)-x$,则$g(x)$在$\mathbb{R}$上连续.\\
    由题意可知$\exists q\in(0,1)\st\forall x,y\in\mathbb{R},\left\lvert g(x)-g(y)-x+y\right\rvert\leqslant q\left\lvert x-y\right\rvert$.\\
    不妨假定$x>y$,则有$(1-q)(x-y)\leqslant g(x)-g(y)\leqslant(1+q)(x-y).$\\
    则有$\forall x>y,g(x)>g(y)$且$\dfrac{g(x)-g(y)}{x-y}\geqslant 1-q$.\\
    现在证明$\exists a,b\in\mathbb{R},g(a)\leqslant 0\leqslant g(b)$.\\
    我们采取反证法.假设$\forall x\in\mathbb{R},g(x)<0$.
    取$g(x_0)=p$,则$g(x_0-\dfrac{p}{1-q})\geqslant g(x_0)+(1-q)\cdot\dfrac{-p}{1-q}=0$.\\
    这与$g(x)<0$不符.同理$\forall x\in\mathbb{R},g(x)>0$也是不成立的.\\
    根据介值定理,$\exists\xi\in[a,b]\st g(\xi)=0$,即$f(x)$存在不动点.\\
    综上所述,$f(x)$在$\mathbb{R}$上连续且存在唯一不动点.
\end{solution}
\end{document}