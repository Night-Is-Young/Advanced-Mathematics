\documentclass{ctexart}
\usepackage{geometry}
\usepackage[dvipsnames,svgnames]{xcolor}
\usepackage{framed}
\usepackage{enumerate}
\usepackage{amsmath,amsthm,amssymb}
\usepackage{enumitem}
\usepackage{template}

\allowdisplaybreaks
\geometry{left=2cm, right=2cm, top=2.5cm, bottom=2.5cm}

\begin{document}
\pagestyle{empty}
\begin{center}\large 梯度、散度和Laplace算子\end{center}
"在重力场中,您将沿着负梯度最大的地方向下滚动."\\
(如无额外的说明,以下的讨论都将以二元函数$f:\R^2\to\R$为例.)\\
\tbf{1.\songti 方向导数和梯度}\\
我们知道,函数$f(x,y)$的偏导数$\dfrac{\p{f}}{\p{x}},\dfrac{\p{f}}{\p{y}}$反映了$f(x,y)$分别沿$x,y$轴正方向的变化率.
然而,有时候函数在别的方向上的导数并不能简单地通过这两个偏导数反映.为此,我们引入了方向导数.
\begin{formal}[1.1 \songti 方向导数的定义]
    设函数$z=f(x,y)$在点$P_0(x_0,y_0)$及其周围的一个邻域内有定义,又设$\vect{l}$是一个给定的方向向量,其方向余弦为$(\cos\alpha,\cos\beta)$.
    若极限$$\lim_{t\to0}\dfrac{f(x_0+t\cos\alpha,y_0+t\cos\beta)-f(x_0,y_0)}{t}$$
    存在,则称此极限为$z=f(x,y)$在$P_0$处沿方向$\vect{l}$的\textbf{方向导数},记作
    $$\left.\dfrac{\p{z}}{\p\vect{l}}\right|_{(x_0,y_0)}\ \ \ \ \ \left.\dfrac{\p{z}}{\p\vect{l}}\right|_{P_0}\ \ \ \ \ \left.\dfrac{\p{f}}{\p\vect{l}}\right|_{(x_0,y_0)}\ \ \ \ \ \left.\dfrac{\p{f}}{\p\vect{l}}\right|_{P_0}$$
\end{formal}\noindent
使用函数极限的方法计算方向导数仍然是比较麻烦的.我们有以下定理计算方向导数.
\begin{theorem}[1.2 \songti 方向导数的计算方法]
    若函数$z=f(x,y)$在$P_0(x_0,y_0)$处可微,则$f(x,y)$在$P_0$处的任意方向$\vect{l}$上的方向导数均存在,且满足
    $$\left.\dfrac{\p{f}}{\p\vect{l}}\right|_{P_0}=\left.\dfrac{\p{f}}{\p{x}}\right|_{P_0}\cos\alpha+\left.\dfrac{\p{f}}{\p{y}}\right|_{P_0}\cos\beta$$
    其中$(\cos\alpha,\cos\beta)$为$\vect{l}$的方向余弦.
\end{theorem}
\begin{solution}[Proof.]
    考虑$\vect{l}$上的一点$P_t(x_0+t\cos\alpha,y_0+t\cos\beta)$.
    由于$f(x,y)$在$(x_0,y_0)$处可微,于是
    $$\begin{aligned}
        f(P_t)-f(P_0)
        &= f(x_0+t\cos\alpha,y_0+t\cos\beta)-f(x_0,y_0) \\
        &= \left.\dfrac{\p{f}}{\p{x}}\right|_{P_0}\cos\alpha+\left.\dfrac{\p{f}}{\p{y}}\right|_{P_0}\cos\beta+o(\rho)
    \end{aligned}$$
    由于$\rho=\left|t\right|$,于是
    $$\begin{aligned}
        \left.\dfrac{\p{f}}{\p\vect{l}}\right|_{P_0}
        &= \lim_{t\to0}\dfrac{f(x_0+t\cos\alpha,y_0+t\cos\beta)-f(x_0,y_0)}{t} \\
        &= \left.\dfrac{\p{f}}{\p{x}}\right|_{P_0}\cos\alpha+\left.\dfrac{\p{f}}{\p{y}}\right|_{P_0}\cos\beta
    \end{aligned}$$
    证毕.\\
    上述定理告诉了我们在函数可微的情形下计算方向导数的方法.
\end{solution}\noindent
我们知道方向导数反应了函数沿某一方向上的变化率,那么我们不禁开始思考$f(x,y)$在$P_0$的各个方向上的方向导数是否存在最大值?
在什么方向上可以取到该最大值?\\
我们引入向量$\vect{g}=\left(\left.\dfrac{\p{f}}{\p{x}}\right|_{P_0},\left.\dfrac{\p{f}}{\p{y}}\right|_{P_0}\right)$和$\vect{l}$的单位向量$\vect{l}_0=(\cos\alpha,\cos\beta)$,于是
$$\left.\dfrac{\p{f}}{\p\vect{l}}\right|_{P_0}=\vect{g}\cdot\vect{l}_0=\left|\vect{g}\right|\left|\vect{l}_0\right|\cos\langle\vect{g},\vect{l}_0\rangle\leqslant\left|\vect{g}\right|$$
等号成立当且仅当$\langle\vect{g},\vect{l}_0\rangle=0$,即两者同向.这就是梯度的定义.
\begin{formal}[1.3 \songti 梯度的定义]
    若函数$f(x,y)$在$P_0(x_0,y_0)$处可微,则称向量
    $$\vect{g}=\left(\left.\dfrac{\p{f}}{\p{x}}\right|_{P_0},\left.\dfrac{\p{f}}{\p{y}}\right|_{P_0}\right)$$
    为$f(x,y)$在$P_0$处的梯度,记作
    $$\left.\mbf{grad}f\right|_{P_0}=\left(\left.\dfrac{\p{f}}{\p{x}}\right|_{P_0},\left.\dfrac{\p{f}}{\p{y}}\right|_{P_0}\right)$$
\end{formal}\noindent
不难发现,当$\langle\vect{g},\vect{l}_0\rangle=\pi$,即两者反向时方向导数最小,且其值恰为$-\left|\vect{g}\right|$.
我们称向量$$-\left.\mbf{grad}f\right|_{P_0}=\left(\left.-\dfrac{\p{f}}{\p{x}}\right|_{P_0},\left.-\dfrac{\p{f}}{\p{y}}\right|_{P_0}\right)$$
为$f(x,y)$在$P_0$的\textbf{\songti 负梯度}.\\
需要注意的是,梯度代表的并不是一个值,而是一个向量.即:梯度运算得到的是一个向量.\\
在物理学和数学中,我们广泛地运用Nabla算子$\nabla=\dfrac{\p}{\p{x}}\vect{i}+\dfrac{\p}{\p{y}}\vect{j}$计算梯度,即
$$\mbf{grad}f=\nabla{f}=\dfrac{\p{f}}{\p{x}}\vect{i}+\dfrac{\p{f}}{\p{y}}\vect{j}$$
需要注意的是,我们这里得到的$\mbf{grad}f$实际上是一个指的是函数$\mbf{grad}f:\R^2\to\R^2$,即一个以二维向量为函数值的函数.
$\mbf{grad}f(x_0,y_0)$即$f(x,y)$在$x_0,y_0$处的梯度.\\
\tbf{2.散度}\\
我们不难发现,对一个函数$f(x,y):\R^2\to\R$求梯度将会得到一个向量场,在任意一点处的向量反映了
$f(x,y)$在该点处的变化方向和幅度.\\
想象空间内流动着的液体,将它与一个向量场对应,其中每一点的流速和流向与该点的向量值相对应.
值得思考的是,这些液体是否有一同流向某一点(想象一个下水口)或一同从某一点流出(想象一个喷泉)的时候?
为此,我们引入了散度来描述这一性质.\\
一种直观理解散度的方式是对于任意一点$P_0(x,y)$,观察指向这一点的向量多还是从这一点指出的向量多.\\
经过推导,我们可以得出以下定义.
\begin{formal}[2.1 散度的定义]
    对于向量场$\vect{F}(x,y):\R^2\to\R^2$,其散度为
$$\mbf{div}\vect{F}=\dfrac{\p{F_x}}{\p{x}}+\dfrac{\p{F_y}}{\p{y}}$$
其中${F_x},{F_y}$分别代表$\vect{F}$在$x,y$方向上的分量,即$\vect{F}(x,y)=F_x(x,y)\vect{i}+F_y(x,y)\vect{j}$.
\end{formal}\noindent
从而我们有
$$\mbf{div}\vect{F}=\nabla\vect{F}$$
所以Nabla算子也可以作用于向量场求散度.\\
\tbf{3.Laplace算子}\\
假定我们有函数$f(x,y):\R^2\to\R$,对$f$的梯度求散度有
$$\mbf{div\ grad}f=\nabla\cdot\nabla f=\dfrac{\p^2{f}}{\p{x^2}}+\dfrac{\p^2{f}}{\p{y^2}}$$
这就是Laplace算子$\Delta=\nabla^2=\dfrac{\p^2}{\p{x^2}}+\dfrac{\p^2}{\p{y^2}}$.\\
关于其物理意义,这里就不再赘述.我们主要关注一些有关的二阶偏微分的计算.
\begin{problem}[Example 3.1]
    设函数$u(x,y):\R^2\to\R$有连续的二阶偏导数且满足Laplacian方程
    $$\dfrac{\p^2u}{\p{x^2}}+\dfrac{\p^2u}{\p{y^2}}=0$$
    作变量代换$$x=\e^s\cos t,y=\e^s\sin t$$
    后,试证明$u$依然满足对$s,t$的Laplacian方程
    $$\dfrac{\p^2u}{\p{s^2}}+\dfrac{\p^2u}{\p{t^2}}=0$$
\end{problem}
\begin{solution}[Proof.]
    由复合函数的二阶偏微分公式可得
    $$\dfrac{\p{u}}{\p{s}}=\dfrac{\p{u}}{\p{x}}\cdot\dfrac{\p{x}}{\p{s}}+\dfrac{\p{u}}{\p{y}}\cdot\dfrac{\p{y}}{\p{s}}=\dfrac{\p{u}}{\p{x}}\e^s\cos t+\dfrac{\p{u}}{\p{y}}\e^s\sin t$$
    同理
    $$\dfrac{\p{u}}{\p{t}}=\dfrac{\p{u}}{\p{x}}\cdot\dfrac{\p{x}}{\p{t}}+\dfrac{\p{u}}{\p{y}}\cdot\dfrac{\p{y}}{\p{t}}=-\dfrac{\p{u}}{\p{x}}\e^s\sin t+\dfrac{\p{u}}{\p{y}}\e^s\cos t$$
    于是
    $$\begin{aligned}
        \dfrac{\p^2u}{\p{s^2}}
        &= \dfrac{\p}{\p{s}}\left(\dfrac{\p{u}}{\p{s}}\right) \\
        &= \dfrac{\p}{\p{s}}\left(\dfrac{\p{u}}{\p{x}}\e^s\cos t+\dfrac{\p{u}}{\p{y}}\e^s\sin t\right) \\
        &= \e^s\cos t\left(\dfrac{\p{u}}{\p{x}}+\dfrac{\p^2u}{\p{s}\p{x}}\right)+\e^s\sin t\left(\dfrac{\p{u}}{\p{y}}+\dfrac{\p^2u}{\p{s}\p{y}}\right) \\
        &= \dfrac{\p{u}}{\p{x}}\e^s\cos t+\dfrac{\p{u}}{\p{y}}\e^s\sin t+\dfrac{\p^2{u}}{\p{x^2}}\left(\e^s\cos t\right)^2+\dfrac{\p^2{u}}{\p{y^2}}\left(\e^s\sin t\right)^2
    \end{aligned}$$
    上述变形中作了代换$\displaystyle\dfrac{\p^2u}{\p{s}\p{x}}=\dfrac{\p{x}}{\p{s}}\cdot\dfrac{\p^2u}{\p{x^2}}$以化简.\\
    同理有
    $$\begin{aligned}
        \dfrac{\p^2u}{\p{t^2}}
        &= \dfrac{\p}{\p{t}}\left(\dfrac{\p{u}}{\p{t}}\right) \\
        &= \dfrac{\p}{\p{t}}\left(-\dfrac{\p{u}}{\p{x}}\e^s\sin t+\dfrac{\p{u}}{\p{y}}\e^s\cos t\right) \\
        &= -\dfrac{\p{u}}{\p{x}}\e^s\cos t-\dfrac{\p{u}}{\p{y}}\e^s\sin t+\dfrac{\p^2{u}}{\p{x^2}}\left(\e^s\sin t\right)^2+\dfrac{\p^2{u}}{\p{y^2}}\left(\e^s\cos t\right)^2
    \end{aligned}$$
    从而
    $$\dfrac{\p^2u}{\p{s^2}}+\dfrac{\p^2u}{\p{t^2}}=\e^{2s}\left(\sin^2t+\cos^2t\right)\left(\dfrac{\p^2u}{\p{x^2}}+\dfrac{\p^2u}{\p{y^2}}\right)=0$$
    证毕.
\end{solution}
\begin{problem}[Example 3.2]
    试证明:球坐标系下Laplace算子为
    $$\Delta=\dfrac{1}{r^2}\dfrac{\p}{\p{r}}\left(r^2\dfrac{\p}{\p{r}}\right)+\dfrac{1}{r^2\sin\theta}\dfrac{\p}{\p\theta}\left(\sin\theta\dfrac{\p}{\p\theta}\right)+\dfrac{1}{r^2\sin^2\theta}\dfrac{\p^2}{\p\phi^2}$$
\end{problem}
\begin{solution}[Proof.]
    我们知道笛卡尔坐标系下的Laplace算子为
    $$\Delta=\dfrac{\p^2}{\p{x^2}}+\dfrac{\p^2}{\p{y^2}}+\dfrac{\p^2}{\p{z^2}}$$
    以及笛卡尔坐标系向球坐标系的变换
    $$\left\{\begin{array}{l}
    x=r\sin\theta\cos\phi\\
    y=r\sin\theta\sin\phi\\
    z=r\cos\theta
    \end{array}\right.$$
    考虑函数$f(x,y,z):\R^3\to\R$,则有
    $$\dfrac{1}{r^2}\dfrac{\p}{\p{r}}\left(r^2\dfrac{\p{f}}{\p{r}}\right)
    =\dfrac{1}{r^2}\left(2r\dfrac{\p{f}}{\p{r}}+r^2\dfrac{\p^2 f}{\p{r^2}}\right)
    =\dfrac{2}{r}\dfrac{\p{f}}{\p{r}}+\dfrac{\p^2f}{\p{r^2}}
    $$
    $$\dfrac{1}{r^2\sin\theta}\dfrac{\p}{\p\theta}\left(\sin\theta\dfrac{\p{f}}{\p\theta}\right)
    =\dfrac{1}{r^2\sin\theta}\left(\cos\theta\dfrac{\p{f}}{\p{\theta}}+\sin\theta\dfrac{\p^2f}{\p\theta^2}\right)
    =\dfrac{1}{r^2\tan\theta}\dfrac{\p{f}}{\p{\theta}}+\dfrac{1}{r^2}\dfrac{\p^2f}{\p\theta^2}$$
    而
    $$\dfrac{\p^2{f}}{\p{r^2}}=\dfrac{\p}{\p{r}}\left(\dfrac{\p{f}}{\p{r}}\right)=\dfrac{\p}{\p{r}}\left(\dfrac{\p{f}}{\p{x}}\cdot\dfrac{\p{x}}{\p{r}}+\dfrac{\p{f}}{\p{y}}\cdot\dfrac{\p{y}}{\p{r}}+\dfrac{\p{f}}{\p{z}}\cdot\dfrac{\p{z}}{\p{r}}\right)$$
    我们知道$$\dfrac{\p}{\p{r}}\left(\dfrac{\p{f}}{\p{x}}\cdot\dfrac{\p{x}}{\p{r}}\right)=\dfrac{\p^2x}{\p{r^2}}\cdot\dfrac{\p{f}}{\p{x}}+\dfrac{\p^2f}{\p{x^2}}\cdot\left(\dfrac{\p{x}}{\p{r}}\right)^2$$
    而
    $$\dfrac{\p{x}}{\p{r}}=\sin\theta\cos\phi,\dfrac{\p^2x}{\p{r^2}}=0$$
    $$\dfrac{\p{y}}{\p{r}}=\sin\theta\sin\phi,\dfrac{\p^2y}{\p{r^2}}=0$$
    $$\dfrac{\p{z}}{\p{r}}=\cos\theta,\dfrac{\p^2z}{\p{r^2}}=0$$
    代入可得
    $$\begin{aligned}
        \dfrac{1}{r^2}\dfrac{\p}{\p{r}}\left(r^2\dfrac{\p{f}}{\p{r}}\right)
        &= \sin^2\theta\cos^2\phi\dfrac{\p^2f}{\p{x^2}}+\sin^2\theta\sin^2\phi\dfrac{\p^2f}{\p{y^2}}+\sin^2\theta\dfrac{\p^2f}{\p{z^2}}\\
        &+ \dfrac{2}{r}\left(\sin\theta\cos\phi\dfrac{\p{f}}{\p{x}}+\sin\theta\sin\phi\dfrac{\p{f}}{\p{y}}+\cos\theta\dfrac{\p{f}}{\p{z}}\right)
    \end{aligned}$$
    又
    $$\dfrac{\p{x}}{\p{\theta}}=r\cos\theta\cos\phi,\dfrac{\p^2x}{\p{\theta^2}}=-r\sin\theta\cos\phi$$
    $$\dfrac{\p{y}}{\p{\theta}}=r\cos\theta\sin\phi,\dfrac{\p^2y}{\p{\theta^2}}=-r\sin\theta\sin\phi$$
    $$\dfrac{\p{z}}{\p{\theta}}=-r\sin\theta,\dfrac{\p^2z}{\p{\theta^2}}=-r\cos\theta$$
    于是
    $$\begin{aligned}
        \dfrac{1}{r^2\tan\theta}\dfrac{\p{f}}{\p{\theta}}+\dfrac{1}{r^2}\dfrac{\p^2f}{\p\theta^2}
        &= \cos^2\theta\cos^2\phi\dfrac{\p^2f}{\p{x^2}}+\cos^2\theta\sin^2\phi\dfrac{\p^2f}{\p{y^2}}+\cos^2\theta\dfrac{\p^2f}{\p{z^2}} \\
        &+ \dfrac{1}{r}\left(\dfrac{\cos2\theta}{\sin\theta}\cos\phi\dfrac{\p{f}}{\p{x}}+\dfrac{\cos2\theta}{\sin\theta}\sin\phi\dfrac{\p{f}}{\p{y}}-2\cos\theta\dfrac{\p{f}}{\p{z}}\right)
    \end{aligned}$$
    又
    $$\dfrac{\p{x}}{\p{\phi}}=-r\sin\theta\sin\phi,\dfrac{\p^2{x}}{\p{\phi^2}}=-r\sin\theta\cos\phi$$
    $$\dfrac{\p{y}}{\p{\phi}}=r\sin\theta\cos\phi,\dfrac{\p^2{y}}{\p{\phi^2}}=-r\sin\theta\sin\phi$$
    $$\dfrac{\p{z}}{\p\phi}=\dfrac{\p^2z}{\p\phi^2}=0$$
    于是
    $$\begin{aligned}
        \dfrac{1}{r^2\sin^2\theta}\dfrac{\p^2f}{\p\phi^2}
        &=\sin^2\phi\dfrac{\p^2f}{\p{x^2}}+\cos^2\phi\dfrac{\p^2f}{\p{y^2}}-\dfrac{1}{r}\left(\dfrac{\cos\phi}{\sin\theta}\dfrac{\p{f}}{\p{x}}+\dfrac{\sin\phi}{\sin\theta}\dfrac{\p{f}}{\p{y}}\right)
    \end{aligned}$$
    将三项相加,考虑如下几项偏导数的系数
    $$\begin{aligned}
        \dfrac{\p{f}}{\p{x}}&:\dfrac{1}{r}\left(2\sin\theta\cos\phi+\dfrac{\cos2\theta}{\sin\theta}\cos\phi-\dfrac{\cos\phi}{\sin\theta}\right)=\dfrac{\cos\phi}{r\sin\theta}\left(2\sin^2\theta-1+\cos2\theta\right)=0\\
        \dfrac{\p{f}}{\p{y}}&:\dfrac{1}{r}\left(2\sin\theta\sin\phi+\dfrac{\cos2\theta}{\sin\theta}\sin\phi-\dfrac{\sin\phi}{\sin\theta}\right)=\dfrac{\sin\phi}{r\sin\theta}\left(2\sin^2\theta-1+\cos2\theta\right)=0\\
        \dfrac{\p{f}}{\p{z}}&:\dfrac{2}{r}\cos\theta-\dfrac{1}{r}\cdot2\cos\theta=0\\
        \dfrac{\p^2{f}}{\p{x^2}}&:\sin^2\theta\cos^2\phi+\cos^2\theta\cos^2\phi+\sin^2\phi=1\\
        \dfrac{\p^2{f}}{\p{y^2}}&:\sin^2\theta\sin^2\phi+\cos^2\theta\sin^2\phi+\cos^2\phi=1\\
        \dfrac{\p^2{f}}{\p{y^2}}&:\sin^2\theta+\cos^2\theta=1
    \end{aligned}$$
    于是
    $$\dfrac{1}{r^2}\dfrac{\p}{\p{r}}\left(r^2\dfrac{\p{f}}{\p{r}}\right)+\dfrac{1}{r^2\sin\theta}\dfrac{\p}{\p\theta}\left(\sin\theta\dfrac{\p{f}}{\p\theta}\right)+\dfrac{1}{r^2\sin^2\theta}\dfrac{\p^2f}{\p\phi^2}
    =\dfrac{\p^2f}{\p{x^2}}+\dfrac{\p^2f}{\p{y^2}}+\dfrac{\p^2f}{\p{z^2}}$$
    从而原命题得证.
\end{solution}
\end{document}