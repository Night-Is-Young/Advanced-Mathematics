\documentclass[a4paper,oneside]{ctexart}
\usepackage{amsmath, amsthm, amssymb,geometry,enumerate,color,xcolor}
\geometry{left=2cm, right=2cm, top=2.5cm, bottom=2.5cm}
\newcommand{\e}{\mathrm{e}}
\newcommand{\di}{\mathrm{d}}
\linespread{1.55}
\begin{document}
\pagestyle{empty}
\begin{center}\large Stirling 近似\end{center}
\textbf{前言}:在统计物理学等多个领域中,我们常常需要估计$N!$的值,其中$N$是一个相当大的整数.
按照定义的连乘算法通常并不高效,其复杂度为$O(N)$,而且一般的计算工具也难以存储如此巨大的数据.
因此,我们需要一个良好的近似来估计${N!}$的值.Stirling近似正是实现了这一目标的方法,
它以$O(\log{n})$的复杂度对$N!$进行了较为精确的估计.\\
下面是Stirling近似的具体内容和推广.
我们首先采取积分近似法对$\ln{N!}$进行估计.\\
以下是采用矩形放缩得到的结果.\\
\colorbox{lightgray}{\textbf{证明}:$\ln{N!}\approx N\ln{N}-N$}
$$\begin{aligned}
    \ln{N!} &=\sum_{i=1}^{N}{\ln i}\approx \int_{1}^{N} \ln{x}\mathrm{d}x \\
            &=\left.x\ln{x}-x\right|_{1}^{N} \\
            &=N\ln{N}-N+1\approx N\ln{N}-N.
\end{aligned}$$
这就是最常用的Stirling近似的结果.
为了更精确的进行估计,我们可以采取梯形放缩.\\
\colorbox{lightgray}{\textbf{证明}:$\ln{N!}\approx (N+\frac{1}{2})\ln{N}-N$}
$$\begin{aligned}
    \int_{1}^{N} \ln{x}\mathrm{d}x  &\approx \sum_{i=1}^{N-1}{\frac{\ln{i}+\ln{(i+1)}}{2}} \\
    N\ln{N}-N+1                     &\approx \sum_{i=1}^{N}{\ln i}-\frac{\ln{N}}{2} \\
    N\ln{N}-N+1                     &\approx \ln{N!}-\frac{\ln{N}}{2}.
\end{aligned}$$
对上式整理并忽略常数$1$可得$\ln{N!}\approx (N+\frac{1}{2})\ln{N}-N$.\\
然而,积分放缩法仍然不是最精确的方法.我们在用积分代替离散图形的面积求和时总有剩余的部分.
为了确定这一偏差,我们可以对梯形放缩的结果进一步优化,即更精准地确定上面式子中的常数项.\\
不妨设$\ln{N!}\approx (N+\frac{1}{2})\ln{N}-N+C$,如果常数$C$是收敛的,那么我们就相当于找到了一个合理的近似方法.
对上式两边取指数运算有$N!=\frac{N^N\sqrt{N}\e^C}{\e^N}.$\\
不妨记序列$\left\{a_n\right\}$满足$a_n=\frac{\e^nn!}{n^n\sqrt{n}}$.\\
\colorbox{lightgray}{\textbf{证明}:$\lim_{n\to\infty}{a_n}$存在,并求其值.}\\
下面首先证明$\lim_{n\to\infty}{a_n}$存在.\\
$$\begin{aligned}
    \frac{a_{n+1}}{a_n} &= \dfrac{\frac{\e^{n+1}(n+1)!}{(n+1)^{n+1}\sqrt{n+1}}}{\frac{e^nn!}{n^n\sqrt{n}}}
                         = \dfrac{\e^{n+1}(n+1)!n^{n+\frac{1}{2}}}{\e^nn!(n+1)^{n+\frac{3}{2}}} \\
                        &= \e\cdot\left(\frac{n}{n+1}\right)^{n+\frac{1}{2}}
\end{aligned}$$
上述式子与$1$的大小关系并不是显然的.我们把它改写为$\dfrac{\e}{\left(1+\frac{1}{n}\right)^{n+\frac{1}{2}}}$,
记$\phi(x)=\left(1+\frac{1}{x}\right)^{x+\frac{1}{2}}.$\\
\colorbox{lightgray}{\textbf{引理}:$\forall x\in [1,+\infty),\phi(x)>\e.$}\\
\textbf{证明}:$\phi(x)>e \Leftrightarrow \ln\left(1+\frac{1}{x}\right)>\frac{2}{2x+1}.$\\
记$\varphi(x)=\ln\left(1+\frac{1}{x}\right)-\frac{x}{2x+1},\varphi(1)=\ln{2}-\frac{1}{3}>0$.\\
令$u=\frac{1}{x}\in(0,1]$,则$\varphi(x)=\ln{(1+u)-\frac{2u}{u+2}}\equiv v(u).$\\
$v'(u)=\frac{1}{u+1}-\frac{4}{(u+2)^2}=\frac{u^2}{(u+1)(u+2)^2}>0.$\\
则$\forall u\in(0,1],v(u)>v(1)>0$.从而原命题得证.\\
因此$\frac{a_{n+1}}{a_n}<1$,又$a_n>0$,故$\left\{a_n\right\}$为单调递减且有下界的序列,其极限必然存在.\\
不妨设$\lim_{n\to\infty}{a_n}=A$.\\
为了求出$A$,我们先来考虑Wallis公式.\\
\colorbox{lightgray}{\textbf{证明}:$\dfrac{\pi}{2}=\lim_{n\to\infty}\dfrac{\left[(2n)!!\right]^2}{(2n-1)!!(2n+1)!!}$}\\
此处暂时略去证明.\\
我们有$(2n)!!=n!\cdot2^n,(2n-1)!!=\dfrac{(2n)!}{(2n)!!}=\dfrac{(2n)!}{n!\cdot2^n},(2n+1)!!=\dfrac{(2n+1)!}{n!\cdot2^n}$.\\
将它们代入Wallis公式中有
$$\begin{aligned}
    \dfrac{\pi}{2} &= \lim_{n\to\infty}{\dfrac{n!\cdot2^n}{\dfrac{(2n)!}{n!\cdot2^n}}\cdot\dfrac{n!\cdot2^n}{\dfrac{(2n+1)!}{n!\cdot2^n}}} \\
                   &= \lim_{n\to\infty}{\dfrac{(n!)^4\cdot 2^{4n}}{\left[(2n)!\right]^2\cdot (2n+1)}} \\
                   &= \lim_{n\to\infty}{\dfrac{\left[\dfrac{n^{n+\frac{1}{2}}}{\e^n}\right]^4\cdot 2^{4n}}{\left[\dfrac{(2n)^{2n+1}}{\e^{2n}}\right]^2\cdot(2n+1)}}
                      \cdot\left(\lim_{n\to\infty}{\dfrac{n!\e^n}{n^{n+\frac{1}{2}}}}\right)^4
                      \cdot\dfrac{1}{\left(\lim_{n\to\infty}{\dfrac{(2n)!\e^{2n}}{(2n)^{2n+1}}}\right)^2} \\
                   &= A^2\cdot\lim_{n\to\infty}{\dfrac{n}{2(2n+1)}} \\
                   &= \dfrac{A^2}{4}
\end{aligned}$$
从而有$A=\sqrt{2\pi}$.因此$C=\ln{\lim_{n\to\infty}{a_n}}=\ln{\sqrt{2\pi}}$.\\
回到Stirling近似中,我们相当于证明了$n!\sim \dfrac{\sqrt{2\pi n}\cdot n^n}{\e^n}$,即这两者是等价无穷大.
实际上在$n>9$时使用这一式子估计的误差就已经小于$1\%$,在取对数之后误差更是可以忽略不计.所以用这个式子是十分合适的.\\
对于Stirling近似的各种形式,我们还要做一些额外的说明.\\
\colorbox{lightgray}{\textbf{引理}:$\lim_{n\to\infty}{\dfrac{\ln{n!}}{n\ln{n}-n}}=\lim_{n\to\infty}{\dfrac{\ln{n!}}{n\ln{n}-n+\ln{\sqrt{2\pi n}}}}=1$}.\\
注意到$$\begin{aligned}
    \lim_{n\to\infty}{\dfrac{\ln{n!}}{n\ln{n}-n+\ln{\sqrt{2\pi n}}}} &= \lim_{n\to\infty}{\dfrac{\ln{n!}}{n\ln{n}-n}}\cdot\lim_{n\to\infty}{\dfrac{n\ln{n}-n}{n\ln{n}-n+\ln{\sqrt{2\pi n}}}} \\
                                                                     &= \lim_{n\to\infty}{\dfrac{\ln{n!}}{n\ln{n}-n}}
\end{aligned}$$
然而我们却知道$\lim_{n\to\infty}{\dfrac{n!\e^n}{n^n}}$是不存在的.
因此,不能在极限的四则运算中使用$\lim_{n\to\infty}{\dfrac{n!\e^n}{n^n}}=1$.这两者不是等价无穷大.\\
与之对应的是$\lim_{n\to\infty}{\dfrac{n}{\sqrt[n]{n!}}}=\e$.
\end{document}