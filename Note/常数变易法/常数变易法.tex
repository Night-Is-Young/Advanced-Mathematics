\documentclass[a4paper,oneside]{ctexart}
\usepackage{amsmath, amsthm, amssymb,geometry,enumerate,color,xcolor}
\geometry{left=2cm, right=2cm, top=2.5cm, bottom=2.5cm}
\newcommand{\e}{\mathrm{e}}
\newcommand{\di}{\mathrm{d}}
\linespread{1.55}
\begin{document}
\pagestyle{empty}
\begin{center}\large 一阶线性微分方程的通解\end{center}
\textbf{前言:}无论是物理还是化学,总是会涉及一些微分方程.下面我们来探讨一下一阶线性微分方程的通解.\\
所谓一阶线性微分方程,是指形如$\dfrac{\di y}{\di x}+P(x)y=Q(x)$的微分方程.\\
其中$P(x),Q(x)$均为关于$x$的已知函数,$Q(x)$称为自由项.\\
一阶,是指方程中关于$y$的导数是一阶,
线性,是指方程化简后关于$y,y'$的次数为$0$或$1$.\\
若$Q(x)=0$,则称该方程为一阶齐次线性微分方程.
若$Q(x)\neq 0$,则称该方程为一阶非齐次线性微分方程.\\
让我们从简单的情况入手,先来求一阶齐次线性微分方程的通解.\\
\colorbox{lightgray}{\textbf{求解微分方程:}$\dfrac{\di y}{\di x}+P(x)y=0$.}\\
\textbf{解:}我们不妨对该方程进行分离变量得到$\dfrac{\di y}{y}=-P(x)\di x$.\\
两边积分得到$\ln{y}=-\int{P(x)\di x}+\ln{C}$.\\
化简后得到$y=C\e^{-\int{P(x)\di x}}$.\\
现在让我们考虑一阶非齐次线性微分方程.首先,我们来看比较直接的做法.\\
\colorbox{lightgray}{\textbf{求解微分方程:}$\dfrac{\di y}{\di x}+P(x)y=Q(x)$.}\\
\textbf{解法一:}先求得$Q(x)=0$时对应的微分方程的解:$y=C\e^{-\int{P(x)\di x}}$.\\
将原微分方程改写为$\dfrac{\di y}{y}=\left(-P(x)+\dfrac{Q(x)}{y}\right)\di x$.\\
两边积分得到$\ln{y}=-\int{P(x)\di x}+\int{\dfrac{Q(x)}{y}\di x}+\ln{C_1}$.\\
即$y=C_1\e^{-\int{P(x)\di x}}\cdot\e^{\int{\frac{Q(x)}{y}\di x}}$.\\
我们不妨再假定$C_1\e^{\int{\frac{Q(x)}{y}\di x}}\equiv C(x)$,从而$y=C(x)\e^{-\int{P(x)\di x}}$.\\
因此,可以得出$\dfrac{\di y}{\di x}=\dfrac{\di C(x)}{\di x}e^{-\int{P(x)\di x}}-C(x)P(x)e^{-\int{P(x)\di x}}$.\\
将上式代入原微分方程后有$\dfrac{\di C(x)}{\di x}e^{-\int{P(x)\di x}}-C(x)P(x)e^{-\int{P(x)\di x}}+P(x)C(x)e^{-\int{P(x)\di x}}=Q(x)$.\\
即$\dfrac{\di C(x)}{\di x}e^{-\int{P(x)\di x}}=Q(x)$,移项可得$\dfrac{\di C(x)}{\di x}=Q(x)e^{\int{P(x)\di x}}$.\\
对上式两边积分有$C(x)=\int{Q(x)e^{\int{P(x)\di x}}\di x}+C$.\\
最终代回原式有:
$$\begin{aligned}
    y &= \e^{-\int{P(x)\di x}}\left(\int{Q(x)e^{\int{P(x)\di x}}\di x}+C\right) \\
      &= C\e^{-\int{P(x)\di x}}+\e^{-\int{P(x)\di x}}\int{Q(x)e^{\int{P(x)\di x}}\di x}
\end{aligned}$$
从上面的式子可以看出,一阶非齐次线性微分方程的通解包含两部分:
一部分为对应的线性齐次方程的通解,另一部分为该线性非齐次微分方程的特解.
因此,一阶线性非齐次方程的通解等于对应的线性齐次方程的通解与线性非齐次方程的一个特解之和.\\
我们所说的常数变易法其实就是将齐次方程中的积分常数$C$替换成待定的函数$C(x)$,随后通过回代的方式进行求解的过程.\\
而你可能看着这个方法陷入了长久的疑惑:为什么我想不出来这么精妙的方法呢?\\
\colorbox{lightgray}{“我们所用的仅是他的结论,并无过程.”——来自百度百科“常数变易法”词条.}\\
事实上,我们要从解微分方程的基本思想开始讲起.\\
我们知道,解微分方程最基本的思想就是分离变量——只要将两个变量分别放在方程两端,
就可以根据我们所学的积分知识对两边进行积分,也就得出了两个变量的关系.
这个办法在解决上面提到的齐次方程是十分有用的.\\
因此,我们解决非齐次方程$\dfrac{\di y}{\di x}+P(x)y=Q(x)$时也需要坚持相同的思想——还是分离变量.\\
直接分离肯定是宣告失败的.我们需要一点点转换的思想.比如令$y=u\cdot x$,其中$u$是关于$x$的函数.\\
将$\dfrac{\di y}{\di x}=u+\dfrac{\di u}{\di x}$代回原微分方程中有$u+\dfrac{\di u}{\di x}+u\cdot x\cdot P(x)=Q(x)$.\\
我们发现上述式子仍然无法将$u$和$x$分离.这个尝试失败了.不过我们仍然也许获得了一点启示:想要让变量能分离,只需要让不好分离的那一项是0就可以了.
于是数学家们一直在寻找一个合适的代换方法使得分离变量成功.
最后,伟大的Lagrange找到了一个异常简单的代换:$y=u\cdot v$,其中$u,v$均为关于$x$的函数.\\
我们将$\dfrac{\di y}{\di x}=u\dfrac{\di v}{\di x}+v\dfrac{\di u}{\di x}$代入原微分方程,可以得到:
$u\dfrac{\di v}{\di x}+v\dfrac{\di u}{\di x}+u\cdot v\cdot P(x)=Q(x)$.\\
即$v\dfrac{\di u}{\di x}+u\left(\dfrac{\di v}{\di x}+ v\cdot P(x)\right)=Q(x)$.\\
既然不好分离变量,我们就干脆让$\dfrac{\di v}{\di x}+ v\cdot P(x)=0$就好了.
这时你会惊奇地发现这就是一个一阶齐次线性微分方程,它的解是$v=C_1\e^{-\int{P(x)\di x}}$.\\
然后我们代回上述式子中,自然便可以得到$\dfrac{\di u}{\di x}=\dfrac{Q(x)}{v}=\dfrac{Q(x)\e^{\int{P(x)\di x}}}{C_1}$.\\
至此,你会惊奇的发现(好吧算不上多惊奇)我们成功完成了分离变量的工作.
只需要把上述式子两边积分,最后整理表达式,就可以得到最终的结果了.\\
所以,在看教科书时难免会产生诸如"为什么可以当成齐次来解""为什么可以把那个常数换成$C(x)$"等等的问题,
而睿智的防自学设计又不会告诉你我们如何得到常数变易法的结果.
我们之所以先解齐次方程,是因为齐次方程的解恰好能让分离变量的某一项为0,从而能解出一个合理的解.\\
至于书上说的"分成某某两个部分",实际上是为了解更高阶的微分方程用的,现在也许可以暂时不考虑.
其实常数变易法和变量代换法在一阶时并无明显差距,直到高阶的情况下用常数变易法才比较简捷.
\end{document}