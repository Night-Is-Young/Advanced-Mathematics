\documentclass[a4paper]{ctexart}
\usepackage{template}
\geometry{left=2cm, right=2cm, top=2.5cm, bottom=2.5cm}
\begin{document}
\pagestyle{empty}
\begin{center}
    \large Cauchy命题
\end{center}
关于数列$\left\{ a_n\right\}$前$n$项的均值与$\left\{ a_n\right\}$的关系,我们有如下命题:
\begin{formal}[Cauchy's Proposition]
    若数列$\left\{ a_n\right\}$满足$\displaystyle\lim_{n\to\infty}{a_n}=A$,则$\displaystyle\lim_{n\to\infty}{\dfrac{\sum_{i=1}^{n}{a_i}}{n}}=A$.
\end{formal}
\begin{analyze}[Analysis.]
    我们可以从直觉感受到这个命题是成立的,然而不难发现均值的收敛速度总是慢于原数列的.
    因此在证明时,我们需要通过一定的分割使均值变成可控制大小的两项,从而证明该命题.
\end{analyze}
\begin{solution}[Proof.]
    由$\displaystyle\lim_{n\to\infty}{a_n}=A$有
    $$\forall\varepsilon_a>0,\exists N\in\mathbb{N}^*\st\forall n\geqslant N,\left\lvert a_n-A\right\rvert\leqslant\varepsilon_a.$$
    由收敛序列的有界性可知$\exists M_a>0\st\forall n\in\N^*,\left|a_n-A\right|\leqslant M_a$.则
    $$\sum_{i=1}^{n}{\left\lvert a_i-A\right\rvert}\leqslant M_a(N_a-1)+\varepsilon_a(n+1-N_a)=(M_a-\varepsilon_a)(N_a-1)+n\varepsilon_a$$
    对于任意$\varepsilon>0$,取$0<\varepsilon_a<\varepsilon$和对应的$N_a,M_a$.\\
    要使$\left\lvert \dfrac{\sum_{i=1}^{n}{a_i}}{n}-A\right\rvert\leqslant\varepsilon$,根据绝对值三角不等式有
    $$\left\lvert \dfrac{\sum_{i=1}^{n}{a_i}}{n}-A\right\rvert=\dfrac{1}{n}\left\lvert \sum_{i=1}^{n}{(a_i-A)}\right\rvert\leqslant\dfrac{1}{n}\sum_{i=1}^{n}{\left\lvert a_i-A\right\rvert}$$
    又$n\geqslant N_a$时$$\dfrac{1}{n}\sum_{i=1}^{n}{\left\lvert a_i-A\right\rvert}\leqslant\dfrac{(M_a-\varepsilon_a)(N_a-1)}{n}+\varepsilon_a$$
    令$$\dfrac{(M_a-\varepsilon_a)(N_a-1)}{n}+\varepsilon_a\leqslant\varepsilon$$
    解得$$n\geqslant\dfrac{(M_a-\varepsilon_a)(N_a-1)}{\varepsilon-\varepsilon_a}$$\\
    则$\exists N=\max\left\{\left[\dfrac{(M_a-\varepsilon_a)(N_a-1)}{\varepsilon-\varepsilon_a}\right]+1,N_a\right\}\st$
    $$\forall n\geqslant N,\left\lvert \dfrac{\sum_{i=1}^{n}{a_i}}{n}-A\right\rvert
    \leqslant\dfrac{1}{n}\sum_{i=1}^{n}{\left\lvert a_i-A\right\rvert}<\dfrac{(M_a-\varepsilon_a)(N_a-1)+N\varepsilon_a}{N}\leqslant\varepsilon$$
    从而原命题得证.
\end{solution}\noindent
下面我们来看两道与Cauchy命题有一定相关的问题.
\begin{problem}[例1(24.10.09 SJTU数分小测).]
    数列$\left\{ x_n\right\}$满足对于$\left\{ x_n\right\}$的任意子列$\left\{ x_{n_k}\right\}$均有$\displaystyle\lim_{k\to\infty}{\dfrac{\sum_{i=1}^{k}{x_{n_i}}}{k}}=1$.\textbf{试证明:}$\displaystyle\lim_{n\to\infty}{x_n}=1$.
\end{problem}
\begin{solution}[Proof.]
    采取反证法.假定$\left\{ x_n\right\}$不收敛或不收敛于$1$,则有
    $$\exists\varepsilon>0,\forall N\in\mathbb{N}^*,s.t.\exists n\geqslant N,\left\lvert x_n-1\right\rvert>\varepsilon$$
    也即$\left\{ x_n\right\}$中有无穷多项$x_i$满足$\left\lvert x_n-1\right\rvert>\varepsilon$.\\
    我们将这些项分为$\left\{ x_n\right\}$的两个子序列$\left\{ x_{n_+}\right\},\left\{ x_{n_-}\right\}$,
    满足$$\begin{aligned}
    \forall x_i\in\left\{ x_{n_+}\right\},x_i>1+\varepsilon \\
    \forall x_j\in\left\{ x_{n_-}\right\},x_j<1-\varepsilon
    \end{aligned}$$
    则$\left\{ x_{n_+}\right\},\left\{ x_{n_-}\right\}$中至少有一个为无穷序列.\\
    当$\left\{ x_{n_+}\right\}$或$\left\{ x_{n_-}\right\}$为无穷序列时有
    $$\lim_{k\to\infty}{\dfrac{\sum_{i=1}^{k}{x_{n_{+,i}}}}{k}}\geqslant \lim_{k\to\infty}{\dfrac{\sum_{i=1}^{k}{1+\varepsilon}}{k}}=1+\varepsilon>1$$
    $$\lim_{k\to\infty}{\dfrac{\sum_{i=1}^{k}{x_{n_{-,i}}}}{k}}\leqslant \lim_{k\to\infty}{\dfrac{\sum_{i=1}^{k}{1-\varepsilon}}{k}}=1-\varepsilon<1$$
    而根据题意,若$\left\{ x_{n_+}\right\}$或$\left\{ x_{n_-}\right\}$为无穷序列,则有
    $$\lim_{k\to\infty}{\dfrac{\sum_{i=1}^{k}{x_{n_{+,i}}}}{k}}=\lim_{k\to\infty}{\dfrac{\sum_{i=1}^{k}{x_{n_{-,i}}}}{k}=1}$$
    矛盾.故$\displaystyle\lim_{n\to\infty}{x_n}=1$.
\end{solution}
\begin{theorem}
    由上可知,Cauchy命题的逆命题并不成立,需要把它加强至$\left\{x_n\right\}$的任意子列才可以推出正确的结果.
\end{theorem}
\begin{problem}[例2(24.10.09 SJTU数分小测).]
    正项数列$\left\{ x_n\right\}$满足$\displaystyle\lim_{n\to\infty}{\dfrac{\sum_{i=1}^{n}{x_i}}{n}}=a,a\in\mathbb{R}$.\textbf{试证明:}$\displaystyle\lim_{n\to\infty}{\dfrac{\sum_{i=1}^{n}{x_i^2}}{n^2}}=0$.
\end{problem}
\begin{solution}[Proof.]
    若$\left\{x_n\right\}$有界,设其上界为$M$,则有
    $$0\leqslant \dfrac{\sum_{i=1}^{n}{x_i^2}}{n^2}\leqslant \dfrac{M^2}{n}$$
    两边夹逼可得$\displaystyle\lim_{n\to\infty}{\dfrac{\sum_{i=1}^{n}{x_i^2}}{n^2}}=0.$\\
    若$\left\{x_n\right\}$无界,则有
    $$\forall\ep>0,\exists n\in\N^*\st x_n>\ep$$
    设$\displaystyle M_n=\max_{1\leqslant i\leqslant n}\left\{x_i\right\}$,在$i=k$处取到,则有
    $$0\leqslant\dfrac{\sum_{i=1}^{n}{x_i^2}}{n^2}\leqslant\dfrac{M_n}{n}\cdot\dfrac{\sum_{i=1}^{n}{x_i}}{n}$$
    由于$\left\{x_n\right\}$是无界的,则$n\to\infty$时有$k\to\infty$.若不然,则$x_k$是$\left\{x_n\right\}$的上界,矛盾.\\
    我们有
    \begin{align*}
        \lim_{n\to\infty}{\dfrac{x_n}{n}} 
        &= \lim_{n\to\infty}{\left(\dfrac{\sum_{i=1}^{n}{x_i}}{n}-\dfrac{n-1}{n}\cdot\dfrac{\sum_{i=1}^{n-1}{x_i}}{n-1}\right)} \\\
        &= a-1\cdot a= 0
    \end{align*}
    从而$$0\leqslant \lim_{n\to\infty}{\dfrac{M_n}{n}}=\lim_{n\to\infty}{\dfrac{x_k}{n}}\leqslant\lim_{k\to\infty}{\dfrac{x_k}{k}}=0$$
    则$$\lim_{n\to\infty}{\left(\dfrac{M_n}{n}\cdot\dfrac{\sum_{i=1}^{n}{x_i}}{n}\right)}=0\cdot a=0$$
    对上面提到的不等式夹逼可得$\displaystyle\lim_{n\to\infty}{\dfrac{\sum_{i=1}^{n}{x_i^2}}{n^2}}=0$,从而原命题得证.
\end{solution}
\end{document}