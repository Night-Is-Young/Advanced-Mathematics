\documentclass{ctexart}
\usepackage{template}
\newcommand{\arccot}{\mathrm{ arccot}}
\begin{document}\pagestyle{empty}
\begin{center}\large
    反三角函数及其在不定积分中的运用
\end{center}
\textbf{前言:}在高中时,我们已经熟知正弦,余弦,正切三种三角函数.在高等数学中,
我们将会学习更多三角函数以及它们的反函数,并以此为工具解决一些问题.\\
首先,我们扩展一下所学的三角函的类型.
\begin{theorem}[六种三角函数]
        $$\text{正弦函数}\ \ \sin x\ \ \ \ \ \ \ \ \ \text{正割函数}\ \ \sec{x}=\dfrac{1}{\cos{x}}$$
        $$\text{余弦函数}\ \ \cos x\ \ \ \ \ \ \ \ \ \text{余割函数}\ \ \csc{x}=\dfrac{1}{\sin{x}}$$
        $$\text{正切函数}\ \ \tan x\ \ \ \ \ \ \ \ \ \text{余切函数}\ \ \cot{x}=\dfrac{1}{\tan{x}}$$
\end{theorem}\noindent
根据反函数的求导法则,我们可以写出反三角函数的导数.
\begin{theorem}[反三角函数的导数]
    $$\left(\arcsin x\right)'=\dfrac{1}{\sqrt{1-x^2}}\ \ \ \ \ \ \ \ \ \ \left(\arccos x\right)'=-\dfrac{1}{\sqrt{1-x^2}}\ \ \ \ \ \ \ \ \ \ \left(\arctan x\right)'=\dfrac{1}{1+x^2}$$
\end{theorem}\noindent
现在,我们来进行一系列不定积分的推导.
\begin{problem}[Example 1.]
    求不定积分$$\int{\tan{x}\di x}$$
\end{problem}
\begin{solution}[Solution.]
    $$
    \int \tan{x}\di x 
    = \int \dfrac{\sin{x}}{\cos{x}}\dx
    = -\int \dfrac{\di(\cos{x})}{\cos{x}}
    = -\ln\left\lvert{\cos{x}}\right\rvert+C
    $$
\end{solution}
\begin{problem}[Example 2.]
    求不定积分$$\int{\dfrac{\di x}{a^2-x^2}}$$
\end{problem}
\begin{solution}[Solution.]
    $$
    \begin{aligned}
        \int \dfrac{\di x}{a^2-x^2} 
        &= \dfrac{1}{2a}\int\left(\dfrac{1}{a+x}+\dfrac{1}{a-x}\right)\di x\\
        &= \dfrac{1}{2a}\left(\int\dfrac{1}{a+x}\di(a+x)-\int\dfrac{1}{a-x}\di(a-x)\right)\\
        &= \dfrac{1}{2a}\left(\ln\left\lvert a+x\right\rvert-\ln\left\lvert a-x\right\rvert\right)+C\\
        &= \dfrac{1}{2a}\ln{\left\lvert\dfrac{a+x}{a-x}\right\rvert}+C
    \end{aligned}
    $$
\end{solution}
\begin{problem}[Example 3.]
    求不定积分$$\int{\dfrac{\di x}{a^2+x^2}}$$
\end{problem}
\begin{solution}[Solution.]
    $$
    \int \dfrac{\di x}{a^2+x^2} 
    = \dfrac{1}{a^2}\int\dfrac{\di x}{1+\left(\dfrac{x}{a}\right)^2}\\
    = \dfrac{1}{a}\int\dfrac{\di\dfrac{x}{a}}{1+\left(\dfrac{x}{a}\right)^2}\\
    = \dfrac{1}{a}\arctan{\dfrac{x}{a}}+C
    $$
\end{solution}
\begin{problem}[Example 4.]
    求不定积分$$\int{\dfrac{\di x}{\sqrt{a^2-x^2}}}$$
\end{problem}
\ \\
\begin{solution}[Solution.]
    $$
    \int{\dfrac{\di x}{\sqrt{a^2-x^2}}} 
    = \int{\dfrac{\di\left(\dfrac{x}{a}\right)}{\sqrt{1-\left(\dfrac{x}{a}\right)^2}}}
    = \arcsin{\dfrac{x}{a}}+C
    $$
\end{solution}
\begin{problem}[Example 5.]
    求不定积分$$\int{\dfrac{\di x}{\sin{x}}}$$
\end{problem}
\begin{solution}[Solution.]
    $$
    \int{\dfrac{\di x}{\sin x}} 
    = \int{\dfrac{\sin{x}\di x}{\sin^2{x}}}
    = -\int{\dfrac{\di(\cos{x})}{1-\cos^2{x}}}
    = \dfrac{1}{2}\ln\left\lvert\dfrac{1-\cos{x}}{1+\cos{x}}\right\rvert+C
    $$
\end{solution}
\begin{problem}[Example 6.]
    求不定积分$$\int{\dfrac{\di x}{\cos{x}}}$$
\end{problem}
\begin{solution}[Solution.]
    $$
    \int{\dfrac{\di x}{\cos x}} 
    = \int{\dfrac{\cos{x}\di x}{\cos^2{x}}}
    = \int{\dfrac{\di(\sin{x})}{1-\sin^2{x}}}
    = \dfrac{1}{2}\ln\left\lvert\dfrac{1+\sin{x}}{1-\sin{x}}\right\rvert+C
    = \ln\left\lvert\sec{x}+\tan{x}\right\rvert+C
    $$
\end{solution}
\begin{problem}[Example 7.]
    求不定积分$$\int{\sqrt{a^2-x^2}\di x}$$
\end{problem}
\begin{solution}[Solution(Method I).]
    采取换元法.令$x=a\sin{t}$,则$\di x=a\cos{t}\di t$.
    $$\begin{aligned}
        \int{\sqrt{a^2-x^2}\di x} 
        &= \int{a\cos{t}\cdot a\cos{t}\di t} = a^2\int{\cos^2{t}\di t} \\
        &= \dfrac{a^2}{2}\int{(1+\cos{2t})\di t} \\
        &= \dfrac{a^2}{2}\left(t+\dfrac{1}{2}\sin{2t}\right)+C \\
        &= \dfrac{a^2}{2}\left(\arcsin{\dfrac{x}{a}}+\dfrac{x}{a}\sqrt{1-\dfrac{x^2}{a^2}}\right)+C \\
        &= \dfrac{a^2}{2}\arcsin{\dfrac{x}{a}}+\dfrac{x}{2}\sqrt{a^2-x^2}+C
    \end{aligned}$$
\end{solution}
\begin{solution}[Solution(Method II).]
    采取分部积分法.
    $$\begin{aligned}
        \int{\sqrt{a^2-x^2}\di x} 
        &= x\sqrt{a^2-x^2}-\int{x\di\left(\sqrt{a^2-x^2}\right)} \\
        &= x\sqrt{a^2-x^2}+\int{\dfrac{x^2\di x}{\sqrt{a^2-x^2}}} \\
        &= x\sqrt{a^2-x^2}+\int{\dfrac{(x^2-a^2)+a^2}{\sqrt{a^2-x^2}}\di x} \\
        &= x\sqrt{a^2-x^2}-\int{\sqrt{a^2-x^2}\di x}+a^2\int{\dfrac{\di x}{\sqrt{a^2-x^2}}}
    \end{aligned}$$
    可知
    $$\begin{aligned}
        \int{\sqrt{a^2-x^2}\di x} 
        &= \dfrac{1}{2}\left( x\sqrt{a^2-x^2}+a^2\int{\dfrac{\di x}{\sqrt{a^2-x^2}}}\right) \\
        &= \dfrac{a^2}{2}\arcsin{\dfrac{x}{a}}+\dfrac{x}{2}\sqrt{a^2-x^2}+C
    \end{aligned}$$
\end{solution}
\begin{problem}[Example 8.]
    求不定积分$$\int{\dfrac{\di x}{\sqrt{a^2+x^2}}}$$
\end{problem}
\begin{solution}[Solution.]
    采取换元法.令$x=a\tan{t}$,则$\di x=\dfrac{a\di t}{\cos^2{t}}$.
    $$\begin{aligned}
        \int{\dfrac{\di x}{\sqrt{a^2+x^2}}} 
        &= \int{\dfrac{\dfrac{a\di t}{\cos^2{t}}}{a\cdot\dfrac{1}{\cos{t}}}} = \int{\dfrac{\di t}{\cos{t}}} \\
        &= \dfrac{1}{2}\ln\left\lvert\dfrac{1+\sin{t}}{1-\sin{t}}\right\rvert+C \\
        &= \ln{\left\lvert\dfrac{1}{\cos{t}}+\tan{t}\right\rvert}+C \\
        &= \ln{\left\lvert\dfrac{x}{a}+\sqrt{1+\dfrac{x^2}{a^2}}\right\rvert}+C \\
    \end{aligned}$$
    亦可以写作$\ln{\left\lvert x+\sqrt{a^2+x^2}\right\rvert}+C$.
\end{solution}
\begin{problem}[Example 9.]
    求不定积分$$\int{\sqrt{a^2+x^2}\di x}$$
\end{problem}
\begin{solution}[Solution]
    采取换元法和分部积分法结合的方法.置$x=a\tan{t}$,则$\di x=\dfrac{a\di t}{\cos^2{t}}$,从而
    $$\int{\sqrt{a^2+x^2}\di x}=\int \dfrac{a}{\cos{t}}\cdot\dfrac{a\di t}{\cos^2{t}}=\int{\dfrac{a^2\di t}{\cos^3{t}}}$$
    我们记$\displaystyle I=\int{\dfrac{\di t}{\cos^3{t}}}$,则
    $$\begin{aligned}
        I 
        &= \int{\dfrac{\di t}{\cos^3{t}}} \\
        &= \int{\dfrac{\di\left(\tan{t}\right)}{\cos{t}}} \\
        &= \dfrac{\tan{t}}{\cos{t}}-\int{\tan{t}\di\left(\dfrac{1}{\cos{t}}\right)} \\
        &= \dfrac{\tan{t}}{\cos{t}}-\int{\dfrac{\sin{t}}{\cos{t}}\cdot\dfrac{\sin{t}}{\cos^2{t}}\di t} \\
        &= \dfrac{\tan{t}}{\cos{t}}-\int{\dfrac{1-\cos^2{t}}{\cos^3{t}}\di t}\\
        &= \dfrac{\tan{t}}{\cos{t}}+\int{\dfrac{\di t}{\cos{t}}}-I
    \end{aligned}$$
    则有
    $$\begin{aligned}
        \int{\sqrt{a^2+x^2}\di x} = a^2I &= \dfrac{a^2}{2}\left(\dfrac{\tan{t}}{\cos{t}}+\int{\dfrac{\di t}{\cos{t}}}\right) \\
                                        &= \dfrac{a^2}{2}\left(\dfrac{x}{a}\sqrt{1+\dfrac{x^2}{a^2}}+\ln{\left\lvert\dfrac{x}{a}+\sqrt{1+\dfrac{x^2}{a^2}}\right\rvert}\right)+C\\
                                        &= \dfrac{x}{2}\sqrt{a^2+x^2}+\dfrac{a^2}{2}\ln{\left\lvert\dfrac{x}{a}+\sqrt{1+\dfrac{x^2}{a^2}}\right\rvert}+C
    \end{aligned}$$
    亦可以写作$\dfrac{x}{2}\sqrt{a^2+x^2}+\dfrac{a^2}{2}\ln{\left\lvert x+\sqrt{a^2+x^2}\right\rvert}+C$.\\
\end{solution}
\begin{problem}[Example 10.]
    求不定积分$$\int{\dfrac{\di x}{\sqrt{x^2-a^2}}}$$
\end{problem}
\begin{solution}[Solution.]
    分$x>a$和$x<-a$两种情况考虑.
    当$x>a$时,设$x=\dfrac{a}{\cos{t}}$,其中$t\in \left(0,\dfrac{\pi}{2}\right)$.则有
    $$\begin{aligned}
        \int{\dfrac{\di x}{\sqrt{x^2-a^2}}}
        &= \int{\dfrac{1}{\sqrt{a^2\tan^2{t}}}\cdot\dfrac{a\sin{t}}{\cos^2{t}}\di t} \\
        &= \int{\dfrac{\di t}{\cos{t}}} \\
        &= \ln{\left\lvert \dfrac{1}{\cos{t}}+\tan{t}\right\rvert}+C
    \end{aligned}$$
    此时我们有$\tan{t}=\sqrt{\dfrac{1}{\cos^2{t}}-1}=\dfrac{1}{a}\sqrt{x^2-a^2}$.\\
    当$x<-a$时,令$x=-\dfrac{a}{\cos{t}}$可得到相同的结果.\\
    综上有$\displaystyle\int{\dfrac{\di x}{\sqrt{x^2-a^2}}}=\ln\left\lvert x+\sqrt{x^2-a^2}\right\rvert+C$.\\
\end{solution}
\begin{problem}[Example 11.]
    求不定积分$$\int{\sqrt{x^2-a^2}\di x}$$
\end{problem}
\begin{solution}[Solution(Method I).]
    采取分部积分法.
    $$\begin{aligned}
        \int{\sqrt{x^2-a^2}\di x} 
        &= x\sqrt{x^2-a^2}-\int{x\di\left(\sqrt{x^2-a^2}\right)} \\
        &= x\sqrt{x^2-a^2}-\int{\dfrac{x^2\di x}{\sqrt{x^2-a^2}}} \\
        &= x\sqrt{x^2-a^2}-\int{\dfrac{\left(x^2-a^2+a^2\right)\di x}{\sqrt{x^2-a^2}}} \\
        &= x\sqrt{x^2-a^2}-\int{\sqrt{x^2-a^2}\di x}-a^2\int{\dfrac{\di x}{\sqrt{x^2-a^2}}}
    \end{aligned}$$
    则有
    $$\begin{aligned}
        \int{\sqrt{x^2-a^2}\di x} 
        &= \dfrac{1}{2}\left( x\sqrt{x^2-a^2}-a^2\int{\dfrac{\di x}{\sqrt{x^2-a^2}}}\right) \\
        &= \dfrac{x}{2}\sqrt{x^2-a^2}-\dfrac{a^2}{2}\ln\left\lvert x+\sqrt{x^2-a^2}\right\rvert+C
    \end{aligned}$$
\end{solution}
\begin{solution}[Solution(Method II).]
    采取换元法.当$x>a$时,置$x=\dfrac{a}{\cos t},t\in\left(0,\dfrac{\pi}{2}\right)$.则
    $$\begin{aligned}
        \int{\sqrt{x^2-a^2}\dx}
        &= \int{\sqrt{a^2\tan t}\cdot\dfrac{a\sin t}{\cos^2t}\di t} \\
        &= a^2\int{\dfrac{\sin^2t}{\cos^3t}\di t} \\
        &= a^2\left(\int{\dfrac{\di t}{\cos^3t}}-\int{\dfrac{\di t}{\cos t}}\right)
    \end{aligned}$$
    由\textbf{Example 9.}可知$\displaystyle\int{\dfrac{\di t}{\cos^3t}}=\dfrac{1}{2}\left(\dfrac{\tan t}{\cos t}+\int{\dfrac{\di t}{\cos t}}\right)$
    从而
    $$\begin{aligned}
        \int{\sqrt{x^2-a^2}\dx}
        &= \dfrac{a^2}{2}\left(\dfrac{\tan t}{\cos t}-\int{\dfrac{\di t}{\cos t}}\right) \\
        &= \dfrac{a^2}{2}\left(\dfrac{\sqrt{1-\dfrac{a^2}{x^2}}}{\dfrac{a^2}{x^2}}+\ln{\left|\dfrac{x}{a}+\sqrt{1-\dfrac{a^2}{x^2}}\right|}\right)+C_0 \\
        &= \dfrac{x^2}{2}\sqrt{x^2-a^2}-\dfrac{a^2}{2}\ln{\left|x+\sqrt{x^2-a^2}\right|}+C
    \end{aligned}$$
    当$x<-a$时亦可以得到相同的结果.于是
    $$\int{\sqrt{x^2-a^2}\dx}=\dfrac{x^2}{2}\sqrt{x^2-a^2}-\dfrac{a^2}{2}\ln{\left|x+\sqrt{x^2-a^2}\right|}+C$$
\end{solution}
\begin{theorem}[Integral Table]
    \begin{enumerate}[leftmargin=*,label=\textbf{\arabic*.\ }]
        \item $\displaystyle\int{\dfrac{\dx}{\sqrt{x^2\pm a^2}}}=\ln{\left|x+\sqrt{x^2\pm a^2}\right|}+C$
        \item $\displaystyle\int{\sqrt{x^2\pm a^2}\dx}=\dfrac{x^2}{2}\sqrt{x^2\pm a^2}\pm\dfrac{a^2}{2}\ln{\left|x+\sqrt{x^2\pm a^2}\right|}+C$
        \item $\displaystyle\int{\dfrac{\dx}{\sqrt{a^2-x^2}}}=\arcsin\dfrac{x}{a}+C$
        \item $\displaystyle\int{\sqrt{a^2-x^2}}=\dfrac{a^2}{2}\arcsin{\dfrac{x}{a}}+\dfrac{x}{2}\sqrt{a^2-x^2}+C$
        \item $\displaystyle\int{\dfrac{\dx}{a^2-x^2}}=\dfrac{1}{2a}\ln{\left|\dfrac{a+x}{a-x}\right|}+C$
        \item $\displaystyle\int{\dfrac{\dx}{a^2+x^2}}=\dfrac{1}{a}\arctan{\dfrac{x}{a}}+C$
        \item $\displaystyle\int{\tan x\dx}=-\ln{\left|\cos x\right|}+C$
        \item $\displaystyle\int{\dfrac{\dx}{\sin x}}=\dfrac{1}{2}\ln{\left|\dfrac{1-\cos x}{1+\cos x}\right|}+C$
        \item $\displaystyle\int{\dfrac{\dx}{\cos x}}=\dfrac{1}{2}\ln{\left|\dfrac{1+\sin x}{1-\sin x}\right|}+C=\ln{\left|\sec x+\tan x\right|}+C$
    \end{enumerate}
\end{theorem}
\end{document}