\documentclass{ctexart}
\usepackage{geometry}
\usepackage[dvipsnames,svgnames]{xcolor}
\usepackage{framed}
\usepackage{enumerate}
\usepackage{amsmath,amsthm,amssymb}
\usepackage{enumitem}
\usepackage{yhmath}
\usepackage{template}

\allowdisplaybreaks
\geometry{left=2cm, right=2cm, top=2.5cm, bottom=2.5cm}

\begin{document}
\pagestyle{empty}
\begin{center}\large 曲线积分\end{center}
\tbf{1.第一型曲线积分}\\
在考虑曲线的质量,质心,转动惯量等问题时,都要用到第一型曲线积分.
\begin{problem}[1.1 物质曲线的质量]
    \\设有一不均匀的物质曲线$L$,设$L$上一点$M(x,y,z)$的密度为$\rho(x,y,z)$,求$L$的质量$m$.
\end{problem}
\begin{solution}
    我们仍然用分割,近似代替,求和,取极限的方法求$m$的值.\\
    将曲线$L$分成$n$段,第$i$段的弧长为$\Delta s_i$.任取第$i$段上的一点$M_i(x_i,y_i,z_i)$.\\
    当分割的足够精细时,可以用$M_i$处的密度近似代替第$i$段的密度.于是第$i$段的质量为
    \[\Delta m_i\approx \rho(x_i,y_i,z_i)\Delta s_i\]
    对$\Delta m_i$求和,即可得到曲线质量$m$的近似值
    \[m=\sum_{i=1}^{n}\Delta m_i\approx\sum_{i=1}^{n}\rho(x_i,y_i,z_i)\Delta s_i\]
    令$\displaystyle\lambda=\max_{1\leqslant i\leqslant n}\left\{\Delta s_i\right\}$.类比定积分的定义,若极限
    \[\lim_{\lambda\to0}\sum_{i=1}^{n}\rho(x_i,y_i,z_i)\Delta s_i\]
    存在,我们就认为这极限值是$L$的质量$m$,即
    \[m=\lim_{\lambda\to0}\sum_{i=1}^{n}\rho(x_i,y_i,z_i)\Delta s_i\]
\end{solution}\noindent
实际问题中不只是物质曲线的质量需要计算上面形式的极限,还有很多问题也是类似的.\\
于是,我们着手定义第一型曲线积分.
\begin{definition}[1.2 定义:第一型曲线积分]
    设函数$f(x,y,z)$在分段光滑的曲线段$L$上有定义.将曲线$L$任意分成$n$段,第$i$段的弧长为$\Delta s_i$,在其上任取一点$M_i(x_i,y_i,z_i)$.%
    令$\displaystyle\lambda=\max_{1\leqslant i\leqslant n}\{\Delta s_i\}$.若极限
    \[\lim_{\lambda\to0}\sum_{i=1}^{n}f(x_i,y_i,z_i)\Delta s_i\]
    对于曲线$L$的任意分割方法和各中间点$M_i$的任意取法都存在,则称此极限为$f(x,y,z)$沿曲线$L$的\tbf{第一型曲线积分},%
    也称为对\tbf{弧长的曲线积分},记作
    \[\int_{L}f(x,y,z)\di s\]
\end{definition}\noindent
如果上述极限存在,我们就称函数$f(x,y,z)$在$L$上可积.\\
第一型曲线积分不依赖曲线的走向.例如,对于曲线$L$和它的端点$A,B$,积分路径从$A$到$B$和从$B$到$A$不改变结果,即
\[\int_{\wideparen{AB}}f(x,y,z)\di s=\int_{\wideparen{BA}}f(x,y,z)\di s\]
其余的性质,例如可加性等,在此略去.\\
我们现在来讨论第一型曲线积分的计算.为了简单起见,我们先讨论平面第一型曲线积分的计算.
\begin{problem}[1.3 平面第一型曲线积分的计算I]
    \\设$L$是$Oxy$平面上的一条曲线,其方程由函数$y=\mbf y(x),a\leqslant x\leqslant b$给出,并假定$y=\mbf{y}(x)$在$[a,b]$上有连续的导数.%
    设$f(x,y)$是在$L$上定义的连续函数,试计算积分$\displaystyle\int_{L}f(x,y)\di s$.
\end{problem}
\begin{solution}
    根据第一型曲线积分的定义有
    \[\int_{L}f(x,y)\di s=\lim_{\lambda\to0}\sum_{i=1}^{n}f(x_i,y_i)\Delta s_i\]
    在这种情况下,对$L$的任意分割都相当于对区间$[a,b]$的分割.于是上述和式可以改写为
    \[\sum_{i=1}^{n}f(x_i,y_i)\Delta s_i=\sum_{i=1}^{n}f(x_i,\mbf{y}(x_i))\Delta s_i\]
    另一方面,在分割地足够精细的情形下,又有
    \[\Delta s_i\approx\sqrt{\left(\Delta x_i\right)^2+\left[\mbf{y}'(x_i)\Delta x_i\right]^2}=\sqrt{1+\left[\mbf{y}'(x_i)\right]^2}\Delta x_i\]
    其中$\Delta x_i=x_i-x_{i-1}$.又因为$\displaystyle\lambda=\max_{1\leqslant i\leqslant n}\{\Delta s_i\}\to0$时,$\displaystyle\lambda'=\max_{1\leqslant i\leqslant n}\{\Delta x_i\}\to0$,于是根据Riemann积分的定义有
    \[\begin{aligned}
        \int_{L}f(x,y)\di s
        &= \lim_{\lambda\to0}\sum_{i=1}^{n}f(x_i,y_i)\Delta s_i \\
        &= \lim_{\lambda'\to0}\sum_{i=1}^{n}f(x_i,\mbf{y}(x_i))\sqrt{1+\left[\mbf{y}'(x_i)\right]^2}\Delta x_i \\
        &= \int_a^bf(x,\mbf{y}(x))\sqrt{1+\left[\mbf{y}'(x)\right]^2}\di x
    \end{aligned}\]
    于是我们有
    \[\int_{L}f(x,y)\di s=\int_a^bf(x,\mbf{y}(x))\sqrt{1+\left[\mbf{y}'(x)\right]^2}\di x\]
\end{solution}\noindent
上述结果的得出是十分符合直觉的,因为$L$由$y=\mbf y(x)$确定,自然可以将$f(x,y)$写成$f(x,\mbf{y}(x))$.\\
我们已经知道,弧微分$\di s=\sqrt{1+[\mbf y'(x)]^2}\di x$,两者结合自然可以得到上面的公式.
\begin{problem}[1.4 平面第一型曲线积分的计算II]
    \\设$L$是$Oxy$平面上的一条曲线,由参数方程
    \[\left\{\begin{array}{l}
        x=\varphi(t)\\y=\psi(t)
    \end{array}\right.\ \ \ \ \ (\alpha\leqslant t\leqslant\beta)\]
    确定,其中$\varphi(t)$和$\psi(t)$在$[\alpha,\beta]$上有连续的一阶导数.%
    设$f(x,y)$是在$L$上定义的连续函数,试计算积分$\displaystyle\int_{L}f(x,y)\di s$.
\end{problem}
\begin{solution}
    我们采取相似的方法计算之.对曲线$L$的任意分割都相当于对区间$[\alpha,\beta]$的分割
    \[\alpha=t_0<t_1<\cdots<t_{n-1}<t_n=\beta\]
    点$M_n(\varphi(t_i),\psi(t_i))(0\leqslant i\leqslant n)$构成了$L$的分割点.可以证明,第$i$段的弧长$\Delta s_i$有如下近似
    \[\Delta s_i\approx\sqrt{\left[\varphi'(t_i)\right]^2+\left[\psi'(t_i)\right]^2}\Delta t_i\]
    其中$\Delta t_i=t_i-t_{i-1}$.令$\displaystyle\lambda'=\max_{1\leqslant i\leqslant n}\{\Delta t_i\}$,于是$\lambda'\to0$时上述近似的误差是高阶无穷小量.因此
    \[\begin{aligned}
        \int_{L}f(x,y)\di s
        &= \lim_{\lambda\to0}\sum_{i=1}^{n}f(x_i,y_i)\Delta s_i \\
        &= \lim_{\lambda'\to0}\sum_{i=1}^{n}f(\varphi(t_i),\psi(t_i))\sqrt{\left[\varphi'(t_i)\right]^2+\left[\psi'(t_i)\right]^2}\Delta t_i \\
        &= \int_\alpha^\beta f(\varphi(t),\psi(t))\sqrt{\left[\varphi'(t)\right]^2+\left[\psi'(t)\right]^2}\di t
    \end{aligned}\]
\end{solution}\noindent
观察上述式子,也可以与参数方程的弧微分对应.总结来说,我们有如下定理.
\begin{formal}[1.5 平面第一型曲线积分的计算]
    我们按曲线的解析式分为如下两类.
    \begin{enumerate}[label=\tbf{\alph*.}]
        \item 设$L$是$Oxy$平面上的一条曲线,其方程由函数$y=\mbf y(x),a\leqslant x\leqslant b$给出,并假定$y=\mbf{y}(x)$在$[a,b]$上有连续的导数.%
            设$f(x,y)$是在$L$上定义的连续函数,则有
            \[\int_{L}f(x,y)\di s=\int_a^bf(x,\mbf{y}(x))\sqrt{1+\left[\mbf{y}'(x)\right]^2}\di x\]
        \item 设$L$是$Oxy$平面上的一条曲线,由参数方程
            \[\left\{\begin{array}{l}
                x=\varphi(t)\\y=\psi(t)
            \end{array}\right.\ \ \ \ \ (\alpha\leqslant t\leqslant\beta)\]
            确定,其中$\varphi(t)$和$\psi(t)$在$[\alpha,\beta]$上有连续的一阶导数.%
            设$f(x,y)$是在$L$上定义的连续函数,则有
            \[\int_{L}f(x,y)\di s=\int_\alpha^\beta f(\varphi(t),\psi(t))\sqrt{\left[\varphi'(t)\right]^2+\left[\psi'(t)\right]^2}\di t\]
    \end{enumerate}
\end{formal}\noindent
总的来说,考虑对应情形下的弧微分即可得到相应的计算公式.\\
类似地,我们也可以给出三维空间中的第一型曲线积分的计算公式.
\begin{formal}[1.6 空间第一型曲线积分的计算]
    设$L$是$\R^3$空间中的一条曲线,由参数方程
    \[\left\{\begin{array}{l}
        x=\mbf{x}(t)\\y=\mbf{y}(t)\\z=\mbf{z}(t)
    \end{array}\right.\ \ \ \ \ (\alpha\leqslant t\leqslant\beta)\]
    确定,其中$\mbf x(t),\mbf y(t)$和$\mbf z(t)$在$[\alpha,\beta]$上有连续的一阶导数.%
    设$f(x,y,z)$是在$L$上定义的连续函数,则有
    \[\int_{L}f(x,y,z)\di s=\int_{\alpha}^{\beta}f\left(\mbf x(t),\mbf y(t),\mbf z(t)\right)\sqrt{\left[\mbf x'(t)\right]^2+\left[\mbf y'(t)\right]^2+\left[\mbf z'(t)\right]^2}\di t\]
\end{formal}\noindent
定理的证明与在平面中是完全相似的,在此就不再赘述.\\
\tbf{2.第二型曲线积分}\\
计算一个受力的质点沿曲线运动的功需要用到第二型曲线积分.
\begin{problem}[2.1 质点沿曲线运动所做的功]
    \\设平面上有一光滑的曲线$L$和$L$的一个走向,其起点为$A$,终点为$B$.设想一质点沿$L$运动,它在点$(x,y)\in L$受到的力为
    \[\mbf F(x,y)=\left(P(x,y),Q(x,y)\right)\]
    试计算该质点从$A$运动到$B$时外力$\mbf F$所做的功$W$.
\end{problem}
\begin{solution}
    我们将$\wideparen{AB}$以分点$A=M_0.M_1.\cdots,M_{n-1},M_n=B$分成$n$段弧$\wideparen{M_{i-1}M_{i}}(i=1,\cdots,n)$.\\
    设第$i$段弧$\wideparen{M_{i-1}M_{i}}$的弧长为$\Delta s_i$.\\
    当分割的足够精细时,外力$\mbf F$在$\wideparen{M_{i-1}M_{i}}$上变化不大,可以近似看作常力$\F(x_i,y_i)$,其中$(x_i,y_i)$为这弧上任意取定的一点.%
    同理,质点的运动路径也可以近似看作从$M_{i-1}$到$M_i$的线段,于是$\mbf F$在这弧上做的功$\Delta W_i$近似为
    \[\Delta W_i\approx \mbf F(x_i,y_i)\cdot\overrightarrow{M_{i-1}M_i}\]
    对$\mbf F$做正交分解,有$\mbf F(x,y)=P(x,y)\mbf i+Q(x,y)\mbf j$.对有向线段$\overrightarrow{M_{i-1}M_{i}}$做正交分解,有$\overrightarrow{M_{i-1}M_i}=\Delta x_i\mbf i+\Delta y_i\mbf j$.%
    其中$\Delta x_i,\Delta y_i$为这有向线段在$x,y$方向上对应的的位移.于是我们有
    \[\Delta W_i\approx \mbf F(x_i,y_i)\cdot\overrightarrow{M_{i-1}M_i}=P(x_i,y_i)\Delta x_i+Q(x_i,y_i)\Delta y_i\]
    于是所求的总功$W$近似为
    \[W\approx\sum_{i=1}^{n}\left[P(x_i,y_i)\Delta x_i+Q(x_i,y_i)\Delta y_i\right]\]
    令$\displaystyle\lambda=\max_{1\leqslant i\leqslant n}\{\Delta s_i\}$,当分割的足够精细时有$\lambda\to0$.若极限
    \[\lim_{\lambda\to0}\sum_{i=1}^{n}\left[P(x_i,y_i)\Delta x_i+Q(x_i,y_i)\Delta y_i\right]\]
    存在,这极限就是所求的功.
\end{solution}\noindent
于是我们可以对第二型曲线积分做如下定义.
\begin{definition}[2.2 定义:第二型曲线积分]
    设$L$是从$A$到$B$的分段光滑有向曲线,向量函数$\mbf F(x,y)=P(x,y)\mbf i+Q(x,y)\mbf j$在$L$上有定义.%
    按照$L$的方向,依次用分点$A=M_0.M_1.\cdots,M_{n-1},M_n=B$将$L$分成$n$条有向弧$\wideparen{M_{i-1}M_i}(i=1,\cdots,n)$,%
    $\wideparen{M_{i-1}M_i}$的弧长记为$\Delta s_i$,并令$\lambda=\displaystyle\max_{1\leqslant i\leqslant n}\{\Delta s_i\}$.%
    在$\wideparen{M_{i-1}M_i}$上任取一点$(\xi_i,\eta_i)$.若极限
    \[\lim_{\lambda\to0}\sum_{i=1}^{n}\mbf F(\xi_i,\eta_i)\cdot\overrightarrow{M_{i-1}M_i}
    =\lim_{\lambda\to0}\sum_{i=1}^{n}\left[P(\xi_i,\eta_i)\Delta x_i+Q(\xi_i,\eta_i)\Delta y_i\right]\]
    存在(不依赖于分割方法和中间点的取法),则称此极限为向量函数$\mbf F(x,y)$沿曲线$L$从$A$到$B$的\tbf{第二型曲线积分},%
    也称作\tbf{对坐标的曲线积分},记作
    \[\int_{\wideparen{AB}}P\di x+Q\di y\ \ \ \ \ \text{或}\ \ \ \ \ \int_{\wideparen{AB}}\mbf F(x,y)\di\mbf r\]
    其中$\di\mbf r=\left(\di x,\di y\right)$.
\end{definition}\noindent
与第一型曲线积分类似,我们可以通过计算定积分的一般方法计算第二型积分曲线,只不过现在要注意曲线的走向.
\begin{formal}[2.3 第二型曲线积分的计算]
    设$L$是$Oxy$平面上的一条曲线,由参数方程
    \[\left\{\begin{array}{l}
        x=\varphi(t)\\y=\psi(t)
    \end{array}\right.\ \ \ \ \ (\alpha\leqslant t\leqslant\beta)\]
    确定,其中$\varphi(t)$和$\psi(t)$在$[\alpha,\beta]$上有连续的一阶导数.\\
    当$t$单调地由$\alpha$变化至$\beta$时,曲线上的点由$A$变化至$B$.%
    设$P(x,y),Q(x,y)$是在$L$上定义的连续函数,则有
    \[\int_{\wideparen{AB}}P(x,y)\di x+Q(x,y)\di y=\int_{\alpha}^{\beta}\left[P(\varphi(t),\psi(t))\varphi'(t)+Q(\varphi(t),\psi(t))\psi'(t)\right]\di t\]
\end{formal}
\begin{proof}
    我们略去详细的证明过程(需要用到Lagrange中值定理),而给出一个理解此式的方法.\\
    根据参数方程可知$\di x=\varphi'(t)\di t,\di y=\psi'(t)\di t$.将$x,y,\dx,\di y$代入定义式中即可得到答案.
\end{proof}\noindent
应特别强调的是,第二型曲线积分与给定曲线的方向有关,于是在所求的定积分中可能出现下限大于上限的情形.不管怎样,都需要注意下限对应起点,上限对应终点.\\
在空间中的第二型曲线积分也是完全类似的,在此也不再赘述.
\end{document}