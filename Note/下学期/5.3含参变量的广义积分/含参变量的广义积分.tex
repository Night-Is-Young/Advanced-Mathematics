\documentclass{ctexart}
\usepackage{geometry}
\usepackage[dvipsnames,svgnames]{xcolor}
\usepackage{framed}
\usepackage{enumerate}
\usepackage{amsmath,amsthm,amssymb}
\usepackage{enumitem}
\usepackage{template}

\allowdisplaybreaks[4]
\geometry{left=2cm, right=2cm, top=2.5cm, bottom=2.5cm}

\begin{document}
\pagestyle{empty}
\begin{center}\large\tbf{含参变量的广义积分}\end{center}
\tbf{1.含参变量的无穷积分}\\
\indent 仿照正常的无穷积分,我们可以定义含参变量的无穷积分.
\begin{definition}[1.1 含参变量的无穷积分]
    设二元函数$f(x,y)$在$[a,+\infty)\times[c,d]$上有定义.如果对于任意$y\in[c,d]$,无穷积分
    \[\int_a^{+\infty}f(x,y)\di x\]
    都收敛,这就在$[c,d]$上确定了$y$的函数
    \[g(y)=\int_a^{+\infty}f(x,y)\di x\]
    该函数称为\tbf{含参变量的无穷积分}.
\end{definition}
上述定义只是保证$g(y)$在$[c,d]$上逐点收敛.我们还要进一步研究其一致收敛性.
\begin{formal}[1.2 含参变量无穷积分的一致收敛性]
    设无穷积分
    \[g(y)=\int_a^{+\infty}f(x,y)\di x\]
    在区间$Y$上逐点收敛.如果对于任意$\ep>0$,存在一个与$y$无关的$N>a$,%
    使得当$A>N$时对任意$y\in Y$都有
    \[\left|\int_A^{+\infty}f(x,y)\dx\right|<\ep\]
    则称$g(y)$在$Y$上一致收敛.
\end{formal}
由此可以得到含参变量无穷积分的Cauchy判别法.
\begin{formal}[1.3 Cauchy判别法]
    设无穷积分
    \[g(y)=\int_a^{+\infty}f(x,y)\di x\]
    在区间$Y$上逐点收敛.如果对于任意$\ep>0$,存在一个与$y$无关的$N>a$,%
    使得当$A>N,A'>N$时对任意$y\in Y$都有
    \[\left|\int_A^{A'}f(x,y)\dx\right|<\ep\]
    则$g(y)$在$Y$上一致收敛.
\end{formal}
由此又可以导出$M$判别法.
\begin{formal}[1.4 $M$判别法]
    设当$y\in Y$时,对任意$A>a$,函数$f(x,y)$关于$x$在区间$[a,A]$上可积.%
    又当$x\geqslant a$时,对任意$y\in Y$有
    \[\left| f(x,y)\right|\leqslant\phi(x)\]
    且无穷积分
    \[\int_a^{+\infty}\phi(x)\di x\]
    收敛,则含参变量的积分
    \[g(y)=\int_a^{+\infty}f(x,y)\di x\]
    在$Y$上一致收敛.
\end{formal}
$M$判别法事实上和比较判别法的原理一致.将含参的被积函数$f(x,y)$%
通过放缩得到绝对值更大的$\phi(x)$,如果$\phi(x)$的无穷积分收敛,那么原含参积分一定也收敛.\\
\indent 对于非绝对一致收敛的无穷积分,需要用狄利克雷判别法和阿贝尔判别法.
\begin{formal}[1.5 Dirichlet判别法]
    如果二元函数$f(x,y),g(x,y)$满足
    \begin{enumerate}[topsep=0pt,parsep=0pt,itemsep=0pt,partopsep=0pt,leftmargin=*,label=\tbf{(\arabic*)}]
        \item 当$x$充分大后$g(x,y)$对任意$y\in Y$都是$x$的单调函数,且$x\to+\infty$时对任意$y\in Y$,$g(x,y)$一致趋于$0$.
        \item 对任意$A>a$,积分
            \[\int_a^Af(x,y)\di x\]
            存在且对任意$y\in Y$一致有界.
    \end{enumerate}
    那么含参变量的积分
    \[I(y)=\int_a^{+\infty}f(x,y)g(x,y)\di x\]
    在$Y$上一致收敛.
\end{formal}
\begin{formal}[1.5 Abel判别法]
    如果二元函数$f(x,y),g(x,y)$满足
    \begin{enumerate}[topsep=0pt,parsep=0pt,itemsep=0pt,partopsep=0pt,leftmargin=*,label=\tbf{(\arabic*)}]
        \item 当$x$充分大后$g(x,y)$对任意$y\in Y$都是$x$的单调函数,且$x\to+\infty$时对任意$y\in Y$,$g(x,y)$一致有界.
        \item 含参变量的无穷积分
            \[\int_a^{+\infty}f(x,y)\di x\]
            在$Y$上一致收敛.
    \end{enumerate}
    那么含参变量的无穷积分
    \[I(y)=\int_a^{+\infty}f(x,y)g(x,y)\di x\]
    在$Y$上一致收敛.
\end{formal}
与含参变量的正常积分一样,无穷积分也可以进行求导操作.
\begin{formal}[1.6 含参变量无穷积分的可微性]
    设函数$f(x,y)$和其对$y$的偏导函数$f_y(x,y)$在$[a,+\infty)\times[c,d]$上连续,并且积分
    \[I(y)=\int_a^{+\infty}f(x,y)\di x\]
    在$[c,d]$上点点收敛.又设积分
    \[\int_a^{+\infty}f_y(x,y)\di x\]
    在$[c,d]$上一致收敛,那么$I(y)$在$[c,d]$上可导,且
    \[I'(y)=\int_a^{+\infty}f_y(x,y)\di x\]

\end{formal}
瑕积分的相关定理与性质和无穷积分是类似的,在这里不再叙述.\\
\tbf{2.$\Gamma$函数和$\text{B}$函数}
\end{document}