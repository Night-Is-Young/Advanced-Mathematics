\documentclass{ctexart}
\usepackage{geometry}
\usepackage[dvipsnames,svgnames]{xcolor}
\usepackage{framed}
\usepackage{enumerate}
\usepackage{amsmath,amsthm,amssymb}
\usepackage{enumitem}
\usepackage{template}

\allowdisplaybreaks[4]
\geometry{left=2cm, right=2cm, top=2.5cm, bottom=2.5cm}

\begin{document}
\pagestyle{empty}
\begin{center}\Large\tbf{佛脚}\end{center}
\begin{theorem}[判断正项级数收敛的方法]
    \begin{enumerate}[label=\tbf{(\arabic*)},topsep=0pt,parsep=0pt,itemsep=0pt,partopsep=0pt]
        \item 比较判别法.
        \item 比较判别法的极限形式.如果$\left\{u_n\right\}$和$\left\{v_n\right\}$满足
            \[\lim_{n\to\infty}\dfrac{u_n}{v_n}=h\]
            当$0<h<+\infty$时,级数$\displaystyle\sum_{n=1}^{\infty}u_n$和$\displaystyle\sum_{n=1}^{\infty}v_n$同敛散.
        \item 达朗贝尔判别法.如果$\left\{u_n\right\}$满足
            \[\lim_{n\to\infty}\dfrac{u_{n+1}}{u_n}=l\]
            $l>1$时$\displaystyle\sum_{n=1}^{\infty}u_n$收敛,$l<1$时发散,$l=1$时敛散性不定.
        \item 柯西判别法.将\tbf{(2)}中的$\dfrac{u_{n+1}}{u_n}$替换为$\sqrt[n]{u_n}$,其余不变.
        \item 拉比判别法.如果$\left\{u_n\right\}$满足
            \[\lim_{n\to\infty}n\left(\dfrac{u_n}{u_{n+1}}-1\right)=R\]
            $R<1$时$\displaystyle\sum_{n=1}^{\infty}u_n$收敛,$R>1$时发散,$R=1$时敛散性不定.
    \end{enumerate}
\end{theorem}
\begin{theorem}[判断任意项级数收敛的方法]
    \begin{enumerate}[label=\tbf{(\arabic*)},topsep=0pt,parsep=0pt,itemsep=0pt,partopsep=0pt]
        \item 莱布尼茨判别法.如果$\left\{u_n\right\}$满足
            \[\lim_{n\to\infty}u_n=0\]
            且对充分大的$n$单调递减,那么交错级数$\displaystyle\sum_{n=1}^{\infty}(-1)^nu_n$收敛.
        \item 狄利克雷判别法.
            \begin{enumerate}[label=\tbf{\roman*.},topsep=0pt,parsep=0pt,itemsep=0pt,partopsep=0pt]
                \item $\left\{a_n\right\}$单调趋于$0$.
                \item $\left\{b_n\right\}$的部分和序列有界,即存在$M\in\R$使得对任意$k\in\N$有
                    \[\left|\sum_{n=1}^{k}b_n\right|\leqslant M\]
            \end{enumerate}
            那么任意项级数$\displaystyle\sum_{n=1}^{\infty}a_nb_n$收敛.
        \item 阿贝尔判别法.
            \begin{enumerate}[label=\tbf{\roman*.},topsep=0pt,parsep=0pt,itemsep=0pt,partopsep=0pt]
                \item $\left\{a_n\right\}$单调有界.
                \item $\displaystyle\sum_{n=1}^{\infty}b_n$收敛.
            \end{enumerate}
            那么任意项级数$\displaystyle\sum_{n=1}^{\infty}a_nb_n$收敛.
    \end{enumerate}
\end{theorem}
\begin{theorem}[判断函数项级数收敛的方法]
    对于给定的$x$,当作数项级数判断即可.常见于求含有$\sin(x),\cos(x)$等项的函数项级数的收敛域中.
\end{theorem}
\begin{theorem}[判断函数项级数一致收敛的方法]
     \begin{enumerate}[label=\tbf{(\arabic*)},topsep=0pt,parsep=0pt,itemsep=0pt,partopsep=0pt]
        \item 强级数判别法.如果对任意$x\in X$都有
            \[\left|u_n(x)\right|<a_n\]
            并且$\displaystyle\sum_{n=1}^{\infty}a_n$收敛,那么函数项级数$\displaystyle\sum_{n=1}^{\infty}u_n(x)$在$x\in X$上一致收敛.
        \item 狄利克雷判别法.
            \begin{enumerate}[label=\tbf{\roman*.},topsep=0pt,parsep=0pt,itemsep=0pt,partopsep=0pt]
                \item 对任意取定的$x\in X$,$a_n(x)$对$n$单调趋于$0$.
                \item $\left\{b_n\right\}$的部分和序列在$x\in X$上一致有界,即存在$M\in\R$使得对任意$x\in X$和任意$k\in\N$有
                    \[\left|\sum_{n=1}^{k}b_n(x)\right|\leqslant M\]
            \end{enumerate}
            那么函数项级数$\displaystyle\sum_{n=1}^{\infty}a_n(x)b_n(x)$在$x\in X$上一致收敛.
        \item 阿贝尔判别法.
            \begin{enumerate}[label=\tbf{\roman*.},topsep=0pt,parsep=0pt,itemsep=0pt,partopsep=0pt]
                \item 对任意取定的$x\in X$,$a_n(x)$对$n$单调且一致有界.
                \item $\displaystyle\sum_{n=1}^{\infty}b_(x)$在$x\in X$上一致收敛.
            \end{enumerate}
            那么函数项级数$\displaystyle\sum_{n=1}^{\infty}a_n(x)b_(x)$在$x\in X$上一致收敛.
    \end{enumerate}
\end{theorem}
\end{document}