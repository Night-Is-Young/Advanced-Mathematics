 \documentclass{ctexart}
\usepackage{geometry}
\usepackage[dvipsnames,svgnames]{xcolor}
\usepackage{framed}
\usepackage{enumerate}
\usepackage{amsmath,amsthm,amssymb}
\usepackage{enumitem}
\usepackage{template}

\allowdisplaybreaks[4]
\geometry{left=2cm, right=2cm, top=2.5cm, bottom=2.5cm}

\begin{document}
\pagestyle{empty}
\begin{center}\large 幂级数和泰勒级数\end{center}
\tbf{1.幂级数和幂级数的收敛半径}
\begin{definition}[1.1 定义:幂级数]
    形如
    \[\sum_{n=0}^\infty a_n\left(x-x_0\right)^n\]
    的无穷级数称作\tbf{幂级数}.其中$a_1,a_2,\cdots$是常数.
\end{definition}
\begin{formal}[1.2 幂级数的收敛半径]
    幂级数的收敛域要么是一个点$\left\{x_0\right\}$,要么是一个以$x_0$为中心的一个区间(两端开闭不定).%
    该区间长度的一半称作幂级数的\tbf{收敛半径}.
\end{formal}
\begin{lemma}
    \begin{enumerate}[label=\tbf{\arabic*.}]
        \item 如果幂级数$\displaystyle\sum_{n=0}^\infty a_nx^n$在$x_1$处收敛,那么对于任意$x$满足$|x|<|x_1|$,$\displaystyle\sum_{n=0}^\infty a_nx^n$都在$x$处绝对收敛.
        \item 如果幂级数$\displaystyle\sum_{n=0}^\infty a_nx^n$在$x_2$处发散,那么对于任意$x$满足$|x|>|x_1|$,$\displaystyle\sum_{n=0}^\infty a_nx^n$都在$x$处发散.
    \end{enumerate}
\end{lemma}\noindent
引理的证明从略,可以由比较审敛法不难得到.这样,我们就得到如下定律.
\begin{formal}[1.3 幂级数的收敛半径]
    对于幂级数$\displaystyle\sum_{n=0}^\infty a_nx^n$,存在一个非负数$R(R\text{可以为}+\infty)$,使得
    \begin{enumerate}[label=\tbf{\arabic*.}]
        \item 当$|x|<R$时,$\displaystyle\sum_{n=0}^\infty a_nx^n$绝对收敛.
        \item 当$|x|>R$时,$\displaystyle\sum_{n=0}^\infty a_nx^n$发散.
        \item 当$|x|=R$时,$\displaystyle\sum_{n=0}^\infty a_nx^n$可能收敛也可能发散.
    \end{enumerate}
\end{formal}
\begin{definition}[1.4 幂级数的收敛半径与收敛区间]
    \tbf{1.3}中的$R$称为$\displaystyle\sum_{n=0}^\infty a_nx^n$的\tbf{收敛半径},$(-R,R)$称为$\displaystyle\sum_{n=0}^\infty a_nx^n$的\tbf{收敛区间}.
\end{definition}\noindent
\tbf{2.收敛半径的求法}
\begin{formal}[2.1 基于D'Alembert判别法的收敛半径求法]
    若幂级数$\displaystyle\sum_{n=0}^\infty a_nx^n$的相邻两项系数之比满足
    \[\lim_{n\to\infty}\left|\dfrac{a_{n+1}}{a_n}\right|=l\]
    设该幂级数的收敛半径为$R$,那么有
    \begin{enumerate}[label=\tbf{\arabic*.}]
        \item 如果$0<l<+\infty$,那么$R=\dfrac1l$.
        \item 如果$l=0$,那么$R=\infty$.
        \item 如果$l=+\infty$,那么$R=0$.
    \end{enumerate}
    广义地说,你可以认为$R=\dfrac1l$.
\end{formal}
\begin{formal}[2.2 基于Cauchy判别法的收敛半径求法]
    若幂级数$\displaystyle\sum_{n=0}^\infty a_nx^n$系数满足
    \[\lim_{n\to\infty}\sqrt[n]{\left|a_n\right|}=l\]
    设该幂级数的收敛半径为$R$,那么(广义地)有$R=\dfrac1l$.
\end{formal}\noindent
\tbf{3.幂级数的性质}
\begin{formal}[3.1 幂级数的内闭一致性]
    设幂级数$\displaystyle\sum_{n=0}^\infty a_nx^n$的收敛半径$R>0$,则有
    \begin{enumerate}[label=\tbf{\arabic*.}]
        \item 对任意$0<b<R$,$\displaystyle\sum_{n=0}^\infty a_nx^n$在$[-b,b]$上一致收敛.
        \item 若$\displaystyle\sum_{n=0}^\infty a_nx^n$在$x=R$收敛,那么它在$[0,R]$上一致收敛.
        \item 若$\displaystyle\sum_{n=0}^\infty a_nx^n$在$x=-R$收敛,那么它在$[-R,0]$上一致收敛.
    \end{enumerate}
\end{formal}
\begin{formal}[3.2 幂级数的连续性]
    幂级数$\displaystyle\sum_{n=0}^\infty a_nx^n$的和函数$S(x)$在$(-R,R)$连续.特别地,%
    如果$\displaystyle\sum_{n=0}^\infty a_nx^n$在$x=R$(或$x=-R$)收敛,那么$S(x)$在在$x=R$(或$x=-R$)右(左)连续.
\end{formal}
\begin{formal}[3.3 幂级数的和函数的逐项积分]
    幂级数$\displaystyle\sum_{n=0}^\infty a_nx^n$的和函数$S(x)$在$(-R,R)$内任意闭区间上可积,且可逐项求积分,并有
    \[\int_0^xS(t)\di t=\sum_{n=0}^\infty\int_0^xa_nt^n\di t=\sum_{n=0}^\infty\dfrac{a_n}{n+1}x^{n+1}\ \ \ \ \ -R<x<R\]
\end{formal}
\begin{formal}[3.4 幂级数的和函数的逐项求导]
    幂级数$\displaystyle\sum_{n=0}^\infty a_nx^n$的和函数$S(x)$在$(-R,R)$内任意闭区间上可导,且可逐项求导,并有
    \[S'(x)=\sum_{n=0}^\infty\left(a_nt^n\right)'=\sum_{n=0}^\infty na_nx^{n-1}\ \ \ \ \ -R<x<R\]
\end{formal}
\begin{lemma}\\
    幂级数$\displaystyle\sum_{n=0}^\infty a_nx^n$的收敛半径记为$R$,%
    由幂级数逐项积分所得的新幂级数$\displaystyle\sum_{n=0}^\infty\dfrac{a_n}{n+1}x^{n+1}$的收敛半径记为$R_1$,%
    由幂级数逐项求导所得的新幂级数$\displaystyle\sum_{n=0}^\infty na_nx^{n-1}$的收敛半径记为$R_2$,%
    则有$R=R_1=R_2$.
\end{lemma}
\noindent\tbf{4.泰勒级数}\\
关于泰勒级数的理论这里就不再多说.以下是几个常见函数的泰勒级数.
\begin{enumerate}[label=\tbf{(\arabic*)},leftmargin=*]
    \item $\e^x=1+x+\dfrac{x^2}{2!}+\dfrac{x^3}{3!}+\cdots=\displaystyle\sum_{n=0}^\infty\dfrac{x^n}{n!}$\ \ \ 收敛域$(-\infty,+\infty)$.
    \item $\sin x=x-\dfrac{x^3}{3!}+\dfrac{x^5}{5!}-\cdots=\displaystyle\sum_{n=0}^\infty\dfrac{(-1)^{n}x^{2n+1}}{(2n+1)!}$\ \ \ 收敛域$(-\infty,+\infty)$.
    \item $\cos x=1-\dfrac{x^2}{2}+\dfrac{x^4}{4!}+\cdots=\displaystyle\sum_{n=0}^\infty\dfrac{(-1)^{n}x^{2n}}{(2n)!}$\ \ \ 收敛域$(-\infty,+\infty)$.
    \item $\arctan x=x-\dfrac{x^3}{3}+\dfrac{x^5}{5}-\cdots=\displaystyle\sum_{n=0}^\infty\dfrac{(-1)^nx^{2n+1}}{(2n+1)!}$\ \ \ 收敛域$[-1,1]$.
    \item $\ln(1+x)=x-\dfrac{x^2}{2}+\dfrac{x^3}{3}-\cdots=\displaystyle\sum_{n=0}^\infty\dfrac{(-1)^nx^{n+1}}{n+1}$\ \ \ 收敛域$(-1,1]$.
    \item $\sqrt{x+1}=1+\dfrac{1}{2}x-\dfrac{1}{8}x^2+\cdots=1+\displaystyle\sum_{n=1}^\infty\dfrac{(2n-3)!!}{(2n)!!}(-1)^{n-1}x^n$\ \ \ 收敛域$[-1,1]$.
    \item $\dfrac{1}{\sqrt{x+1}}=1-\dfrac12x+\dfrac{3}{8}x^2+\cdots=\displaystyle\sum_{n=0}^\infty\dfrac{(2n-1)!!}{(2n)!!}(-1)^nx^n$\ \ \ 收敛域$(-1,1)$.
    \item $\dfrac{1}{1+x}=1-x+x^2-\cdots=\displaystyle\sum_{n=0}^\infty(-1)^nx^n$\ \ \ 收敛域$(-1,1)$.
    \item $(1+x)^\alpha=1+\alpha x+\dfrac{\alpha(\alpha-1)}{2!}x^2+\cdots=1+\displaystyle\sum_{n=1}^\infty\dfrac{\alpha(\alpha-1)\cdots(\alpha-n)}{n!}x^n$\ \ \ 收敛域$(-1,1)$.
\end{enumerate}
\end{document}