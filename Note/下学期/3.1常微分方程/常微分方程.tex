\documentclass{ctexart}
\usepackage{geometry}
\usepackage[dvipsnames,svgnames]{xcolor}
\usepackage{framed}
\usepackage{enumerate}
\usepackage{amsmath,amsthm,amssymb}
\usepackage{enumitem}
\usepackage{template}

\allowdisplaybreaks
\geometry{left=2cm, right=2cm, top=2.5cm, bottom=2.5cm}

\begin{document}
\pagestyle{empty}
\begin{center}\large 常微分方程\end{center}
\tbf{1.基本概念}
\begin{definition}[1.1 定义:常微分方程]
    一般来说,一个联系自变量$x$,未知的一元函数$y=y(x)$及其导数$y^{(j)}(x)(0<j\leqslant n)$的方程
    \[F\left(x,y,y',\cdots,y^{(n)}\right)=0\]
    称为一个$n$\tbf{阶常微分方程}.\\
    如果区间$(a,b)$上存在一个函数$y=y(x)$,它有$n$阶导数且满足
    \[F\left(x,y,y',\cdots,y^{(n)}\right)\equiv0,\forall x\in(a,b)\]
    那么称$y=y(x)$是该常微分方程在$(a,b)$上的一个\tbf{解}.
\end{definition}
\begin{definition}[1.2 定义:通解与特解]
    若$n$阶常微分方程$F\left(x,y,y',\cdots,y^{(n)}\right)=0$有解
    \[y=\varphi\left(x;\li C,n\right)\]
    其中$\li C,n$是独立的常数,那么我们称上面的函数为该方程的\tbf{通解}.\\
    反之,上述方程的任意一个不包含任意常数的解都被称作方程的\tbf{特解}.
\end{definition}\noindent
需要注意的是,通解中的独立常数的数目恰好等于方程的阶数.\\
我们所说的独立,浅显的理解是每个常数都是不可替代的.更精确的说法是:如果$\varphi\left(x;\li C,n\right)$中的$n$个常数是独立的,那么
\[\dfrac{\text{D}\left(\varphi,\varphi',\cdots,\varphi^{(n-1)}\right)}{\text{D}\left(\li C,n\right)}\neq0,\forall x\in (a,b)\]
虽然通解包含了大部分特解,但是仍有一些特解是不被包含于其中的.
\begin{definition}[1.3 定义:通积分]
    有时,微分方程的解并不能写成显式的形式,而是用隐函数$\Phi\left(x,y;\li C,n\right)=0$给出.这时我们把该隐函数称作微分方程的\tbf{通积分}.
\end{definition}\noindent
\tbf{2.初等积分法}\\
对于一系列的具有一定特殊结构的微分方程都是可以通过初等积分法来解决的.下面我们介绍这些方法.
\begin{theorem}[2.1 变量分离的方程]
    形如
    \[\dfrac{\di y}{\dx}=f(x)\cdot g(y)\]
    的方程被称为\tbf{变量分离的方程}.\\
    变量分离的方程的通常解法为移项,使其成为
    \[\dfrac{\di y}{g(y)}=f(x)\di x\]
    然后在上式两边求积分,即可得到
    \[\int\dfrac{1}{g(y)}\di y=\int{f(x)\di x}\]
    上式两端分别是关于$y$和$x$的函数.求解此函数方程即可得到微分方程的解.
\end{theorem}
\begin{theorem}[2.2 可化为变量分离的方程的几类方程]
    有些方程可以通过适当的代换将其转化为变量分离的方程.
    \begin{enumerate}[label=\tbf{\arabic*.}]
        \item 形如
            \[\dfrac{\di y}{\di x}=f\left(ax+by+c\right)\]
            的方程,其中$a,b,c$为常数.我们做代换$z=ax+by+cz$,即可得到
            \[\dfrac{\di z}{\di x}=a+bf(z)\]
            这是一个变量分离的方程,按正常方法求解即可.
        \item 形如
            \[\dfrac{\di y}{\dx}=f(x,y)\]
            的方程,其中$f(x,y)$是关于$x,y$的齐次函数(我们略去齐次函数的严格定义).\\
            做代换$u=\dfrac yx$,并令$h(u)=f(x,y)$,则有
            \[\dfrac{\di u}{\dx}=\dfrac{h(u)-u}{x}\]
            这是一个变量分离的方程,按正常方法求解即可.
        \item 形如
            \[\dfrac{\di y}{\di x}=f\left(\dfrac{a_1x+b_1y+c_1}{a_2x+b_2y+c_2}\right)\]
            的方程(其中各$a,b,c$均为常数).我们再分情况讨论.
            \begin{enumerate}[label=\tbf{\alph*.}]
                \item $c_1=c_2=0$,则上述微分方程即\tbf{2.}所述的齐次方程.
                \item $\Delta=\begin{vmatrix}
                        a_1&b_1\\a_2&b_2
                    \end{vmatrix}\neq0$.则方程组
                    \[\left\{\begin{array}{l}
                        a_1x+b_1y+c_1=0\\a_2x+b_2y+c_2=0
                    \end{array}\right.\]
                    有唯一解,记此解为$\left(x_0,y_0\right)$.做代换$u=x-x_0,v=y-y_0$,则上述微分方程即
                    \[\dfrac{\di v}{\di u}=f\left(\dfrac{a_1u+b_1v}{a_2u+b_2v}\right)\]
                    这也是一个齐次方程,将其解出后回代即可.
                \item $\Delta=0$.则\tbf{b.}所示的方程组无解或有无数组解.我们再分情况讨论.
                    \begin{enumerate}[label=\tbf{\roman*.}]
                        \item 当$a_1\cdot b_1\neq0$时,存在常数$k$使得$\left(a_2,b_2\right)=k\left(a_1,b_1\right)$.令$z=a_1x+b_1y$,则有
                            \[\dfrac{\di z}{\di x}=a_1+b_1\dfrac{\di y}{\di x}=a_1+b_1f\left(\dfrac{z+c_1}{kz+c_2}\right)\]
                            这也是一个齐次方程.
                        \item 当$a_1\neq b_1=0$时,可知$b_2=0$.这时方程即为
                            \[\dfrac{\di y}{\di x}=f\left(\dfrac{a_1x+c_1}{a_2x+c_2}\right)\]
                            这是一个变量分离的方程.\\
                            当$b_1\neq a_1=0$时亦可采用相似的方法处理.
                        \item 当$a_1=b_1=0$时,上述方程即与\tbf{2.}相同.
                    \end{enumerate}
            \end{enumerate}
    \end{enumerate}
\end{theorem}
\end{document}