 \documentclass{ctexart}
\usepackage{geometry}
\usepackage[dvipsnames,svgnames]{xcolor}
\usepackage{framed}
\usepackage{enumerate}
\usepackage{amsmath,amsthm,amssymb}
\usepackage{enumitem}
\usepackage{template}

\allowdisplaybreaks[4]
\geometry{left=2cm, right=2cm, top=2.5cm, bottom=2.5cm}

\begin{document}
\pagestyle{empty}
\begin{center}\large 函数项级数\end{center}
\tbf{1.函数项级数}
\begin{definition}[函数项级数]
   设$u_n(x)(n=1,2,\cdots)$是定义在$D$上的函数,和式
   \[\sum_{n=1}^{\infty}u_n(x)=u_1(x)+u_2(x)+\cdots\]
   被称为定义在$D$上的\tbf{函数项级数}.在$D$上取定一点$x_0$,如果数项级数
   \[\sum_{n=1}^{\infty}u_n\left(x_0\right)\]
   收敛(发散),则称$x_0$为该函数项级数的\tbf{收敛点}(\tbf{发散点}).因此,函数项级数的敛散性是以数项级数的敛散性为基础的.\\
   函数项级数的收敛点的全体称为它的\tbf{收敛域},发散点的全体称为它的\tbf{发散域}.%
   对收敛域$X$内的任意一点$x$,上述级数的和记为$S(x)$.显然,$S$是定义在$X$上的函数,称为级数的\tbf{和函数}.
\end{definition}
\noindent\tbf{2.函数序列及函数项级数的一致收敛性}
\begin{definition}[2.1 函数序列的收敛性和极限函数]
    设有一个函数序列$f_1(x),f_2(x),\cdots$,其中的每一项$f_n(x)$在集合$D$上有定义.\\
    若一点$x_0\in D$使得序列$\left\{f_n\left(x_0\right)\right\}$收敛,即极限$\displaystyle\lim_{n\to\infty}f_x\left(x_0\right)$%
    存在,则称序列$\left\{f_n(x)\right\}$\tbf{在$x_0$处收敛},$x_0$称为该序列的\tbf{收敛点},该序列的全体收敛点构成的集合称作序列的\tbf{收敛域}.\\
    另外,序列$\left\{f_n(x)\right\}$在其收敛域$X$内定义了一个函数
    \[f(x)=\lim_{n\to\infty}f_n(x)\ \ \ \ \ \forall x\in X\]
    称$f(x)$为该序列的\tbf{极限函数}.
\end{definition}\noindent
显然,函数项级数$\displaystyle\sum_{n=1}^{\infty}f_n(x)$的收敛域就是其部分和序列的收敛域,其和函数就是部分和序列的极限函数.
\begin{definition}[2.2 一致收敛]
    设函数序列$\left\{f_n(x)\right\}$在集合$X$上收敛于极限函数$f(x)$.若对于任意$\ep>0$,都存在一个%
    只依赖于$\ep$而不依赖于$x$的正整数$N$使得$\forall n>N$都有$\left|f_n(x)-f(x)\right|<\ep$对任意$x\in X$成立,%
    则称函数序列$\left\{f_n(x)\right\}$在$X$上\tbf{一致收敛}于$f(x)$,记作$f_n(x)\rightrightarrows f(x),x\in X(n\to\infty)$.
\end{definition}\noindent
一致收敛的几何意义是,对于任意给定的$\ep>0$,都存在足够大的$n$使得$f_n(x)$落在带状区域
\[f(x)-\ep<f_n(x)<f(x)+\ep\]
中.需要注意的是,上述定义中的$X$不一定是序列的收敛域,可能只是收敛域的一个子集.
\begin{formal}[2.3 一致收敛的判据I]
    设函数序列$\left\{f_n(x)\right\}$在区间$X$上收敛于极限函数$f(x)$,若存在序列$\{a_n\}$使得
    \[\left|f_n(x)-f(x)\right|\leqslant a_n\ \ \ \ \ x\in X,n\geqslant N\]
    且$\displaystyle\lim_{n\to\infty}a_n=0$,则$\left\{f_n(x)\right\}$在$X$上一致收敛于$f(x)$.
\end{formal}
\begin{proof}
    对于任意$\ep>0$,由于$\displaystyle\lim_{n\to\infty}a_n=0$,则存在$N\in\N^*$使得对任意$n\geqslant N$有
    \[\left|a_n\right|<\ep\]
    于是对任意$\ep>0$,考虑满足上述条件的$N$,对任意$n\geqslant N$和$x\in X$有
    \[\left|f_n(x)-f(x)\right|\leqslant a_n<\ep\]
    因而$\left\{f_n\left(x\right)\right\}$一致收敛于$f(x)$.
\end{proof}
\begin{formal}[2.4 不一致收敛的判据I]
    设函数序列$\left\{f_n(x)\right\}$在$X$上收敛到极限函数$f(x)$.若存在常数$l>0$及点列$x_n\in X(n=1,2,\cdots)$使得当$n\geqslant N(N\in\N^*)$时有
    \[\left|f_n(x_n)-f(x_n)\right|\geqslant l\]
    则$\left\{f_n(x)\right\}$在$X$上不一致收敛.\\
    这一定理还有一极限版本.\\
    设函数序列$\left\{f_n(x)\right\}$在$X$上收敛到极限函数$f(x)$.若存在常数$l>0$及点列$x_n\in X(n=1,2,\cdots)$使得
    \[\lim_{n\to\infty}\left[f_n(x_n)-f(x_n)\right]=k\neq0\]
    则$\left\{f_n(x)\right\}$在$X$上不一致收敛.
\end{formal}
\noindent\tbf{3.函数项级数一致收敛的必要条件和判别法}
\begin{formal}[3.1]
    若函数项级数$\displaystyle\sum_{n=1}^\infty u_n(x)$在$X$上一致收敛,那么其一般项序列$\left\{u_n(x)\right\}$也在$X$上一致收敛于$0$.
\end{formal}
\begin{formal}[3.2 一致收敛的Cauchy准则]
    函数项级数$\displaystyle\sum_{n=1}^\infty u_n(x)$在$X$上一致收敛,当且仅当对于任意给定的$\ep>0$,存在一个只依赖于$\ep$的$N\in\N^*$,使得对任意$n>N$和任意$p\in\N^*$都有
    \[\left|\sum_{k=n+1}^{n+p}u_k(x)\right|<\ep\]
    对任意$x\in X$成立.
\end{formal}
\begin{formal}[3.3 强级数判别法(Weierstrass判别法)]
    若函数项级数$\displaystyle\sum_{n=1}^\infty u_n(x)$的一般项满足
    \[\left|u_n(x)\right|<a_n\ \ \ \ \ \forall x\in X,n=1,2,\cdots\]
    且正项级数$\displaystyle\sum_{n=1}^\infty a_n$收敛,那么该函数项级数在$X$上一致收敛.
\end{formal}
\begin{definition}[3.4 一致有界]
    设函数序列$\left\{f_n(x)\right\}$在$X$上有定义.若存在常数$M$,使得对任意$n=1,2,\cdots$和任意$x\in X$都有
    \[\left|f_n(x)\right|<M\]
    则称该函数序列在$X$上一致有界.
\end{definition}
\begin{formal}[3.5 Dirichlet判别法]
    设函数项级数$\displaystyle\sum_{n=1}^\infty u_n(x)$在$X$上有定义,且通项$u_n(x)$可以写成
    \[u_n(x)=a_n(x)\cdot b_n(x)\ \ \ \ \ \forall x\in X\]
    若$u_n(x)$满足
    \begin{enumerate}[label=\tbf{\arabic*.}]
        \item 在$X$中任意取定$x$,数列$\left\{a_n(x)\right\}$对$n$单调,且函数序列$\left\{a_n(x)\right\}$在$X$上一致收敛于$0$.
        \item 函数项级数$\displaystyle\sum_{n=1}^{\infty}b_n(x)$的部分和序列$\left\{B_n(x)\right\}$在$X$上一致有界.
    \end{enumerate}
    则$\displaystyle\sum_{n=1}^\infty u_n(x)$在$X$上一致收敛.
\end{formal}\noindent
只需将数项级数的Dirichlet判别法中$\{a_n\}$收敛于$0$和$\{B_n\}$有界分别换成$\{a_n(x)\}$\tbf{一致}收敛于$0$和%
$\{B_n(x)\}$\tbf{一致}有界,就可以得出$\displaystyle\sum_{n=1}^\infty a_n(x)b_n(x)$\tbf{一致}收敛.
\begin{formal}[3.6 Abel判别法]
    设函数项级数$\displaystyle\sum_{n=1}^\infty u_n(x)$在$X$上有定义,且通项$u_n(x)$可以写成
    \[u_n(x)=a_n(x)\cdot b_n(x)\ \ \ \ \ \forall x\in X\]
    若$u_n(x)$满足
    \begin{enumerate}[label=\tbf{\arabic*.}]
        \item 在$X$中任意取定$x$,数列$\left\{a_n(x)\right\}$对$n$单调,且函数序列$\left\{a_n(x)\right\}$在$X$上一致有界.
        \item 函数项级数$\displaystyle\sum_{n=1}^{\infty}b_n(x)$在$X$上一致收敛.
    \end{enumerate}
    则$\displaystyle\sum_{n=1}^\infty u_n(x)$在$X$上一致收敛.
\end{formal}\noindent
同样地,只需将数项级数的Abel判别法中的有界和收敛换成一致有界和一致收敛即可.\\
\tbf{4.一致收敛级数的性质}
\begin{formal}[4.1 和函数的连续性]
    设函数项级数$\displaystyle\sum_{n=1}^\infty u_n(x)$在$[a,b]$上一致收敛,且其每一个通项$u_n(x)(n=1,2,\cdots)$在$[a,b]$上都连续,则%
    其和函数$S(x)=\displaystyle\sum_{n=1}^\infty u_n(x)$在$[a,b]$上也连续.
\end{formal}\noindent
这就说明当级数一致收敛且各项连续时,无穷多项的求和运算与求极限的运算可以交换次序.
\begin{formal}[4.2 逐项求积分]
    设函数项级数$\displaystyle\sum_{n=1}^\infty u_n(x)$在$[a,b]$上一致收敛,且其每一个通项$u_n(x)(n=1,2,\cdots)$在$[a,b]$上都连续,则%
    其和函数$S(x)=\displaystyle\sum_{n=1}^\infty u_n(x)$在$[a,b]$上可积,而且可以逐项积分,即
    \[\int_a^b S(x)\dx=\sum_{n=1}^\infty\int_a^b u_n(x)\dx\]
\end{formal}
\begin{formal}[4.3 逐项求导]
    设函数项级数$\displaystyle\sum_{n=1}^\infty u_n(x)$在$[a,b]$上点点收敛,且通项$u_n(x)(n=1,2,\cdots)$的导函数$u'(x)$在$[a,b]$上都连续,且级数%
    $\displaystyle\sum_{n=1}^\infty u_n'(x)$在$[a,b]$上一直连续,则和函数$S(x)=\displaystyle\sum_{n=1}^\infty u_n(x)$在$[a,b]$上可导,且
    \[S'(x)=\sum_{n=1}^\infty u_n'(x)\ \ \ \ \ \forall x\in[a,b]\]
    并且$S'(x)$在$[a,b]$上也连续.
\end{formal}
\end{document}